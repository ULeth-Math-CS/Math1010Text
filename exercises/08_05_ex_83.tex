{ \label{eulerformulaexercise} Another way to express the polar form of a complex number is to use the exponential function.  For real numbers $t$, \href{http://en.wikipedia.org/wiki/Leonhard_Euler}{\underline{Euler}}'s Formula defines $e^{it} = \cos(t) + i \sin(t)$.  
\begin{enumerate} 
\item  Use Theorem \ref{prodquotpolarcomplex} to show that $e^{ix} e^{iy} = e^{i(x+y)}$ for all real numbers $x$ and $y$. 
\item  Use Theorem \ref{prodquotpolarcomplex} to show that $\left(e^{ix}\right)^{n} = e^{i(nx)}$ for any real number $x$ and any natural number $n$. 
\item  Use Theorem \ref{prodquotpolarcomplex} to show that $\dfrac{e^{ix}}{e^{iy}} = e^{i(x-y)}$ for all real numbers $x$ and $y$. 
\item  If $z = r\operatorname{cis}(\theta)$ is the polar form of $z$, show that $z = re^{it}$ where $\theta = t$ radians. 
\item  Show that $e^{i\pi} + 1 = 0$.  (This famous equation relates the five most important constants in all of Mathematics with the three most fundamental operations in Mathematics.) 
\item   \label{expformcosandsin} Show that $\cos(t) = \dfrac{e^{it} + e^{-it}}{2}$ and that $\sin(t) = \dfrac{e^{it} - e^{-it}}{2i}$ for all real numbers $t$. 
\end{enumerate}}
{}
