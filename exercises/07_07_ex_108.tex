{With the help of your classmates, determine the number of solutions to $\sin(x) = \frac{1}{2}$ in $[0,2\pi)$.  Then find the number of solutions to $\sin(2x) = \frac{1}{2}$,  $\sin(3x) = \frac{1}{2}$ and $\sin(4x) = \frac{1}{2}$ in $[0,2\pi)$.  A pattern should emerge.  Explain how this pattern would help you solve equations like $\sin(11x) = \frac{1}{2}$.  Now consider $\sin\left(\frac{x}{2}\right)  = \frac{1}{2}$,  $\sin\left(\frac{3x}{2}\right)  = \frac{1}{2}$ and $\sin\left(\frac{5x}{2}\right)  = \frac{1}{2}$.  What do you find?  Replace $\dfrac{1}{2}$ with $-1$ and repeat the whole exploration.}
{} 
