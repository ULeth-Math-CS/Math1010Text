{ Given any natural number $n \geq 2$, the $n$ complex $n^{\text{th}}$ roots of the number $z = 1$ are called the \textbf{\boldmath $n^{\mbox{\textbf{\scriptsize th}}}$ Roots of Unity}. \index{$n^{\textrm{th}}$ Roots of Unity} \index{complex number ! $n^{\textrm{th}}$ Roots of Unity} \index{Roots of Unity} In the following exercises, assume that $n$ is a fixed, but arbitrary, natural number such that $n \geq 2$. 
\begin{enumerate} 
\item  Show that $w = 1$ is an $n^{\text{th}}$ root of unity. 
\item  Show that if both $w_{\text{\tiny$j$}}$ and $w_{\text{\tiny$k$}}$ are $n^{\text{th}}$ roots of unity then so is their product $w_{\text{\tiny$j$}}w_{\text{\tiny$k$}}$. 
\item  Show that if $w_{\text{\tiny$j$}}$ is an $n^{\text{th}}$ root of unity then there exists another $n^{\text{th}}$ root of unity $w_{\text{\tiny$j'$}}$ such that $w_{\text{\tiny$j$}}w_{\text{\tiny$j'$}} = 1$.  Hint: If $w_{\text{\tiny$j$}} = \operatorname{cis}(\theta)$ let $w_{\text{\tiny$j'$}} = \operatorname{cis}(2\pi - \theta)$. You'll need to verify that $w_{\text{\tiny$j'$}} = \operatorname{cis}(2\pi - \theta)$ is indeed an $n^{\text{th}}$ root of unity. 
\end{enumerate}}
{}
