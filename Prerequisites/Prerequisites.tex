The course material for Math 1010 consists of two texts. The first, \textit{Precalculus} (this text) covers the theory of functions needed to proceed with the material in the second text, \textit{Calculus I}. However, history has taught us that before proceeding with our study of functions, there are certain prerequisite skills\footnote{Working with fractions, for example.} that are sometimes lacking in students and therefore must be reviewed before we proceed.  The goal of Chapter \ref{Prerequisites} is exactly that: to review the concepts, skills and vocabulary we believe are prerequisite to a rigorous, college-level Precalculus course.  This review is not designed to teach the material to students who have never seen it before thus the presentation is more succinct and the exercise sets are shorter than those usually found in an high school algebra text.  An outline of the chapter is given below.

\smallskip

Section \ref{SetTheory} (Basic Set Theory and Interval Notation) contains a brief summary of the set theory terminology used throughout the text including sets of real numbers and interval notation.

\smallskip

Section \ref{RealNumberArithmetic} (Real Number Arithmetic) lists the properties of real number arithmetic.\footnote{You know, the stuff students mess up all of the time like fractions and negative signs.  The collection is close to exhaustive and definitely exhausting!}

\smallskip

Section \ref{LinearEqIneq} (Linear Equations and Inequalities) focuses on solving linear equations and linear inequalities from a strictly algebraic perspective.  The geometry of graphing lines in the plane is deferred until Section \ref{LinearFunctions} (Linear Functions).

\smallskip

Section \ref{AbsValEqIneq} (Absolute Value Equations and Inequalities) begins with a definition of absolute value as a distance.  Fundamental properties of absolute value are listed and then basic equations and inequalities involving absolute value are solved using the `distance definition' and those properties.  Absolute value is revisited in much greater depth in Section \ref{AbsoluteValueFunctions} (Absolute Value Functions).

\smallskip

Section \ref{PolyArith} (Polynomial Arithmetic) covers the addition, subtraction, multiplication and division of polynomials as well as the vocabulary which is used extensively when the graphs of polynomials are studied in Chapter \ref{Polynomials} (Polynomials).

\smallskip

Section \ref{Factoring} (Factoring) covers basic factoring techniques and how to solve equations using those techniques along with the Zero Product Property of Real Numbers.

\smallskip

Section \ref{QuadEqus} (Quadratic Equations) discusses solving quadratic equations using the technique of `completing the square' and by using the Quadratic Formula.  Equations which are `quadratic in form' are also discussed.

\smallskip

Section \ref{RatExpEqus} (Rational Expressions and Equations) starts with the basic arithmetic of rational expressions and the simplifying of compound fractions.  Solving equations by clearing denominators and the handling negative integer exponents are presented but the graphing of rational functions is deferred until Chapter \ref{Rationals}  (Rational Functions).

\smallskip

Section \ref{RadEqus} (Radicals and Equations) covers simplifying radicals as well as the solving of basic equations involving radicals.  


\newpage

\section{Basic Set Theory and Interval Notation}

\mfpicnumber{1}

\opengraphsfile{SetTheory}

\setcounter{footnote}{0}

\label{SetTheory}

\subsection{Some Basic Set Theory Notions}
\label{SetTheoryNotions}

Like all good Math books, we begin with a definition.

\medskip

\colorbox{ResultColor}{\bbm

\begin{defn} \label{setdef}

A \textbf{set}\index{set ! definition of} is a well-defined collection of objects which are called the `elements' of the set.  Here, `well-defined' means that it is possible to determine if something belongs to the collection or not, without prejudice. 

\end{defn}

\ebm}

\medskip

The collection of letters that make up the word ``smolko'' is well-defined and is a set, but the collection of the worst Math teachers in the world is \textbf{not} well-defined and therefore is \textbf{not} a set.\footnote{For a more thought-provoking example, consider the collection of all things that do not contain themselves - this leads to the famous \href{http://en.wikipedia.org/wiki/Russell's_paradox}{\underline{Russell's Paradox}}.}  In general, there are three ways to describe sets and those methods are listed below.

\medskip

\phantomsection \label{waystodescribesets}

\colorbox{ResultColor}{\bbm

\centerline{\textbf{Ways to Describe Sets}}

\begin{enumerate}

\item \textbf{The Verbal Method:} Use a sentence to define the set.\index{set ! verbal description}

\item \textbf{The Roster Method:}  Begin with a left brace `$\{$', list each element of the set \textit{only once} and then end with a right brace `$\}$'.\index{set ! roster method}

\item \textbf{The Set-Builder Method:} A combination of the verbal and roster methods using a ``dummy variable'' such as $x$.\index{set ! set-builder notation}\index{set-builder notation}

\end{enumerate}

\ebm}

\medskip

For example, let $S$ be the set described \textit{verbally} as the set of letters that make up the word ``smolko''.  A  \textbf{roster} description of $S$ is $\left\{ s, m, o, l, k \right\}$. Note that we listed `o' only once, even though it appears twice in the word ``smolko''.  Also, the \textit{order} of the elements doesn't matter, so $\left\{ k,  l,  m, o, s \right\}$ is also a roster description of $S$.  Moving right along,  a \textbf{set-builder} description of $S$ is: $\{ x \, | \, \mbox{$x$ is a letter in the word ``smolko''}\}$.  The way to read this is `The set of elements $x$ \underline{such that} $x$ is a letter in the word ``smolko''.'   In each of the above cases, we may use the familiar equals sign `$=$' and write  $S = \left\{ s, m, o, l, k \right\}$ or $S = \{ x \, | \, \mbox{$x$ is a letter in the word ``smolko''}\}$.  

\smallskip

Notice that  $m$ is in $S$ but many other letters, such as $q$, are not in $S$.  We express these ideas of set inclusion and exclusion mathematically using the symbols $m \in S$ (read `$m$ \underline{is in} $S$') and $q \notin S$ (read `$q$ \underline{is not in} $S$').  More precisely, we have the following.

\medskip

\colorbox{ResultColor}{\bbm

\begin{defn} \label{notationforsetinclusion}  Let $A$ be a set.

\begin{itemize}

\item If $x$ is an element of $A$ then we write $x \in A$\index{set ! inclusion}\index{$\in$} which is read `$x$ is in $A$'.

\item If $x$ is \emph{not} an element of $A$ then we write $x \notin A$\index{set ! exclusion}\index{$\notin$} which is read `$x$ is not in $A$'.

\end{itemize}

\end{defn}

\ebm}

\medskip

Now let's consider the set $C =  \{ x \, | \, \mbox{$x$ is a consonant in the word ``smolko''}\}$.  A roster description of $C$ is  $C = \{ s, m, l, k\}$.  Note that by construction, every element of $C$ is also in $S$.  We express this relationship by stating that the set $C$ is a \textbf{subset} of the set $S$, which is written in symbols as $C \subseteq S$.  The more formal definition is given below.

\medskip

\colorbox{ResultColor}{\bbm

\begin{defn} \label{subsetdef}

Given sets $A$ and $B$, we say that the set $A$ is a \textbf{subset}\index{subset ! definition of} of the set $B$ and write `$A \subseteq B$' if every element in $A$ is also an element of $B$.  

\end{defn}

\ebm}

\medskip

Note that in our example above $C \subseteq S$, but not vice-versa, since $o \in S$ but $o \notin C$.  Additionally, the set of vowels $V = \{ a, e, i, o, u\}$, while it does have an element in common with $S$, is not a subset of $S$. (As an added note,  $S$ is not a subset of $V$, either.)  We could, however, \textit{build} a set which contains both $S$ and $V$ as subsets by gathering all of the elements in both $S$ and $V$ together into a single set, say $U = \{ s, m, o, l, k, a, e, i, u\}$.   Then $S \subseteq U$ and $V \subseteq U$.  The set $U$ we have built is called the \textbf{union} of the sets $S$ and $V$ and is denoted $S \cup V$.    Furthermore, $S$ and $V$ aren't completely \textit{different} sets since they both contain the letter `o.'  The \textbf{intersection} of two sets is the set of elements (if any) the two sets have in common. In this case, the intersection of $S$ and $V$ is $\{ o\}$, written $S \cap V = \{ o \}$.  We formalize these ideas below.

\medskip

\colorbox{ResultColor}{\bbm

\begin{defn} \label{intersectionunion}  Suppose $A$ and $B$ are sets.

\begin{itemize}

\item The \textbf{intersection}\index{set ! intersection}\index{intersection of two sets} of $A$ and $B$ is $A \cap B = \{ x \, | \, x \in A \, \text{and} \,\, x \in B \}$

\item The \textbf{union}\index{set ! union}\index{union of two sets} of $A$ and $B$ is $A \cup B = \{ x \, | \, x \in A \, \text{or} \,\, x \in B \, \, \text{(or both)} \}$

\end{itemize}

\end{defn}

\ebm}

\medskip

The key words in Definition \ref{intersectionunion} to focus on are the conjunctions:  `intersection' corresponds to `and' meaning the elements have to be in \textit{both} sets to be in the intersection, whereas `union' corresponds to `or' meaning the elements have to be in one set, or the other set (or both).  In other words, to belong to the union of two sets an element must belong to \textit{at least one} of them.

\smallskip

Returning to the sets $C$ and $V$  above, $C \cup V = \{ s, m, l, k, a, e, i, o, u\}$.\footnote{Which just so happens to be the same set as $S \cup V$.}  When it comes to their intersection, however, we run into a bit of notational awkwardness since $C$ and $V$ have no elements in common.  While we could write $C \cap V = \{ \}$, this sort of thing happens often enough that we give the set with no elements a name. 

\medskip

\colorbox{ResultColor}{\bbm

\begin{defn} \label{emptysetdefn}

The \textbf{Empty Set} $\emptyset$ is the set which contains no elements.\index{set ! empty}\index{empty set}  That is, \[\emptyset=\{ \}=\{x\,|\,\mbox{$x \neq x$}\}.\]  

\end{defn}

\ebm}

\medskip

As promised, the empty set is the set containing no elements since no matter what `$x$' is, `$x = x$.'  Like the number `$0$,'  the empty set plays a vital role in mathematics.\footnote{Sadly, the full extent of the empty set's role will not be explored in this text.} We introduce it here more as a symbol of convenience as opposed to a contrivance.\footnote{Actually, the empty set can be used to generate numbers -  mathematicians can create something from nothing!}   Using this new bit of notation, we have for the sets $C$ and $V$ above that $C \cap V = \emptyset$.    A nice way to visualize relationships between sets and set operations is to draw a  \href{http://en.wikipedia.org/wiki/Venn_diagram}{\underline{\textbf{Venn Diagram}}}\index{diagram ! Venn Diagram}\index{Venn Diagram}.  A Venn Diagram for the sets $S$, $C$ and $V$ is drawn at the top of the next page.  

\phantomsection
\label{venndiagram}

\begin{center}

\begin{mfpic}[10]{-10}{10}{-10}{10} 

\rect{(-10,-8),(10,8)}
\circle{(-3,0), 5}
\circle{(3,0), 5}
\circle{(-4,0), 1.5}
\tlabel[cc](-4,0){\scriptsize $s$ $m$ $l$ $k$}
\tlabel[cc](0,0){\scriptsize $o$}
\tlabel[cc](4,0){\scriptsize $a$ $e$ $i$ $u$}
\tlabel[cc](-3,-6){\scriptsize $S$}
\tlabel[cc](3,-6){\scriptsize $V$}
\tlabel[cc](-4,2){\scriptsize $C$}
\tlabel[cc](9,7){\scriptsize $U$}

\end{mfpic}

A Venn Diagram for $C$, $S$ and $V$.  

\end{center}

In the Venn Diagram above we have three circles - one for each of the sets $C$, $S$ and $V$.  We visualize the area enclosed by each of these circles as the elements of each set.  Here, we've spelled out the elements for definitiveness.  Notice that the circle representing the set $C$ is completely inside the circle representing $S$.  This is a geometric way of  showing that $C \subseteq S$.  Also, notice that the circles representing $S$ and $V$ overlap on the letter `o'.  This common region is how we visualize $S \cap V$.  Notice that since $C \cap V = \emptyset$, the circles which represent $C$ and $V$ have no overlap whatsoever.  

\medskip

All of these circles lie in a rectangle labeled $U$ (for `universal' set).  A universal set contains all of the elements under discussion, so it could always be taken as the union of all of the sets in question, or an even larger set.  In this case, we could take $U = S \cup V$ or $U$ as the set of letters in the entire alphabet.  The reader may well wonder if there is an ultimate universal set which contains \textit{everything}.  The short answer is `no' and we refer you once again to \href{http://en.wikipedia.org/wiki/Russell's_paradox}{\underline{Russell's Paradox}}.  The usual triptych of Venn Diagrams indicating generic sets $A$ and  $B$ along with $A \cap B$ and $A \cup B$ is given below.

\begin{center}

\begin{tabular}{ccc}


\begin{mfpic}[40]{-2}{2}{-2}{2}

\fillcolor[gray]{0.7}

 \circle{(0.4375,0),1}
 \circle{(-0.4375,0),1}
\tlabel[cc](-0.4375,-1.25){\scriptsize $A$}
\tlabel[cc](0.4375,-1.25){\scriptsize $B$}
\tlabel[cc](1.25,1.25){\scriptsize $U$}
\rect{(-1.5, -1.5), (1.5, 1.5)}

    
\end{mfpic}



&

\hspace{0.15in}

\begin{mfpic}[40]{-2}{2}{-2}{2}



\fillcolor[gray]{0.7}
\gfill\circle{(0.4375,0),1}
 \gclip\circle{(-0.4375,0),1}
 \circle{(0.4375,0),1}
 \circle{(-0.4375,0),1}

\tlabel[cc](0,0){\scriptsize $A \cap B$}
\tlabel[cc](-0.4375,-1.25){\scriptsize $A$}
\tlabel[cc](0.4375,-1.25){\scriptsize $B$}
\tlabel[cc](1.25,1.25){\scriptsize $U$}
\rect{(-1.5, -1.5), (1.5, 1.5)}
    
\end{mfpic}

&

\hspace{0.15in}

\begin{mfpic}[40]{-2}{2}{-2}{2}



\fillcolor[gray]{0.7}
\gfill\circle{(0.4375,0),1}
\gfill\circle{(-0.4375,0),1}
 \circle{(0.4375,0),1}
 \circle{(-0.4375,0),1}
\tlabel[cc](0,0){\scriptsize $A \cup B$}
\tlabel[cc](-0.4375,-1.25){\scriptsize $A$}
\tlabel[cc](0.4375,-1.25){\scriptsize $B$}
\tlabel[cc](1.25,1.25){\scriptsize $U$}
\rect{(-1.5, -1.5), (1.5, 1.5)}
    
\end{mfpic}

\\

Sets $A$ and $B$. 



&

\hspace{0.15in}

$A \cap B$ is shaded.

&

\hspace{0.15in}

$A \cup B$ is shaded.

\\
\end{tabular}

\end{center}

\subsection{Sets of Real Numbers}
\label{SetsofRealNumbers}

The playground for most of this text is the set of \textbf{Real Numbers}.  Many quantities in the `real world' can be quantified using real numbers: the temperature at a given time, the revenue generated by selling a certain number of products and the maximum population of Sasquatch which can inhabit a particular region are just three basic examples.   A succinct, but nonetheless incomplete\footnote{Math pun intended!} definition of a real number is given below.

\medskip

\colorbox{ResultColor}{\bbm

\begin{defn} \label{realnumberdefn}

A \textbf{real number}\index{real number ! set of}\index{real number ! definition of} is any number which possesses a decimal representation.  The set of real numbers is denoted by the character $\R$.  

\end{defn}

\ebm}

\medskip

Certain subsets of the real numbers are worthy of note and are listed below.  In fact, in more advanced texts,\footnote{See, for instance, Landau's \underline{Foundations of Analysis}.}   the real numbers are \textit{constructed} from some of these subsets.  

\medskip

\phantomsection
\label{setsofnumbersboxonthispage}

\colorbox{ResultColor}{\bbm

\centerline{\textbf{Special Subsets of Real Numbers}}\index{set ! sets of real numbers}

\begin{enumerate}

\item The \textbf{Natural Numbers}:\index{natural number ! set of}\index{natural number ! definition of} $\N= \{ 1, 2, 3,  \ldots\}$ The periods of ellipsis `$\ldots$' here indicate that the natural numbers contain $1$, $2$, $3$ `and so forth'.

\item The \textbf{Whole Numbers}:\index{whole number ! set of}\index{whole number ! definition of} $\W = \{ 0, 1, 2, \ldots \}.$

\item The \textbf{Integers}:\index{integer ! set of}\index{integer ! definition of} $\mathbb Z=\{ \ldots, -3, -2, -1, 0, 1, 2, 3, \ldots \} = \{ 0, \pm 1, \pm 2, \pm 3, \ldots\}.$\footnote{The symbol $\pm$ is read `plus or minus' and it is a shorthand notation which appears throughout the text.  Just remember that $x = \pm 3$ means $x = 3$ or $x = -3$.}

\item The \textbf{Rational Numbers}:\index{rational number ! set of}\index{rational number ! definition of} $\Q=\left\{\frac{a}{b} \, | \, a \in \mathbb Z \, \mbox{and} \, b \in \mathbb Z \right\}$.  \underline{Ratio}nal numbers are the \underline{ratio}s of integers where the denominator is not zero.  It turns out that another way to describe the rational numbers\footnote{See Section \ref{Summation}.} is: \[\Q=\{x\,|\,\mbox{$x$ possesses a repeating or terminating decimal representation}\}\]

\item The \textbf{Irrational Numbers}:\index{irrational number ! set of}\index{irrational number ! definition of} $\mathbb P = \{x\,|\,\mbox{$x \in \R$ but $x \notin \Q$}\}$.\footnote{Examples here include number $\pi$ (See Section \ref{Angles}), $\sqrt{2}$ and $0.101001000100001\ldots$.}  That is, an \underline{ir}rational number is a real number which isn't rational.  Said differently, \[\mathbb P = \{x\,|\,\mbox{$x$ possesses a decimal representation which neither repeats nor terminates}\}\]

\end{enumerate}

\ebm}

\medskip

Note that every natural number is a whole number which, in turn, is an integer.   Each integer is a rational number (take $b =1$ in the above definition for $\Q$) and since every rational number is a real number\footnote{Thanks to long division!}  the sets $\N$, $\W$, $\mathbb Z$, $\Q$, and  $\R$ are nested like \href{http://en.wikipedia.org/wiki/Matryoshka_doll}{\underline{Matryoshka dolls}}. More formally, these sets form a subset chain:  $\N \subseteq \W \subseteq \mathbb Z \subseteq \Q \subseteq \R$.  The reader is encouraged to sketch a Venn Diagram depicting $\R$ and all of the subsets mentioned above.  It is time for an example.

\begin{ex} \label{numbersetex}   $~$

\begin{enumerate}

\item  Write a roster description for $P = \{ 2^{n} \, | \, n \in \N \}$  and $E = \{ 2n \, | \, n \in \mathbb Z \}$.

\item Write a verbal description for $S = \{ x^2 \, | \, x \in \R \}$.

\item Let $A = \{-117, \frac{4}{5}, 0.20\overline{2002}, 0.202002000200002 \ldots\}$. 

\begin{enumerate}

\item Which elements of $A$ are natural numbers?  Rational numbers?  Real numbers?

\item Find $A \cap \W$, $A \cap \mathbb Z$ and $A \cap \mathbb P$.

\end{enumerate}

\item  What is another name for $\N \cup \Q$?  What about  $\Q \cup \mathbb P$?

\end{enumerate}

{\bf Solution.}

\begin{enumerate}

\item  To find a roster description for these sets, we need to list their elements.   Starting with $P = \{ 2^{n} \, | \, n \in \N \}$, we substitute natural number values $n$ into the formula $2^n$.  For $n = 1$ we get $2^1 = 2$,  for $n = 2$ we get $2^2 = 4$, for $n = 3$ we get $2^3 = 8$ and for $n = 4$ we get $2^4 = 16$.  Hence  $P$ describes the powers of $2$, so a roster description for $P$ is $P = \{ 2, 4, 8, 16, \ldots \}$ where the `$\ldots$' indicates the that pattern continues.\footnote{This isn't the most \textit{precise} way to describe this set - it's always dangerous to use `$\ldots$' since we assume that the pattern is clearly demonstrated and thus made evident to the reader.  Formulas are more precise because the pattern is clear.}  

\smallskip

Proceeding in the same way, we generate elements in $E = \{ 2n \, | \, n \in \mathbb Z \}$ by plugging in integer values of $n$ into the formula $2n$.  Starting with $n = 0$ we obtain $2(0) = 0$.  For $n = 1$ we get $2(1) = 2$, for $n = -1$ we get $2(-1) = -2$ for $n = 2$, we get $2(2) = 4$ and for $n = -2$ we get $2(-2) = -4$.  As $n$  moves through the integers, $2n$ produces all of the \textit{even} integers.\footnote{This shouldn't be too surprising, since an even integer is \textit{defined} to be an integer multiple of $2$.} A roster description for  $E$ is $E = \{ 0, \pm 2, \pm 4, \ldots \}$.

\item  One way to verbally describe $S$ is to say that $S$ is the `set of all squares of real numbers'.  While this isn't incorrect, we'd like to take this opportunity to delve a little deeper.\footnote{Think of this as an opportunity to stop and smell the mathematical roses.}  What makes the set $S = \{ x^2 \, | \, x \in \R \}$ a little trickier to wrangle than the sets $P$ or $E$ above is that the dummy variable here, $x$, runs through all \textit{real} numbers.  Unlike the natural numbers or the integers, the real numbers cannot be listed in any methodical way.\footnote{This is a nontrivial statement.  Interested readers are directed to a discussion of \href{http://en.wikipedia.org/wiki/Cantor's_diagonal_argument}{\underline{Cantor's Diagonal Argument}}.}  Nevertheless, we can select some real numbers, square them and get a sense of what kind of numbers lie in $S$.  For $x = -2$, $x^2 = (-2)^2 = 4$ so $4$ is in $S$, as are $\left(\frac{3}{2}\right)^2 = \frac{9}{4}$ and $(\sqrt{117})^2 = 117$.  Even things like $(-\pi)^2$ and $(0.101001000100001 \ldots)^2$ are in $S$.  

\smallskip

So suppose $s \in S$.  What can be said about $s$?  We know there is some real number $x$ so that $s = x^2$.  Since $x^2 \geq 0$ for any real number $x$, we know $s \geq 0$.  This tells us that everything in $S$ is a non-negative real number.\footnote{This means $S$ is a subset of the non-negative real numbers.}  This begs the question:  are \underline{all} of the non-negative real numbers in $S$?  Suppose $n$ is a non-negative real number, that is, $n \geq 0$.  If $n$ were in $S$, there would be a real number $x$ so that $x^2=n$.  As you may recall, we can solve $x^2 = n$ by `extracting square roots':  $x = \pm \sqrt{n}$.  Since $n \geq 0$, $\sqrt{n}$ is a real number.\footnote{This is called the `square root closed' property of the non-negative real numbers.}  Moreover, $(\sqrt{n})^2 = n$ so $n$ is the square of a real number which means $n \in S$. Hence, $S$ is the set of non-negative real numbers.

\item  \begin{enumerate} \item The set $A$ contains no natural numbers.\footnote{Carl was tempted to include $0.\overline{9}$ in the set $A$, but thought better of it.  See Section \ref{Summation} for details.}  Clearly, $\frac{4}{5}$ is a rational number as is $-117$ (which can be written as $\frac{-117}{1}$). It's the last two numbers listed in $A$, $0.20\overline{2002}$ and $0.202002000200002 \ldots$, that warrant some discussion.  First, recall that the `line' over the digits $2002$ in $0.20\overline{2002}$ (called the vinculum) indicates that these digits repeat, so it is a rational number.\footnote{So $0.20\overline{2002} = 0.20200220022002 \ldots$.}  As for the number $0.202002000200002 \ldots$, the `$\ldots$' indicates the pattern of adding an extra `$0$' followed by a `$2$' is what defines this real number.  Despite the fact there is a \textit{pattern} to this decimal, this decimal is \textit{not repeating}, so it is not a rational number - it is, in fact, an irrational number.  All of the elements of $A$ are real numbers, since all of them can be expressed as decimals (remember that $\frac{4}{5} = 0.8$).

\item  The set $A \cap \W = \{ x \, | \, \text{$x \in A$ and $x \in \W$} \}$ is another way of saying we are looking for the set of numbers in $A$ which are whole numbers.  Since $A$ contains no whole numbers, $A \cap \W = \emptyset$.  Similarly, $A \cap \mathbb Z$ is looking for the set of numbers in $A$ which are integers.  Since $-117$ is the only integer in $A$,  $A \cap \mathbb Z = \{ -117 \}$. As for the set $A \cap \mathbb P$, as discussed in part (a), the number $0.202002000200002 \ldots$ is irrational, so $A \cap \mathbb P = \{ 0.202002000200002 \ldots \}$.

\end{enumerate}

\item  The set $\N \cup \Q = \{ x \, | \, \text{$x \in \N$ or $x \in \Q$} \}$ is the union of the set of natural numbers with the set of rational numbers.  Since every natural number is a rational number, $\N$ doesn't contribute any new elements to $\Q$, so $\N \cup \Q = \Q$.\footnote{In fact, anytime $A \subseteq B$, $A \cup B = B$ and vice-versa.  See the exercises.}  For the  set $\Q \cup \mathbb P$, we note that every real number is either rational or not, hence $\Q \cup \mathbb P = \R$, pretty much by the definition of the set $\mathbb P$.  \qed


\end{enumerate}


\end{ex}

As you may recall, we often visualize the set of real numbers $\R$ as a line where each point on the line corresponds to one and only one real number.  Given two different real numbers $a$ and $b$,  we write $a < b$ if $a$ is located to the left of $b$ on the number line, as shown below.

\begin{center}

\begin{mfpic}[10]{-5}{5}{-2}{2} 


\tlpointsep{4pt}
\axislabels {x}{{$a\vphantom{b} \hspace{4pt} $} -3, {$b$} 3}

\arrow \reverse \arrow \polyline{(-5,0), (5,0)}
\point[3pt]{(3,0), (-3,0)}

\end{mfpic}

The real number line with two numbers $a$ and $b$ where $a < b$.

\end{center}

While this notion seems innocuous, it is worth pointing out that this convention is rooted in two deep properties of real numbers.  The first property is that $\R$ is  \href{http://en.wikipedia.org/wiki/Complete_metric_space}{\underline{complete}}. This means that there are no `holes' or `gaps' in the real number line.\footnote{Alas, this intuitive feel for what it means to be `complete' is as good as it gets at this level.  Completeness does get a much more precise meaning later in courses like Analysis and Topology.} Another way to think about this is that if you choose any two distinct (different) real numbers, and look between them, you'll find a solid line segment (or interval) consisting of infinitely many real numbers.  The next result tells us what types of numbers we can expect to find.

\medskip

\phantomsection
\label{densityofqandp}

\colorbox{ResultColor}{\bbm

\centerline{\textbf{Density Property of $\Q$ and $\mathbb P$ in $\R$}}

Between any two distinct real numbers, there is at least one rational number and irrational number.  It then follows that between any two distinct real numbers there will be \underline{infinitely many} rational and irrational numbers.

\ebm}

\medskip

The root word `dense' here communicates the idea that rationals and irrationals are `thoroughly mixed' into $\R$.   The reader is encouraged to think about how one would find both a rational and an irrational number between, say, $0.9999$ and $1$. Once you've done that, try doing the same thing for the numbers $0.\overline{9}$ and $1$. (`Try' is the operative word, here.\footnote{Again, see Section \ref{Summation} for details.})

\smallskip

The second property $\R$ possesses that lets us view it as a line is that the set is \href{http://en.wikipedia.org/wiki/Total_order}{\underline{totally ordered}}. This means that given any two real numbers $a$ and $b$, either $a < b$, $a > b$ or $a = b$ which allows us to arrange the numbers from least (left) to greatest (right). You may have heard this property given as the `Law of Trichotomy'.\index{trichotomy}

\medskip

\phantomsection
\label{trichotomy}

\colorbox{ResultColor}{\bbm

\centerline{\textbf{Law of Trichotomy}}

If $a$ and $b$ are real numbers then exactly one of the following statements is true: \vspace{-.15in} \[ \begin{array}{lclcl} a < b & \hspace*{1.25in} & a > b & \hspace*{1.25in} & a = b \end{array} \]

\ebm}

\medskip

Segments of the real number line are called \textbf{intervals}.\index{interval ! definition of}  They play a huge role not only in this text but also in the Calculus curriculum so we need a concise way to describe them.  We start by examining a few examples of the \textbf{interval notation}\index{interval ! notation for} associated with some specific sets of numbers.  

\begin{center}
\begin{tabular}{|c|c|c|} \hline

Set of Real Numbers & Interval Notation &  Region on the Real Number Line  \\
\hline

& &  \\
\shortstack{$\{x\,|\,1\leq x< 3\}$ \\ \hfill} & \shortstack{$[1,3)$ \\ \hfill} & 

\begin{mfpic}[10]{-3}{3}{-2}{2} 


\tlpointsep{4pt}
\axislabels {x}{{$1 \hspace{4pt} $} -3, {$3$} 3}

\polyline{(-3,0), (3,0)}
\point[3pt]{(-3,0)}
\pointfillfalse
\point[3pt]{(3,0)}

\end{mfpic}   \\
\hline

 &  & \\
\shortstack{$\{x\,|\,-1\leq x \leq 4\}$ \\ \hfill}& \shortstack{$[-1,4]$ \\ \hfill} & 

\begin{mfpic}[10]{-3}{3}{-2}{2} 


\tlpointsep{4pt}
\axislabels {x}{{$-1 \hspace{8pt} $} -3, {$4$} 3}

\polyline{(-3,0), (3,0)}
\point[3pt]{(-3,0), (3,0)}

\end{mfpic}   \\
\hline

&  & \\

\shortstack{$\{x\,| \, x \leq 5 \}$ \\ \hfill} & \shortstack{$(-\infty, 5]$ \\ \hfill} &

\begin{mfpic}[10]{-3}{3}{-2}{2} 


\tlpointsep{4pt}
\axislabels {x}{{$5$} 3}

\arrow \polyline{(3,0), (-3,0)}
\point[3pt]{(3,0)}

\end{mfpic}   \\
\hline

 &  & \\
\shortstack{$\{x\,| \, x > -2 \}$ \\ \hfill} & \shortstack{$(-2, \infty)$ \\ \hfill} &  

\begin{mfpic}[10]{-3}{3}{-2}{2} 


\tlpointsep{4pt}
\axislabels {x}{{$-2 \hspace{8pt} $} -3}

\arrow \polyline{(-3,0), (3,0)}
\pointfillfalse
\point[3pt]{(-3,0)}

\end{mfpic}   \\
\hline

\end{tabular}

\end{center}

As you can glean from the table, for intervals with finite endpoints we start by writing `left endpoint, right endpoint'.  We use square brackets, `$[$' or `$]$', if the endpoint is included in the interval. This corresponds to a `filled-in' or `closed' dot on the number line to indicate that the number is included in the set.  Otherwise, we use parentheses, `$($' or `$)$' that correspond to an `open' circle which indicates that the endpoint is not part of the set.  If the interval does not have finite endpoints, we use the symbol $-\infty$ to indicate that the interval extends indefinitely to the left and the symbol $\infty$ to indicate that the interval extends indefinitely to the right.  Since infinity is a concept, and not a number, we always use parentheses when using these symbols in interval notation, and use the appropriate arrow to indicate that the interval extends indefinitely in one or both directions. We summarize all of the possible cases in one convenient table below.\footnote{The importance of understanding interval notation in Calculus cannot be overstated so please do yourself a favor and memorize this chart.}

\medskip

\phantomsection
\label{intervalnotationsummary}

\colorbox{ResultColor}{\bbm

%\smallskip

\centerline{\textbf{Interval Notation}}

\medskip

\hspace{.5in} Let $a$ and $b$ be real numbers with $a<b$.

\smallskip

\begin{center}

\begin{tabular}{|c|c|c|} \hline

Set of Real Numbers & Interval Notation &  Region on the Real Number Line  \\
\hline

 &  & \\
\shortstack{$\{x\,|\,a<x<b\}$ \\ \hfill}& \shortstack{$(a,b)$ \\ \hfill} & 

\begin{mfpic}[10]{-3}{3}{-2}{2} 
\backgroundcolor[gray]{.95}

\tlpointsep{4pt}
\axislabels {x}{{$a\vphantom{b} \hspace{4pt} $} -3, {$b$} 3}

\polyline{(-3,0), (3,0)}
\pointfillfalse
\point[3pt]{(3,0), (-3,0)}

\end{mfpic}  \\ \hline

& &  \\
\shortstack{$\{x\,|\,a\leq x<b\}$ \\ \hfill}& \shortstack{$[a,b)$ \\ \hfill} & 

\begin{mfpic}[10]{-3}{3}{-2}{2} 
\backgroundcolor[gray]{.95}

\tlpointsep{4pt}
\axislabels {x}{{$a\vphantom{b} \hspace{4pt} $} -3, {$b$} 3}

\polyline{(-3,0), (3,0)}
\point[3pt]{(-3,0)}
\pointfillfalse
\point[3pt]{(3,0)}

\end{mfpic}   \\
\hline

 &  & \\
\shortstack{$\{x\,|\,a<x\leq b\}$ \\ \hfill}&\shortstack{$(a,b]$ \\ \hfill} & 

\begin{mfpic}[10]{-3}{3}{-2}{2} 
\backgroundcolor[gray]{.95}

\tlpointsep{4pt}
\axislabels {x}{{$a\vphantom{b} \hspace{4pt} $} -3, {$b$} 3}

\polyline{(-3,0), (3,0)}
\point[3pt]{(3,0)}
\pointfillfalse
\point[3pt]{(-3,0)}

\end{mfpic}   \\
\hline

 &  & \\
\shortstack{$\{x\,|\,a\leq x \leq b\}$ \\ \hfill}& \shortstack{$[a,b]$ \\ \hfill}& 

\begin{mfpic}[10]{-3}{3}{-2}{2} 
\backgroundcolor[gray]{.95}

\tlpointsep{4pt}
\axislabels {x}{{$a\vphantom{b} \hspace{4pt} $} -3, {$b$} 3}

\polyline{(-3,0), (3,0)}
\point[3pt]{(3,0), (-3,0)}

\end{mfpic}   \\
\hline

 & & \\
\shortstack{$\{x\,| \, x<b\}$ \\ \hfill}& \shortstack{$(-\infty,b)$ \\ \hfill}& 

\begin{mfpic}[10]{-3}{3}{-2}{2} 
\backgroundcolor[gray]{.95}

\tlpointsep{4pt}
\axislabels {x}{{$b$} 3}

\arrow \polyline{(3,0), (-3,0)}

\pointfillfalse
\point[3pt]{(3,0)}

\end{mfpic}   \\
\hline


&  & \\

\shortstack{$\{x\,| \, x \leq b\}$ \\ \hfill} & \shortstack{$(-\infty,b]$ \\ \hfill}& 

\begin{mfpic}[10]{-3}{3}{-2}{2} 
\backgroundcolor[gray]{.95}

\tlpointsep{4pt}
\axislabels {x}{{$b$} 3}

\arrow \polyline{(3,0), (-3,0)}

\point[3pt]{(3,0)}

\end{mfpic}   \\
\hline

 &  & \\
\shortstack{$\{x\,| \, x>a\}$ \\ \hfill}& \shortstack{$(a,\infty)$ \\ \hfill}& 

\begin{mfpic}[10]{-3}{3}{-2}{2} 
\backgroundcolor[gray]{.95}

\tlpointsep{4pt}
\axislabels {x}{{$a\vphantom{b} \hspace{4pt}$} -3}

\arrow \polyline{(-3,0), (3,0)}

\pointfillfalse
\point[3pt]{(-3,0)}

\end{mfpic}   \\
\hline

 &  & \\
\shortstack{$\{x\,| \, x \geq a \}$ \\ \hfill}& \shortstack{$[a,\infty)$ \\ \hfill} & 

\begin{mfpic}[10]{-3}{3}{-2}{2} 
\backgroundcolor[gray]{.95}

\tlpointsep{4pt}
\axislabels {x}{{$a\vphantom{b} \hspace{4pt}$} -3}

\arrow \polyline{(-3,0), (3,0)}

\point[3pt]{(-3,0)}

\end{mfpic}   \\
\hline

&  & \\
\shortstack{$\R$ \\ \hfill}& \shortstack{$(-\infty,\infty)$ \\ \hfill} & 

\begin{mfpic}[10]{-3}{3}{-2}{2} 
\backgroundcolor[gray]{.95}

\tlpointsep{4pt}
\axislabels {x}{{$\vphantom{b} \hspace{4pt}$} -3}

\arrow \reverse \arrow \polyline{(-3,0), (3,0)}

\end{mfpic}   \\
\hline

\end{tabular}

\end{center}

\ebm}

\medskip

We close this section with an example that ties together several concepts presented earlier.  Specifically, we demonstrate how to use interval notation along with the concepts of `union' and `intersection' to describe a variety of sets on the real number line.

\begin{ex} \label{intervalex} $~$

\begin{enumerate} \item Express the following sets of numbers using interval notation.

\begin{multicols}{2}

\begin{enumerate}

\item  $\{ x \, | \, x \leq -2 \, \, \text{or} \, \,  x \geq 2 \}$

\item  $\{ x \, | \, x \neq 3 \}$

\setcounter{HW}{\value{enumii}}

\end{enumerate}

\end{multicols}

\begin{multicols}{2}

\begin{enumerate}

\setcounter{enumii}{\value{HW}}

\item  $\{ x \, | \, x \neq \pm 3 \}$

\item  $\{ x \, | \, -1 < x \leq 3 \,\, \text{or} \,\, x = 5\}$

\end{enumerate}

\end{multicols}

\item  Let $A = [-5,3)$ and $B = (1, \infty)$.  Find  $A \cap B$ and $A\cup B$. 

\end{enumerate}


{\bf Solution.}

\begin{enumerate}

\item 

\begin{enumerate}

\item  The best way to proceed here is to graph the set of numbers on the number line and glean the answer from it.  The inequality $x \leq -2$ corresponds to the interval $(-\infty, -2]$ and the inequality $x \geq 2$ corresponds to the interval $[2, \infty)$. The `or' in $\{ x \, | \, x \leq -2 \, \, \text{or} \, \,  x \geq 2 \}$ tells us that we are looking for the union of these two intervals, so our answer is $(-\infty, -2] \cup [2, \infty)$.

\begin{center}

\begin{mfpic}[10]{-5}{5}{-2}{2}
\polyline{(-5,0), (5,0)}
\tlpointsep{5pt}
\axislabels {x}{{$-2 \hspace{8pt}$} -2, {$2$} 2}
\penwd{1.15pt}
\arrow \polyline{(2,0), (5,0)}
\arrow \polyline{(-2,0), (-5,0)}
\point[4pt]{(-2,0), (2,0)}
\end{mfpic}  \\
$(-\infty, -2] \cup [2, \infty)$ 

\end{center}

\item For the set $\{ x \, | \, x \neq 3 \}$, we shade the entire real number line except $x=3$, where we leave an open circle.  This divides the real number line into two intervals, $(-\infty, 3)$ and $(3,\infty)$.  Since the values of $x$ could be in one of these intervals \textit{or} the other, we once again use the union symbol to get $\{ x \, | \, x \neq 3 \} = (-\infty, 3) \cup (3,\infty)$.
 
\begin{center}

\begin{mfpic}[10]{-5}{5}{-2}{2}
\tlpointsep{5pt}
\axislabels {x}{{$3$} 0}
\penwd{1.15pt}
\arrow \reverse \arrow \polyline{(-5,0), (5,0)}
\pointfillfalse
\point[3pt]{(0,0)}
\end{mfpic}  \\



 $(-\infty, 3) \cup (3, \infty)$ 
 
 

\end{center}

\item  For the set $\{ x \, | \, x \neq \pm 3 \}$, we proceed as before and exclude both $x=3$ and $x=-3$ from our set. (Do you remember what we said back on \pageref{setsofnumbersboxonthispage} about $x = \pm 3$?)  This breaks the number line into \textit{three} intervals, $(-\infty, -3)$, $(-3,3)$ and $(3, \infty)$.   Since the set describes real numbers which come from the first, second \textit{or} third interval, we have $\{ x \, | \, x \neq \pm 3 \} = (-\infty, -3) \cup (-3,3) \cup (3, \infty)$.


\begin{center}

\begin{mfpic}[10]{-5}{5}{-2}{2}
\tlpointsep{5pt}
\axislabels {x}{{$-3 \hspace{8pt}$} -3, {$3$} 3}
\penwd{1.15pt}
\arrow \reverse \arrow \polyline{(-5,0), (5,0)}
\pointfillfalse
\point[3pt]{(-3,0), (3,0)}
\end{mfpic} \\

 $(-\infty, -3) \cup (-3,3) \cup (3, \infty)$
 
 \end{center}



\item  Graphing the set $\{ x \, | \, -1 < x \leq 3 \,\, \text{or} \,\, x = 5\}$ yields the interval $(-1,3]$ along with the single number 5.  While we \textit{could} express this single point as $[5,5]$, it is customary to write a single point as a `singleton set', so in our case we have the set $\{ 5\}$. Thus our final answer is $\{ x \, | \, -1 < x \leq 3 \,\, \text{or} \,\, x = 5\} = (-1,3] \cup \{ 5\}$.


\begin{center}

\begin{mfpic}[10]{-5}{5}{-2}{2}
\arrow \reverse \arrow \polyline{(-5,0), (5,0)}
\tlpointsep{5pt}
\axislabels {x}{{$-1 \hspace{8pt}$} -3, {$3$} 1, {$5$} 3}
\penwd{1.15pt}
\polyline{(-3,0), (1,0)}
\point[4pt]{(3,0), (1,0)}
\pointfillfalse
\point[3pt]{(-3,0)}
\end{mfpic} \\
  
 $(-1,3] \cup \{ 5\}$
 
\end{center}


\end{enumerate}

\item  We start by graphing $A = [-5,3)$ and $B = (1, \infty)$ on the number line.  To find $A\cap B$, we need to find the numbers in common to both $A$ and $B$, in other words, the overlap of the two intervals.  Clearly, everything between $1$ and $3$ is in both $A$ and $B$.  However, since $1$ is in $A$ but not in $B$, $1$ is not in the intersection.  Similarly, since $3$ is in $B$ but not in $A$, it isn't in the intersection either.  Hence, $A \cap B = (1,3)$.  To find $A \cup B$, we need to find the numbers in at least one of $A$ or $B$.  Graphically, we shade $A$ and $B$ along with it.   Notice here that even though $1$ isn't in $B$, it is in $A$, so it's the union along with all the other elements of $A$ between $-5$ and $1$.  A similar argument goes for the inclusion of $3$ in the union.  The result of shading both $A$ and $B$ together gives us $A \cup B = [-5,\infty)$.

\begin{center}

\begin{tabular}{ccc}

\begin{mfpic}[10]{-5}{7}{-2}{2}
\arrow \polyline{(-5.5,0),(7.5,0)}
\axismarks{x}{-5,1,3}
\tlpointsep{4pt}
\axislabels {x}{{$-5 \hspace{8pt} $} -5, {$1$} 1,{$3$} 3,}
\penwd{1.15pt}
\polyline{(-5,2), (3,2)}
\arrow \polyline{ (1,1), (7,1)}
\point[4pt]{(-5,2)}
\pointfillfalse
\point[3pt]{(1,1), (3,2)}
\tcaption{$A = [-5,3)$,  $B = (1, \infty)$ }
\end{mfpic}  &

\begin{mfpic}[10]{-5}{7}{-2}{2}
\arrow \polyline{(-5,0),(7,0)}
\axismarks{x}{-5,3}
\tlpointsep{5pt}
\axislabels {x}{{$-5 \hspace{8pt} $} -5, {$1$} 1,{$3$} 3,}
\penwd{1.15pt}
\polyline{(-5,2), (3,2)}
\arrow \polyline{ (1,1), (7,1)}
\polyline{(1,0), (3,0)}
\point[4pt]{(-5,2)}
\pointfillfalse
\point[3pt]{(1,0),(1,1), (3,2), (3,0)}
\tcaption{$A \cap B = (1,3)$}
\end{mfpic}  &

\begin{mfpic}[10]{-5}{7}{-2}{2}
\tlpointsep{5pt}
\axislabels {x}{{$-5 \hspace{8pt} $} -5, {$1$} 1,{$3$} 3,}
\penwd{1.15pt}
\polyline{(-5,2), (3,2)}
\arrow \polyline{ (1,1), (7,1)}
\arrow \polyline{(-5,0), (7,0)}
\point[4pt]{(-5,0),(-5,2)}
\pointfillfalse
\point[3pt]{(1,1), (3,2)}
\tcaption{$A \cup B = [-5,\infty)$}
\end{mfpic} \\

\end{tabular}

\end{center}

\end{enumerate}

\vspace*{-.4in} \qed

\end{ex}

\newpage

\subsection{Exercises}

\begin{enumerate}


\item  Find a verbal description for $O = \{ 2n-1 \, | \, n \in \N\}$


\item  Find a roster description for $X = \{ z^2 \, | \, z \in \mathbb{Z}\}$


\item  Let $A = \left\{ -3, -1.02, -\dfrac{3}{5}, 0.57, 1.\overline{23}, \sqrt{3}, 5.2020020002 \ldots, \dfrac{20}{10}, 117 \right\}$ 

\begin{enumerate}

\item  List the elements of $A$ which are natural numbers.
\item  List the elements of $A$ which are irrational numbers.
\item  Find $A \cap \mathbb{Z}$
\item  Find $A \cap \Q$


\end{enumerate}


\item Fill in the chart below. 

\begin{center}
\begin{tabular}{|c|c|c|} \hline

Set of Real Numbers & Interval Notation &  Region on the Real Number Line  \\
\hline

& &  \\

\shortstack{$\{x\,|\,-1\leq x< 5\}$ \\ \hfill} &  &  \\ \hline

& &  \\

 & \shortstack{$[0,3)$ \\ \hfill} &   \\ \hline


& &  \\

 &  & 

\begin{mfpic}[10]{-3}{3}{-2}{2} 
\tlpointsep{4pt}
\axislabels {x}{{$2 \hspace{4pt} $} -3, {$7$} 3}
\polyline{(-3,0), (3,0)}
\point[3pt]{(3,0)}
\pointfillfalse
\point[3pt]{(-3,0)}
\end{mfpic}   \\
\hline

 &  & \\
 
\shortstack{$\{x\,|\, -5 <  x \leq 0 \}$ \\ \hfill} &  & \\ \hline

 &  & \\
 
  & \shortstack{$(-3,3)$ \\ \hfill} &  \\ \hline

&  & \\
 
& & 

\begin{mfpic}[10]{-3}{3}{-2}{2} 
\tlpointsep{4pt}
\axislabels {x}{{$5 \hspace{4pt} $} -3, {$7$} 3}
\polyline{(-3,0), (3,0)}
\point[3pt]{(-3,0), (3,0)}

\end{mfpic}   \\
\hline

&  & \\

\shortstack{$\{x\,| \, x \leq 3 \}$ \\ \hfill} &  &  \\ \hline

 &  & \\
 
& \shortstack{$(-\infty, 9)$ \\ \hfill} &  \\ \hline

 &  & \\

 &  &  

\begin{mfpic}[10]{-3}{3}{-2}{2} 
\tlpointsep{4pt}
\axislabels {x}{{$4 \hspace{4pt} $} -3}
\arrow \polyline{(-3,0), (3,0)}
\pointfillfalse
\point[3pt]{(-3,0)}

\end{mfpic}   \\
\hline

 &  & \\
 
 
\shortstack{$\{x\,| \, x \geq  -3 \}$ \\ \hfill} & &    \\ \hline

\end{tabular}

\end{center}

\setcounter{HW}{\value{enumi}}
\end{enumerate}

\newpage

In Exercises \ref{findunionintfirst} - \ref{findunionintlast}, find the indicated intersection or union and simplify if possible.  Express your answers in interval notation. 

\begin{multicols}{3}
\begin{enumerate}
\setcounter{enumi}{\value{HW}}

\item  $(-1,5] \cap [0,8)$ \label{findunionintfirst}
\item  $(-1,1) \cup [0,6]$
\item $(-\infty,4]\cap (0,\infty)$

\setcounter{HW}{\value{enumi}}
\end{enumerate}
\end{multicols}

\begin{multicols}{3}
\begin{enumerate}
\setcounter{enumi}{\value{HW}}

\item $(-\infty,0) \cap [1,5]$
\item $(-\infty, 0) \cup [1,5]$
\item $(-\infty, 5] \cap [5,8)$ \label{findunionintlast}

\setcounter{HW}{\value{enumi}}
\end{enumerate}
\end{multicols}

In Exercises \ref{writeintervalfirst} - \ref{writeintervallast}, write the set using interval notation.   

\begin{multicols}{3}
\begin{enumerate}
\setcounter{enumi}{\value{HW}}

\item  $\{x\,|\, x \neq 5 \}$ \label{writeintervalfirst}
\item  $\{x\,|\, x \neq -1 \}$
\item  $\{x\,|\, x \neq -3,\, 4 \}$

\setcounter{HW}{\value{enumi}}
\end{enumerate}
\end{multicols}

\begin{multicols}{3}
\begin{enumerate}
\setcounter{enumi}{\value{HW}}

\item  $\{x\,|\, x \neq 0, \, 2 \}$
\item  $\{x\,|\, x \neq 2, \, -2 \}$
\item  $\{x\,|\, x \neq 0,\, \pm 4 \}$

\setcounter{HW}{\value{enumi}}
\end{enumerate}
\end{multicols}

\begin{multicols}{3}
\begin{enumerate}
\setcounter{enumi}{\value{HW}}

\item $\{x\,|\, x \leq -1 \, \text{or} \, x \geq 1 \}$
\item $\{x\,|\, x < 3 \, \text{or} \, x \geq 2 \}$
\item $\{x\,|\, x \leq -3 \, \text{or} \, x > 0 \}$

\setcounter{HW}{\value{enumi}}
\end{enumerate}
\end{multicols}

\begin{multicols}{3}
\begin{enumerate}
\setcounter{enumi}{\value{HW}}

\item $\{x\,|\, x \leq 5 \, \text{or} \, x = 6 \}$
\item $\{x\,|\, x > 2 \, \text{or} \, x = \pm 1 \}$
\item $\{x\,|\,  -3 < x < 3 \, \text{or} \, x = 4 \}$ \label{writeintervallast}

\setcounter{HW}{\value{enumi}}
\end{enumerate}
\end{multicols}

For Exercises \ref{shadevennfirst} - \ref{shadevennlast}, use the blank Venn Diagram below $A$, $B$, and $C$ as a guide for you to shade the following sets.

\begin{center}
\begin{mfpic}[40]{-2}{2}{-2}{2}
  
	\circle{(.5,-.875),1} %Circle C
   \circle{(0,0),1} % Circle A
   \circle{(.875,0),1} % Circle B
   \tlabel[cc](-1.25, 0){$A$}
   \tlabel[cc](2.125, 0){$B$}
   \tlabel[cc](0.5,-2.125){$C$}
	\tlabel[cc](2.375, 1.5){$U$}
	\rect{(-1.75,-2.625), (2.625, 1.75)}
\end{mfpic}

\end{center}

\begin{multicols}{3}
\begin{enumerate}
\setcounter{enumi}{\value{HW}}

\item  $A \cup C$ \label{shadevennfirst}

\item  $B \cap C$

\item  $(A \cup B) \cup C$



\setcounter{HW}{\value{enumi}}
\end{enumerate}
\end{multicols}

\begin{multicols}{3}
\begin{enumerate}
\setcounter{enumi}{\value{HW}}

\item  $(A \cap B) \cap C$ 

\item  $A \cap (B \cup C)$ \label{intoverunion}

\item  $(A \cap B) \cup (A \cap C)$ \label{shadevennlast}

\setcounter{HW}{\value{enumi}}
\end{enumerate}
\end{multicols}

\begin{enumerate}
\setcounter{enumi}{\value{HW}}

\item  Explain how your answers to problems \ref{intoverunion} and \ref{shadevennlast} show $A \cap (B \cup C) = (A \cap B) \cup (A \cap C)$.  Phrased differently, this shows `intersection \textit{distributes} over union.'  Discuss with your classmates if  `union' distributes over `intersection.'  Use a Venn Diagram to support your answer.

\setcounter{HW}{\value{enumi}}
\end{enumerate}

\newpage

\subsection{Answers}

\begin{enumerate}


\item  $O$ is the odd natural numbers.


\item  $X = \{ 0, 1, 4, 9, 16, \ldots \}$


\item  \begin{enumerate} \item  $\dfrac{20}{10} = 2$ and $117$
\item  $\sqrt{3}$ and $5.2020020002$
\item $\left\{ -3, \dfrac{20}{10}, 117\right\}$
\item  $\left\{ -3, -1.02, -\dfrac{3}{5}, 0.57, 1.\overline{23},\dfrac{20}{10}, 117 \right \}$

\end{enumerate}

\item $~$

\begin{center}
\begin{tabular}{|c|c|c|} \hline

Set of Real Numbers & Interval Notation &  Region on the Real Number Line  \\
\hline

& &  \\

\shortstack{$\{x\,|\,-1\leq x< 5\}$ \\ \hfill} & \shortstack{$[-1,5)$ \\ \hfill} & 

\begin{mfpic}[10]{-3}{3}{-2}{2} 
\tlpointsep{4pt}
\axislabels {x}{{$-1 \hspace{8pt} $} -3, {$5$} 3}
\polyline{(-3,0), (3,0)}
\point[3pt]{(-3,0)}
\pointfillfalse
\point[3pt]{(3,0)}
\end{mfpic}   \\
\hline

& &  \\

\shortstack{$\{x\,|\,0\leq x < 3\}$ \\ \hfill} & \shortstack{$[0,3)$ \\ \hfill} & 

\begin{mfpic}[10]{-3}{3}{-2}{2} 
\tlpointsep{4pt}
\axislabels {x}{{$0 \hspace{4pt} $} -3, {$3$} 3}
\polyline{(-3,0), (3,0)}
\point[3pt]{(-3,0)}
\pointfillfalse
\point[3pt]{(3,0)}
\end{mfpic}   \\
\hline


& &  \\

\shortstack{$\{x\,|\, 2 <  x \leq 7 \}$ \\ \hfill} & \shortstack{$(2,7]$ \\ \hfill} & 

\begin{mfpic}[10]{-3}{3}{-2}{2} 
\tlpointsep{4pt}
\axislabels {x}{{$2 \hspace{4pt} $} -3, {$7$} 3}
\polyline{(-3,0), (3,0)}
\point[3pt]{(3,0)}
\pointfillfalse
\point[3pt]{(-3,0)}
\end{mfpic}   \\
\hline

 &  & \\
 
 \shortstack{$\{x\,|\, -5 <  x \leq 0 \}$ \\ \hfill} & \shortstack{$(-5,0]$ \\ \hfill} & 

\begin{mfpic}[10]{-3}{3}{-2}{2} 
\tlpointsep{4pt}
\axislabels {x}{{$-5 \hspace{8pt} $} -3, {$0$} 3}
\polyline{(-3,0), (3,0)}
\point[3pt]{(3,0)}
\pointfillfalse
\point[3pt]{(-3,0)}
\end{mfpic}   \\
\hline

 &  & \\
 
 \shortstack{$\{x\,|\, -3 <  x < 3 \}$ \\ \hfill} & \shortstack{$(-3,3)$ \\ \hfill} & 

\begin{mfpic}[10]{-3}{3}{-2}{2} 
\tlpointsep{4pt}
\axislabels {x}{{$-3 \hspace{8pt} $} -3, {$3$} 3}
\polyline{(-3,0), (3,0)}
\pointfillfalse
\point[3pt]{(-3,0), (3,0)}
\end{mfpic}   \\
\hline

 &  & \\
 
\shortstack{$\{x\,|\,5\leq x \leq 7\}$ \\ \hfill}& \shortstack{$[5,7]$ \\ \hfill} & 

\begin{mfpic}[10]{-3}{3}{-2}{2} 
\tlpointsep{4pt}
\axislabels {x}{{$5 \hspace{4pt} $} -3, {$7$} 3}
\polyline{(-3,0), (3,0)}
\point[3pt]{(-3,0), (3,0)}

\end{mfpic}   \\
\hline

&  & \\

\shortstack{$\{x\,| \, x \leq 3 \}$ \\ \hfill} & \shortstack{$(-\infty, 3]$ \\ \hfill} &
\begin{mfpic}[10]{-3}{3}{-2}{2} 
\tlpointsep{4pt}
\axislabels {x}{{$3$} 3}
\arrow \polyline{(3,0), (-3,0)}
\point[3pt]{(3,0)}

\end{mfpic}   \\
\hline

 &  & \\
 
 \shortstack{$\{x\,| \, x < 9 \}$ \\ \hfill} & \shortstack{$(-\infty, 9)$ \\ \hfill} &
\begin{mfpic}[10]{-3}{3}{-2}{2} 
\tlpointsep{4pt}
\axislabels {x}{{$9$} 3}
\arrow \polyline{(3,0), (-3,0)}
\pointfillfalse
\point[3pt]{(3,0)}

\end{mfpic}   \\
\hline

 &  & \\
 
 
\shortstack{$\{x\,| \, x >  4 \}$ \\ \hfill} & \shortstack{$(4, \infty)$ \\ \hfill} &  

\begin{mfpic}[10]{-3}{3}{-2}{2} 
\tlpointsep{4pt}
\axislabels {x}{{$4 \hspace{4pt} $} -3}
\arrow \polyline{(-3,0), (3,0)}
\pointfillfalse
\point[3pt]{(-3,0)}

\end{mfpic}   \\
\hline

 &  & \\
 
 
\shortstack{$\{x\,| \, x \geq  -3 \}$ \\ \hfill} & \shortstack{$[-3, \infty)$ \\ \hfill} &  

\begin{mfpic}[10]{-3}{3}{-2}{2} 
\tlpointsep{4pt}
\axislabels {x}{{$-3 \hspace{8pt} $} -3}
\arrow \polyline{(-3,0), (3,0)}
\point[3pt]{(-3,0)}

\end{mfpic}   \\
\hline

\end{tabular}

\end{center}

\setcounter{HW}{\value{enumi}}
\end{enumerate}

\begin{multicols}{2}
\begin{enumerate}
\setcounter{enumi}{\value{HW}}

\item  $(-1,5] \cap [0,8) = [0,5]$

\item  $(-1,1) \cup [0,6] = (-1,6]$

\setcounter{HW}{\value{enumi}}
\end{enumerate}
\end{multicols}

\begin{multicols}{2}
\begin{enumerate}
\setcounter{enumi}{\value{HW}}

\item $(-\infty,4]\cap (0,\infty) = (0,4]$


\item $(-\infty,0) \cap [1,5] = \emptyset$

\setcounter{HW}{\value{enumi}}
\end{enumerate}
\end{multicols}

\begin{multicols}{2}
\begin{enumerate}
\setcounter{enumi}{\value{HW}}

\item $(-\infty, 0) \cup [1,5] = (-\infty,0) \cup [1,5]$

\item $(-\infty, 5] \cap [5,8) = \left\{ 5\right\}$

\setcounter{HW}{\value{enumi}}
\end{enumerate}
\end{multicols}

\begin{multicols}{2}
\begin{enumerate}
\setcounter{enumi}{\value{HW}}

\item  $(-\infty, 5) \cup (5, \infty)$

\item  $(-\infty, -1) \cup (-1, \infty)$

\setcounter{HW}{\value{enumi}}
\end{enumerate}
\end{multicols}


\begin{multicols}{2}
\begin{enumerate}
\setcounter{enumi}{\value{HW}}

\item  $(-\infty, -3) \cup (-3, 4)\cup (4, \infty)$


\item   $(-\infty, 0) \cup (0, 2)\cup (2, \infty)$

\setcounter{HW}{\value{enumi}}
\end{enumerate}
\end{multicols}


\begin{multicols}{2}
\begin{enumerate}
\setcounter{enumi}{\value{HW}}

\item  $(-\infty, -2) \cup (-2, 2)\cup (2, \infty)$

\item  $(-\infty, -4) \cup (-4, 0) \cup (0, 4) \cup (4, \infty)$

\setcounter{HW}{\value{enumi}}
\end{enumerate}
\end{multicols}

\begin{multicols}{2}
\begin{enumerate}
\setcounter{enumi}{\value{HW}}

\item $(-\infty, -1] \cup [1, \infty)$

\item $(-\infty, \infty)$


\setcounter{HW}{\value{enumi}}
\end{enumerate}
\end{multicols}


\begin{multicols}{2}
\begin{enumerate}
\setcounter{enumi}{\value{HW}}

\item $(-\infty, -3] \cup (0, \infty)$

\item $(-\infty, 5] \cup \{6\}$

\setcounter{HW}{\value{enumi}}
\end{enumerate}
\end{multicols}

\begin{multicols}{2}
\begin{enumerate}
\setcounter{enumi}{\value{HW}}


\item $\{-1\} \cup \{1\} \cup (2, \infty)$

\item $(-3,3) \cup \{4\}$

\setcounter{HW}{\value{enumi}}
\end{enumerate}
\end{multicols}


\begin{multicols}{2}
\begin{enumerate}
\setcounter{enumi}{\value{HW}}

\item $A \cup C$

\begin{mfpic}[40]{-2}{2}{-2}{2}
  \fillcolor[gray]{0.7}
	\gfill\circle{(.5,-.875),1} %Circle C
   \gfill\circle{(0,0),1} % Circle A
	\circle{(.5,-.875),1} %Circle C
   \circle{(0,0),1} % Circle A
   \circle{(.875,0),1} % Circle B
   \tlabel[cc](-1.25, 0){$A$}
   \tlabel[cc](2.125, 0){$B$}
   \tlabel[cc](0.5,-2.125){$C$}
	\tlabel[cc](2.375, 1.5){$U$}
	\rect{(-1.75,-2.625), (2.625, 1.75)}
  \end{mfpic}

\item $B \cap C$

\begin{mfpic}[40]{-2}{2}{-2}{2}
  \fillcolor[gray]{0.7}
	\gfill\circle{(.5,-.875),1} %Circle C
	\gclip\circle{(.875,0),1} % Circle B
	\circle{(.5,-.875),1} %Circle C
	 
	 \circle{(.875,0),1} % Circle B
   \circle{(0,0),1} % Circle A
  
   \tlabel[cc](-1.25, 0){$A$}
   \tlabel[cc](2.125, 0){$B$}
   \tlabel[cc](0.5,-2.125){$C$}
	\tlabel[cc](2.375, 1.5){$U$}
	\rect{(-1.75,-2.625), (2.625, 1.75)}
\end{mfpic}



\setcounter{HW}{\value{enumi}}
\end{enumerate}
\end{multicols}

\pagebreak


\begin{multicols}{2}
\begin{enumerate}
\setcounter{enumi}{\value{HW}}


\item $(A \cup B) \cup C$

\begin{mfpic}[40]{-2}{2}{-2}{2}
  \fillcolor[gray]{0.7}
	\gfill\circle{(.5,-.875),1} %Circle C
	\gfill\circle{(.875,0),1} % Circle B
	 \gfill\circle{(0,0),1} % Circle A
	\circle{(.5,-.875),1} %Circle C
	 
	 \circle{(.875,0),1} % Circle B
   \circle{(0,0),1} % Circle A
  
   \tlabel[cc](-1.25, 0){$A$}
   \tlabel[cc](2.125, 0){$B$}
   \tlabel[cc](0.5,-2.125){$C$}
	\tlabel[cc](2.375, 1.5){$U$}
	\rect{(-1.75,-2.625), (2.625, 1.75)}
\end{mfpic}


\item $(A \cap B) \cap C$

\begin{mfpic}[40]{-2}{2}{-2}{2}
  \fillcolor[gray]{0.7}
	\gfill\circle{(.5,-.875),1} %Circle C
	\gclip\circle{(.875,0),1} % Circle B
	\gclip\circle{(0,0),1} % Circle A
	\circle{(.5,-.875),1} %Circle C
	 
	 \circle{(.875,0),1} % Circle B
   \circle{(0,0),1} % Circle A
  
   \tlabel[cc](-1.25, 0){$A$}
   \tlabel[cc](2.125, 0){$B$}
   \tlabel[cc](0.5,-2.125){$C$}
	\tlabel[cc](2.375, 1.5){$U$}
	\rect{(-1.75,-2.625), (2.625, 1.75)}
\end{mfpic}



\setcounter{HW}{\value{enumi}}
\end{enumerate}
\end{multicols}

\begin{multicols}{2}
\begin{enumerate}
\setcounter{enumi}{\value{HW}}


\item $A \cap (B \cup C)$

\begin{mfpic}[40]{-2}{2}{-2}{2}
  \fillcolor[gray]{0.7}
	
	\gfill  \circle{(0,0),1} % Circle A
	\gclip  \circle{(.875,0),1} % Circle B
	\gfill \circle{(.5,-.875),1} %Circle C
	\gclip  \circle{(0,0),1} % Circle A
	
	\circle{(.5,-.875),1} %Circle C
	 \circle{(.875,0),1} % Circle B
   \circle{(0,0),1} % Circle A
  
   \tlabel[cc](-1.25, 0){$A$}
   \tlabel[cc](2.125, 0){$B$}
   \tlabel[cc](0.5,-2.125){$C$}
	\tlabel[cc](2.375, 1.5){$U$}
	\rect{(-1.75,-2.625), (2.625, 1.75)}
\end{mfpic}


\item $(A \cap B) \cup (A \cap C)$ 

\begin{mfpic}[40]{-2}{2}{-2}{2}
  \fillcolor[gray]{0.7}
	
	\gfill  \circle{(0,0),1} % Circle A
	\gclip  \circle{(.875,0),1} % Circle B
	\gfill \circle{(.5,-.875),1} %Circle C
	\gclip  \circle{(0,0),1} % Circle A
	
	\circle{(.5,-.875),1} %Circle C
	 \circle{(.875,0),1} % Circle B
   \circle{(0,0),1} % Circle A
  
   \tlabel[cc](-1.25, 0){$A$}
   \tlabel[cc](2.125, 0){$B$}
   \tlabel[cc](0.5,-2.125){$C$}
	\tlabel[cc](2.375, 1.5){$U$}
	\rect{(-1.75,-2.625), (2.625, 1.75)}
\end{mfpic}




\setcounter{HW}{\value{enumi}}
\end{enumerate}
\end{multicols}
\enlargethispage{2in}

\begin{enumerate}
\setcounter{enumi}{\value{HW}}

\item  Yes, $A \cup (B \cap C) = (A \cup B) \cap (A \cup C)$.

\begin{center}

\begin{mfpic}[40]{-2}{2}{-2}{2}
  \fillcolor[gray]{0.7}
	
  \gfill \circle{(.5,-.875),1} %Circle C
	\gclip  \circle{(.875,0),1} % Circle B
	\gfill  \circle{(0,0),1} % Circle A
	
	\circle{(.5,-.875),1} %Circle C
	 \circle{(.875,0),1} % Circle B
   \circle{(0,0),1} % Circle A
  
   \tlabel[cc](-1.25, 0){$A$}
   \tlabel[cc](2.125, 0){$B$}
   \tlabel[cc](0.5,-2.125){$C$}
	\tlabel[cc](2.375, 1.5){$U$}
	\rect{(-1.75,-2.625), (2.625, 1.75)}
\end{mfpic}


\end{center}

\setcounter{HW}{\value{enumi}}
\end{enumerate}

\closegraphsfile

\newpage

\section{Real Number Arithmetic}

\documentclass[11pt]{article}
\usepackage[margin=1in,letterpaper]{geometry}
\usepackage{amssymb,amsmath,amsthm,fancyhdr,supertabular,longtable,hhline}
\usepackage{colortbl}
\usepackage{import, multicol,boxedminipage}
\usepackage{graphicx}
\usepackage[colorlinks, hyperindex, plainpages=false, linkcolor=blue, urlcolor=blue, pdfpagelabels]{hyperref}
\usepackage[all]{hypcap}
\definecolor{ResultColor}{gray}{0.9}
\theoremstyle{definition}  % this prevents the text in definitions, theorems, and corollaries from being italicized
\newtheorem{defn}{\bf Definition}
\newtheorem{thm}{\bf Theorem}
\newtheorem{cor}[thm]{\bf Corollary}
\newtheorem{eqn}{\bf Equation}
\newtheorem{ex}{\bf Example}
\newtheorem{fig}{\bf Figure}
\setlength{\parindent}{0in}
\newcommand{\bbm}{\begin{boxedminipage}{6.41in}}
\newcommand{\ebm}{\end{boxedminipage}}
\usepackage{array}
\setlength{\extrarowheight}{2pt}
\allowdisplaybreaks[2]
\usepackage{cancel}
\usepackage{sectsty}
\usepackage{textcomp}
\usepackage{multirow}
\usepackage[sfdefault,lf]{carlito}
	%% The 'lf' option for lining figures
	%% The 'sfdefault' option to make the base font sans serif
	%\usepackage[T1]{fontenc}
	\renewcommand*\oldstylenums[1]{\carlitoOsF #1}
\usepackage[nottoc]{tocbibind}
\allsectionsfont{\mdseries \scshape}
\makeatletter
\renewcommand\l@section{\@dottedtocline{1}{1.5em}{3em}}
\renewcommand\l@subsection{\@dottedtocline{2}{4.5em}{3.5em}}
\makeatother
\pagestyle{fancy}
\newcounter{HW}
\newcounter{HWindent}

\title{Review \#2: Real Number Arithmetic}
\author{Carl Stitz and Jeff Zeager\\
Edited by Sean Fitzpatrick}


\begin{document}
\maketitle


\renewcommand{\headrulewidth}{0pt}
\renewcommand{\headheight}{14pt}
\lhead[\fancyplain{}{\sc\thepage}]%
      {\fancyplain{}{\sc \nouppercase{\rightmark}}}
\rhead[\fancyplain{}{\sc \nouppercase{\leftmark}}]%
      {\fancyplain{}{\sc\thepage}}
\cfoot{}

In this handout we list the properties of real number arithmetic.  This is meant to be a succinct, targeted review so we'll resist the temptation to wax poetic about these axioms and their subtleties and refer the interested reader to a more formal course in Abstract Algebra.  There are two (primary) operations one can perform with real numbers:  addition and multiplication.  

\medskip

\phantomsection
\label{realnumberaddition}

\colorbox{ResultColor}{\bbm

\centerline{\textbf{Properties of Real Number Addition}}

\begin{itemize}

\item  \textbf{Closure:}  For all real numbers $a$ and $b$,  $a+b$ is also a real number.

\item  \textbf{Commutativity:}  For all real numbers $a$ and $b$, $a+b = b+a$.

\item  \textbf{Associativity:}  For all real numbers $a$, $b$ and $c$, $a+(b+c) = (a+b)+c$.

\item  \textbf{Identity:}  There is a real number `$0$' so that for all real numbers $a$, $a+0 = a$.

\item  \textbf{Inverse:}  For all real numbers $a$, there is a real number $-a$ such that $a + (-a) = 0$.

\item \textbf{Definition of Subtraction:}  For all real numbers $a$ and $b$, $a - b = a + (-b)$.

\end{itemize}

\ebm}

\medskip

Next, we give real number multiplication a similar treatment.  Recall that we may denote the product of two real numbers $a$ and $b$ a variety of ways:  $ab$, $a \cdot b$, $a(b)$, $(a)(b)$ and so on.  We'll refrain from using $a \times b$ for real number multiplication in this text with one notable exception in Definition \ref{scientificnotation}.

\medskip

\phantomsection
\label{realnumbermultiplication}

\colorbox{ResultColor}{\bbm

\centerline{\textbf{Properties of Real Number Multiplication}}

\begin{itemize}

\item  \textbf{Closure:}  For all real numbers $a$ and $b$,  $ab$ is also a real number.

\item  \textbf{Commutativity:}  For all real numbers $a$ and $b$, $ab = ba$.

\item  \textbf{Associativity:}  For all real numbers $a$, $b$ and $c$, $a(bc) = (ab)c$.

\item  \textbf{Identity:}  There is a real number `$1$' so that for all real numbers $a$, $a \cdot 1 = a$.

\item  \textbf{Inverse:}  For all real numbers $a \neq 0$, there is a real number $\dfrac{1}{a}$ such that $a \left(\dfrac{1}{a}\right) = 1$.

\item \textbf{Definition of Division:}  For all real numbers $a$ and $b \neq 0$, $a \div b = \dfrac{a}{b} = a  \left(\dfrac{1}{b}\right)$.

\end{itemize}

\ebm}

\medskip

While most students and some faculty tend to skip over these properties or give them a cursory glance at best, it is important to realize that the properties stated above are what drive the symbolic manipulation for all of Algebra.  When listing a tally of more than two numbers, $1 + 2 + 3$\label{howtoaddonetwothree} for example, we don't need to specify the order in which those numbers are added. Notice though, try as we might, we can add only two numbers at a time and it is the associative property of addition which assures us that we could organize this sum as $(1+2) + 3$ or $1+(2+3)$.  This brings up a note about `grouping symbols'.  Recall that parentheses and brackets are used in order to specify which operations are to be performed first.  In the absence of such grouping symbols, multiplication (and hence division) is given priority over addition (and hence subtraction). For example, $1 + 2 \cdot 3 = 1+6 = 7$, but $(1+2) \cdot 3 = 3 \cdot 3 = 9$.  As you may recall, we can `distribute' the $3$ across the addition if we really wanted to do the multiplication first:  $(1+2) \cdot 3 = 1\cdot 3 + 2 \cdot 3 = 3 + 6 = 9$. More generally, we have the following.

\medskip

\phantomsection
\label{distributiveproperty}

\colorbox{ResultColor}{\bbm

\centerline{\textbf{The Distributive Property and Factoring}}
%\smallskip
For all real numbers $a$, $b$ and $c$:

\begin{itemize}

\item  \textbf{Distributive Property:}   $a(b+c) = ab + ac$ and $(a+b)c = ac + bc$.

\item  \textbf{Factoring:}\footnote{Or, as Carl calls it, `reading the Distributive Property from right to left.'}   $ab+ac = a(b+c)$ and $ac + bc = (a+b)c$.

\end{itemize}

\ebm}

\medskip

It is worth pointing out that we didn't really need to list the Distributive Property both for $a(b+c)$ (distributing from the left) and $(a+b)c$ (distributing from the right), since the commutative property of multiplication gives us one from the other.  Also, `factoring' really is the same equation as the distributive property, just read from right to left. These are the first of many redundancies in this section, and they exist in this review section for one reason only - in our experience, many students \textit{see} these things differently so we will list them as such.   

\smallskip

It is hard to overstate the importance of the Distributive Property.  For example, in the expression $5(2+x)$, without knowing the value of $x$, we cannot perform the addition inside the parentheses first;  we must rely on the distributive property here to get  $5(2+x) = 5\cdot 2 + 5 \cdot x = 10 + 5x$.  The Distributive Property is also responsible for combining `like terms'.  Why is $3x + 2x = 5x$?  Because  $3x + 2x = (3+2)x = 5x$.  

\smallskip

We continue our review with summaries of other properties of arithmetic, each of which can be derived from the properties listed above.  First up are properties of the additive identity $0$.

\medskip

\phantomsection
\label{propertiesofzero}

\colorbox{ResultColor}{\bbm

\centerline{\textbf{Properties of Zero}}

Suppose $a$ and $b$ are real numbers.

\begin{itemize}

\item  \textbf{Zero Product Property:} $ab = 0$ if and only if $a=0$ or $b=0$ (or both)

\textbf{Note:} This not only says that $0 \cdot a = 0$ for any real number $a$, it also says that the \textit{only} way to get an answer of `$0$' when multiplying two real numbers  is to have one (or both) of the numbers be `$0$' in the first place.

\item  \textbf{Zeros in Fractions:}  If $a \neq 0$, $\dfrac{0}{a} = 0 \cdot \left(\dfrac{1}{a}\right) = 0$.

\textbf{Note:}  The quantity $\dfrac{a}{0}$ is undefined.\footnote{The expression $\frac{0}{0}$ is technically an `indeterminant form' as opposed to being strictly `undefined' meaning that with Calculus we can make some sense of it in certain situations.}

\end{itemize}

\ebm}

\medskip

The Zero Product Property drives most of the equation solving algorithms in Algebra because it allows us to take complicated equations and reduce them to simpler ones.  For example, you may recall that one way to solve  $x^2+x-6=0$ is by factoring\footnote{Don't worry.  We'll review this in due course.  And, yes, this is our old friend the Distributive Property!} the left hand side of this equation to get  $(x-2)(x+3) = 0$.  From here, we apply the Zero Product Property and set each factor equal to zero.  This yields  $x-2=0$ or $x+3=0$ so $x=2$ or $x=-3$.  This application to solving equations leads, in turn,  to some deep and profound structure theorems about polynomials (see the Math 1010 textbook). 

\smallskip

Next up is a review of the arithmetic of `negatives'. On page \pageref{realnumberaddition} we first introduced the dash which we all recognize as the `negative' symbol in terms of the additive inverse.  For example, the number $-3$ (read `negative $3$') is defined so that $3 + (-3) = 0$.  We then defined subtraction using the concept of the additive inverse again so that, for example, $5 - 3 = 5 + (-3)$.  In this text we do not distinguish typographically between the dashes in the expressions `$5-3$' and `$-3$' even though they are mathematically quite different.\footnote{We're not just being lazy here.  We looked at many of the big publishers' Precalculus books and none of them use different dashes, either.} In the expression `$5-3$,' the dash is a \textit{binary} operation (that is, an operation requiring \textit{two} numbers) whereas in `$-3$', the dash is a \textit{unary} operation (that is, an operation requiring only one number).  You might ask, `Who cares?'  Your calculator does - that's who!  In the text we can write $-3 - 3 = -6$ but that will not work in your calculator.  Instead you'd need to type $^{-}3 - 3$ to get $-6$ where the first dash comes from the `$+/-$' key.

\medskip

\phantomsection
\label{propertiesofnegatives}

 \colorbox{ResultColor}{\bbm

\centerline{\textbf{Properties of Negatives}}
\smallskip
Given real numbers $a$ and $b$ we have the following.  

\begin{itemize}

\item  \textbf{Additive Inverse Properties:}  $-a = (-1)a$ and $-(-a) = a$

\item  \textbf{Products of Negatives:} $(-a)(-b) = ab$. 

\item  \textbf{Negatives and Products:} $-ab = -(ab) = (-a)b = a(-b)$.

\item  \textbf{Negatives and Fractions:} If $b$ is nonzero, $-\dfrac{a}{b} = \dfrac{-a}{b} = \dfrac{a}{-b}$ and $\dfrac{-a}{-b} = \dfrac{a}{b}$.

\item  \textbf{`Distributing' Negatives:}  $-(a+b) = -a-b$ and $-(a-b) = -a + b = b-a$.

\item  \textbf{`Factoring' Negatives:}\footnote{Or, as Carl calls it, reading `Distributing' Negatives from right to left.} $-a-b = -(a+b)$ and $b-a = -(a-b)$.

\end{itemize}

\ebm}

\medskip

An important point here is that when we `distribute' negatives, we do so across addition or subtraction only.  This is because we are really distributing a factor of $-1$ across each of these terms:  $-(a+b) = (-1)(a+b) = (-1)(a) + (-1)(b) = (-a)+(-b) = -a-b$. Negatives do not `distribute' across multiplication:  $- (2 \cdot 3) \neq (-2)\cdot(-3)$. Instead, $-(2\cdot 3) = (-2)\cdot (3) = (2) \cdot (-3) = -6$.  The same sort of thing goes for fractions:  $- \frac{3}{5}$ can be written as $\frac{-3}{5}$ or $\frac{3}{-5}$, but not $\frac{-3}{-5}$.  Speaking of fractions, we now review their arithmetic.


\phantomsection
\label{fractionarithmetic}

\colorbox{ResultColor}{\bbm

\centerline{\textbf{Properties of Fractions}}

Suppose $a$, $b$, $c$ and $d$ are real numbers.  Assume them to be nonzero whenever necessary; for example,  when they appear in a denominator.

\begin{itemize}

\item  \textbf{Identity Properties:}  $a = \dfrac{a}{1}$ and $\dfrac{a}{a} = 1$.

\item  \textbf{Fraction Equality:}  $\dfrac{a}{b} = \dfrac{c}{d}$ if and only if $ad = bc$. 

\item  \textbf{Multiplication of Fractions:}  $\dfrac{a}{b} \cdot \dfrac{c}{d} = \dfrac{ac}{bd}$. In particular:  $\dfrac{a}{b} \cdot c = \dfrac{a}{b} \cdot \dfrac{c}{1} = \dfrac{ac}{b}$

\textbf{Note:}  A common denominator is \textbf{not} required to \textbf{multiply} fractions!

\item  \textbf{Division\footnote{The old `invert and multiply' or `fraction gymnastics' play.} of Fractions:}  $\dfrac{a}{b} \div \dfrac{c}{d} = \dfrac{a}{b} \cdot \dfrac{d}{c} = \dfrac{ad}{bc}$. 

In particular: $1 \div \dfrac{a}{b} = \dfrac{b}{a}$ and  $\dfrac{a}{b} \div c = \dfrac{a}{b} \div \dfrac{c}{1}  = \dfrac{a}{b} \cdot \dfrac{1}{c} = \dfrac{a}{bc}$

\textbf{Note:}  A common denominator is \textbf{not} required to \textbf{divide} fractions!

\item  \textbf{Addition and Subtraction of Fractions:}  $\dfrac{a}{b} \pm \dfrac{c}{b} = \dfrac{a \pm c}{b}$.  

\textbf{Note:}  A common denominator \textbf{is} required to \textbf{add or subtract} fractions!

\item  \textbf{Equivalent Fractions:}  $\dfrac{a}{b} = \dfrac{ad}{bd}$, since $ \dfrac{a}{b} = \dfrac{a}{b} \cdot 1 = \dfrac{a}{b} \cdot \dfrac{d}{d} = \dfrac{ad}{bd}$

\textbf{Note:}  The \textit{only} way to change the denominator is to multiply both it and the numerator by the same nonzero value because we are, in essence, multiplying the fraction by $1$.

\item  \textbf{`Reducing'\footnote{Or `Cancelling' Common Factors - this is really just reading the previous property `from right to left'.} Fractions:} $\dfrac{a\cancel{d}}{b\cancel{d}} = \dfrac{a}{b}$, since  $\dfrac{ad}{bd} = \dfrac{a}{b} \cdot \dfrac{d}{d} = \dfrac{a}{b} \cdot 1 = \dfrac{a}{b}$.

In particular, $\dfrac{ab}{b} = a$ since $\dfrac{ab}{b} = \dfrac{ab}{1 \cdot b} =  \dfrac{a \cancel{b}}{1 \cdot \cancel{b}} = \dfrac{a}{1} = a$ and $\dfrac{b-a}{a-b} = \dfrac{(-1)\cancel{(a-b)}}{\cancel{(a-b)}} = -1$.

\textbf{Note:}  We may only cancel common \textbf{factors} from both numerator and denominator.

\end{itemize}

\ebm}

\medskip

Students make so many mistakes with fractions that we feel it is necessary to pause a moment in the narrative and offer you the following example.

\begin{ex} \label{fractionreview}  Perform the indicated operations and simplify. By `simplify' here, we mean to have the final answer written in the form $\frac{a}{b}$ where $a$ and $b$ are integers which have no common factors.  Said another way, we want $\frac{a}{b}$ in `lowest terms'.

\begin{multicols}{4}
\begin{enumerate}

\item $\dfrac{1}{4} + \dfrac{6}{7}$\vphantom{$\dfrac{\dfrac{12}{5} - \dfrac{7}{24}}{1 + \left(\dfrac{12}{5}\right) \left(\dfrac{7}{24}\right)}$}
\item $\dfrac{5}{12} - \left(\dfrac{47}{30} - \dfrac{7}{3}\right)$\vphantom{$\dfrac{\dfrac{12}{5} - \dfrac{7}{24}}{1 + \left(\dfrac{12}{5}\right) \left(\dfrac{7}{24}\right)}$}
\item $\dfrac{\dfrac{7}{3-5} - \dfrac{7}{3-5.21}}{5-5.21}$\vphantom{$\dfrac{\dfrac{12}{5} - \dfrac{7}{24}}{1 + \left(\dfrac{12}{5}\right) \left(\dfrac{7}{24}\right)}$}
\item $\dfrac{\dfrac{12}{5} - \dfrac{7}{24}}{1 + \left(\dfrac{12}{5}\right) \left(\dfrac{7}{24}\right)}$ 

\setcounter{HW}{\value{enumi}}
\end{enumerate}
\end{multicols}


\begin{multicols}{2}
\begin{enumerate}
\setcounter{enumi}{\value{HW}}

\item $\dfrac{(2(2)+1)(-3-(-3)) - 5(4-7)}{4-2(3)}$\vphantom{$\left(\dfrac{3}{5} \right) \left(\dfrac{5}{13} \right) - \left(\dfrac{4}{5}\right) \left( - \dfrac{12}{13}\right)$}
\item $\left(\dfrac{3}{5} \right) \left(\dfrac{5}{13} \right) - \left(\dfrac{4}{5}\right) \left( - \dfrac{12}{13}\right)$

\setcounter{HW}{\value{enumi}}
\end{enumerate}
\end{multicols}
\pagebreak

{\bf Solution.}

\begin{enumerate}

\item It may seem silly to start with an example this basic but experience has taught us not to take much for granted.  We start by finding the lowest common denominator and then we rewrite the fractions using that new denominator.  Since $4$ and $7$ are {\bf relatively prime}, meaning they have no factors in common, the lowest common denominator is $4 \cdot 7 = 28$.\[ \begin{array}{rclr}

\dfrac{1}{4} + \dfrac{6}{7} & = & \dfrac{1}{4} \cdot \dfrac{7}{7} + \dfrac{6}{7} \cdot \dfrac{4}{4} &  \text{Equivalent Fractions} \\ [10pt]
                                           & = & \dfrac{7}{28}  + \dfrac{24}{28} & \text{Multiplication of Fractions}\\ [10pt]
																					 & = & \dfrac{31}{28}                  & \text{Addition of Fractions} \\ \end{array} \]

The result is in lowest terms because $31$ and $28$ are relatively prime so we're done.

%%%%%%%%%%%%%%%%%%%

\item  We could begin with the subtraction in parentheses, namely $\frac{47}{30} - \frac{7}{3}$, and then subtract that result from $\frac{5}{12}$.  It's easier, however, to first distribute the negative across the quantity in parentheses and then use the Associative Property to perform all of the addition and subtraction in one step.\footnote{See the remark on page \pageref{howtoaddonetwothree} about how we add $1 + 2 + 3$.}  The lowest common denominator\footnote{We could have used $12 \cdot 30 \cdot 3 = 1080$ as our common denominator but then the numerators would become unnecessarily large.  It's best to use the \emph{lowest} common denominator.} for all three fractions is $60$.\[ \begin{array}{rclr}

\dfrac{5}{12} - \left(\dfrac{47}{30} - \dfrac{7}{3}\right) & = & \dfrac{5}{12} - \dfrac{47}{30} + \dfrac{7}{3} & \text{Distribute the Negative}\\ [10pt]
& = & \dfrac{5}{12} \cdot \dfrac{5}{5} - \dfrac{47}{30} \cdot \dfrac{2}{2} + \dfrac{7}{3} \cdot \dfrac{20}{20} & \text{Equivalent Fractions}\\ [10pt]
& = & \dfrac{25}{60} - \dfrac{94}{60} + \dfrac{140}{60} & \text{Multiplication of Fractions} \\ [10pt]
& = & \dfrac{71}{60} & \text{Addition and Subtraction of Fractions} \\ \end{array}\]

The numerator and denominator are relatively prime so the fraction is in lowest terms and we have our final answer.

%%%%%%%%%%%%%%%%%%%%%%%%%%%%%%%

\item What we are asked to simplify in this problem is known as a  `complex' or `compound' fraction.  Simply put, we have fractions within a fraction.  The longest division line\footnote{Also called a `vinculum'.} acts as a grouping symbol, quite literally dividing the compound fraction into a numerator (containing fractions) and a denominator (which in this case does not contain fractions).  The first step to simplifying a compound fraction like this one is to see if you can simplify the little fractions inside it.  To that end, we clean up the fractions in the numerator as follows.\[ \begin{array}{rclr}

 \dfrac{\dfrac{7}{3-5} - \dfrac{7}{3-5.21}}{5-5.21} & = & \dfrac{\dfrac{7}{-2} - \dfrac{7}{-2.21}}{-0.21} & \\ [10pt]
                                                    & = & \dfrac{-\left(-\dfrac{7}{2} + \dfrac{7}{2.21}\right)}{0.21} & \text{Properties of Negatives} \\ [10pt]
																										& = & \dfrac{\dfrac{7}{2} - \dfrac{7}{2.21}}{0.21} & \text{Distribute the Negative} \\ \end{array}\]
																										
We are left with a compound fraction with decimals.  We could replace $2.21$ with $\frac{221}{100}$ but that would make a mess.\footnote{Try it if you don't believe us.}  It's better in this case to eliminate the decimal by multiplying the numerator and denominator of the fraction with the decimal in it by $100$ (since $2.21 \cdot 100 = 221$ is an integer) as shown below.\[ \begin{array}{rclcl}

\dfrac{\dfrac{7}{2} - \dfrac{7}{2.21}}{0.21} & = & \dfrac{ \dfrac{7}{2} - \dfrac{7 \cdot 100}{2.21 \cdot 100}}{0.21} & = & \dfrac{\dfrac{7}{2} - \dfrac{700}{221}}{0.21}\\ \end{array}\]

We now perform the subtraction in the numerator and replace $0.21$ with $\frac{21}{100}$ in the denominator.  This will leave us with one fraction divided by another fraction.  We finish by performing the `division by a fraction is multiplication by the reciprocal' trick and then cancel any factors that we can.\[ \begin{array}{rclcl}
																										
\dfrac{\dfrac{7}{2}-\dfrac{700}{221}}{0.21} & = & \dfrac{\dfrac{7}{2}\cdot\dfrac{221}{221} - \dfrac{700}{221}\cdot\dfrac{2}{2}}{\dfrac{21}{100}} & = & \dfrac{\dfrac{1547}{442} -\dfrac{1400}{442}}{\dfrac{21}{100}} \\[10pt] 
		                                        & = & \dfrac{\dfrac{147}{442}}{\dfrac{21}{100}} = \dfrac{147}{442} \cdot \dfrac{100}{21} & = & \dfrac{14700}{9282} = \dfrac{350}{221} \\ \end{array}\] The last step comes from the factorizations $14700 = 42 \cdot 350$ and $9282 = 42 \cdot 221$.

%%%%%%%%%%%%%%%%%%%%%%%%%%%%%%%%%%%%

\item We are given another compound fraction to simplify and this time both the numerator and denominator contain fractions.  As before, the longest division line acts as a grouping symbol to separate the numerator from the denominator.\[ \begin{array}{rclr}

\dfrac{\dfrac{12}{5} - \dfrac{7}{24}}{1 + \left(\dfrac{12}{5}\right) \left(\dfrac{7}{24}\right)} & = & \dfrac{\left(\dfrac{12}{5} - \dfrac{7}{24}\right)}{\left(1 + \left(\dfrac{12}{5}\right) \left(\dfrac{7}{24}\right)\right)} & \end{array} \] 

Hence, one way to proceed is as before: simplify the numerator and the denominator then perform the `division by a fraction is the multiplication by the reciprocal' trick.  While there is nothing wrong with this approach, we'll use our Equivalent Fractions property to rid ourselves of the `compound' nature of this fraction straight away.  The idea is to multiply both the numerator and denominator by the lowest common denominator of each of the `smaller' fractions - in this case, $24 \cdot 5 = 120$.\[ \begin{array}{rclr}

 \dfrac{\left(\dfrac{12}{5} - \dfrac{7}{24}\right)}{\left(1 + \left(\dfrac{12}{5}\right) \left(\dfrac{7}{24}\right)\right)} & = &\dfrac{\left(\dfrac{12}{5} - \dfrac{7}{24}\right) \cdot 120}{\left(1 + \left(\dfrac{12}{5}\right) \left(\dfrac{7}{24}\right)\right) \cdot 120} & \text{Equivalent Fractions}\\ [30pt]

& = & \dfrac{\left(\dfrac{12}{5}\right) (120) - \left(\dfrac{7}{24}\right) (120)}{(1)(120) + \left(\dfrac{12}{5}\right) \left(\dfrac{7}{24}\right)(120)} & \text{Distributive Property} \\[30pt]

& = & \dfrac{\dfrac{12 \cdot 120}{5} - \dfrac{7 \cdot 120}{24}}{120 + \dfrac{12 \cdot 7 \cdot 120}{5 \cdot 24}} & \text{Multiply fractions} \\ [25pt]

& = & \dfrac{\dfrac{12 \cdot 24 \cdot \cancel{5}}{\cancel{5}} - \dfrac{7 \cdot 5 \cdot \cancel{24}}{\cancel{24}}}{120 + \dfrac{12 \cdot 7 \cdot \cancel{5} \cdot \cancel{24}}{\cancel{5} \cdot \cancel{24}}} & \text{Factor and cancel} \\[25pt]
 & = & \dfrac{(12 \cdot 24) - (7 \cdot 5)}{120 + (12 \cdot 7)} & \\[10pt]
 & = & \dfrac{288 - 35}{120 + 84} & \\[10pt]
 & = & \dfrac{253}{204} & \\
  \end{array} \] 
 
Since $253 = 11 \cdot 23$ and $204 = 2 \cdot 2 \cdot 3 \cdot 17$ have no common factors our result is in lowest terms which means we are done.

%%%%%%%%%%%%%%%%%%%%%%%%%%%%%%%%%%%%%%%%

																					
\item  This fraction may look simpler than the one before it, but the negative signs and parentheses mean that we shouldn't get complacent.  Again we note that the division line here acts as a grouping symbol.  That is, 

\[ \dfrac{(2(2)+1)(-3-(-3)) - 5(4-7)}{4-2(3)} = \dfrac{\left((2(2)+1)(-3-(-3)) - 5(4-7) \right)}{(4-2(3))} \]

This means that we should simplify the numerator and denominator first, then perform the division last.  We tend to what's in parentheses first, giving multiplication priority over addition and subtraction.\[ \begin{array}{rclr}


\dfrac{(2(2)+1)(-3-(-3)) - 5(4-7)}{4-2(3)} & = & \dfrac{(4+1)(-3+3)-5(-3)}{4 - 6} &  \\ [8pt]
                                           & = & \dfrac{(5)(0) + 15}{-2}  & \\ [8pt]
																					 & = & \dfrac{15}{-2} & \\ [8pt]
																					 & = & -\dfrac{15}{2} & \text{Properties of Negatives} \\ \end{array} \]
Since $15 = 3\cdot 5$ and $2$ have no common factors, we are done.
																			

%%%%%%%%%%%%%%%%%%%%%%%%%%%%%%


\item  In this problem, we have multiplication and subtraction.  Multiplication takes precedence so we perform it first.  Recall that to multiply fractions, we do \textit{not} need to obtain common denominators;  rather, we multiply the corresponding numerators together along with the corresponding denominators.  Like the previous example, we have parentheses and negative signs for added fun!\[ \begin{array}{rclr}

\left(\dfrac{3}{5} \right) \left(\dfrac{5}{13} \right) - \left(\dfrac{4}{5}\right) \left( - \dfrac{12}{13}\right) & = & \dfrac{3 \cdot 5}{5 \cdot 13} - \dfrac{4\cdot (-12)}{5 \cdot 13} & \text{Multiply fractions}\\ [8pt]

& = & \dfrac{15}{65} - \dfrac{-48}{65} & \\[10pt]
& = & \dfrac{15}{65} + \dfrac{48}{65} & \text{Properties of Negatives}\\[10pt]
& = & \dfrac{15+48}{65}  & \text{Add numerators} \\ [10pt]
& = & \dfrac{63}{65}  & \\ \end{array} \]

Since $64 = 3 \cdot 3 \cdot 7$ and $65 = 5 \cdot 13$ have no common factors, our answer $\frac{63}{65}$ is in lowest terms and we are done.\qed

\end{enumerate}

\end{ex} 

Of the issues discussed in the previous set of examples none causes students more trouble than simplifying compound fractions.  We presented two different methods for simplifying them:  one in which we simplified the overall numerator and denominator and then performed the division and one in which we removed the compound nature of the fraction at the very beginning.   We encourage the reader to go back and use both methods on each of the compound fractions presented.  Keep in mind that when a compound fraction is encountered in the rest of the text it will usually be simplified using only one method and we may not choose your favorite method.  Feel free to use the other one in your notes.

\smallskip

Next, we review exponents and their properties.  Recall that $2 \cdot 2 \cdot 2$  can be written as $2^3$ because exponential notation expresses repeated multiplication.  In the expression $2^3$, $2$ is called the \textbf{base} and $3$ is called the \textbf{exponent}. In order to generalize exponents from natural numbers to the integers, and eventually to rational and real numbers, it is helpful to think of the exponent as a count of the number of factors of the base we are multiplying by $1$.  For instance, \[2^3 = 1 \cdot (\text{three factors of two}) = 1 \cdot (2 \cdot 2 \cdot 2) = 8.\] From this, it makes sense that \[2^{0} = 1 \cdot (\text{zero factors of two}) = 1.\]  What about $2^{-3}$?  The `$-$' in the exponent indicates that we are `taking away' three factors of two, essentially dividing by three factors of two.  So, \[2^{-3} = 1 \div (\text{three factors of two}) = 1 \div (2 \cdot 2 \cdot 2) = \frac{1}{2 \cdot 2 \cdot 2} = \frac{1}{8}.\]  We summarize the properties of integer exponents below.

\medskip

\phantomsection
\label{propertiesofintegerexponents}

\colorbox{ResultColor}{\bbm

\centerline{\textbf{Properties of Integer Exponents}}

\vspace{.05in}

Suppose $a$ and $b$ are nonzero real numbers and $n$ and $m$ are integers.

\begin{itemize}

\item  \textbf{Product Rules:} $(ab)^{n} = a^n b^n$ and $a^n a^m = a^{n+m}$.

\item  \textbf{Quotient Rules:} $\left(\dfrac{a}{b}\right)^n = \dfrac{a^n}{b^n}$ and $\dfrac{a^n}{a^m} = a^{n-m}$. 

\item \textbf{Power Rule:}  $\left(a^{n}\right)^{m} = a^{nm}$.

\item  \textbf{Negatives in Exponents:}  $a^{-n} = \dfrac{1}{a^n}$.

 In particular, $\left(\dfrac{a}{b}\right)^{-n} = \left(\dfrac{b}{a}\right)^{n} = \dfrac{b^n}{a^n}$ and $\dfrac{1}{a^{-n}} = a^{n}$.

\item  \textbf{Zero Powers:}  $a^{0} = 1$.

\textbf{Note:}  The expression $0^{0}$ is an indeterminate form.\footnote{See the comment regarding `$\frac{0}{0}$' on page \pageref{propertiesofzero}.}

\item  \textbf{Powers of Zero:}  For any \textit{natural} number $n$, $0^{n} = 0$.

\textbf{Note:}  The expression $0^{n}$ for integers $n \leq 0$ is not defined.

\end{itemize}

\ebm}

While it is important the state the Properties of Exponents, it is also equally important to take a moment to discuss one of the most common errors in Algebra.  It is true that $(ab)^2 = a^2 b^2$ (which some students refer to as `distributing' the exponent to each factor) but you cannot do this sort of thing with addition.  That is, in general,   $(a+b)^2 \neq a^2 + b^2$. (For example, take $a= 3$ and $b = 4$.)  The same goes for any other powers.

\smallskip

With exponents now in the mix, we can now state the Order of Operations Agreement.

\medskip

\phantomsection
\label{orderofoperations}

\colorbox{ResultColor}{\bbm

\centerline{\textbf{Order of Operations Agreement}}

\vspace{.05in}

When evaluating an expression involving real numbers:

\begin{enumerate}

\item  Evaluate any expressions in \textbf{p}arentheses (or other grouping symbols.)
\item  Evaluate \textbf{e}xponents.
\item  Evaluate \textbf{m}ultiplication and \textbf{d}ivision as you read from left to right.
\item  Evaluate \textbf{a}ddition and \textbf{s}ubtraction as you read from left to right.

\end{enumerate}

 We note that there are many useful mnemonic device for remembering the order of operations.\footnote{Our favourite is  `\textbf{P}lease \textbf{e}ntertain \textbf{m}y \textbf{d}ear \textbf{a}uld \textbf{S}asquatch.'}  

\ebm}

\medskip

For example, $2 + 3\cdot 4^2 = 2 + 3\cdot 16 = 2 + 48 = 50$.  Where students get into trouble is with things like $-3^2$.  If we think of this as $0 - 3^2$, then it is clear that we evaluate the exponent first:  $-3^2 =0 -3^2 =0 -9 = -9$.  In general, we interpret $-a^n = -\left(a^n\right)$.  If we want the `negative' to also be raised to a power, we must  write $(-a)^n$ instead.  To summarize, $-3^2 = -9$ but $(-3)^2  = 9$. 

\smallskip

Of course, many of the `properties' we've stated in this section can be viewed as ways to circumvent the order of operations. We've already seen how the distributive property allows us to simplify $5(2+x)$ by performing the indicated multiplication \textbf{before} the addition that's in parentheses.  Similarly, consider trying to evaluate $2^{30172}\cdot 2^{-30169}$.  The Order of Operations Agreement demands that the exponents be dealt with first, however, trying to compute $2^{30172}$ is a challenge, even for a calculator.  One of the Product Rules of Exponents, however, allow us to rewrite this product, essentially performing the multiplication first, to get:  $2^{30172-30169} = 2^{3} = 8$.  

\smallskip

Let's take a break and enjoy another example.

\smallskip

\begin{ex} \label{exponentreview}  Perform the indicated operations and simplify.

\begin{multicols}{2}

\begin{enumerate}

\item  $\dfrac{(4-2)(2 \cdot 4)-(4)^2}{(4-2)^2}$

\item $12(-5)(-5+3)^{-4}+6(-5)^2(-4)(-5+3)^{-5}$\vphantom{$\dfrac{(4-2)(2 \cdot 4)-(4)^2}{(4-2)^2}$}

\setcounter{HW}{\value{enumi}}

\end{enumerate}

\end{multicols}

\begin{multicols}{2}

\begin{enumerate}

\setcounter{enumi}{\value{HW}}

\item  $\dfrac{\left(\dfrac{5\cdot 3^{51}}{4^{36}}\right)}{\left(\dfrac{5 \cdot 3^{49}}{4^{34}}\right)}$

\item $\dfrac{2 \left(\dfrac{5}{12}\right)^{-1}}{1 - \left(\dfrac{5}{12}\right)^{-2}}$\vphantom{$\dfrac{\left(\dfrac{5\cdot 3^{51}}{4^{36}}\right)}{\left(\dfrac{5 \cdot 3^{49}}{4^{34}}\right)}$}

\end{enumerate}

\end{multicols}


{\bf Solution.}

\begin{enumerate}

\item  We begin working inside parentheses then deal with the exponents before working through the other operations.  As we saw in Example \ref{fractionreview}, the division here acts as a grouping symbol, so we save the division to the end.\[ \begin{array}{rclcl}

\dfrac{(4-2)(2 \cdot 4)-(4)^2}{(4-2)^2} & = & \dfrac{(2)(8)-(4)^2}{(2)^2} & = & \dfrac{(2)(8)-16}{4} \\ [10pt]
                                        & = & \dfrac{16-16}{4} = \dfrac{0}{4} & = & 0 \\ \end{array}\]

\item  As before, we simplify what's in the parentheses first, then work our way through the exponents, multiplication, and finally, the addition.\[ \begin{array}{rclr}

12(-5)(-5+3)^{-4}+6(-5)^2(-4)(-5+3)^{-5} & = & 12(-5)(-2)^{-4} + 6(-5)^{2}(-4)(-2)^{-5} \\ [10pt]
                                         & = & 12(-5)\left(\dfrac{1}{(-2)^4}\right) + 6(-5)^{2}(-4)\left(\dfrac{1}{(-2)^5}\right)& \\ [10pt]
                                        
                                         & = & 12(-5)\left(\dfrac{1}{16}\right) + 6(25)(-4)\left(\dfrac{1}{-32}\right)& \\ [10pt]
																				
								& = & (-60)\left(\dfrac{1}{16}\right) + (-600)\left(\dfrac{1}{-32}\right)& \\ [10pt]

	& = & \dfrac{-60}{16} + \left(\dfrac{-600}{-32}\right) =\dfrac{-15\cdot \cancel{4}}{4 \cdot \cancel{4}} + \dfrac{-75 \cdot \cancel{8}}{-4 \cdot \cancel{8}}  & \\ [10pt]
				& = & \dfrac{-15}{4} + \dfrac{-75}{-4}  =\dfrac{-15}{4} + \dfrac{75}{4} &\\ [10pt]
				& = & \dfrac{-15 + 75}{4}  =\dfrac{60}{4} &\\ [10pt]
	       & = & 15 & \\  \end{array}\]
				
\pagebreak

\item  The Order of Operations Agreement mandates that we work within each set of parentheses first, giving precedence to the exponents, then the multiplication, and, finally the division.  The trouble with this approach is that the exponents are so large that computation becomes a trifle unwieldy.   What we observe, however, is that the bases of the exponential expressions, $3$ and $4$, occur in both the numerator and denominator of the compound fraction, giving us hope that we can use some of the Properties of Exponents (the Quotient Rule, in particular) to help us out. Our first step here is to invert and multiply.  We see immediately that the $5$'s cancel after which we group the powers of $3$ together and the powers of $4$ together and apply the properties of exponents.\[ \begin{array}{rclclcl}

\dfrac{\left(\dfrac{5\cdot 3^{51}}{4^{36}}\right)}{\left(\dfrac{5 \cdot 3^{49}}{4^{34}}\right)} & = & \dfrac{5\cdot 3^{51}}{4^{36}} \cdot \dfrac{4^{34}}{5 \cdot 3^{49}} & = & \dfrac{\cancel{5} \cdot 3^{51} \cdot 4^{34}}{\cancel{5} \cdot 3^{49} \cdot 4^{36}} & = & \dfrac{3^{51}}{3^{49}} \cdot\dfrac{4^{34}}{4^{36}} \\

& = & 3^{51-49} \cdot 4^{34-36} & = & 3^{2} \cdot 4^{-2} & = & 3^{2} \cdot \left( \dfrac{1}{4^2}\right) \\

& = & 9 \cdot \left(\dfrac{1}{16} \right) & = & \dfrac{9}{16} & & \\ \end{array} \]

\item We have yet another instance of a compound fraction so our first order of business is to rid ourselves of the compound nature of the fraction like we did in Example \ref{fractionreview}.  To do this, however, we need to tend to the exponents first so that we can determine what common denominator is needed to simplify the fraction.\[ \begin{array}{rclclcl} \dfrac{2 \left(\dfrac{5}{12}\right)^{-1}}{1 - \left(\dfrac{5}{12}\right)^{-2}} & = & \dfrac{2 \left(\dfrac{12}{5}\right)}{1 - \left(\dfrac{12}{5}\right)^{2}} & = & \dfrac{\left(\dfrac{24}{5}\right)}{1 - \left(\dfrac{12^2}{5^2}\right)} & = & \dfrac{\left(\dfrac{24}{5}\right)}{1 - \left(\dfrac{144}{25}\right)} \\ [30pt]

& = & \dfrac{\left(\dfrac{24}{5}\right) \cdot 25}{\left(1 - \dfrac{144}{25}\right)\cdot 25} & = & \dfrac{\left(\dfrac{24\cdot 5 \cdot \cancel{5}}{\cancel{5}}\right)}{\left(1 \cdot 25 - \dfrac{144 \cdot \cancel{25}}{\cancel{25}}\right)} & = & \dfrac{120}{25-144} \\ [30pt]
& = & \dfrac{120}{-119} = -\dfrac{120}{119} & & & & \\  \end{array} \]

Since $120$ and $119$ have no common factors, we are done.  \qed

\end{enumerate}

\end{ex}

\medskip

One of the places where the properties of exponents play an important role is in the use of \textbf{Scientific Notation}.  The basis for scientific notation is that since we use \underline{dec}imals (base ten numerals) to represent real numbers, we can adjust where the decimal point lies by multiplying by an appropriate power of 10.  This allows scientists and engineers to focus in on the `significant' digits\footnote{Awesome pun!} of a number - the nonzero values - and adjust for the decimal places later.  For instance, $-621 = -6.21 \times 10^2$ and $0.023 = 2.3 \times 10^{-2}$.  Notice here that we revert to using the familiar `$\times$' to indicate multiplication.\footnote{This is the `notable exception' we alluded to earlier.}   In general, we arrange the real number so exactly one non-zero digit appears to the left of the decimal point.  We make this idea precise in the following: 

\medskip

\colorbox{ResultColor}{\bbm

\begin{defn} \label{scientificnotation}

A real number is written in \textbf{Scientific Notation} if it has the form $\pm n . d_{1} d_{2} \ldots \times 10^{k}$ where $n$ is a natural number, $d_{1}$, $d_{2}$, etc., are whole numbers, and $k$ is an integer.

\end{defn}

\ebm}

\medskip

On calculators, scientific notation may appear using an `E' or `EE' as opposed to the $\times$ symbol.  For instance, while we will write $6.02 \times 10^{23}$ in the text, the calculator may display $6.02\, \text{E} \, 23$ or $6.02\, \text{EE} \, 23$. 

\begin{ex} \label{scientificnotationex} Perform the indicated operations and simplify.  Write your final answer in scientific notation, rounded to two decimal places.

\begin{multicols}{2}

\begin{enumerate}

\item  $\dfrac{\left(6.626 \times 10^{-34} \right) \left(3.14 \times 10^{9}\right)}{1.78 \times 10^{23}}$

\item  $\left(2.13 \times 10^{53}\right)^{100}$\vphantom{$\dfrac{\left(6.626 \times 10^{-34} \right) \left(3.14 \times 10^{9}\right)}{6.02 \times 10^{23}}$}

\end{enumerate}

\end{multicols}

{\bf Solution.}

\begin{enumerate}

\item  As mentioned earlier, the point of scientific notation is to separate out the `significant' parts of a calculation and deal with the powers of $10$ later.  In that spirit, we separate out the powers of $10$ in both the numerator and the denominator and proceed as follows \[ \begin{array}{rclr}

\dfrac{\left(6.626 \times 10^{-34} \right) \left(3.14 \times 10^{9}\right)}{1.78 \times 10^{23}} & = & \dfrac{(6.626)(3.14)}{1.78} \cdot \dfrac{10^{-34} \cdot 10^{9}}{10^{23}} & \\[8pt]
& = & \dfrac{20.80564}{1.78} \cdot \dfrac{10^{-34 + 9}}{10^{23}} & \\ [8pt]
& = & 11.685 \ldots \cdot \dfrac{10^{-25}}{10^{23}} & \\ [8pt]
& = & 11.685 \ldots \times 10^{-25-23} & \\
& = & 11.685 \ldots \times 10^{-48} & \\
\end{array} \]

We are asked to write our final answer in scientific notation, rounded to two decimal places.  To do this, we note that  $11.685 \ldots = 1.1685 \ldots \times 10^{1}$, so\[ 11.685 \ldots \times 10^{-48} = 1.1685 \ldots \times 10^{1} \times 10^{-48} = 1.1685 \ldots \times 10^{1-48} = 1.1685 \ldots \times 10^{-47} \] Our final answer, rounded to two decimal places, is $1.17 \times 10^{-47}$.  

\smallskip

We could have done that whole computation on a calculator so why did we bother doing any of this by hand in the first place?  The answer lies in the next example.

\item If you try to compute  $\left(2.13 \times 10^{53}\right)^{100}$ using most hand-held calculators, you'll most likely get an `overflow' error.  It is possible, however, to use the calculator in combination with the properties of exponents to compute this number.  Using properties of exponents, we get:

\[ \begin{array}{rclr}

\left(2.13 \times 10^{53}\right)^{100} & = & (2.13)^{100} \left(10^{53}\right)^{100} & \\
																			 & = & \left(6.885 \ldots \times 10^{32}\right) \left(10^{53 \times 100}\right) & \\
																			 & = & \left(6.885 \ldots \times 10^{32}\right) \left(10^{5300}\right) & \\
																			 & = & 6.885 \ldots \times 10^{32} \cdot 10^{5300} & \\
																			 & = & 6.885 \ldots \times 10^{5332} & \\ \end{array} \]
To two decimal places our answer is $6.88 \times 10^{5332}$. \qed


\end{enumerate}

\end{ex}

We close our review of real number arithmetic with a discussion of roots and radical notation.  Just as subtraction and division were defined in terms of the inverse of addition and multiplication, respectively, we define roots by undoing natural number exponents.

\medskip

\colorbox{ResultColor}{\bbm

\begin{defn} \label{principalnthrootdefn} Let $a$ be a real number and let $n$ be a natural number.  If $n$ is odd, then the \textbf{principal \boldmath $n^{\textbf{th}}$ root} of $a$ (denoted $\sqrt[n]{a}$) is the unique real number satisfying $\left(\sqrt[n]{a}\right)^n = a$.  If $n$ is even, $\sqrt[n]{a}$ is defined similarly provided  $a \geq 0$ and $\sqrt[n]{a} \geq 0$.  The number $n$ is called the \textbf{index} of the root and the the number $a$ is called the \textbf{radicand}.  For $n=2$, we write $\sqrt{a}$ instead of $\sqrt[2]{a}$.

\end{defn}

\ebm}

\medskip

The reasons for the added stipulations for even-indexed roots in Definition \ref{principalnthrootdefn} can be found in the Properties of Negatives.  First, for all real numbers,  $x^{\text{even power}} \geq 0$, which means it is never negative.  Thus if $a$ is a \textit{negative} real number, there are no real numbers $x$ with $x^{\text{even power}} = a$.  This is why if $n$ is even, $\sqrt[n]{a}$ only exists if $a \geq 0$.  The second restriction for even-indexed roots is that $\sqrt[n]{a} \geq 0$.  This comes from the fact that $x^{\text{even power}} = (-x)^{\text{even power}}$, and we require $\sqrt[n]{a}$ to have just one value.  So even though $2^{4} = 16$ and $(-2)^{4} = 16$, we require $\sqrt[4]{16} = 2$ and ignore $-2$.  

\smallskip

Dealing with odd powers is much easier. For example, $x^3 = -8$ has one and only one real solution, namely $x = -2$, which means not only does $\sqrt[3]{-8}$ exist, there is only one choice, namely $\sqrt[3]{-8} = -2$. Of course, when it comes to solving $x^{5213} = -117$, it's not so clear that there is one and only one real solution, let alone that the solution is $\sqrt[5213]{-117}$. Such pills are easier to swallow once we've thought a bit about such equations graphically, and ultimately, these things come from the completeness property of the real numbers mentioned earlier.  

\smallskip

We list properties of radicals below as a `theorem' since they can be justified using the properties of exponents.

\medskip

\colorbox{ResultColor}{\bbm
\begin{thm}  \textbf{Properties of Radicals:} Let $a$ and $b$ be real numbers and let $m$ and $n$ be natural numbers.  If $\sqrt[n]{a}$ and $\sqrt[n]{b}$ are real numbers, then

\label{radicalprops}

\begin{itemize}

\item  \textbf{Product Rule:}  $\sqrt[n]{ab} = \sqrt[n]{a} \, \sqrt[n]{b}$ 

\item  \textbf{Quotient Rule:}  $\sqrt[n]{\dfrac{a}{b}} = \dfrac{\sqrt[n]{a}}{\sqrt[n]{b}}$, provided $b \neq 0$. 

\item  \textbf{Power Rule:} $\sqrt[n]{a^m} = \left(\sqrt[n]{a}\right)^m$ 

\end{itemize}

\end{thm}

\ebm}

\medskip

The proof of Theorem \ref{radicalprops} is based on the definition of the principal $n^{\textbf{th}}$ root and the Properties of Exponents.  To establish the product rule, consider the following.  If $n$ is odd, then by definition $\sqrt[n]{ab}$ is the \underline{unique} real number such that $(\sqrt[n]{ab})^{n} = ab$.  Given that $( \sqrt[n]{a} \, \sqrt[n]{b})^n = (\sqrt[n]{a})^n (\sqrt[n]{b})^n = ab$ as well, it must be the case that $\sqrt[n]{ab} = \sqrt[n]{a} \, \sqrt[n]{b}$. If $n$ is even, then $\sqrt[n]{ab}$ is the unique non-negative real number such that $(\sqrt[n]{ab})^{n} = ab$.  Note that since $n$ is even, $\sqrt[n]{a}$ and $\sqrt[n]{b}$ are also non-negative thus $\sqrt[n]{a}\sqrt[n]{b} \geq 0$ as well.  Proceeding as above, we find that $\sqrt[n]{ab} = \sqrt[n]{a} \, \sqrt[n]{b}$.  The quotient rule is proved similarly and is left as an exercise.  The power rule results from repeated application of the product rule, so long as $\sqrt[n]{a}$ is a real number to start with.  We leave that as an exercise as well.

\smallskip

We pause here to point out one of the most common errors students make when working with radicals.  Obviously $\sqrt{9} = 3$, $\sqrt{16} = 4$ and $\sqrt{9 + 16} = \sqrt{25} = 5$.  Thus we can clearly see that $5 = \sqrt{25} = \sqrt{9 + 16} \neq \sqrt{9} + \sqrt{16} = 3 + 4 = 7$ because we all know that $5 \neq 7$.  The authors urge you to never consider `distributing' roots or exponents.  It's wrong and no good will come of it because in general $\sqrt[n]{a+b} \neq \sqrt[n]{a} + \sqrt[n]{b}$. 

\phantomsection
\label{donotdistributeexponents}

\smallskip

Since radicals have properties inherited from exponents, they are often written as such.  We define rational exponents in terms of radicals in the box below.

\medskip

\colorbox{ResultColor}{\bbm

\begin{defn}  \label{rationalexponentdefn} Let $a$ be a real number, let $m$ be an integer and let $n$ be a natural number. 

\begin{itemize}

\item  $a^{\frac{1}{n}} = \sqrt[n]{a}$ whenever $\sqrt[n]{a}$ is a real number.\footnote{If $n$ is even we need $a \geq 0$.}

\item  $a^{\frac{m}{n}}  = \left(\sqrt[n]{a}\right)^m = \sqrt[n]{a^m}$ whenever $\sqrt[n]{a}$ is a real number.

\end{itemize}
\end{defn}

\ebm}

\medskip

It would make life really nice if the rational exponents defined in Definition \ref{rationalexponentdefn} had all of the same properties that integer exponents have as listed on page \pageref{propertiesofintegerexponents}  - but they don't.  Why not?  Let's look at an example to see what goes wrong.  Consider the Product Rule which says that $(ab)^{n} = a^{n}b^{n}$ and let $a = -16$, $b = -81$ and $n = \frac{1}{4}$.  Plugging the values into the Product Rule yields the equation $((-16)(-81))^{1/4} = (-16)^{1/4}(-81)^{1/4}$.  The left side of this equation is $1296^{1/4}$ which equals $6$ but the right side is undefined because neither root is a real number.  Would it help if, when it comes to even roots (as signified by even denominators in the fractional exponents), we ensure that everything they apply to is non-negative?  That works for some of the rules - we leave it as an exercise to see which ones - but does not work for the Power Rule.

\smallskip
 
Consider the expression $\left(a^{2/3}\right)^{3/2}$.  Applying the usual laws of exponents, we'd be tempted to simplify this as $\left(a^{2/3}\right)^{3/2} = a^{\frac{2}{3} \cdot \frac{3}{2}} = a^{1} = a$.  However, if we substitute $a=-1$ and apply Definition \ref{rationalexponentdefn}, we find $(-1)^{2/3} = \left(\sqrt[3]{-1}\right)^2 = (-1)^2 = 1$ so that $\left((-1)^{2/3}\right)^{3/2} = 1^{3/2} = \left(\sqrt{1}\right)^3 = 1^3 = 1$.  Thus in this case we have $\left(a^{2/3}\right)^{3/2} \neq a$ even though all of the roots were defined.  It is true, however, that $\left(a^{3/2}\right)^{2/3} = a$  and we leave this for the reader to show.  The moral of the story is that when simplifying powers of rational exponents where the base is negative or worse, unknown, it's usually best to rewrite them as radicals.


\begin{ex}  Perform the indicated operations and simplify. 

  
\begin{multicols}{2}

\begin{enumerate}

\item  $\dfrac{-(-4) - \sqrt{(-4)^2-4(2)(-3)}}{2(2)}$\vphantom{$\dfrac{2 \left( \dfrac{\sqrt{3}}{3}\right)}{1 - \left( \dfrac{\sqrt{3}}{3} \right)^2}$}

\item  $\dfrac{2 \left( \dfrac{\sqrt{3}}{3}\right)}{1 - \left( \dfrac{\sqrt{3}}{3} \right)^2}$

\setcounter{HW}{\value{enumi}}

\end{enumerate}

\end{multicols}

\begin{multicols}{2}

\begin{enumerate}

\setcounter{enumi}{\value{HW}}

\item  $(\sqrt[3]{-2} - \sqrt[3]{-54})^2$\vphantom{ $2 \left(\dfrac{9}{4} - 3\right)^{1/3} + 2\left(\dfrac{9}{4}\right)\left(\dfrac{1}{3}\right)\left(\dfrac{9}{4}-3\right)^{-2/3}$}

\item  $2 \left(\dfrac{9}{4} - 3\right)^{1/3} + 2\left(\dfrac{9}{4}\right)\left(\dfrac{1}{3}\right)\left(\dfrac{9}{4}-3\right)^{-2/3}$

\end{enumerate}

\end{multicols}


{\bf Solution.}  

\begin{enumerate}

\item  We begin in the numerator and note that the radical here acts a grouping symbol,\footnote{The line extending horizontally from the square root symbol `$\sqrt{\vphantom{2}}$ is, you guessed it, another vinculum.} so our first order of business is to simplify the radicand.\[ \begin{array}{rclr}

\dfrac{-(-4) -\sqrt{(-4)^2-4(2)(-3)}}{2(2)}  & = & \dfrac{-(-4) - \sqrt{16-4(2)(-3)}}{2(2)} & \\[8pt]
                                             & = & \dfrac{-(-4) - \sqrt{16-4(-6)}}{2(2)} & \\[8pt]
																						& = & \dfrac{-(-4) - \sqrt{16-(-24)}}{2(2)} & \\[8pt]
				                                     & = & \dfrac{-(-4) - \sqrt{16+24}}{2(2)} & \\[8pt]
																						    & = & \dfrac{-(-4) - \sqrt{40}}{2(2)} & \\ \end{array} \] As you may recall, $40$ can be factored using a perfect square as $40 = 4 \cdot 10$ so we use the product rule of radicals to write $\sqrt{40} = \sqrt{4 \cdot 10} = \sqrt{4} \sqrt{10} = 2 \sqrt{10}$.  This lets us factor a `$2$' out of both terms in the numerator, eventually allowing us to cancel it with a factor of $2$ in the denominator.\[ \begin{array}{rclcl}

 \dfrac{-(-4) - \sqrt{40}}{2(2)} & = &  \dfrac{-(-4) - 2\sqrt{10}}{2(2)} & = &  \dfrac{4  - 2\sqrt{10}}{2(2)} \\ [8pt]
                                 & = &  \dfrac{2 \cdot 2  - 2\sqrt{10}}{2(2)} & = &  \dfrac{2(2  - \sqrt{10})}{2(2)} \\ [8pt]
																& = &  \dfrac{\cancel{2}(2  - \sqrt{10})}{\cancel{2}(2)} & = &  \dfrac{2  - \sqrt{10}}{2} \\ \end{array} \]Since the numerator and denominator have no more common factors,\footnote{Do you see why we aren't `cancelling' the remaining $2$'s?} we are done.

\item  Once again we have a compound fraction, so we first simplify the exponent in the denominator to see which factor we'll need to multiply by in order to clean up the fraction.\[ \begin{array}{rclcl}

\dfrac{2 \left( \dfrac{\sqrt{3}}{3}\right)}{1 - \left( \dfrac{\sqrt{3}}{3} \right)^2} & = & \dfrac{2 \left( \dfrac{\sqrt{3}}{3}\right)}{1 - \left( \dfrac{(\sqrt{3})^2}{3^2} \right)} & = & \dfrac{2 \left( \dfrac{\sqrt{3}}{3}\right)}{1 - \left( \dfrac{3}{9} \right)}\\ [20pt]
																				
& = & \dfrac{2 \left( \dfrac{\sqrt{3}}{3}\right)}{1 - \left( \dfrac{1 \cdot \cancel{3}}{3 \cdot \cancel{3}} \right)} & = & \dfrac{2 \left( \dfrac{\sqrt{3}}{3}\right)}{1 - \left( \dfrac{1}{3} \right)} \\[20pt]
																
& = & \dfrac{2 \left( \dfrac{\sqrt{3}}{3}\right) \cdot 3}{\left(1 - \left( \dfrac{1}{3} \right)\right) \cdot 3} & = & \dfrac{\dfrac{2 \cdot \sqrt{3} \cdot \cancel{3}}{\cancel{3}}}{1\cdot 3 -  \dfrac{1\cdot \cancel{3}}{\cancel{3}}} \\[20pt]

& = & \dfrac{2 \sqrt{3}}{3 - 1} & = & \dfrac{\cancel{2} \sqrt{3}}{\cancel{2}} = \sqrt{3} \\ \end{array} \]

\item  Working inside the parentheses, we first encounter $\sqrt[3]{-2}$.  While the $-2$ isn't a perfect cube,\footnote{Of an integer, that is!} we may think of $-2 = (-1)(2)$.  Since $(-1)^3 = -1$, $-1$ \textit{is} a perfect cube, and we may write $\sqrt[3]{-2} = \sqrt[3]{(-1)(2)} = \sqrt[3]{-1} \sqrt[3]{2} = - \sqrt[3]{2}$. When it comes to $\sqrt[3]{54}$, we may write it as $\sqrt[3]{(-27)(2)} = \sqrt[3]{-27} \sqrt[3]{2} = -3 \sqrt[3]{2}$.  So, \[\sqrt[3]{-2} - \sqrt[3]{-54} = -\sqrt[3]{2} - (-3\sqrt[3]{2}) = -\sqrt[3]{2} + 3 \sqrt[3]{2}.\]  At this stage, we can simplify $-\sqrt[3]{2} + 3 \sqrt[3]{2} = 2 \sqrt[3]{2}$.  You may remember this as being called `combining like radicals,' but it is in fact just another application of the distributive property:  \[-\sqrt[3]{2} + 3\sqrt[3]{2} = (-1)\sqrt[3]{2} + 3 \sqrt[3]{2} = (-1+3)\sqrt[3]{2} = 2\sqrt[3]{2}.\]  Putting all this together, we get:\[ \begin{array}{rclcl}
  (\sqrt[3]{-2} - \sqrt[3]{-54})^2 & = & (-\sqrt[3]{2} + 3 \sqrt[3]{2})^2 & = & (2 \sqrt[3]{2})^2  \\ [5pt]
																	 & = & 2^2 (\sqrt[3]{2})^2 = 4 \sqrt[3]{2^2} & = & 4 \sqrt[3]{4} \\ \end{array} \] Since there are no perfect integer cubes which are factors of $4$ (apart from $1$, of course), we are done.



\item  We start working in parentheses and get a common denominator to subtract the fractions:\[ \dfrac{9}{4} - 3 = \dfrac{9}{4} - \dfrac{3 \cdot 4}{1 \cdot 4} = \dfrac{9}{4} - \dfrac{12}{4} = \dfrac{-3}{4}  \] Since the denominators in the fractional exponents are odd, we can proceed using the properties of exponents:\[ \begin{array}{rclr}

2 \left(\dfrac{9}{4} - 3\right)^{1/3} + 2\left(\dfrac{9}{4}\right)\left(\dfrac{1}{3}\right)\left(\dfrac{9}{4}-3\right)^{-2/3} & = &2 \left(\dfrac{-3}{4} \right)^{1/3} + 2\left(\dfrac{9}{4}\right)\left(\dfrac{1}{3}\right)\left(\dfrac{-3}{4}\right)^{-2/3} & \\ [3pt]

& = & 2 \left(\dfrac{(-3)^{1/3}}{(4)^{1/3}} \right) + 2\left(\dfrac{9}{4}\right)\left(\dfrac{1}{3}\right)\left(\dfrac{4}{-3}\right)^{2/3} & \\ [3pt]

& = & 2 \left(\dfrac{(-3)^{1/3}}{(4)^{1/3}} \right) + 2\left(\dfrac{9}{4}\right)\left(\dfrac{1}{3}\right)\left(\dfrac{(4)^{2/3}}{(-3)^{2/3}}\right) & \\ [3pt]

& = & \dfrac{2 \cdot (-3)^{1/3}}{4^{1/3}} + \dfrac{2 \cdot 9 \cdot 1 \cdot 4^{2/3}}{4 \cdot 3 \cdot (-3)^{2/3}} & \\ [3pt]

& = & \dfrac{2 \cdot (-3)^{1/3}}{4^{1/3}} + \dfrac{\cancel{2} \cdot 3 \cdot \cancel{3} \cdot 4^{2/3}}{2 \cdot \cancel{2} \cdot \cancel{3} \cdot (-3)^{2/3}} & \\ [3pt]

& = & \dfrac{2 \cdot (-3)^{1/3}}{4^{1/3}} + \dfrac{3 \cdot 4^{2/3}}{2 \cdot (-3)^{2/3}} & \\ \end{array} \] At this point, we could start looking for common denominators but it turns out that these fractions reduce even further.  Since $4 = 2^2$, $4^{1/3} = (2^2)^{1/3} = 2^{2/3}$.  Similarly, $4^{2/3} = (2^2)^{2/3} = 2^{4/3}$. The expressions $(-3)^{1/3}$ and $(-3)^{2/3}$ contain negative bases so we proceed with caution and convert them back to radical notation to get:  $(-3)^{1/3} = \sqrt[3]{-3} = -\sqrt[3]{3} = - 3^{1/3}$ and  $(-3)^{2/3} = (\sqrt[3]{-3})^2 = (-\sqrt[3]{3})^2 =(\sqrt[3]{3})^2 = 3^{2/3}$.  Hence:\[ \begin{array}{rclr}

\dfrac{2 \cdot (-3)^{1/3}}{4^{1/3}} + \dfrac{3 \cdot 4^{2/3}}{2 \cdot (-3)^{2/3}} & = & \dfrac{2 \cdot (-3^{1/3})}{2^{2/3}} + \dfrac{3 \cdot 2^{4/3}}{2 \cdot 3^{2/3}}  & \\ [3pt]

& = & \dfrac{2^{1} \cdot (-3^{1/3})}{2^{2/3}} + \dfrac{3^{1} \cdot 2^{4/3}}{2^{1} \cdot 3^{2/3}}  & \\ [3pt]

& = & 2^{1 - 2/3} \cdot (-3^{1/3}) +3^{1- 2/3} \cdot 2^{4/3 - 1}  & \\ [3pt]

& = & 2^{1/3} \cdot (-3^{1/3}) +3^{1/3} \cdot 2^{1/3}  & \\ [3pt]

& = &  - 2^{1/3} \cdot 3^{1/3} +3^{1/3} \cdot 2^{1/3}  & \\ [3pt]

& = & 0 & \\ \end{array} \] \qed

\end{enumerate}

\end{ex}

\newpage

\subsection{Exercises}

In Exercises \ref{arithexfirst} - \ref{arithexlast}, perform the indicated operations and simplify.


\begin{multicols}{4}
\begin{enumerate}

\item $5 - 2 + 3$\vphantom{$\dfrac{3}{8} + \dfrac{5}{12}$} \label{arithexfirst}
\item $5 - (2+3)$\vphantom{$\dfrac{3}{8} + \dfrac{5}{12}$}
\item  $\dfrac{2}{3} - \dfrac{4}{7}$\vphantom{$\dfrac{3}{8} + \dfrac{5}{12}$}
\item  $\dfrac{3}{8} + \dfrac{5}{12}$

\setcounter{HW}{\value{enumi}}
\end{enumerate}
\end{multicols}

\begin{multicols}{4}
\begin{enumerate}
\setcounter{enumi}{\value{HW}}

\item  $\dfrac{5-3}{-2-4}$\vphantom{$\dfrac{2(3)-(4-1)}{2^2 + 1}$}
\item  $\dfrac{2(-3)}{3 - (-3)}$\vphantom{$\dfrac{2(3)-(4-1)}{2^2 + 1}$}
\item  $\dfrac{2(3)-(4-1)}{2^2 + 1}$\vphantom{$\dfrac{2(3)-(4-1)}{2^2 + 1}$}
\item  $\dfrac{4 - 5.8}{2 - 2.1}$\vphantom{$\dfrac{2(3)-(4-1)}{2^2 + 1}$}

\setcounter{HW}{\value{enumi}}
\end{enumerate}
\end{multicols}

\begin{multicols}{4}
\begin{enumerate}
\setcounter{enumi}{\value{HW}}

\item  $\dfrac{1 - 2(-3)}{5(-3) + 7}$\vphantom{$\dfrac{(-2)^2 - (-2) - 6}{(-2)^2 - 4}$}
\item  $\dfrac{5(3) - 7}{2(3)^2-3(3)-9}$\vphantom{$\dfrac{(-2)^2 - (-2) - 6}{(-2)^2 - 4}$}
\item  $\dfrac{2((-1)^2-1)}{((-1)^2+1)^2}$\vphantom{$\dfrac{(-2)^2 - (-2) - 6}{(-2)^2 - 4}$}
\item  $\dfrac{(-2)^2 - (-2) - 6}{(-2)^2 - 4}$


\setcounter{HW}{\value{enumi}}
\end{enumerate}
\end{multicols}



\begin{multicols}{4}
\begin{enumerate}
\setcounter{enumi}{\value{HW}}

\item  $\dfrac{3 - \frac{4}{9}}{-2 - (-3)}$\vphantom{$\dfrac{2\left(\frac{4}{3}\right)}{1 - \left(\frac{4}{3}\right)^2}$}
\item  $\dfrac{\frac{2}{3} - \frac{4}{5}}{4 - \frac{7}{10}}$\vphantom{$\dfrac{2\left(\frac{4}{3}\right)}{1 - \left(\frac{4}{3}\right)^2}$}
\item  $\dfrac{2\left(\frac{4}{3}\right)}{1 - \left(\frac{4}{3}\right)^2}$
\item  $\dfrac{1 - \left(\frac{5}{3}\right)\left(\frac{3}{5}\right)}{1 + \left(\frac{5}{3}\right)\left(\frac{3}{5}\right)}$\vphantom{$\dfrac{2\left(\frac{4}{3}\right)}{1 - \left(\frac{4}{3}\right)^2}$}

\setcounter{HW}{\value{enumi}}
\end{enumerate}
\end{multicols}

\begin{multicols}{4}
\begin{enumerate}
\setcounter{enumi}{\value{HW}}

\item  $\left(\dfrac{2}{3}\right)^{-5}$\vphantom{$\dfrac{3\cdot 5^{100}}{12 \cdot 5^{98}}$}
\item  $3^{-1} - 4^{-2}$\vphantom{$\dfrac{3\cdot 5^{100}}{12 \cdot 5^{98}}$}
\item  $\dfrac{1 + 2^{-3}}{3 - 4^{-1}}$ \vphantom{$\dfrac{3\cdot 5^{100}}{12 \cdot 5^{98}}$}
\item  $\dfrac{3\cdot 5^{100}}{12 \cdot 5^{98}}$

\setcounter{HW}{\value{enumi}}
\end{enumerate}
\end{multicols}

\begin{multicols}{4}
\begin{enumerate}
\setcounter{enumi}{\value{HW}}

\item  $\sqrt{3^2 + 4^2}$  \vphantom{$\left(-\frac{32}{9}\right)^{-3/5}$}
\item  $\sqrt{12} - \sqrt{75}$  \vphantom{$\left(-\frac{32}{9}\right)^{-3/5}$}
\item  $(-8)^{2/3} - 9^{-3/2}$ \vphantom{$\left(-\frac{32}{9}\right)^{-3/5}$}
\item  $\left(-\frac{32}{9}\right)^{-3/5}$

\setcounter{HW}{\value{enumi}}
\end{enumerate}
\end{multicols}


\begin{multicols}{3}
\begin{enumerate}
\setcounter{enumi}{\value{HW}}

\item  $\sqrt{(3-4)^2 + (5-2)^2}$
\item  $\sqrt{(2 - (-1))^2 + \left(\frac{1}{2} - 3\right)^2}$ 
\item  $\sqrt{(\sqrt{5} - 2\sqrt{5})^2 + (\sqrt{18} - \sqrt{8})^2}$

\setcounter{HW}{\value{enumi}}
\end{enumerate}
\end{multicols}

\begin{multicols}{3}
\begin{enumerate}
\setcounter{enumi}{\value{HW}}

\item  $\dfrac{-12 + \sqrt{18}}{21}$\vphantom{$\dfrac{-(-4) + \sqrt{(-4)^2 - 4(1)(-1)}}{2(1)}$}
\item  $\dfrac{-2 - \sqrt{(2)^2 - 4(3)(-1)}}{2(3)}$\vphantom{$\dfrac{-(-4) + \sqrt{(-4)^2 - 4(1)(-1)}}{2(1)}$}  
\item  $\dfrac{-(-4) + \sqrt{(-4)^2 - 4(1)(-1)}}{2(1)}$

\setcounter{HW}{\value{enumi}}
\end{enumerate}
\end{multicols}

\begin{enumerate}
\setcounter{enumi}{\value{HW}}

\item $2(-5)(-5+1)^{-1} + (-5)^2(-1)(-5+1)^{-2}$
\item $3\sqrt{2(4)+1} + 3(4)\left(\frac{1}{2}\right)(2(4)+1)^{-1/2}(2)$
\item $2(-7)\sqrt[3]{1-(-7)} + (-7)^2 \left(\frac{1}{3}\right)(1-(-7))^{-2/3}(-1)$ \label{arithexlast}

\item With the help of your calculator, find $(3.14 \times 10^{87})^{117}$.  Write your final answer, using scientific notation, rounded to two decimal places. (See Example \ref{scientificnotationex}.)

\end{enumerate}


\newpage

\subsection{Answers}

\begin{multicols}{4}
\begin{enumerate}

\item $6$\vphantom{$\dfrac{19}{24}$}
\item $0$\vphantom{$\dfrac{19}{24}$}
\item  $\dfrac{2}{21}$\vphantom{$\dfrac{19}{24}$}
\item  $\dfrac{19}{24}$

\setcounter{HW}{\value{enumi}}
\end{enumerate}
\end{multicols}

\begin{multicols}{4}
\begin{enumerate}
\setcounter{enumi}{\value{HW}}

\item  $-\dfrac{1}{3}$\vphantom{$\dfrac{3}{5}$}
\item  $-1$\vphantom{$\dfrac{3}{5}$}
\item  $\dfrac{3}{5}$
\item  $18$\vphantom{$\dfrac{3}{5}$}

\setcounter{HW}{\value{enumi}}
\end{enumerate}
\end{multicols}

\begin{multicols}{4}
\begin{enumerate}
\setcounter{enumi}{\value{HW}}

\item  $-\dfrac{7}{8}$
\item  Undefined.
\item  $0$
\item  Undefined.

\setcounter{HW}{\value{enumi}}
\end{enumerate}
\end{multicols}



\begin{multicols}{4}
\begin{enumerate}
\setcounter{enumi}{\value{HW}}

\item  $\dfrac{23}{9}$
\item  $-\dfrac{4}{99}$\vphantom{$\dfrac{23}{9}$}
\item  $-\dfrac{24}{7}$\vphantom{$\dfrac{23}{9}$}
\item  $0$\vphantom{$\dfrac{23}{9}$}

\setcounter{HW}{\value{enumi}}
\end{enumerate}
\end{multicols}

\begin{multicols}{4}
\begin{enumerate}
\setcounter{enumi}{\value{HW}}

\item  $\dfrac{243}{32}$
\item  $\dfrac{13}{48}$\vphantom{$\dfrac{243}{32}$}
\item  $\dfrac{9}{22}$\vphantom{$\dfrac{243}{32}$}
\item  $\dfrac{25}{4}$\vphantom{$\dfrac{243}{32}$}

\setcounter{HW}{\value{enumi}}
\end{enumerate}
\end{multicols}

\begin{multicols}{4}
\begin{enumerate}
\setcounter{enumi}{\value{HW}}

\item  $5$\vphantom{$-\dfrac{3\sqrt[5]{3}}{8} = -\dfrac{3^{6/5}}{8}$} 
\item  $-3\sqrt{3}$\vphantom{$-\dfrac{3\sqrt[5]{3}}{8} = -\dfrac{3^{6/5}}{8}$} 
\item  $\dfrac{107}{27}$\vphantom{$-\dfrac{3\sqrt[5]{3}}{8} = -\dfrac{3^{6/5}}{8}$}
\item  $-\dfrac{3\sqrt[5]{3}}{8} = -\dfrac{3^{6/5}}{8}$

\setcounter{HW}{\value{enumi}}
\end{enumerate}
\end{multicols}


\begin{multicols}{3}
\begin{enumerate}
\setcounter{enumi}{\value{HW}}

\item  $\sqrt{10}$\vphantom{$\dfrac{\sqrt{61}}{2}$}
\item  $\dfrac{\sqrt{61}}{2}$ 
\item  $\sqrt{7}$\vphantom{$\dfrac{\sqrt{61}}{2}$}

\setcounter{HW}{\value{enumi}}
\end{enumerate}
\end{multicols}

\begin{multicols}{3}
\begin{enumerate}
\setcounter{enumi}{\value{HW}}

\item  $\dfrac{-4 + \sqrt{2}}{7}$
\item  $-1$\vphantom{$\dfrac{-4 + \sqrt{2}}{7}$}
\item  $2 + \sqrt{5}$\vphantom{$\dfrac{-4 + \sqrt{2}}{7}$}

\setcounter{HW}{\value{enumi}}
\end{enumerate}
\end{multicols}

\begin{multicols}{4}
\begin{enumerate}
\setcounter{enumi}{\value{HW}}

\item $\dfrac{15}{16}$
\item $13$
\item $-\dfrac{385}{12}$

\item $1.38 \times 10^{10237}$
\end{enumerate}
\end{multicols}

\end{document}

\newpage

\section{Linear Equations and Inequalities}

\documentclass[11pt]{article}
\usepackage[margin=1in,letterpaper]{geometry}
\usepackage{amssymb,amsmath,amsthm,fancyhdr,supertabular,longtable,hhline}
\usepackage{colortbl}
\usepackage{import, multicol,boxedminipage}
\usepackage{chapterfolder}
\usepackage[metapost,truebbox]{mfpic}
\usepackage[pdflatex]{graphicx}
\usepackage{makeidx}
\usepackage[colorlinks, hyperindex, plainpages=false, linkcolor=blue, urlcolor=blue, pdfpagelabels]{hyperref}
\usepackage[all]{hypcap}
\definecolor{ResultColor}{gray}{0.9}
\theoremstyle{definition}  % this prevents the text in definitions, theorems, and corollaries from being italicized
\newtheorem{defn}{\bf Definition}[section]
\newtheorem{thm}{\bf Theorem}[section]
\newtheorem{cor}[thm]{\bf Corollary}
\newtheorem{eqn}{\bf Equation}[section]
\newtheorem{ex}{\bf Example}[section]
\newtheorem{fig}{\bf Figure}[section]
\setlength{\parindent}{0in}
\newcommand{\bbm}{\begin{boxedminipage}{6.41in}}
\newcommand{\ebm}{\end{boxedminipage}}
\usepackage{array}
\setlength{\extrarowheight}{2pt}
\allowdisplaybreaks[2]
\usepackage{cancel}
\usepackage{sectsty}
\usepackage{textcomp}
\usepackage{multirow}
\usepackage[sfdefault,lf]{carlito}
	%% The 'lf' option for lining figures
	%% The 'sfdefault' option to make the base font sans serif
	\usepackage[T1]{fontenc}
	\renewcommand*\oldstylenums[1]{\carlitoOsF #1}
\usepackage[nottoc]{tocbibind}
\allsectionsfont{\mdseries \scshape}
\makeatletter
\renewcommand\l@section{\@dottedtocline{1}{1.5em}{3em}}
\renewcommand\l@subsection{\@dottedtocline{2}{4.5em}{3.5em}}
\makeatother
\pagestyle{fancy}
\newcounter{HW}
\newcounter{HWindent}
%\makeindex

\title{Review \#1: Linear Equations and Inequalities}

\begin{document}
\maketitle


\renewcommand{\headrulewidth}{0pt}
\lhead[\fancyplain{}{\sc\thepage}]%
      {\fancyplain{}{\sc \nouppercase{\rightmark}}}
\rhead[\fancyplain{}{\sc \nouppercase{\leftmark}}]%
      {\fancyplain{}{\sc\thepage}}
\cfoot{}


%\label{LinearEqIneq}

This is the first of several handouts concentrating on a review of the Algebra skills needed to solve the sorts of basic equations and inequalities encountered in Math 1010.  In general, equations and inequalities fall into one of three categories:  conditional, identity or contradiction, depending on the nature of their solutions.  A \textbf{conditional} equation or inequality is true for only \textit{certain} real numbers.  For example, $2x+1 = 7$ is true precisely when $x = 3$, and $w - 3 \leq 4$ is true precisely when $w \leq 7$.  An \textbf{identity} is an equation or inequality that is true for \textit{all} real numbers.  For example, $2x -3 = 1+x-4+x$ or $2t \leq 2t + 3$.  A \textbf{contradiction} is an equation or inequality that is \textit{never} true.  Examples here include $3x - 4 = 3x + 7$ and $a - 1 > a + 3$.  

\smallskip

As you may recall, solving an equation or inequality means finding all of the values of the variable, if any exist, which make the given equation or inequality true.  This often requires us to manipulate the given equation or inequality from its given form to an easier form.  For example, if we're asked to solve $3 - 2(x-3) = 7x + 3(x+1)$, we get $x = \frac{1}{2}$, but not without a fair amount of algebraic manipulation. In order to obtain the correct answer(s), however, we need to make sure that whatever manoeuvres we apply are reversible in order to guarantee that we maintain a chain of \textbf{equivalent} equations or inequalities.  Two equations or inequalities are called \textbf{equivalent} if they have the same solutions.  We list these `legal moves' below.

\medskip

\phantomsection \label{equivalenteqnineq}

\colorbox{ResultColor}{\bbm

\centerline{\textbf{Procedures which Generate Equivalent Equations}}

\begin{itemize}

\item  Add (or subtract) the same real number to (from) both sides of the equation.

\item  Multiply (or divide) both sides of the equation by the same \textbf{nonzero} real number.\footnote{Multiplying both sides of an equation by $0$ collapses the equation to $0 = 0$, which doesn't do anybody any good.}

\end{itemize}

\centerline{\textbf{Procedures which Generate Equivalent Inequalities}}

\vspace{-0.1in}

\begin{itemize}

\item  Add (or subtract) the same real number to (from) both sides of the equation.

\item  Multiply (or divide) both sides of the equation by the same \textbf{positive} real number.\footnote{Remember that if you multiply both sides of an inequality by a negative real number, the inequality sign is reversed:  $3 \leq 4$, but $(-2)(3) \geq (-2)(4)$.}

\end{itemize}

\ebm}

\section{Linear Equations} \label{LinearEqn}

The first type of equations we need to review are \textbf{linear} equations as defined below.

\medskip

\colorbox{ResultColor}{\bbm

\begin{defn}\label{lineareqndefn} An equation is said to be \textbf{linear} in a variable $X$ if it can be written in the form $AX = B$ where $A$ and $B$ are expressions which do not involve $X$ and $A \neq 0$.

\end{defn}

\ebm}

One key point about Definition \ref{lineareqndefn} is that the exponent on the unknown `$X$' in the equation is $1$, that is $X = X^1$. Our main strategy for solving linear equations is summarized below.

\medskip

\phantomsection \label{strategyforsolvinglineareqns}

\colorbox{ResultColor}{\bbm

\centerline{\textbf{Strategy for Solving Linear Equations}}

\vspace{0.05in}

In order to solve an equation which is linear in a given variable, say $X$:

\vspace{-0.1in}

\begin{enumerate}

\item  Isolate all of the terms containing $X$ on one side of the equation, putting all of the terms not containing $X$ on the other side of the equation.

\item  Factor out the $X$ and divide both sides of the equation by its coefficient.

\end{enumerate}

\ebm}

\medskip

We illustrate this process with a collection of examples below.

\begin{ex}\label{lineareqnreview}  Solve the following equations for the indicated variable.  Check your answer.

\begin{multicols}{2}

\begin{enumerate}

\item  Solve for $x$: $3x - 6 = 7x + 4$\vphantom{$3 - 1.7t = \dfrac{t}{4}$}

\item  Solve for $a$: $\dfrac{1}{18}(7 - 4a) + 2 = \dfrac{a}{3} - \dfrac{4-a}{12}$

\setcounter{HW}{\value{enumi}}

\end{enumerate}

\end{multicols}

\begin{multicols}{2}

\begin{enumerate}

\setcounter{enumi}{\value{HW}}

\item  Solve for $y$:  $8 y \sqrt{3} + 1 = 7 - \sqrt{12}(5 - y)$\vphantom{$\dfrac{1}{18}(7 - 4a) + 2 = \dfrac{a}{3} - \dfrac{4-a}{12}$.}

\item  Solve for $y$: $x(4-y)=8y$.

\setcounter{HW}{\value{enumi}}

\end{enumerate}

\end{multicols}


{\bf Solution.} 

\begin{enumerate}

\item  The variable we are asked to solve for is $x$ so our first move is to gather all of the terms involving $x$ on one side and put the remaining terms on the other.\footnote{In the margin notes, when we speak of operations, e.g.,`Subtract $7x$,' we mean to subtract $7x$ from \textbf{both} sides of the equation.  The `from both sides of the equation' is omitted in the interest of spacing.}\[ \begin{array}{rclr} 3x - 6 &  = & 7x + 4 & \\
                       (3x-6) - 7x + 6 &  = & (7x+4) -7x +6 &  \text{Subtract $7x$, add $6$} \\
											3x - 7x - 6 + 6 & = & 7x - 7x + 4 + 6 & \text{Rearrange terms} \\
											-4x & = & 10 & \text{$3x-7x = (3-7)x = -4x$} \\ [5pt]
											\dfrac{-4x}{-4} & = & \dfrac{10}{-4} & \text{Divide by the coefficient of $x$} \\ [10pt]
											x & = & -\dfrac{5}{2} & \text{Reduce to lowest terms} \\ \end{array}\]
											
To check our answer, we substitute $x = -\frac{5}{2}$ into each side of the \textbf{original} equation to see the equation is satisfied.  Sure enough, $3\left(-\frac{5}{2}\right) - 6 = -\frac{27}{2}$ and $7\left(-\frac{5}{2}\right) + 4 = -\frac{27}{2}$.

\item  To solve this next equation, we begin once again by clearing fractions.  The least common denominator here is $36$:\[ \begin{array}{rclr}

 \dfrac{1}{18}(7 - 4a) + 2 & = & \dfrac{a}{3} - \dfrac{4-a}{12} & \\[8pt]

36 \left(\dfrac{1}{18}(7 - 4a) + 2\right) & = & 36 \left(\dfrac{a}{3} - \dfrac{4-a}{12}\right) & \text{Multiply by $36$} \\[13pt]

\dfrac{36}{18} (7-4a) + (36)(2) & = & \dfrac{36a}{3} - \dfrac{36(4-a)}{12} & \text{Distribute} \\[5pt]

2(7-4a)  + 72 & = & 12 a - 3(4-a) & \text{Distribute} \\

14  - 8a + 72 & = & 12a - 12 + 3a & \\

86 - 8a & = & 15 a - 12 & \text{$12 a + 3a = (12+3)a = 15a$} \\

(86-8a)+8a+12 & = & (15a-12) + 8a + 12 & \text{Add $8a$ and $12$} \\

86 + 12 - 8a + 8a & = & 15a + 8a - 12 + 12 & \text{Rearrange terms} \\

98 & = & 23 a & \text{$15a + 8a = (15+8)a = 23a$} \\ [5pt]

\dfrac{98}{23} & = & \dfrac{23a}{23} & \text{Divide by the coefficient of $a$} \\[8pt]

\dfrac{98}{23} & = & a &

\end{array} \]

The check, as usual, involves substituting $a = \frac{98}{23}$ into both sides of the original equation.  The reader is encouraged to work through the (admittedly messy) arithmetic.  Both sides work out to $\frac{199}{138}$.


\item  The square roots may dishearten you but we treat them just like the real numbers they are.  Our strategy is the same:  get everything with the variable (in this case $y$) on one side, put everything else on the other and divide by the coefficient of the variable.  We've added a few steps to the narrative that we would ordinarily omit just to help you see that this equation is indeed linear.\[ \begin{array}{rclr}

8 y \sqrt{3} + 1 & = & 7 - \sqrt{12}(5 - y) & \\ [3pt]

8 y \sqrt{3} + 1 & = & 7 - \sqrt{12}(5)  + \sqrt{12} y & \text{Distribute} \\ [3pt]

8 y \sqrt{3} + 1 & = & 7 - (2 \sqrt{3})5 + (2 \sqrt{3})y & \text{$\sqrt{12} = \sqrt{4\cdot 3} = 2 \sqrt{3}$} \\ [3pt]

8 y \sqrt{3} + 1 & = & 7 - 10 \sqrt{3} + 2y \sqrt{3}  &  \\ [3pt]

(8 y \sqrt{3} + 1) - 1 - 2y\sqrt{3} & = & (7 - 10 \sqrt{3} + 2y \sqrt{3}) - 1 - 2y\sqrt{3}  & \text{Subtract $1$ and $2y\sqrt{3}$}  \\ [3pt]

8 y \sqrt{3} - 2y\sqrt{3} + 1 - 1 & = & 7 - 1 - 10 \sqrt{3} + 2y \sqrt{3}  - 2y\sqrt{3}  & \text{Rearrange terms}  \\ [3pt]

(8\sqrt{3}-2\sqrt{3})y & = & 6 - 10 \sqrt{3}  & \\ [3pt]

6 y \sqrt{3} & = & 6 - 10 \sqrt{3}  &  \text{See note below} \\ [3pt]

\dfrac{6 y \sqrt{3}}{6 \sqrt{3}}  & = & \dfrac{6 - 10 \sqrt{3}}{6 \sqrt{3}}  & \text{Divide $6\sqrt{3}$} \\ [12pt]

y & = & \dfrac{2 \cdot \sqrt{3} \cdot \sqrt{3} - 2 \cdot 5 \cdot \sqrt{3}}{2 \cdot 3 \cdot \sqrt{3}} & \\ [12pt]

y & = & \dfrac{\cancel{2} \cancel{\sqrt{3}}(\sqrt{3} - 5)}{\cancel{2} \cdot 3 \cdot \cancel{\sqrt{3}}} &  \text{Factor and cancel} \\ [12pt]

y & = & \dfrac{\sqrt{3} - 5}{3} & \\ \end{array}\]In the list of computations above we marked the row $6 y \sqrt{3} = 6 - 10 \sqrt{3}$ with a note.  That's because we wanted to draw your attention to this line without breaking the flow of the manipulations.  The equation $6 y \sqrt{3} = 6 - 10 \sqrt{3}$ is in fact linear according to Definition \ref{lineareqndefn}: the variable is $y$, the value of $A$ is $6\sqrt{3}$ and $B = 6 - 10 \sqrt{3}$. Checking the solution, while not trivial, is good mental exercise.  Each side works out to be $\frac{27 - 40 \sqrt{3}}{3}$.



\item  If we were instructed to solve our last equation for $x$, we'd be done in one step: divide both sides by $(4-y)$ - assuming $4-y \neq 0$, that is.  Alas, we are instructed to solve for $y$, which means we have some more work to do.\[ \begin{array}{rclr}

x(4-y) & = & 8y & \\

4x - xy & = & 8y & \text{Distribute} \\

(4x - xy) + xy & = & 8y + xy & \text{Add $xy$} \\

4x & = & (8+x)y & \text{Factor} \\ \end{array}\]In order to finish the problem, we need to divide both sides of the equation by the coefficient of $y$ which in this case is $8+x$.  Since this expression contains a variable, we need to stipulate that we may perform this division only if $8 + x \neq 0$, or, in other words, $x \neq -8$.  Hence, we write our solution as:\[ y = \dfrac{4x}{8+x}, \quad \text{provided $x \neq -8$}\] What happens if $x = -8$?  Substituting $x = -8$ into the original equation gives $(-8)(4-y) = 8y$ or $-32 + 8y = 8y$.  This reduces to $-32 = 0$, which is a contradiction.  This means there is no solution when $x = -8$, so we've covered all the bases.  Checking our answer requires some Algebra we haven't reviewed yet in this text, but the necessary skills \emph{should} be lurking somewhere in the mathematical mists of your mind.  The adventurous reader is invited to show that both sides work out to $\frac{32x}{x+8}$. \qed

\end{enumerate}

\end{ex}

\section{Linear Inequalities}
\label{LinearInequal}

We now turn our attention to linear inequalities.  Unlike linear equations which admit at most one solution, the solutions to linear inequalities are generally intervals of real numbers.  While the solution strategy for solving linear inequalities is the same as with solving linear equations, we need to remind ourselves that, should we decide to multiply or divide both sides of an inequality by a \textbf{negative} number, we need to reverse the direction of the inequality. (See page \pageref{equivalenteqnineq}.)  In the example below, we work not only some `simple' linear inequalities in the sense there is only one inequality present, but also some `compound' linear inequalities which require us to use the notions of intersection and union. 

\begin{ex}\label{linearineqreview}  Solve the following inequalities for the indicated variable. 

\begin{multicols}{2}

\begin{enumerate}

\item  Solve for $x$: $\dfrac{7-8x}{2} \geq 4x + 1$

\item  Solve for $y$: $\dfrac{3}{4} \leq \dfrac{7-y}{2} < 6$\vphantom{$\dfrac{7-8x}{2} \geq 4x + 1$}

\setcounter{HW}{\value{enumi}}

\end{enumerate}

\end{multicols}

\begin{multicols}{2}

\begin{enumerate}

\setcounter{enumi}{\value{HW}}

\item  Solve for $t$:  $2t-1 \leq 4-t < 6t+1$

\item  Solve for $w$: $2.1 - 0.01w \leq -3$ or $2.1-0.01w \geq 3$


\setcounter{HW}{\value{enumi}}

\end{enumerate}

\end{multicols}



{\bf Solution.}

\begin{enumerate}

\item  We begin by clearing denominators and gathering all of the terms containing $x$ to one side of the inequality and putting the remaining terms on the other.\[ \begin{array}{rclr}

\dfrac{7-8x}{2} & \geq & 4x + 1 & \\ [8pt]

2\left(\dfrac{7-8x}{2}\right) & \geq & 2(4x + 1) & \text{Multiply by $2$} \\ [10pt]

\dfrac{\cancel{2}(7-8x)}{\cancel{2}} & \geq & 2(4x) + 2(1) & \text{Distribute} \\ [3pt]

7 - 8x & \geq & 8x + 2 & \\

(7-8x) + 8x-2 & \geq & 8x+2 + 8x -2 & \text{Add $8x$, subtract $2$} \\

7 - 2 - 8x + 8x & \geq & 8x + 8x + 2 - 2 & \text{Rearrange terms} \\

5 & \geq &  16x & \text{$8x + 8x = (8+8)x = 16x$} \\ [3pt]

\dfrac{5}{16} & \geq & \dfrac{16x}{16} & \text{Divide by the coefficient of $x$} \\[8pt]

\dfrac{5}{16} & \geq & x & \\ 

\end{array} \]

We get $\frac{5}{16} \geq x$ or, said differently,  $x \leq \frac{5}{16}$.  We express this set\footnote{Using set-builder notation, our `set' of solutions here is $\{ x \, | \, x \leq \frac{5}{16} \}$.} of real numbers as  $\left(-\infty, \frac{5}{16}\right]$. Though not required to do so, we could partially check our answer by substituting $x = \frac{5}{16}$ and a few other values in our solution set ($x =0$, for instance) to make sure the inequality holds.  (It also isn't a bad idea to choose an $x > \frac{5}{16}$, say $x = 1$, to see that the inequality \textit{doesn't} hold there.)  The only real way to actually show that our answer works for \textit{all} values in our solution set is to start with $x \leq \frac{5}{16}$ and reverse all of the steps in our solution procedure to prove it is equivalent to our original inequality.  

\item  We have our first example of a `compound' inequality.  The solutions to  \[ \dfrac{3}{4} \leq \dfrac{7-y}{2} < 6 \] must satisfy \[ \dfrac{3}{4} \leq \dfrac{7-y}{2} \qquad \text{\underline{and}} \qquad \dfrac{7-y}{2} < 6\]

One approach is to solve each of these inequalities separately, then intersect their solution sets.  While this method works (and will be used later for more complicated problems), since our variable $y$ appears only in the middle expression, we can proceed by essentially working both inequalities at once:\[ \begin{array}{rclr}

\dfrac{3}{4} \leq & \dfrac{7-y}{2} & < 6 & \\ [10pt]

4\left(\dfrac{3}{4} \right) \leq & 4\left( \dfrac{7-y}{2}\right) & < 4(6) & \text{Multiply by $4$} \\ [12pt]

\dfrac{\cancel{4} \cdot 3}{\cancel{4}} \leq & \dfrac{\cancelto{2}{4}(7-y)}{\cancel{2}} &  < 24 & \\ [5pt]

3 \leq & 2(7-y) & < 24 & \\

3 \leq & 2(7)-2y & < 24 & \text{Distrbute}\\

3 \leq & 14-2y & < 24 & \\

3 -14 \leq & (14-2y) - 14 & < 24 - 14 & \text{Subtract $14$}\\

-11 \leq & -2y & < 10 & \\ [3pt]

\dfrac{-11}{-2} \geq & \dfrac{-2y}{-2} & > \dfrac{10}{-2} & \text{Divide by the coefficient of $y$} \\ [-5pt]
                     &                 &                  & \text{Reverse inequalities} \\ [-3pt]

\dfrac{11}{2}  \geq & y & > -5 & \\

\end{array} \]

Our final answer is $\frac{11}{2} \geq y > -5$, or, said differently,  $-5 < y \leq \frac{11}{2}$. In interval notation, this is $\left( -5, \frac{11}{2} \right]$.  We could check the reasonableness of our answer as before, and the reader is encouraged to do so.  

\item  We have another compound inequality and what distinguishes this one from our previous example is that `$t$' appears on both sides of both inequalities.  In this case, we need to create two separate inequalities and find all of the real numbers $t$ which satisfy both  $2t-1 \leq 4-t$ \textit{and} $4-t < 6t + 1$.  The first inequality, $2t-1 \leq 4-t$, reduces to $3t \leq 5$ or $t \leq \frac{5}{3}$.  The second inequality, $4-t < 6t+1$, becomes $3 < 7t$  which reduces to $t > \frac{3}{7}$.  Thus our solution is all real numbers $t$ with $t \leq \frac{5}{3}$ \textit{and}  $t > \frac{3}{7}$, or, writing this as a compound inequality,  $\frac{3}{7} < t \leq \frac{5}{3}$. Using interval notation,\footnote{If we intersect the solution sets of the two individual inequalities, we get the answer, too:  $\left(-\infty, \frac{5}{3}\right] \cap \left(\frac{3}{7}, \infty\right) = \left( \frac{3}{7}, \frac{5}{3} \right]$.} we express our solution as $\left( \frac{3}{7}, \frac{5}{3} \right]$.


\item  Our last example is yet another compound inequality but here, instead of the two inequalities being connected with the conjunction `\textit{and}', they are connected with `\textit{or}', which indicates that we need to find the \textit{union} of the results of each.  Starting with $2.1 - 0.01w \leq -3$, we get $-0.01 w \leq -5.1$, which gives\footnote{Don't forget to flip the inequality!} $w \geq 510$.  The second inequality, $2.1-0.01w \geq 3$, becomes $-0.01w \geq 0.9$, which reduces to  $w \leq -90$.  Our solution set consists of all real numbers $w$ with $w \geq 510$ \textit{or} $w \leq -90$.  In interval notation, this is $(-\infty, -90] \cup [510, \infty)$. \qed

\end{enumerate}

\end{ex}

\newpage

\section{Exercises}

In Exercises \ref{lineareqfirst0} - \ref{lineareqlast0}, solve the given linear equation and check your answer.  


\begin{multicols}{3}
\begin{enumerate}

\item $3x - 4 = 2 - 4(x-3)$\vphantom{$\dfrac{2(w-3)}{5} = \dfrac{4}{15} - \dfrac{3w+1}{9}$}\label{lineareqfirst0} 
\item $\dfrac{3 - 2t}{4} = 7t+1$\vphantom{$\dfrac{2(w-3)}{5} = \dfrac{4}{15} - \dfrac{3w+1}{9}$}

\item  $\dfrac{2(w-3)}{5} = \dfrac{4}{15} - \dfrac{3w+1}{9}$ 

\setcounter{HW}{\value{enumi}}
\end{enumerate}
\end{multicols}

\begin{multicols}{3}
\begin{enumerate}
\setcounter{enumi}{\value{HW}}

\item  $\sqrt{50} y = \dfrac{6 - \sqrt{8} y}{3}$ \vphantom{$4 - (2x+1) = \dfrac{x \sqrt{7}}{9}$} 
\item  $\dfrac{49w - 14}{7}= 3w - (2-4w)$ 
\item  $7 - (4-x) = \dfrac{2x-3}{2}$ \vphantom{$\dfrac{49w - 14}{7}= 3w - (2-4w)$} \label{lineareqlast0} 

\setcounter{HW}{\value{enumi}}
\end{enumerate}
\end{multicols}







In equations \ref{literalexfirst0} - \ref{literalexlast0}, solve each equation for the indicated variable.

\begin{multicols}{2}
\begin{enumerate}
\setcounter{enumi}{\value{HW}}
\item  Solve for $y$:  $3x+2y = 4$  \label{literalexfirst0}
\item  Solve for $C$: $F = \dfrac{9}{5} C + 32$
\setcounter{HW}{\value{enumi}}
\end{enumerate}
\end{multicols}

\begin{multicols}{2}
\begin{enumerate}
\setcounter{enumi}{\value{HW}}
\item  Solve for $y$:  $x= 4(y+1) + 3$ 
\item  Solve for $y$:  $x(y-3) = 2y+1$
\setcounter{HW}{\value{enumi}}
\end{enumerate}
\end{multicols}

\begin{multicols}{2}
\begin{enumerate}
\setcounter{enumi}{\value{HW}}
\item  Solve for $v$:   $vw - 1 = 3v$
\item  Solve for $w$:  $vw - 1 = 3v$ \label{literalexlast0}
\setcounter{HW}{\value{enumi}}
\end{enumerate}
\end{multicols}


In Exercises \ref{subex1} - \ref{subex2}, the subscripts on the variables have no intrinsic mathematical meaning; they're just used to distinguish one variable from another.  In other words, treat `$P_{\text{\tiny $1$}}$' and `$P_{\text{\tiny $2$}}$'  as two different variables as you would `$x$' and `$y$.'  (The same goes for `$x$' and `$x_{\text{\tiny $0$}}$,'  etc.)

\begin{multicols}{2}
\begin{enumerate}
\setcounter{enumi}{\value{HW}}
\item Solve for $V_{\text{\tiny $2$}}$:  $P_{\text{\tiny $1$}}V_{\text{\tiny $1$}} = P_{\text{\tiny $2$}}V_{\text{\tiny $2$}}$  \label{subex1}
\item Solve for $t$:  $x = x_{\text{\tiny $0$}} + at$ 
\setcounter{HW}{\value{enumi}}
\end{enumerate}
\end{multicols}


\begin{multicols}{2}
\begin{enumerate}
\setcounter{enumi}{\value{HW}}
\item Solve for $x$:   $y-y_{\text{\tiny $0$}} = m(x -x_{\text{\tiny $0$}})$
\item Solve for $T_{\text{\tiny $1$}}$:  $q = mc(T_{\text{\tiny $2$}} -T_{\text{\tiny $1$}})$   \label{subex2}
\setcounter{HW}{\value{enumi}}
\end{enumerate}
\end{multicols}

\begin{enumerate}
\setcounter{enumi}{\value{HW}}

\item With the help of your classmates, find values for $c$ so that the equation:  $2x - 5c = 1 - c(x+2)$

\begin{enumerate}

\item  has $x = 42$ as a solution.
\item  has no solution (that is, the equation is a contradiction.)

\end{enumerate}
Is it possible to find a value of $c$ so the equation is an identity?  Explain.

\setcounter{HW}{\value{enumi}}
\end{enumerate}


In Exercises \ref{linineqnexfirst0} - \ref{linineqnexlast0}, solve the given inequality.  Write your answer using interval notation.

\begin{multicols}{3}
\begin{enumerate}
\setcounter{enumi}{\value{HW}}
\item $3 - 4x \geq 0$\vphantom{$\dfrac{7 -y}{4} \geq 3y + 1$}\label{linineqnexfirst0}
\item  $\dfrac{7 -y}{4} \geq 3y + 1$ 
\item $7 - (2-x) \leq x+3$\vphantom{$\dfrac{10m+1}{5} \geq 2m - \dfrac{1}{2}$}

\setcounter{HW}{\value{enumi}}
\end{enumerate}
\end{multicols}

\begin{multicols}{3}
\begin{enumerate}
\setcounter{enumi}{\value{HW}}

\item $x \sqrt{12} - \sqrt{3} > \sqrt{3} x + \sqrt{27}$

\item $-\dfrac{1}{2} \leq 5x - 3 \leq \dfrac{1}{2}$\vphantom{$-\dfrac{3}{2} \leq \dfrac{4 - 2t}{10} < \dfrac{7}{6}$}

\item $-\dfrac{3}{2} \leq \dfrac{4 - 2t}{10} < \dfrac{7}{6}$



\setcounter{HW}{\value{enumi}}
\end{enumerate}
\end{multicols}


\begin{multicols}{3}
\begin{enumerate}
\setcounter{enumi}{\value{HW}}

\item  $3x \geq 4-x \geq 3$

\item   $4-x \leq 0$ \text{or} $2x+7 < x$

\item   $\dfrac{5-2x}{3} > x$ \text{or} $2x + 5 \geq 1$ \label{linineqnexlast0}


\setcounter{HW}{\value{enumi}}
\end{enumerate}
\end{multicols}


\end{document}

\newpage

\section{Absolute Value Equations and Inequalities}

\documentclass[11pt]{article}
\usepackage[margin=1in,letterpaper]{geometry}
\usepackage{amssymb,amsmath,amsthm,fancyhdr,supertabular,longtable,hhline}
\usepackage{colortbl}
\usepackage{import, multicol,boxedminipage}
\usepackage{graphicx}
\usepackage[colorlinks, hyperindex, plainpages=false, linkcolor=blue, urlcolor=blue, pdfpagelabels]{hyperref}
\usepackage[all]{hypcap}
\definecolor{ResultColor}{gray}{0.9}
\theoremstyle{definition}  % this prevents the text in definitions, theorems, and corollaries from being italicized
\newtheorem{defn}{\bf Definition}
\newtheorem{thm}{\bf Theorem}
\newtheorem{cor}[thm]{\bf Corollary}
\newtheorem{eqn}{\bf Equation}
\newtheorem{ex}{\bf Example}
\newtheorem{fig}{\bf Figure}
\setlength{\parindent}{0in}
\newcommand{\bbm}{\begin{boxedminipage}{6.41in}}
\newcommand{\ebm}{\end{boxedminipage}}
\usepackage{array}
\setlength{\extrarowheight}{2pt}
\allowdisplaybreaks[2]
\usepackage{cancel}
\usepackage{sectsty}
\usepackage{textcomp}
\usepackage{multirow}
\usepackage[sfdefault,lf]{carlito}
	%% The 'lf' option for lining figures
	%% The 'sfdefault' option to make the base font sans serif
	\usepackage[T1]{fontenc}
	\renewcommand*\oldstylenums[1]{\carlitoOsF #1}
\usepackage[nottoc]{tocbibind}
\allsectionsfont{\mdseries \scshape}
\makeatletter
\renewcommand\l@section{\@dottedtocline{1}{1.5em}{3em}}
\renewcommand\l@subsection{\@dottedtocline{2}{4.5em}{3.5em}}
\makeatother
\pagestyle{fancy}
\newcounter{HW}
\newcounter{HWindent}
%\makeindex

\title{Review \#2: Absolute Value Equations and Inequalities}
\author{Carl Stitz and Jeff Zeager\\
Edited by Sean Fitzpatrick}
\begin{document}
\maketitle


\renewcommand{\headrulewidth}{0pt}
\renewcommand{\headheight}{14pt}
\lhead[\fancyplain{}{\sc\thepage}]%
      {\fancyplain{}{\sc \nouppercase{\rightmark}}}
\rhead[\fancyplain{}{\sc \nouppercase{\leftmark}}]%
      {\fancyplain{}{\sc\thepage}}
\cfoot{}


In this section, we review some basic concepts involving the absolute value of a real number $x$.  There are a few different ways to define absolute value and in this section we choose the following definition.  (Absolute value will be revisited in much greater depth in the Math 1010 textbook, where we present what one can think of as the ``precise'' definition.)

\medskip

\colorbox{ResultColor}{\bbm

\begin{defn}\label{absvaldistdefn}{\bf Absolute Value as Distance:}  For every real number $x$, the \textbf{absolute value} of $x$, denoted $|x|$, is the distance between $x$ and $0$ on the number line.  More generally, if $x$ and $c$ are real numbers, $|x-c|$ is the distance between the numbers $x$ and $c$ on the number line.

\end{defn}

\ebm}

\medskip

For example, $|5| = 5$ and $|-5| = 5$, since each is $5$ units from $0$ on the number line:

\begin{center}

\includegraphics{AbsValEqIneq-1}

Graphically why $|-5| = 5$ and $|5| = 5$

\end{center}

Computationally, the absolute value `makes negative numbers positive',  though we need to be a little cautious with this description. While $|-7| = 7$, $|5-7| \neq 5+7$.  The absolute value acts as a grouping symbol, so $|5-7| = |-2| = 2$, which makes sense since $5$ and $7$ are two units away from each other on the number line:

\begin{center}

\includegraphics{AbsValEqIneq-2}

Graphical illustration of why $|5-7| = 2$

\end{center}

We list some of the operational properties of absolute value below.

\medskip

\colorbox{ResultColor}{\bbm
\begin{thm}  \textbf{Properties of Absolute Value:} Let $a$, $b$ and $x$ be real numbers and let $n$ be an integer. Then \label{absolutevalueprops} 

\begin{itemize}

\item {\bf Product Rule:} $|ab|= |a||b|$ 

\item {\bf Power Rule:} $\left| a^{n} \right| = |a|^{n}$ whenever $a^{n}$ is defined 

\item {\bf Quotient Rule:} $\left| \dfrac{a}{b} \right| = \dfrac{|a|}{|b|}$, provided $b \neq 0$ 

\end{itemize}

\end{thm}

\ebm}

\medskip

The proof of Theorem \ref{absolutevalueprops} is difficult, but not impossible, using the distance definition of absolute value or even the `it makes negatives positive' notion.  It is, however, much easier if one uses the ``precise'' definition given in the textbook, so we will omit the proof for now. For now, let's focus on how to solve basic equations and inequalities involving the absolute value.

\section{Absolute Value Equations}
\label{basicabsvaleqns}

Thinking of absolute value in terms of distance gives us a geometric way to interpret equations.  For example, to solve $|x| = 3$, we are looking for all real numbers $x$ whose distance from $0$ is $3$ units.  If we move three units to the right of $0$, we end up at $x = 3$.  If we move three units to the left, we end up at $x = -3$.  Thus the solutions to  $|x| = 3$ are $x = \pm 3$.  


\begin{center}

\includegraphics{AbsValEqIneq-3}

The solutions to  $|x| = 3$ are $x = \pm 3$.  

\end{center}



Thinking this way gives us the following.

\medskip

\colorbox{ResultColor}{\bbm

\begin{thm} \textbf{Absolute Value Equations: }\label{absvalequality}  Suppose $x$, $y$ and $c$ are real numbers.

\begin{itemize}

\item  $|x| = 0$ if and only if $x = 0$.

\item  For $c > 0$, $|x| = c$ if and only if $x = c$ or $x = -c$.

\item  For $c < 0$, $|x| = c$ has no solution.

\item  $|x| = |y|$ if and only if $x = y$ or $x = -y$. 

(That is,  if two numbers have the same absolute values, they are either the same number or exact opposites.) 

\end{itemize}

\end{thm}

\ebm}

\medskip

Theorem \ref{absvalequality} is our main tool in solving equations involving the absolute value, since it allows us a way to rewrite such equations as compound linear equations.

\medskip

\phantomsection
\label{strategyforsolvingabseqns}

\colorbox{ResultColor}{\bbm

\centerline{\textbf{Strategy for Solving Equations Involving Absolute Value}}

\vspace{0.05in}

In order to solve an equation involving the absolute value of a quantity $|X|$:

\begin{enumerate}

\item  Isolate the absolute value on one side of the equation so it has the form $|X| = c$.

\item  Apply Theorem \ref{absvalequality}.

\end{enumerate}

\ebm}

\medskip

The techniques we use to `isolate the absolute value' are precisely those we used in the handout on solving linear equations and inequalities (Review \#1) to isolate the variable when solving linear equations.  Time for some practice.

\begin{ex} \label{absvalueeqnex}  Solve each of the following equations.

\begin{multicols}{3}
\begin{enumerate}

\item  $|3x-1| = 6$\vphantom{ $\dfrac{3 - |y+5|}{2} = 1$}
\item  $\dfrac{3 - |y+5|}{2} = 1$
\item  $3|2t+1| - \sqrt{5} = 0$\vphantom{ $\dfrac{3 - |y+5|}{2} = 1$}

\setcounter{HW}{\value{enumi}}
\end{enumerate}
\end{multicols}

\begin{multicols}{3}
\begin{enumerate}
\setcounter{enumi}{\value{HW}}

\item  $4 - |5w+3| = 5$\vphantom{$\left|3 - x \sqrt[3]{12}\right| = |4x+1|$}

\item  $\left|3 - x \sqrt[3]{12}\right| = |4x+1|$\vphantom{$\left|3 - x \sqrt[3]{12}\right| = |4x+1|$}

\item  $|t-1| - 3|t+1| = 0$\vphantom{$\left|3 - x \sqrt[3]{12}\right| = |4x+1|$}


\end{enumerate}
\end{multicols}

{\bf Solution.} 

\begin{enumerate}

\item  The equation  $|3x-1| = 6$ is of already in the form $|X| = c$, so we know  $3x-1=6$ or $3x-1 = -6$.  Solving the former gives us at $x = \frac{7}{3}$ and solving the latter yields $x = -\frac{5}{3}$.  We may check both of these solutions by substituting them into the original equation and showing that the arithmetic works out.

\item  We begin solving  $\frac{3 - |y+5|}{2} = 1$ by isolating the absolute value to put it in the form $|X| = c$.\[ \begin{array}{rclr}
\dfrac{3 - |y+5|}{2} & = & 1 &  \\
3 - |y+5| & = & 2 & \text{Multiply by $2$}\\
-|y+5| & = & -1 & \text{Subtract $3$} \\
|y+5| & = & 1 & \text{Divide by $-1$}  \\ 

\end{array} \] At this point, we have $y+5 = 1$ or $y+5 = -1$, so our solutions are $y = -4$ or $y = -6$.  We leave it to the reader to check both answers in the original equation.

\item As in the previous example, we first isolate the absolute value.  Don't let the $\sqrt{5}$ throw you off - it's just another real number, so we treat it as such:\[ \begin{array}{rclr}

 3|2t+1| - \sqrt{5} & = & 0 & \\
 3|2t+1|  & = &  \sqrt{5} & \text{Add $\sqrt{5}$} \\
 |2t + 1| & = & \dfrac{\sqrt{5}}{3} & \text{Divide by $3$}\\
\end{array} \] From here, we have that $2t+1 = \frac{\sqrt{5}}{3}$ or $2t+1 = -\frac{\sqrt{5}}{3}$. The first equation gives $t = \frac{\sqrt{5}-3}{6}$ while the second gives $t = \frac{-\sqrt{5}-3}{6}$ thus we list our answers as $t = \frac{-3 \pm \sqrt{5}}{6}$.   The reader should enjoy the challenge of substituting both answers into the original equation and following through the arithmetic to see that both answers work.

\item  Upon isolating the absolute value in the equation $4 - |5w+3| = 5$, we get $|5w+3| = -1$.  At this point, we know there cannot be any real solution.  By definition, the absolute value is a \textit{distance}, and as such is never negative.  We write `no solution' and carry on.

\item Our next equation already has the absolute value expressions (plural) isolated, so we work from the principle that if $|x| = |y|$, then $x = y$ or $x = -y$. Thus from $\left|3 - x \sqrt[3]{12}\right| = |4x+1|$ we get two equations to solve:  \[ 3 - x \sqrt[3]{12} = 4x+1, \qquad \text{and} \qquad 3 - x \sqrt[3]{12} = -(4x+1) \] Notice that the right side of the second equation is $-(4x+1)$ and not simply $-4x+1$.  Remember, the expression $4x+1$ represents a single real number so in order to negate it we need to negate the \textit{entire} expression $-(4x+1)$. Moving along, when solving $3 - x \sqrt[3]{12} = 4x+1$, we obtain $x = \frac{2}{4 + \sqrt[3]{12}}$ and the solution to $3 - x \sqrt[3]{12} = -(4x+1)$ is $x = \frac{4}{\sqrt[3]{12}-4}$.  As usual, the reader is invited to check these answers by substituting them into the original equation.

\item We start by isolating one of the absolute value expressions:  $|t-1| - 3|t+1| = 0$ gives $|t-1| = 3|t+1|$.  While this \textit{resembles} the form $|x| = |y|$, the coefficient $3$ in $3|t+1|$ prevents it from being an exact match.  Not to worry - since $3$ is positive, $3 = |3|$ so \[3|t+1| = |3| |t+1| = |3(t+1)| = |3t+3|.\]  Hence, our equation becomes $|t-1| = |3t+3|$ which results in the two equations:  $t-1 = 3t+3$ and $t-1 = -(3t+3)$.  The first equation gives $t = -2$ and the second gives $t = -\frac{1}{2}$.  The reader is encouraged to check both answers in the original equation. \qed

\end{enumerate}

\end{ex}


\section{Absolute Value Inequalities}
\label{basicabsvalineq}

We now turn our attention to solving some basic inequalities involving the absolute value.  Suppose we wished to solve $|x| < 3$.  Geometrically, we are looking for all of the real numbers whose distance from $0$ is \textit{less} than $3$ units.  We get $-3 < x < 3$, or in interval notation, $(-3,3)$.  Suppose we are asked to solve $|x| > 3$ instead.  Now we want the distance between $x$ and $0$ to be \textit{greater} than $3$ units.  Moving in the positive direction, this means $x > 3$.  In the negative direction, this puts $x < -3$.  Our solutions would then satisfy $x < -3$ \textit{or} $x > 3$.  In interval notation, we express this as $(-\infty, -3) \cup (3, \infty)$.  


\begin{center}

\begin{tabular}{cc}

\includegraphics{AbsValEqIneq-4}

& 

\includegraphics{AbsValEqIneq-5}

\\




The solution to  $|x| < 3$ is $(-3,3)$ &   The solution to  $|x| > 3$ is $(-\infty, -3) \cup (3, \infty)$ \\

\end{tabular}

\end{center}

Generalizing this notion, we get the following:

\medskip

\colorbox{ResultColor}{\bbm
\begin{thm}  \label{absolutevalueineq} \textbf{Inequalities Involving Absolute Value:}  Let $c$ be a real number.  

\begin{itemize}

\item   If $c> 0$, $|x| < c$ is equivalent to $-c<x<c$.

\item  If $c \leq 0$, $|x| < c$ has no solution.

\item  If $c > 0$, $|x| > c$ is equivalent to $x < -c$ or $x > c$.

\item If $c \leq 0$, $|x| > c$ is true for all real numbers.

\end{itemize}

\end{thm}
\ebm}

\medskip

If the inequality we're faced with involves `$\leq$' or `$\geq$,' we can combine the results of Theorem \ref{absolutevalueineq}  with Theorem \ref{absvalequality} as needed. 

\medskip

\phantomsection
\label{strategyforsolvingabsineq}

\colorbox{ResultColor}{\bbm

\centerline{\textbf{Strategy for Solving Inequalities Involving  Absolute Value}}

\vspace{0.05in}

In order to solve an inequality involving the absolute value of a quantity $|X|$:

\begin{enumerate}

\item  Isolate the absolute value on one side of the inequality.

\item  Apply Theorem \ref{absolutevalueineq}.

\end{enumerate}

\ebm}

\medskip

\begin{ex}  Solve the following inequalities.

\begin{multicols}{2}
\begin{enumerate}

\item  $\left|x-\sqrt[4]{5} \right| > 1$\vphantom{$\dfrac{4 - 2|2x+1|}{4} > -\sqrt{3}$}

\item  $\dfrac{4 - 2|2x+1|}{4} \geq -\sqrt{3}$

\setcounter{HW}{\value{enumi}}
\end{enumerate}
\end{multicols}

\begin{multicols}{2}
\begin{enumerate}
\setcounter{enumi}{\value{HW}}

\item  $|2x - 1| \leq 3|4 - 8x| - 10$

\item  $|2x - 1| \leq 3|4 - 8x| + 10$

\setcounter{HW}{\value{enumi}}
\end{enumerate}
\end{multicols}

\begin{multicols}{2}
\begin{enumerate}
\setcounter{enumi}{\value{HW}}



\item  $2 < |x-1| \leq 5$\vphantom{$\left|\sqrt{10} x - 5 \right| + \left|\sqrt{10} - 5x \right| \leq 0$
}

\item  $|10 x - 5| +|10 - 5x| \leq 0$


\setcounter{HW}{\value{enumi}}
\end{enumerate}
\end{multicols}

{\bf Solution.}  

\begin{enumerate}

\item  From Theorem \ref{absolutevalueineq}, $\left|x-\sqrt[4]{5} \right| > 1$ is equivalent to $x-\sqrt[4]{5} < -1$ or $x-\sqrt[4]{5} > 1$. Solving this compound inequality, we get $x < -1 + \sqrt[4]{5}$ or $x > 1 + \sqrt[4]{5}$.  Our answer, in interval notation, is:  $\left(-\infty,-1 + \sqrt[4]{5} \right) \cup \left(1 + \sqrt[4]{5},\infty \right)$.  As with linear inequalities, we can partially check our answer by selecting values of $x$ both inside and outside the solution intervals to see which values of $x$ satisfy the original inequality and which do not.

\item  Our first step in solving $\frac{4 - 2|2x+1|}{4} \geq -\sqrt{3}$ is to isolate the absolute value. \[ \begin{array}{rclr}
\dfrac{4 - 2|2x+1|}{4} & \geq & -\sqrt{3} & \\ [5pt]

4 - 2|2x+1| & \geq & -4\sqrt{3} & \text{Multiply by $4$} \\
- 2|2x+1| & \geq & -4-4\sqrt{3} & \text{Subtract $4$} \\
|2x+1| & \leq & \dfrac{-4-4\sqrt{3}}{-2} & \text{Divide by $-2$,  reverse the inequality} \\ [8pt]
|2x+1| & \leq & 2 + 2\sqrt{3} & \text{Reduce} \\ 

\end{array}\] Since we're dealing with `$\leq$' instead of just `$<$,' we can combine Theorems \ref{absolutevalueineq} and  \ref{absvalequality} to rewrite this last inequality as:\footnote{Note the use of parentheses: $-(2+2\sqrt{3})$ as opposed to $-2 + 2\sqrt{3}$.}   $-(2 + 2\sqrt{3}) \leq 2x+1 \leq 2+2\sqrt{3}$. Subtracting the `$1$' across both inequalities gives $-3-2\sqrt{3} \leq 2x \leq 1 + 2\sqrt{3}$, which reduces to $\frac{-3-2\sqrt{3}}{2} \leq x \leq \frac{1+2\sqrt{3}}{2}$.  In interval notation this reads as  $\left[\frac{-3-2\sqrt{3}}{2}, \frac{1+2\sqrt{3}}{2}\right]$.

\item  There are two absolute values in $|2x - 1| \leq 3|4 - 8x| - 10$, so it is unclear how we are to proceed.  However, before jumping in and trying to apply (or misapply) Theorem \ref{absolutevalueineq}, we note that $|4 - 8x| = |(-4)(2x-1)|$.   Using this, we get:\[ \begin{array}{rclr}

|2x - 1| & \leq & 3|4 - 8x| - 10 & \\

|2x - 1| & \leq & 3|(-4)(2x-1)| - 10 & \text{Factor}\\

|2x - 1| & \leq & 3|-4||2x-1| - 10 & \text{Product Rule}\\

|2x - 1| & \leq & 12|2x-1| - 10 & \\

-11|2x - 1| & \leq & - 10 & \text{Subtract $12|2x-1|$} \\

|2x - 1| & \geq  & \dfrac{10}{11} & \text{Divide by $-11$ and reduce} \\

\end{array}\] At this point, we invoke Theorems \ref{absvalequality} and \ref{absolutevalueineq} and write the equivalent compound inequality:  $2x - 1 \leq -\frac{10}{11}$ \textit{or} $2x-1 \geq \frac{10}{11}$.  We get $x \leq \frac{1}{22}$ \textit{or} $x \geq \frac{21}{22}$, which, in interval notation reads $\left(-\infty, \frac{1}{22}\right] \cup \left[\frac{21}{22}, \infty\right)$.

\item  The inequality  $|2x - 1| \leq 3|4 - 8x| + 10$ differs from the previous example in exactly one respect: on the right side of the inequality, we have `$+10$' instead of `$-10$.' The steps to isolate the absolute value here are identical to those in the previous example, but instead of obtaining $|2x - 1| \geq  \frac{10}{11}$ as before, we obtain $|2x - 1| \geq  -\frac{10}{11}$.  This latter inequality is \textit{always} true. (Absolute value is, by definition, a distance and hence always $0$ or greater.)  Thus our solution to this inequality is all real numbers, $(-\infty, \infty)$.


\item  To solve  $2 < |x-1| \leq 5$, we rewrite it as the compound inequality: $2 < |x-1|$ \textit{and} $|x-1| \leq 5$.   The first inequality, $2 < |x-1|$, can be re-written as $|x-1|>2$ so it is equivalent to  $x-1 < -2$ \textit{or} $x-1 > 2$.  Thus the solution to $2 < |x-1|$ is $x<-1$ or $x>3$, which in interval notation is  $(-\infty, -1) \cup (3, \infty)$.  For $|x-1| \leq 5$, we combine the results of Theorems \ref{absvalequality} and \ref{absolutevalueineq} to get $-5 \leq x-1 \leq 5$ so that $-4 \leq x \leq 6$, or $[-4,6]$.  Our solution to   $2 < |x-1| \leq 5$ is comprised of values of $x$ which satisfy both parts of the inequality, so we intersect $(-\infty, -1) \cup (3, \infty)$ with $[-4,6]$ to get our final answer $[-4,-1) \cup (3,6]$. 

\item Our first hope when encountering $|10 x - 5| + |10 - 5x| \leq 0$ is that we can somehow combine the two absolute value quantities as we'd done in earlier examples.  We leave it to the reader to show, however, that no matter what we try to factor out of the absolute value quantities, what remains inside the absolute values will always be different.  At this point, we take a step back and look at the equation in a more general way:  we are adding two absolute values together and wanting the result to be less than or equal to $0$.  Since the absolute value of anything is always $0$ or greater, there are no solutions to:  $|10x - 5| + |10 - 5x| < 0$.  Is it possible that $|10x - 5| + |10 - 5x| = 0$?  Only if there is an $x$ where $|10x-5| = 0$ and $|10-5x| = 0$ \textit{at the same time}.\footnote{Do you see why?}  The first equation holds only when $x = \frac{1}{2}$, while the second holds only when $x = 2$.  Alas, we have no solution.\footnote{Not for lack of trying, however!}   \qed

\end{enumerate}

\end{ex}

We close this section with an example of how the properties in Theorem \ref{absolutevalueprops} are used in Calculus.  Here, `$\varepsilon$' is the Greek letter `epsilon' and it represents a positive real number.  Those of you who will be taking Calculus in the future should become \emph{very} familiar with this type of algebraic manipulation.\[ \begin{array}{rclr}

\left| \dfrac{8-4x}{3} \right| & < & \varepsilon & \\ [12pt]

\dfrac{|8 - 4x|}{|3|} & < & \varepsilon & \text{Quotient Rule}\\ [12pt]

\dfrac{|-4(x-2)|}{3} & < & \varepsilon & \text{Factor} \\ [12pt]

\dfrac{|-4| |x-2|}{3} & < & \varepsilon & \text{Product Rule} \\ [12pt]

\dfrac{4 |x-2|}{3} & < & \varepsilon & \\ [12pt]

\dfrac{3}{4} \cdot \dfrac{4 |x-2|}{3} & < & \dfrac{3}{4} \cdot \varepsilon & \text{Multiply by $\dfrac{3}{4}$} \\ [12pt]

|x -2 | & < & \dfrac{3}{4} \varepsilon & \\  \end{array}\]

\newpage

\section{Exercises}

In Exercises \ref{solveabsvalequfirst} - \ref{solveabsvalequlast}, solve the equation.


\begin{multicols}{3}
\begin{enumerate}

\item  $|x| = 6$ \label{solveabsvalequfirst} 
\item $|3t-1| = 10$
\item $|4-w| = 7$

\setcounter{HW}{\value{enumi}}
\end{enumerate}
\end{multicols}

\begin{multicols}{3}
\begin{enumerate}
\setcounter{enumi}{\value{HW}}

\item  $4 - |y| = 3$
\item $2|5m+1| - 3 = 0$
\item $|7x-1| + 2 = 0$

\setcounter{HW}{\value{enumi}}
\end{enumerate}
\end{multicols}

\begin{multicols}{3}
\begin{enumerate}
\setcounter{enumi}{\value{HW}}

\item $\dfrac{5 - |x|}{2} = 1$ \vphantom{$\dfrac{|2x+1| - 3}{4} = \dfrac{1}{2} - |2x+1|$}
\item $\dfrac{2}{3} |5-2w| - \dfrac{1}{2} = 5$ \vphantom{$\dfrac{|2x+1| - 3}{4} = \dfrac{1}{2} - |2x+1|$}
\item $|3t - \sqrt{2}| + 4 = 6$ 
\setcounter{HW}{\value{enumi}}
\end{enumerate}
\end{multicols}


\begin{multicols}{3}
\begin{enumerate}
\setcounter{enumi}{\value{HW}}

\item $\dfrac{|2v+1| - 3}{4} = \dfrac{1}{2} - |2v+1|$
\item $|2x+1| = \dfrac{|2x+1| - 3}{2}$\vphantom{$\dfrac{|2v+1| - 3}{4} = \dfrac{1}{2} - |2v+1|$}
\item $\dfrac{|3-2y|+ 4}{2} = 2 - |3-2y|$\vphantom{$\dfrac{|2v+1| - 3}{4} = \dfrac{1}{2} - |2v+1|$}

\setcounter{HW}{\value{enumi}}
\end{enumerate}
\end{multicols}



\begin{multicols}{3}
\begin{enumerate}
\setcounter{enumi}{\value{HW}}

\item $|3t - 2| = |2t + 7|$  
\item $|3x+1| = |4x|$
\item $|1-\sqrt{2} y| = |y+1|$

\setcounter{HW}{\value{enumi}}
\end{enumerate}
\end{multicols}


\begin{multicols}{3}
\begin{enumerate}
\setcounter{enumi}{\value{HW}}

\item  $|4-x| - |x+2| = 0$
\item $|2-5z| = 5 |z+1|$
\item $\sqrt{3}|w-1| = 2|w+1|$ \label{solveabsvalequlast}


\setcounter{HW}{\value{enumi}}
\end{enumerate}
\end{multicols}


In Exercises \ref{solveinequabsquadfirst} - \ref{solveinequabsquadlast}, solve the inequality.  Write your answer using interval notation. 

\begin{multicols}{3}
\begin{enumerate}
\setcounter{enumi}{\value{HW}}
\item $|3x - 5| \leq 4$ \label{solveinequabsquadfirst}
\item $|7t + 2| > 10$
\item $|2w+1| - 5 < 0$   
\setcounter{HW}{\value{enumi}}
\end{enumerate}
\end{multicols}

\begin{multicols}{3}
\begin{enumerate}
\setcounter{enumi}{\value{HW}}


\item $|2-y| - 4 \geq -3$
\item $|3z+5| + 2 < 1$   
\item $2|7-v| +4 > 1$

\setcounter{HW}{\value{enumi}}
\end{enumerate}
\end{multicols}

\begin{multicols}{3}
\begin{enumerate}
\setcounter{enumi}{\value{HW}}


\item $3 - |x+\sqrt{5}| < -3$
\item $|5t| \leq |t|+3$   
\item $|w-3| < |3-w|$

\setcounter{HW}{\value{enumi}}
\end{enumerate}
\end{multicols}

\begin{multicols}{3}
\begin{enumerate}
\setcounter{enumi}{\value{HW}}

\item  $2 \leq |4-y| < 7$ 
\item $1 < |2w - 9| \leq 3$ 
\item  $3 > 2|\sqrt{3} - x| > 1$ \label{solveinequabsquadlast}
\setcounter{HW}{\value{enumi}}
\end{enumerate}
\end{multicols}

\begin{enumerate}
\setcounter{enumi}{\value{HW}}
\item  With help from your classmates, solve:
\begin{enumerate}
\item  $|5 - |2x-3|| = 4$
\item   $|5 - |2x-3|| < 4$
 
\end{enumerate}

\end{enumerate}


\newpage

\section{Answers}

\begin{multicols}{3}
\begin{enumerate}

\item  $x = -6$ or $x=6$ \vphantom{$t = -3$ or $t= \dfrac{11}{3}$}

\item $t = -3$ or $t= \dfrac{11}{3}$

\item $w = -3$ or $w= 11$ \vphantom{$t = -3$ or $t= \dfrac{11}{3}$}

\setcounter{HW}{\value{enumi}}
\end{enumerate}
\end{multicols}

\begin{multicols}{3}
\begin{enumerate}
\setcounter{enumi}{\value{HW}}

\item  $y = -1$ or $y= 1$\vphantom{$m=-\dfrac{1}{2}$ or $m= \dfrac{1}{10}$}

\item $m=-\dfrac{1}{2}$ or $m= \dfrac{1}{10}$

\item No solution\vphantom{$m=-\dfrac{1}{2}$ or $m= \dfrac{1}{10}$}

\setcounter{HW}{\value{enumi}}
\end{enumerate}
\end{multicols}

\begin{multicols}{3}
\begin{enumerate}
\setcounter{enumi}{\value{HW}}

\item  $x=-3$ or $x= 3$\vphantom{$t = \dfrac{\sqrt{2} \pm 2}{3}$}

\item $w = -\dfrac{13}{8}$ or $w= \dfrac{53}{8}$\vphantom{$t = \dfrac{\sqrt{2} \pm 2}{3}$}

\item $t = \dfrac{\sqrt{2} \pm 2}{3}$


\setcounter{HW}{\value{enumi}}
\end{enumerate}
\end{multicols}


\begin{multicols}{3}
\begin{enumerate}
\setcounter{enumi}{\value{HW}}

\item $v = -1$ or $v = 0$ \vphantom{$y = \dfrac{3}{2}$}

\item  No solution\vphantom{$y = \dfrac{3}{2}$}

\item  $y = \dfrac{3}{2}$

\setcounter{HW}{\value{enumi}}
\end{enumerate}
\end{multicols}



\begin{multicols}{3} 
\begin{enumerate}
\setcounter{enumi}{\value{HW}}

\item $t = -1$ or $t = 9$\vphantom{$x = -\dfrac{1}{7}$ or $x = 1$}

\item $x = -\dfrac{1}{7}$ or $x = 1$

\item $y = 0$ or $y = \dfrac{2}{\sqrt{2} - 1}$ 

\setcounter{HW}{\value{enumi}}
\end{enumerate}
\end{multicols}

\begin{multicols}{3} 
\begin{enumerate}
\setcounter{enumi}{\value{HW}}

\item $x=1$\vphantom{$w = \dfrac{\sqrt{3} \pm 2}{\sqrt{3} \mp 2}$}

\item $z = -\dfrac{3}{10}$\vphantom{$w = \dfrac{\sqrt{3} \pm 2}{\sqrt{3} \mp 2}$}

\item $w = \dfrac{\sqrt{3} \pm 2}{\sqrt{3} \mp 2}$ 

See footnote\footnote{That is, $w = \dfrac{\sqrt{3} + 2}{\sqrt{3} - 2}$ or $w = \dfrac{\sqrt{3} - 2}{\sqrt{3} + 2}$} 

\setcounter{HW}{\value{enumi}}
\end{enumerate}
\end{multicols}

\begin{multicols}{3}
\begin{enumerate}
\setcounter{enumi}{\value{HW}}

\item $\left[\dfrac{1}{3}, 3\right]$\vphantom{$\left(-\infty, -\dfrac{12}{7} \right) \cup \left(\dfrac{8}{7}, \infty\right)$}  
\item $\left(-\infty, -\dfrac{12}{7} \right) \cup \left(\dfrac{8}{7}, \infty\right)$
\item $(-3,2)$\vphantom{$\left(-\infty, -\dfrac{12}{7} \right) \cup \left(\dfrac{8}{7}, \infty\right)$}  
\setcounter{HW}{\value{enumi}}
\end{enumerate}
\end{multicols}

\begin{multicols}{3}
\begin{enumerate}
\setcounter{enumi}{\value{HW}}

 
\item $(-\infty,1] \cup [3,\infty)$
\item No solution  
\item $(-\infty, \infty)$

\setcounter{HW}{\value{enumi}}
\end{enumerate}
\end{multicols}

\begin{multicols}{2}
\begin{enumerate}
\setcounter{enumi}{\value{HW}}


\item $(-\infty, -6-\sqrt{5}) \cup (6-\sqrt{5}, \infty)$ \vphantom{$\left[ -\dfrac{3}{4}, \dfrac{3}{4}\right]$}
\item $\left[ -\dfrac{3}{4}, \dfrac{3}{4}\right]$   


\setcounter{HW}{\value{enumi}}
\end{enumerate}
\end{multicols}

\begin{multicols}{3}
\begin{enumerate}
\setcounter{enumi}{\value{HW}}

\item No solution 
\item $(-3,2] \cup [6,11)$

\item $[3, 4) \cup (5, 6]$

\setcounter{HW}{\value{enumi}}
\end{enumerate}
\end{multicols}

\begin{enumerate}
\setcounter{enumi}{\value{HW}}

\item $\left(\dfrac{2 \sqrt{3} - 3}{2},  \dfrac{2 \sqrt{3} - 1}{2}   \right) \cup \left(\dfrac{2 \sqrt{3} +1}{2},  \dfrac{2 \sqrt{3} +3}{2}   \right)$

\item  \begin{enumerate} \item $x = -3$, or $x = 1$, or $x = 2$, or $x = 6$

\item $(-3,1) \cup (2,6)$

\end{enumerate}

\setcounter{HW}{\value{enumi}}
\end{enumerate}


\end{document}

\newpage

\section{Polynomial Arithmetic}

\documentclass[11pt]{article}
\usepackage[margin=1in,letterpaper]{geometry}
\usepackage{amssymb,amsmath,amsthm,fancyhdr,supertabular,longtable,hhline}
\usepackage{colortbl}
\usepackage{import, multicol,boxedminipage}
\usepackage{graphicx}
\usepackage[colorlinks, hyperindex, plainpages=false, linkcolor=blue, urlcolor=blue, pdfpagelabels]{hyperref}
\usepackage[all]{hypcap}
\definecolor{ResultColor}{gray}{0.9}
\theoremstyle{definition}  % this prevents the text in definitions, theorems, and corollaries from being italicized
\newtheorem{defn}{\bf Definition}
\newtheorem{thm}{\bf Theorem}
\newtheorem{cor}[thm]{\bf Corollary}
\newtheorem{eqn}{\bf Equation}
\newtheorem{ex}{\bf Example}
\newtheorem{fig}{\bf Figure}
\setlength{\parindent}{0in}
\newcommand{\bbm}{\begin{boxedminipage}{6.41in}}
\newcommand{\ebm}{\end{boxedminipage}}
\usepackage{array}
\setlength{\extrarowheight}{2pt}
\allowdisplaybreaks[2]
\usepackage{cancel}
\usepackage{sectsty}
\usepackage{textcomp}
\usepackage{multirow}
\usepackage[sfdefault,lf]{carlito}
	%% The 'lf' option for lining figures
	%% The 'sfdefault' option to make the base font sans serif
	%\usepackage[T1]{fontenc}
	\renewcommand*\oldstylenums[1]{\carlitoOsF #1}
\usepackage[nottoc]{tocbibind}
\allsectionsfont{\mdseries \scshape}
\makeatletter
\renewcommand\l@section{\@dottedtocline{1}{1.5em}{3em}}
\renewcommand\l@subsection{\@dottedtocline{2}{4.5em}{3.5em}}
\makeatother
\pagestyle{fancy}
\newcounter{HW}
\newcounter{HWindent}

\title{Review \#6: Polynomial Arithmetic}
\author{Carl Stitz and Jeff Zeager\\
Edited by Sean Fitzpatrick}

\begin{document}
\maketitle


\renewcommand{\headrulewidth}{0pt}
\renewcommand{\headheight}{14pt}
\lhead[\fancyplain{}{\sc\thepage}]%
      {\fancyplain{}{\sc \nouppercase{\rightmark}}}
\rhead[\fancyplain{}{\sc \nouppercase{\leftmark}}]%
      {\fancyplain{}{\sc\thepage}}
\cfoot{}



In this handout, we review the arithmetic of \textbf{polynomials}. What precisely is a polynomial?\footnote{In this handout we focus on polynomial \textit{expressions}. When you encounter polynomials in your textbook (and courses) it will usually be in the context of polynomial \textit{functions} of the form $f(x) = a_nx^n+\cdots + a_1x + a_0$.}

\medskip

\colorbox{ResultColor}{\bbm

\begin{defn}  A \textbf{polynomial} is a sum of terms each of which is a real number or a real number multiplied by one or more variables to natural number powers. 
\end{defn}

\ebm}

\medskip

Some examples of polynomials are $x^2 + x\sqrt{3} + 4$, $27x^2y + \frac{7x}{2}$ and $6$. Things like $3\sqrt{x}$, $4x - \frac{2}{x+1}$ and $13x^{2/3}y^{2}$ are \textbf{not} polynomials. (Do you see why not?)  Below we review some of the terminology associated with polynomials.

\medskip

\colorbox{ResultColor}{\bbm

\begin{defn}\label{polynomialterminology} \textbf{Polynomial Vocabulary}

\begin{itemize}

\item  \textbf{Constant Terms:} Terms in polynomials without variables are called \textbf{constant} terms.

\item  \textbf{Coefficient:}   In non-constant terms, the real number factor in the expression is called the \textbf{coefficient} of the term. 

\item  \textbf{Degree:}  The \textbf{degree} of a non-constant term is the sum of the exponents on the variables in the term; non-zero constant terms are defined to have degree $0$. The degree of a polynomial is the highest degree of the nonzero terms.

\item  \textbf{Like Terms:} Terms in a polynomial are called \textbf{like} terms if they have the same variables each with the same corresponding exponents.

\item  \textbf{Simplified:} A polynomial is said to be \textbf{simplified} if all arithmetic operations have been completed and there are no longer any like terms.

\item  \textbf{Classification by Number of Terms:}  A simplified polynomial  is  called a 

\begin{itemize}

\item   \textbf{monomial} if it has exactly one nonzero term

\item   \textbf{binomial} if it has exactly two nonzero terms

\item   \textbf{trinomial} if it has exactly three nonzero terms

\end{itemize}

\end{itemize}

\end{defn}

\ebm}

\medskip

For example, $x^2 + x\sqrt{3} +4$ is a trinomial of degree $2$.  The coefficient of $x^2$ is $1$ and the constant term is $4$.  The polynomial $27x^2y + \frac{7x}{2}$ is a binomial of degree $3$ ($x^2y = x^2 y^1$) with constant term $0$.  

\medskip

The concept of `like' terms really amounts to finding terms which can be combined using the Distributive Property.  For example, in the polynomial $17x^2y - 3xy^2 + 7xy^2$, $-3xy^2$ and $7xy^2$ are like terms, since they have the same variables with the same corresponding exponents. This allows us to combine these two terms as follows:  \[17x^2y -  3xy^2 + 7xy^2 = 17x^2y + (-3)xy^2 + 7xy^2 + 17x^2y +(-3 + 7)xy^2 = 17x^2y + 4xy^2\]  Note that even though $17x^2y$ and $4xy^2$ have the same variables, they are not like terms since in the first term we have $x^2$ and $y = y^1$ but in the second we have $x = x^1$ and $y = y^2$ so the corresponding exponents aren't the same.  Hence,  $17x^2y + 4xy^2$ is the simplified form of the polynomial.  

\smallskip

There are four basic operations we can perform with polynomials:  addition, subtraction, multiplication and division. The first three of these operations follow directly from properties of real number arithmetic and will be discussed together first.  Division, on the other hand, is a bit more complicated and will be discussed separately.

\section{Polynomial Addition, Subtraction and Multiplication.}
\label{polyaddsubtmult}

Adding and subtracting polynomials comes down to identifying like terms and then adding or subtracting the coefficients of those like terms.  Multiplying polynomials comes to us courtesy of the Generalized Distributive Property.

\medskip

\colorbox{ResultColor}{\bbm

\begin{thm}\label{generaldistprop} \textbf{Generalized Distributive Property:}  To multiply a quantity of $n$ terms by a quantity of $m$ terms, multiply each of the $n$ terms of the first quantity by each of the $m$ terms in the second quantity and add the resulting $n \cdot m$ terms together. 

\end{thm}

\ebm}

\medskip

In particular, Theorem \ref{generaldistprop} says that, before combining like terms, a product of an $n$-term polynomial and an $m$-term polynomial will generate $(n \cdot m)$-terms.  For example, a binomial times a trinomial will produce six terms some of which may be like terms.  Thus the simplified end result may have fewer than six terms but you will start with six terms. 

\medskip

A special case of Theorem \ref{generaldistprop}  is the famous \textbf{F.O.I.L.}, listed here:\footnote{We caved to peer pressure on this one.  Apparently all of the cool Precalculus books have FOIL in them even though it's redundant once you know how to distribute multiplication across addition.  In general, we don't like mechanical short-cuts that interfere with a student's understanding of the material and FOIL is one of the worst.}

\medskip

\colorbox{ResultColor}{\bbm

\begin{thm}\label{FOIL} \textbf{F.O.I.L:} The terms generated from the product of two binomials: $(a + b)(c+d)$ can be verbalized as follows ``Take the sum of:

\begin{itemize}

\item the product of the \textbf{F}irst terms $a$ and $c$, $ac$

\item the product of the \textbf{O}uter terms $a$ and $d$, $ad$

\item  the product of the \textbf{I}nner terms $b$ and $c$, $bc$

\item  the product of the \textbf{L}ast terms $b$ and $d$, $bd$.''

\end{itemize}

That is, $(a+b)(c+d) = ac + ad + bc + bd$.

\end{thm}

\ebm}

\medskip

Theorem \ref{generaldistprop} is best proved using the technique known as Mathematical Induction which is covered in Math 2000.  The result is really nothing more than repeated applications of the Distributive Property so it seems reasonable and we'll use it without proof for now.  The other major piece of polynomial multiplication is the law of exponents  $a^n a^m = a^{n+m}$.  The Commutative and Associative Properties of addition and multiplication are also used extensively.  We put all of these properties to good use in the next example.

\pagebreak

\begin{ex}

\label{polyaddsubtmultex} 

Perform the indicated operations and simplify.

\begin{multicols}{2}

\begin{enumerate}

\item  $\left(3x^2 - 2x + 1\right) - (7x-3)$

\item  $4xz^2 - 3z(xz - x + 4)$

\setcounter{HW}{\value{enumi}}

\end{enumerate}

\end{multicols}



\begin{multicols}{2}

\begin{enumerate}

\setcounter{enumi}{\value{HW}}

\item $(2t+1)(3t - 7)$ \vphantom{$\left(3y - \sqrt[3]{2}\right)\left(9y^2 + 3\sqrt[3]{2} y + \sqrt[3]{4}\right)$}

\item  $\left(3y - \sqrt[3]{2}\right)\left(9y^2 + 3\sqrt[3]{2} y + \sqrt[3]{4}\right)$

\setcounter{HW}{\value{enumi}}

\end{enumerate}

\end{multicols}



\begin{multicols}{2}

\begin{enumerate}

\setcounter{enumi}{\value{HW}}

\item  $\left(4w - \dfrac{1}{2} \right)^2$

\item  $\left[2(x+h) - (x+h)^2\right] - \left(2x - x^2 \right)$ \vphantom{ $\left(4w - \dfrac{1}{2} \right)^2$}

\setcounter{HW}{\value{enumi}}

\end{enumerate}

\end{multicols}



{\bf Solution.}

\begin{enumerate}

\item  We begin `distributing the negative', then we rearrange and combine like terms:\[\begin{array}{rclr}

\left(3x^2 - 2x + 1\right) - (7x-3) &  = & 3x^2-2x+1 - 7x + 3 & \text{Distribute} \\
                                    & = & 3x^2  -2x - 7x + 1 + 3 & \text{Rearrange terms} \\
																		& = & 3x^2 - 9x + 4 & \text{Combine like terms}  \\ 
																		\end {array} \] Our answer is $3x^2 - 9x + 4$.

\item  Following in our footsteps from the previous example, we first distribute the $-3z$ through, then rearrange and combine like terms.\[ \begin{array}{rclr}

4xz^2 - 3z(xz - x + 4) & = & 4xz^2 - 3z(xz) + 3z (x) - 3z(4) & \text{Distribute} \\
                       & = & 4xz^2 - 3xz^2 + 3xz - 12 z & \text{Multiply} \\
											 & = & xz^2+ 3xz - 12 z & \text{Combine like terms} \\

\end{array}\] We get our final answer: $xz^2+ 3xz - 12z$


\item  At last, we have a chance to use our F.O.I.L. technique:\[ \begin{array}{rclr}

(2t+1)(3t - 7) & = & (2t)(3t) + (2t)(-7) + (1)(3t) + (1)(-7) & \text{F.O.I.L.} \\
               & = & 6t^2 - 14t + 3t - 7 & \text{Multiply} \\
							 & = & 6t^2 - 11t - 7 & \text{Combine like terms} \\
							\end{array} \] We get $6t^2 - 11t - 7$ as our final answer.

\item  We use the Generalized Distributive Property here, multiplying each term in the second quantity first by $3y$, then by $-\sqrt[3]{2}$:\[ \begin{array}{rclr}

\left(3y - \sqrt[3]{2}\right)\left(9y^2 + 3\sqrt[3]{2} y + \sqrt[3]{4}\right) & = & 3y\left(9y^2\right) +3y\left(3\sqrt[3]{2} y\right) + 3y\left(\sqrt[3]{4}\right) & \\
       & & -\sqrt[3]{2} \left(9y^2\right) - \sqrt[3]{2} \left(3\sqrt[3]{2} y\right) -\sqrt[3]{2} \left(\sqrt[3]{4}\right) & \\
			  & = & 27y^3 + 9y^2 \sqrt[3]{2} + 3y \sqrt[3]{4} - 9y^2\sqrt[3]{2} - 3y \sqrt[3]{4} - \sqrt[3]{8} & \\
				& = & 27y^3 + 9y^2 \sqrt[3]{2} - 9y^2 \sqrt[3]{2} + 3y \sqrt[3]{4} - 3y \sqrt[3]{4} - 2 & \\
				& = & 27y^3 - 2 \\ 

\end{array} \] To our surprise and delight, this product reduces to $27y^3 - 2$.

\item Since exponents do \textbf{not} distribute across powers,  $\left(4w - \frac{1}{2} \right)^2 \neq (4w)^2 - \left(\frac{1}{2}\right)^2$.  (We know you knew that.)  Instead, we proceed as follows:\[ \begin{array}{rclr}

\left(4w - \dfrac{1}{2} \right)^2 & = & \left(4w - \dfrac{1}{2} \right)\left(4w - \dfrac{1}{2} \right) & \hspace*{-.2in} \\[12pt]
                                 & = & (4w)(4w) + (4w)\left(-\dfrac{1}{2}\right) + \left(-\dfrac{1}{2}\right)(4w) + \left(-\dfrac{1}{2}\right)\left(-\dfrac{1}{2}\right) & \hspace*{-.2in} \text{F.O.I.L.} \\[12pt]
																
																& = & 16w^2 - 2w - 2w + \dfrac{1}{4} & \hspace*{-.2in} \text{Multiply} \\ 
                                & = & 16w^2 - 4w + \dfrac{1}{4} & \hspace*{-.2in} \text{Combine like terms} \\ 
\end{array}\]

Our (correct) final answer is $16w^2 - 4w + \frac{1}{4}$.

\item  Our last example has two levels of grouping symbols.  We begin simplifying the quantity inside the brackets, squaring out the binomial $(x+h)^2$ in the same way we expanded the square in our last example: \[ (x+h)^2 = (x+h)(x+h) = (x)(x) + (x)(h) + (h)(x) + (h)(h) = x^2 + 2xh + h^2 \]  When we substitute this into our expression, we envelope it in parentheses, as usual, so we don't forget to distribute the negative.\[ \begin{array}{rclr}
					
\left[2(x+h) - (x+h)^2\right] - \left(2x - x^2 \right) & = & \left[2(x+h) - \left( x^2 + 2xh + h^2\right) \right] - \left(2x - x^2 \right) & \hspace*{-.6in}\text{Substitute} \\
	                                                     & = & \left[2x+2h - x^2-2xh-h^2\right] - \left(2x - x^2 \right) & \hspace*{-6in} \text{Distribute}\\ 
                                                      & = & 2x+2h - x^2-2xh-h^2 -2x + x^2 & \hspace*{-.6in}\text{Distribute} \\ 
																											 & = & 2x - 2x+2h - x^2 + x^2 -2xh-h^2& \hspace*{-.6in}\text{Rearrange terms}\\
																											 & = & 2h-2xh-h^2& \hspace*{-.6in}\text{Combine like terms}\\
																											\end{array} \] We find no like terms in $2h-2xh-h^2$ so we are finished. \qed                                               

\end{enumerate}

\end{ex}



We conclude our discussion of polynomial multiplication by showcasing two special products which happen often enough they should be committed to memory.

\medskip

\colorbox{ResultColor}{\bbm

\begin{thm}\label{SpecialProducts} \textbf{Special Products:} Let $a$ and $b$ be real numbers:

\begin{itemize}

\item \textbf{Perfect Square:}  $(a+b)^2 = a^2 + 2ab + b^2$ and $(a-b)^2 = a^2 - 2ab + b^2$

\item \textbf{Difference of Two Squares:}  $(a-b)(a+b) = a^2 - b^2$ 

\end{itemize}

\end{thm}

\ebm}

\medskip

The formulas in Theorem \ref{SpecialProducts} can be verified by working through the multiplication.\footnote{These are both special cases of F.O.I.L.}

\section{Polynomial Long Division.}
\label{polylongdiv}

We now turn our attention to polynomial long division.  Dividing two polynomials follows the same algorithm, in principle, as dividing two natural numbers so we review that process first.  Suppose we wished to divide $2585$ by $79$.  The standard division tableau is given below. 

\setlength\arraycolsep{0.1pt}
\setlength\extrarowheight{2pt}

\[ \begin{array}{cccccc}

    &             &      &    & 3   & 2  \\ \hhline{~~|----}

  7 & 9 \, \vline & \, 2 & 5 & 8 & 5  \\

    &            -&    2 & 3 & 7 & \downarrow \\ \hhline{~~---} 
    &             &      & 2 & 1 &  5   \\ 
    &             &     - & 1 & 5 & 8    \\ \hhline{~~~---} 
    &             &      &   & 5 & 7    \\

 
\end{array}\]

\setlength\arraycolsep{5pt}
\setlength\extrarowheight{0pt}

In this case, $79$ is called the \textbf{divisor}, $2585$ is called the \textbf{dividend}, $32$ is called the \textbf{quotient} and $57$ is called the \textbf{remainder}.  We can check our answer by showing:  \[ \text{dividend} = (\text{divisor})( \text{quotient}) + \text{remainder}\] or in this case, $2585 = 
 (79)(32) + 57 \checkmark$.  We hope that the long division tableau evokes warm, fuzzy memories of your formative years as opposed to feelings of hopelessness and frustration.  If you experience the latter, keep in mind that the Division Algorithm essentially is a two-step process, iterated over and over again.  First, we guess the number of times the divisor goes into the dividend and  then we subtract off our guess.  We repeat those steps with what's left over until what's left over (the remainder) is less than what we started with (the divisor).  That's all there is to it!

\smallskip

The division algorithm for polynomials has the same basic two steps but when we subtract polynomials, we must take care to subtract \emph{like terms} only.  As a transition to polynomial division, let's write out our previous division tableau in expanded form.


\setlength\arraycolsep{0.1pt}
\setlength\extrarowheight{2pt}

\[ \begin{array}{cccccccccc}

& & & & & & & 3 \cdot 10 & + & 2 \\ \hhline{~~~|-------}

7 \cdot 10 & + & 9 \, \vline& 2\cdot 10^3 & + & 5 \cdot 10^2 & + & 8 \cdot 10 & + & 5 \\

 &  &  -& \left(2 \cdot 10^3 \right. & + &  3 \cdot 10^2  & + & \left. 7 \cdot 10 \right) &  &  \downarrow \\ \hhline{~~~-----~~} 
 &  &  &   &  & 2 \cdot 10^2 & +  & 1 \cdot 10 & + & 5 \\ 
 &  &  &   & - & \left(1 \cdot 10^2 \right. & +  &  5 \cdot 10 &  + &\left.  8 \right) \\ \hhline{~~~~~---~~} 
 &  &  &   &   &  & & 5 \cdot 10  & + & 7 \\

 
\end{array}\]

\setlength\arraycolsep{5pt}
\setlength\extrarowheight{0pt}

Written this way, we see that when we line up the digits we are really lining up the coefficients of the corresponding powers of $10$ - much like how we'll have to keep the powers of $x$ lined up in the same columns.  The big difference between polynomial division and the division of natural numbers is that the value of $x$ is an unknown quantity.  So unlike using the known value of $10$, when we subtract there can be no regrouping of coefficients as in our previous example. (The subtraction $215 - 158$ requires us to `regroup' or `borrow' from the tens digit, then the hundreds digit.) This actually makes polynomial division easier.\footnote{In our opinion - you can judge for yourself.}  Before we dive into examples, we first state a theorem telling us when we can divide two polynomials, and what to expect when we do so.

\medskip

\colorbox{ResultColor}{\bbm

\begin{thm} \label{polydiv} \textbf{Polynomial Division:}  Let $d$ and $p$ be nonzero polynomials where the degree of $p$ is greater than or equal to the degree of $d$.  There exist two unique polynomials, $q$ and $r$, such that $p = d \cdot q + r,\,$ where either $r = 0$ or the degree of $r$ is strictly less than the degree of $d$.
\end{thm}
\ebm}

\medskip

Essentially, Theorem \ref{polydiv} tells us that we can divide polynomials whenever the degree of the divisor is less than or equal to the degree of the dividend.  We know we're done with the division when the polynomial left over (the remainder) has a degree strictly less than the divisor.  It's time to walk through a few examples to refresh your memory.

\begin{ex}\label{polynomiallongdivex}  Perform the indicated division.  Check your answer by showing \[\text{dividend} = (\text{divisor})( \text{quotient}) + \text{remainder}\]

\begin{multicols}{2}

\begin{enumerate}

\item  $\left(x^3 + 4x^2 - 5x - 14\right) \div (x-2)$

\item  $\left(2t +  7\right) \div \left(3t - 4\right)$

\setcounter{HW}{\value{enumi}}

\end{enumerate}

\end{multicols}

\begin{multicols}{2}

\begin{enumerate}

\setcounter{enumi}{\value{HW}}



\item  $\left(6y^2 - 1 \right) \div \left(2y + 5\right)$

\item  $\left(w^3 \right) \div \left(w^2 - \sqrt{2}\right)$.

\setcounter{HW}{\value{enumi}}

\end{enumerate}

\end{multicols}


{\bf Solution.}

\begin{enumerate}

\item  To begin $\left(x^3 + 4x^2 - 5x - 14\right) \div (x-2)$, we divide the first term in the dividend, namely $x^3$, by the first term in the divisor, namely $x$, and get $\frac{x^3}{x} = x^2$. This then becomes the first term in the quotient.  We proceed as in regular long division at this point: we multiply the entire divisor, $x-2$, by this first term in the quotient to get $x^{2}(x - 2) = x^3 - 2x^2$.  We then subtract this result from the dividend.\setlength\arraycolsep{0.1pt}\setlength\extrarowheight{2pt}\[ \begin{array}{cccccccccc}

& & & & & x^2 & & &  &  \\ \hhline{~~~|-------}

x & - & 2 \, \vline& x^3 & + & 4x^2 & - & 5x & - & 14 \\

 &  &  -& \left(x^3 \right. & - & \left.  2x^2\right) &  & \downarrow &  &  \\ \hhline{~~~---~~~~} 
 &  &  &   &  & 6 x^2 & - & 5x &  &  \\ 
% &  &  &   & - & \left(6 x^2 \right. & - & \left. 12x \right) &  &  \\ \hhline{~~~~~---~~} 
% &  &  &   &   &  & & 7x  & - & 14 \\
% &  &  &   &   &  & - & \left( 7x \right. & - & \left. 14 \right) \\ \hhline{~~~~~~~---} 
% &   &  &  &  &  &  &  &  & 0
 
\end{array}\]

\setlength\arraycolsep{5pt}
\setlength\extrarowheight{0pt} 

Now we `bring down' the next term of the quotient, namely $-5x$, and repeat the process. We divide $\frac{6x^2}{x} = 6x$, and add this to the quotient polynomial, multiply it by the divisor (which yields $6x(x - 2) = 6x^{2} - 12x$) and subtract. \setlength\arraycolsep{0.1pt}\setlength\extrarowheight{2pt}\[ \begin{array}{cccccccccc}

& & & & & x^2 & + & 6x &  &  \\ \hhline{~~~|-------}

x & - & 2 \, \vline& x^3 & + & 4x^2 & - & 5x & - & 14 \\

 &  &  -& \left(x^3 \right. & - & \left.  2x^2\right) &  & &  & \downarrow  \\ \hhline{~~~---~~~~} 
 &  &  &   &  & 6 x^2 & - & 5x &  &  \downarrow \\ 
 &  &  &   & - & \left(6 x^2 \right. & - & \left. 12x \right) &  & \downarrow \\ \hhline{~~~~~---~~} 
 &  &  &   &   &  & & 7x  & - & 14 \\
% &  &  &   &   &  & - & \left( 7x \right. & - & \left. 14 \right) \\ \hhline{~~~~~~~---} 
% &   &  &  &  &  &  &  &  & 0
 
\end{array}\]

\setlength\arraycolsep{5pt}
\setlength\extrarowheight{0pt}

Finally, we `bring down' the last term of the dividend, namely $-14$, and repeat the process.  We divide $\frac{7x}{x} = 7$, add this to the quotient, multiply it by the divisor (which yields $7(x - 2) = 7x - 14$) and subtract.\setlength\arraycolsep{0.1pt}\setlength\extrarowheight{2pt}\[ \begin{array}{cccccccccc}

& & & & & x^2 & + & 6x & + & 7 \\ \hhline{~~~|-------}

x & - & 2 \, \vline& x^3 & + & 4x^2 & - & 5x & - & 14 \\

 &  &  -& \left(x^3 \right. & - & \left.  2x^2\right) &  &  &  &  \\ \hhline{~~~---~~~~} 
 &  &  &   &  & 6 x^2 & - & 5x &  &  \\ 
 &  &  &   & - & \left(6 x^2 \right. & - & \left. 12x \right) &  &  \\ \hhline{~~~~~---~~} 
 &  &  &   &   &  & & 7x  & - & 14 \\
 &  &  &   &   &  & - & \left( 7x \right. & - & \left. 14 \right) \\ \hhline{~~~~~~~---} 
 &   &  &  &  &  &  &  &  & 0
 
\end{array}\]
\setlength\arraycolsep{5pt}
\setlength\extrarowheight{0pt}

In this case, we get a quotient of $x^2 + 6x + 7$ with a remainder of $0$.  To check our answer, we compute  \[(x-2)\left(x^2 + 6x + 7\right) + 0 = x^3 + 6x^2 + 7x - 2x^2 - 12x -14 =  x^3 + 4x^2 - 5x - 14 \, \checkmark \]


\item    To compute  $\left(2t +  7\right) \div \left(3t - 4\right)$, we start as before.  We find $\frac{2t}{3t} = \frac{2}{3}$, so that becomes the first (and only) term in the quotient.  We multiply the divisor $(3t-4)$ by $\frac{2}{3}$ and get $2t - \frac{8}{3}$.  We subtract this from the divided and get $\frac{29}{3}$.\setlength\arraycolsep{0.1pt}\setlength\extrarowheight{5pt}\[ \begin{array}{cccccc}

& & & & & \dfrac{2}{3} \\ \hhline{~~~|---}

3t & - & 4 \, \vline& 2t & + & 7  \\

 &  &  -& \left(2t\vphantom{\dfrac{8}{3}} \right. & - & \left.  \dfrac{8}{3}\right)  \\ \hhline{~~~---}
 &  &  &   &  & \dfrac{29}{3} \vphantom{\sqrt{\dfrac{7}{7}}} \\ 

 
\end{array}\]
\setlength\arraycolsep{5pt}
\setlength\extrarowheight{0pt}

Our answer is $\frac{2}{3}$ with a remainder of $\frac{29}{3}$.  To check our answer, we compute \[(3t-4) \left(\frac{2}{3}\right) + \frac{29}{3} = 2t - \frac{8}{3} + \frac{29}{3} = 2t + \frac{21}{3} = 2t + 7 \, \checkmark\]

\item When we set-up the tableau for   $\left(6y^2 - 1 \right) \div \left(2y + 5\right)$, we must first issue a `placeholder' for the `missing' $y$-term in the dividend, $6y^2 -1 = 6y^2 + 0y - 1$.  We then proceed as before.  Since $\frac{6y^2}{2y} = 3y$, $3y$ is the first term in our quotient. We multiply $(2y+5)$ times $3y$ and subtract it from the dividend.  We bring down the $-1$, and repeat.  \setlength\arraycolsep{0.1pt}\setlength\extrarowheight{5pt}\[ \begin{array}{cccccccc}

& & & & & 3y & - & \dfrac{15}{2}  \\ \hhline{~~~|-----}

2y& + & 5 \, \vline& 6y^2 & + & 0y & - & 1  \\

 &  &  -& \left(6y^2 \right. & + & \left.  15y\right) &  & \downarrow  \\ \hhline{~~~---~~} 
 &  &  &   &  & -15y & - & 1  \\ 
 &  &  &   & - & \left(-15y\vphantom{\dfrac{75}{2}} \right. & - & \left. \dfrac{75}{2} \right) \\ \hhline{~~~~~---} 
 &  &  &   &   &  & & \dfrac{73}{2} \vphantom{\sqrt{\dfrac{73}{2}}}\\
 
\end{array}\]
\setlength\arraycolsep{5pt}
\setlength\extrarowheight{0pt}
 
Our answer is $3y - \frac{15}{2}$ with a remainder of $\frac{73}{2}$.  To check our answer, we compute:

\[ (2y + 5)\left(3y - \dfrac{15}{2}\right) + \dfrac{73}{2} = 6y^2 - 15y + 15y - \dfrac{75}{2} + \dfrac{73}{2} = 6y^2 - 1 \, \checkmark\]


\item For our last example, we need `placeholders' for both the divisor  $w^2 - \sqrt{2} = w^2 + 0w -\sqrt{2}$ and the dividend $w^3 = w^3 + 0w^2 + 0w + 0$.  The first term in the quotient is $\frac{w^3}{w^2} = w$, and when we multiply and subtract this from the dividend, we're left with just $0w^2 + w\sqrt{2} + 0 = w\sqrt{2}$.

\setlength\arraycolsep{0.1pt}
\setlength\extrarowheight{2pt}

\[ \begin{array}{cccccccccccc}

    &   &    &   &                    &     &   &      &   &  w &   & \\ \hhline{~~~~~|-------}

w^2 & + & 0w & - & \sqrt{2} \, \vline & w^3 & + & 0w^2 & + & 0w & + & 0  \\
    
		&   &    &    &                  -&\left(w^3\vphantom{w\sqrt{2}} \right. & + & 0w^2 & - & \left.  w\sqrt{2} \right) & & \downarrow \\ \hhline{~~~~~-----~~}
    &   &    &    &                   &                                       &  &  0w^2     &  + &   w\sqrt{2}  & + & 0\\ 
 
\end{array}\]
\setlength\arraycolsep{5pt}
\setlength\extrarowheight{0pt}

Since the degree of $w\sqrt{2}$ (which is $1$) is less than the degree of the divisor (which is $2$), we are done.\footnote{Since $\frac{0w^2}{w^2} = 0$, we could proceed, write our quotient as $w+0$, and move on\ldots but even pedants have limits.}  Our answer is $w$ with a remainder of $w \sqrt{2}$.  To check, we compute:

\[ \left(w^2 - \sqrt{2}\right)w + w\sqrt{2} = w^3 - w\sqrt{2} + w\sqrt{2} = w^3 \, \checkmark\]

\vspace{-0.3in}\qed

\end{enumerate}

\end{ex}

\newpage

\section{Exercises}

In Exercises \ref{polyarithexfirst} - \ref{polyarithexlast}, perform the indicated operations and simplify.

\begin{multicols}{3}
\begin{enumerate}

\item  $(4-3x) + (3x^2 + 2x + 7)$ \label{polyarithexfirst}
\item $t^2 + 4t - 2(3-t)$
\item $q(200-3q) - (5q + 500)$

\setcounter{HW}{\value{enumi}}
\end{enumerate}
\end{multicols}

\begin{multicols}{3}
\begin{enumerate}
\setcounter{enumi}{\value{HW}}

\item $(3y-1)(2y+1)$\vphantom{$\left(3-\dfrac{x}{2}\right)(2x+5)$}
\item $\left(3-\dfrac{x}{2}\right)(2x+5)$
\item $-(4t+3)(t^2-2)$\vphantom{$\left(3-\dfrac{x}{2}\right)(2x+5)$}

\setcounter{HW}{\value{enumi}}
\end{enumerate}
\end{multicols}

\begin{multicols}{3}
\begin{enumerate}
\setcounter{enumi}{\value{HW}}

\item $2w(w^3-5)(w^3+5)$
\item $(5a^2 - 3)(25a^4 + 15a^2 + 9)$
\item $(x^2-2x+3)(x^2+2x+3)$


\setcounter{HW}{\value{enumi}}
\end{enumerate}
\end{multicols}

\begin{multicols}{3}
\begin{enumerate}
\setcounter{enumi}{\value{HW}}

\item $(\sqrt{7} - z)(\sqrt{7} + z)$
\item $(x - \sqrt[3]{5})^3$
\item $(x - \sqrt[3]{5})(x^2 + x\sqrt[3]{5} + \sqrt[3]{25})$

\setcounter{HW}{\value{enumi}}
\end{enumerate}
\end{multicols}

\begin{multicols}{3}
\begin{enumerate}
\setcounter{enumi}{\value{HW}}


\item $(w-3)^2 - (w^2 + 9)$
\item $(x+h)^2 - 2(x+h) - (x^2 - 2x)$
\item $(x-[2+\sqrt{5}])(x-[2-\sqrt{5}])$ \label{polyarithexlast}

\setcounter{HW}{\value{enumi}}
\end{enumerate}
\end{multicols}

In Exercises \ref{polydivexfirst} - \ref{polydivexlast}, perform the indicated division.  Check your answer by showing \[\text{dividend} = (\text{divisor})( \text{quotient}) + \text{remainder}\]

\begin{multicols}{2}
\begin{enumerate}
\setcounter{enumi}{\value{HW}}

\item $(5x^2 - 3x + 1) \div (x + 1)$ \label{polydivexfirst}
\item $(3y^2 + 6y - 7) \div (y-3)$

\setcounter{HW}{\value{enumi}}
\end{enumerate}
\end{multicols}


\begin{multicols}{2}
\begin{enumerate}
\setcounter{enumi}{\value{HW}}

\item $(6w - 3) \div (2w+5)$
\item $(2x+1) \div (3x-4)$


\setcounter{HW}{\value{enumi}}
\end{enumerate}
\end{multicols}


\begin{multicols}{2}
\begin{enumerate}
\setcounter{enumi}{\value{HW}}

\item $(t^2 - 4) \div (2t + 1)$

\item $(w^3 - 8) \div (5w-10)$

\setcounter{HW}{\value{enumi}}
\end{enumerate}
\end{multicols}

\begin{multicols}{2}
\begin{enumerate}
\setcounter{enumi}{\value{HW}}

\item $(2x^2 - x + 1) \div (3x^2 + 1)$

\item $(4y^4+3y^2+1) \div (2y^2-y+1)$

\setcounter{HW}{\value{enumi}}
\end{enumerate}
\end{multicols}

\begin{multicols}{2}
\begin{enumerate}
\setcounter{enumi}{\value{HW}}

\item $w^4 \div (w^3 - 2)$
\item $(5t^3 - t + 1) \div (t^2 + 4)$



\setcounter{HW}{\value{enumi}}
\end{enumerate}
\end{multicols}

\begin{multicols}{2}
\begin{enumerate}
\setcounter{enumi}{\value{HW}}
\item $(t^3 - 4) \div (t - \sqrt[3]{4})$

\item $(x^2-2x-1) \div (x-[1-\sqrt{2}])$  \label{polydivexlast}

\setcounter{HW}{\value{enumi}}
\end{enumerate}
\end{multicols}

In Exercises \ref{specialformexfirst} - \ref{specialformexlast} verify the given formula by showing the left hand side of the equation simplifies to the right hand side of the equation.

\begin{enumerate}
\setcounter{enumi}{\value{HW}}

\item \textbf{Perfect Cube:} $(a+b)^3 = a^3 + 3a^2b + 3ab^2 + b^3$ \label{specialformexfirst}

\item  \textbf{Difference of Cubes:} $(a - b)(a^2 + ab + b^2) = a^3 - b^3$

\item  \textbf{Sum of Cubes:} $(a + b)(a^2 - ab + b^2) = a^3 + b^3$

\item  \textbf{Perfect Quartic:} $(a+b)^4 = a^4 + 4a^3b + 6a^2b^2 + 4ab^3 + b^4$

\item  \textbf{Difference of Quartics:} $(a-b)(a+b)(a^2+b^2) = a^4 - b^4$

\item  \textbf{Sum of Quartics:}  $(a^2 + ab \sqrt{2} + b^2)(a^2 - ab \sqrt{2} + b^2) = a^4 + b^4$ \label{specialformexlast}

\item  With help from your classmates, determine under what conditions $(a+b)^2 = a^2 + b^2$.  What about $(a+b)^3 = a^3 + b^3$? In general, when does $(a+b)^n = a^n + b^n$ for a natural number $n \geq 2$?

\setcounter{HW}{\value{enumi}}
\end{enumerate}


\newpage

\section{Answers}
\begin{multicols}{3}
\begin{enumerate}

\item $3x^2 - x + 11$
\item $t^2 + 6t-6$
\item $-3q^2+195q-500$

\setcounter{HW}{\value{enumi}}
\end{enumerate}
\end{multicols}

\begin{multicols}{3}
\begin{enumerate}
\setcounter{enumi}{\value{HW}}

\item $6y^2+y-1$\vphantom{$-x^2 + \dfrac{7}{2} x + 15$}
\item $-x^2 + \dfrac{7}{2} x + 15$
\item $-4t^3-3t^2+8t+6$\vphantom{$-x^2 + \dfrac{7}{2} x + 15$}

\setcounter{HW}{\value{enumi}}
\end{enumerate}
\end{multicols}

\begin{multicols}{3}
\begin{enumerate}
\setcounter{enumi}{\value{HW}}

\item $2w^7 - 50w$
\item $125a^6 - 27$
\item $x^4+2x^2+9$


\setcounter{HW}{\value{enumi}}
\end{enumerate}
\end{multicols}

\begin{multicols}{3}
\begin{enumerate}
\setcounter{enumi}{\value{HW}}

\item $7-z^2$
\item $x^3 - 3x^2\sqrt[3]{5} + 3x\sqrt[3]{25} - 5$
\item $x^3 - 5$

\setcounter{HW}{\value{enumi}}
\end{enumerate}
\end{multicols}

\begin{multicols}{3}
\begin{enumerate}
\setcounter{enumi}{\value{HW}}


\item $-6w$
\item $h^2 + 2xh - 2h$
\item $x^2 - 4x - 1$ 

\setcounter{HW}{\value{enumi}}
\end{enumerate}
\end{multicols}



\begin{multicols}{2}
\begin{enumerate}
\setcounter{enumi}{\value{HW}}

\item quotient: $5x-8$, remainder: $9$ 
\item quotient: $3y+15$, remainder: $38$

\setcounter{HW}{\value{enumi}}
\end{enumerate}
\end{multicols}


\begin{multicols}{2}
\begin{enumerate}
\setcounter{enumi}{\value{HW}}

\item quotient: $3$, remainder: $18$ \vphantom{$\dfrac{11}{3}$}
\item quotient: $\dfrac{2}{3}$, remainder: $\dfrac{11}{3}$

\setcounter{HW}{\value{enumi}}
\end{enumerate}
\end{multicols}


\begin{multicols}{2}
\begin{enumerate}
\setcounter{enumi}{\value{HW}}

\item quotient: $\dfrac{t}{2} - \dfrac{1}{4}$, remainder: $-\dfrac{15}{4}$ \vphantom{ $\dfrac{w^2}{5} + \dfrac{2w}{5} + \dfrac{4}{5}$, remainder: $0$}

\item quotient: $\dfrac{w^2}{5} + \dfrac{2w}{5} + \dfrac{4}{5}$, remainder: $0$

\setcounter{HW}{\value{enumi}}
\end{enumerate}
\end{multicols}

\begin{multicols}{2}
\begin{enumerate}
\setcounter{enumi}{\value{HW}}

\item quotient: $\dfrac{2}{3}$, remainder: $-x + \dfrac{1}{3}$

\item quotient:  $2y^2+y+1$, remainder: $0$ \vphantom{$\dfrac{2}{3}$, remainder: $-x + \dfrac{1}{3}$}

\setcounter{HW}{\value{enumi}}
\end{enumerate}
\end{multicols}

\begin{multicols}{2}
\begin{enumerate}
\setcounter{enumi}{\value{HW}}

\item quotient: $w$, remainder: $2w$
\item quotient: $5t$, remainder: $-21t + 1$


\setcounter{HW}{\value{enumi}}
\end{enumerate}
\end{multicols}

\begin{multicols}{2}
\begin{enumerate}
\setcounter{enumi}{\value{HW}}
\item quotient:\footnote{Note: $\sqrt[3]{16} = 2\sqrt[3]{2}$.} $t^2 + t \sqrt[3]{4} + 2\sqrt[3]{2}$, remainder: $0$

\item quotient: $x -1 - \sqrt{2}$, remainder: 0  
\setcounter{HW}{\value{enumi}}
\end{enumerate}
\end{multicols}

\end{document}

\newpage

\section{Factoring}

\documentclass[11pt]{article}
\usepackage[margin=1in,letterpaper]{geometry}
\usepackage{amssymb,amsmath,amsthm,fancyhdr,supertabular,longtable,hhline}
\usepackage{colortbl}
\usepackage{import, multicol,boxedminipage}
\usepackage{graphicx}
\usepackage[colorlinks, hyperindex, plainpages=false, linkcolor=blue, urlcolor=blue, pdfpagelabels]{hyperref}
\usepackage[all]{hypcap}
\definecolor{ResultColor}{gray}{0.9}
\theoremstyle{definition}  % this prevents the text in definitions, theorems, and corollaries from being italicized
\newtheorem{defn}{\bf Definition}
\newtheorem{thm}{\bf Theorem}
\newtheorem{cor}[thm]{\bf Corollary}
\newtheorem{eqn}{\bf Equation}
\newtheorem{ex}{\bf Example}
\newtheorem{fig}{\bf Figure}
\setlength{\parindent}{0in}
\newcommand{\bbm}{\begin{boxedminipage}{6.41in}}
\newcommand{\ebm}{\end{boxedminipage}}
\usepackage{array}
\setlength{\extrarowheight}{2pt}
\allowdisplaybreaks[2]
\usepackage{cancel}
\usepackage{sectsty}
\usepackage{textcomp}
\usepackage{multirow}
\usepackage[sfdefault,lf]{carlito}
	%% The 'lf' option for lining figures
	%% The 'sfdefault' option to make the base font sans serif
	\usepackage[T1]{fontenc}
	\renewcommand*\oldstylenums[1]{\carlitoOsF #1}
\usepackage[nottoc]{tocbibind}
\allsectionsfont{\mdseries \scshape}
\makeatletter
\renewcommand\l@section{\@dottedtocline{1}{1.5em}{3em}}
\renewcommand\l@subsection{\@dottedtocline{2}{4.5em}{3.5em}}
\makeatother
\pagestyle{fancy}
\newcounter{HW}
\newcounter{HWindent}

\title{Review \#5: Factoring}
\author{Carl Stitz and Jeff Zeager\\
Edited by Sean Fitzpatrick}

\begin{document}
\maketitle


\renewcommand{\headrulewidth}{0pt}
\renewcommand{\headheight}{14pt}
\lhead[\fancyplain{}{\sc\thepage}]%
      {\fancyplain{}{\sc \nouppercase{\rightmark}}}
\rhead[\fancyplain{}{\sc \nouppercase{\leftmark}}]%
      {\fancyplain{}{\sc\thepage}}
\cfoot{}


Now that we have reviewed the basics of polynomial arithmetic it's time to review the basic techniques of factoring polynomial expressions.  Our goal is to apply these techniques to help us solve certain specialized classes of non-linear equations.  Given that `factoring' literally means to resolve a product into its factors, it is, in the purest sense, `undoing' multiplication.  If this sounds like division to you then you've been paying attention.  Let's start with a numerical example.  

\smallskip

Suppose we are asked to factor $16337$.  We could write $16337 = 16337 \cdot 1$, and while this is technically a factorization of $16337$,  it's probably not an answer the poser of the question would accept.  Usually, when we're asked to factor a natural number, we are being asked to resolve it into to a product of so-called 	`prime' numbers.\footnote{This is possible (up to ordering) in only one way, thanks to the \href{https://en.wikipedia.org/wiki/Fundamental_theorem_of_arithmetic}{\underline{Fundamental Theorem of Arithmetic}.}}  Recall that \textbf{prime numbers} are defined as natural numbers whose only (natural number) factors are themselves and $1$. They are, in essence, the `building blocks' of natural numbers as far as multiplication is concerned.  Said differently, we can build - via multiplication - any natural number given enough primes.  So how do we find the prime factors of $16337$?  We start by dividing each of the primes: $2$, $3$, $5$, $7$, etc., into $16337$ until we get a remainder of $0$.  Eventually, we find that $16337 \div 17 = 961$ with a remainder of $0$, which means $16337 = 17 \cdot 961$.  So factoring and division are indeed closely related - factors of a number are precisely the divisors of that number which produce a zero remainder.  We continue our efforts to see if $961$ can be factored down further, and we find that $961 = 31 \cdot 31$.  Hence, $16337$ can be `completely factored' as $17 \cdot 31^2$.  (This factorization is called the \textbf{prime factorization} of $16337$.)  

\smallskip

In factoring natural numbers, our building blocks are prime numbers, so to be completely factored means that every number used in the factorization of a given number is prime. One of the challenges when it comes to factoring polynomial expressions is to explain what it means to be `completely factored'. In this section, our `building blocks' for factoring polynomials are `irreducible' polynomials as defined below.

\medskip

\colorbox{ResultColor}{\bbm \begin{defn}\label{irreduciblepoly}  A polynomial is said to be \textbf{irreducible} if it cannot be written as the product of polynomials of lower degree.
\end{defn}
\ebm}

\medskip

While Definition \ref{irreduciblepoly} seems straightforward enough, sometimes a greater level of specificity is required. For example, $x^2 - 3 = (x-\sqrt{3})(x + \sqrt{3})$.  While $x-\sqrt{3}$ and $x+\sqrt{3}$ are perfectly fine polynomials, factoring which requires irrational numbers is usually saved for a more advanced treatment of factoring.  For now, we will restrict ourselves to factoring using rational coefficients. So, while the polynomial $x^2 - 3$ can be factored using irrational numbers, it is called irreducible \textbf{over the rationals}, since there are no polynomials with \textit{rational} coefficients of smaller degree which can be used to factor it.\footnote{If this isn't immediately obvious, don't worry - in some sense, it shouldn't be.  We'll talk more about this later.} 

\medskip

Since polynomials involve terms, the first step in any factoring strategy involves pulling out factors which are common to all of the terms.  For example, in the polynomial $18x^2y^3 - 54x^3y^2 - 12xy^2$,  each coefficient is a multiple of $6$ so we can begin the factorization as $6(3x^2y^3 - 9x^3y^2 - 2xy^2)$.  The remaining coefficients: $3$, $9$ and $2$, have no common factors so $6$ was the greatest common factor.  What about the variables? Each term contains an $x$, so we can factor an $x$ from each term.  When we do this, we are effectively dividing each term by $x$ which means the exponent on  $x$ in each term is reduced by $1$:  $6x(3xy^3 - 9x^2y^2 - 2y^2)$.  Next, we see that each term has a factor of $y$ in it.  In fact, each term has at least \textit{two} factors of $y$ in it, since the lowest exponent on $y$ in each term is $2$. This means that we can factor $y^2$ from each term. Again, factoring out $y^2$ from each term is tantamount to dividing each term by $y^2$ so the exponent on $y$ in each term is reduced by \textit{two}:  $6xy^2(3xy - 9x^2 - 2)$.  Just like we checked our division by multiplication in the previous section, we can check our factoring here by multiplication, too.  $6xy^2(3xy - 9x^2 - 2) = (6xy^2)(3xy) - (6xy^2)(9x^2) - (6xy^2)(2) = 18x^2y^3 - 54x^3y^2  - 12xy^2 \, \checkmark$.  We summarize how to find the Greatest Common Factor (G.C.F.) of a polynomial expression below.

\medskip

\phantomsection
\label{PolynomialGCF}
\colorbox{ResultColor}{\bbm
\centerline{\textbf{Finding the G.C.F. of a Polynomial Expression}}

\begin{itemize}

\item If the coefficients are integers, find the G.C.F. of the coefficients. 

\textbf{NOTE 1:}  If all of the coefficients are negative, consider the negative as part of the G.C.F..

\textbf{NOTE 2:}  If the coefficients involve fractions, get a common denominator, combine numerators, reduce to lowest terms and apply this step to the polynomial in the numerator.

\item  If a variable is common to all of the terms, the G.C.F. contains that variable to the smallest exponent which appears among the terms.

\end{itemize}

\ebm}

\smallskip

For example, to factor $-\frac{3}{5}z^3 - 6z^2$, we would first get a common denominator and factor as: \[ -\frac{3}{5}z^3 - 6z^2 = \frac{-3z^3 - 30z^2}{5} = \frac{-3z^2(z + 10)}{5} = -\frac{3z^2(z + 10)}{5}\]

We now list some common factoring formulas, each of which can be verified by multiplying out the right side of the equation.  While they all should look familiar - this is a review section after all - some should look more familiar than others since they appeared as `special product' formulas in the previous section.  

\medskip

\phantomsection
\label{CommonFactoringFormulas}
\colorbox{ResultColor}{\bbm
\centerline{\textbf{Common Factoring Formulas}}

\begin{itemize}

\item \textbf{Perfect Square Trinomials:}  $a^2 + 2ab + b^2 = (a+b)^2$ and $a^2 - 2ab + b^2 = (a-b)^2$
\item  \textbf{Difference of Two Squares:}  $a^2 - b^2 = (a-b)(a+b)$

\textbf{NOTE:}  In general, the sum of squares, $a^2 + b^2$ is irreducible over the rationals.

\item  \textbf{Sum of Two Cubes:}  $a^3 + b^3 = (a + b)(a^2 - ab + b^2)$

\textbf{NOTE:}  In general, $a^2 - ab + b^2$ is irreducible over the rationals.

\item   \textbf{Difference of Two Cubes:}  $a^3 - b^3 = (a - b)(a^2 + ab + b^2)$ 

\textbf{NOTE:}  In general, $a^2 + ab + b^2$ is irreducible over the rationals.
\end{itemize}

\ebm}

\medskip

Our next example gives us practice with these formulas.

\begin{ex}\label{FormulaFactoring}  Factor the following polynomials completely over the rationals.  That is, write each polynomial as a product polynomials of lowest degree which are irreducible over the rationals. 

\begin{multicols}{3}

\begin{enumerate}

\item  $18x^2 - 48x + 32$  

\item  $64y^2 - 1$ 

\item  $75t^4 + 30t^3 + 3t^2$


\setcounter{HW}{\value{enumi}}

\end{enumerate}

\end{multicols}

\begin{multicols}{3}

\begin{enumerate}

\setcounter{enumi}{\value{HW}}


\item  $w^4 z - w z^4$

\item  \label{quadinform1} $81 - 16t^4$ 

\item  \label{quadinform2} $x^6 - 64$

\setcounter{HW}{\value{enumi}}

\end{enumerate}

\end{multicols}

{\bf Solution.}

\begin{enumerate}
\item  Our first step is to factor out the G.C.F. which in this case is $2$.  To match what is left with one of the special forms, we rewrite $9x^2 = (3x)^2$ and $16 = 4^2$. Since the `middle' term is $-24x = -2(4)(3x)$, we see that we have a perfect square trinomial.\[ \begin{array}{rclr}

18x^2 - 48x + 32 & = & 2(9x^2 - 24x + 16) & \text{Factor out G.C.F.}\\
                 & = & 2((3x)^2 - 2(4)(3x) + (4)^2) & \\
								 & = & 2(3x-4)^2 & \text{Perfect Square Trinomial:  $a = 3x$, $b=4$} \\ \end{array}\]Our final answer is $2(3x-4)^2$.  To check, we multiply out $2(3x-4)^2$ to show that it equals $18x^2 - 48x + 32$.
 
\item  For $64y^2 - 1$, we note that the G.C.F. of the terms is just $1$, so there is nothing (of substance) to factor out of both terms. Since $64y^2 - 1$ is the difference of two terms, one of which is a square, we look to the Difference of Squares Formula for inspiration.   By identifying $64y^2 = (8y)^2$ and $1 = 1^2$, we get \[ \begin{array}{rclr} 

64y^2 - 1 & = & (8y)^2 - 1^2 & \\
          & = & (8y-1)(8y+1) & \text{Difference of Squares, $a = 8y$, $b = 1$} \end{array} \] As before, we can check our answer by multiplying out $(8y-1)(8y+1)$ to show that it equals $64y^2 - 1$.

\item  The G.C.F. of the terms in $75t^4 + 30t^3 + 3t^2$ is $3t^2$, so we factor that out first.  We identify what remains as a perfect square trinomial:\[ \begin{array}{rclr}
 
75t^4 + 30t^3 + 3t^2 & = & 3t^2 (25t^2+10t + 1) & \text{Factor out G.C.F.}\\
                     & = & 3t^2 ((5t)^2 + 2(1)(5t) + 1^2) & \\
                     & = & 3t^2 (5t+1)^2 & \text{Perfect Square Trinomial, $a = 5t$, $b = 1$} \\

\end{array}\] Our final answer is $3t^2 (5t+1)^2$, which the reader is invited to check.

\item  For $w^4 z - w z^4$, we identify the G.C.F. as $wz$ and once we factor it out a difference of cubes is revealed: \[ \begin{array}{rclr}

w^4 z - w z^4 & = & wz(w^3 - z^3) & \text{Factor out G.C.F.} \\
              & = & wz(w-z)(w^2+wz+z^2) & \text{Difference of Cubes, $a=w$, $b = z$} \\
							\end{array} \] Our final answer is $wz(w-z)(w^2+wz+z^2)$.  The reader is strongly encouraged to multiply this out to see that it reduces to $w^4 z - w z^4$.

\item  The G.C.F. of the terms in $81 - 16t^4$ is just $1$ so there is nothing of substance to factor out from both terms.  With just a difference of two terms, we are limited to fitting this polynomial into either the Difference of Two Squares or Difference of Two Cubes formula. Since the variable here is $t^4$, and $4$ is a multiple of $2$, we can think of $t^4 = (t^2)^2$.  This means that we can write $16t^4 = (4t^2)^2$ which is a perfect square.  (Since $4$ is not a multiple of $3$, we cannot write $t^4$ as a perfect cube of a polynomial.)  Identifying $81 = 9^2$ and $16t^4 = (4t^2)^2$, we apply the Difference of Squares Formula to get: \[ \begin{array}{rclr}

81 - 16t^4 & = & 9^2 - (4t^2)^2 & \\
           & = & (9-4t^2)(9+4t^2) & \text{Difference of Squares, $a = 9$, $b = 4t^2$} \\ \end{array}\] At this point, we have an opportunity to proceed further.  Identifying $9 = 3^2$ and $4t^2 = (2t)^2$, we see that we have another difference of squares in the first quantity, which we can reduce. (The sum of two squares in the second quantity \underline{cannot} be factored over the rationals.) \[ \begin{array}{rclr}
81 - 16t^4 & = & (9-4t^2)(9+4t^2) & \\
           & = & (3^2 - (2t)^2) (9 + 4t^2) & \\
					 & = & (3 - 2t)(3+2t)(9 + 4t^2) & \text{Difference of Squares, $a = 3$, $b = 2t$} \\ \end{array} \] As always, the reader is encouraged to multiply out $(3 - 2t)(3+2t)(9 + 4t^2)$ to check the result.

\item  With a G.C.F. of $1$ and just two terms, $x^6 - 64$ is a candidate for both the Difference of Squares and the Difference of Cubes formulas.  Notice that we can identify $x^6 = (x^3)^2$ and $64 = 8^2$ (both perfect squares), but also $x^6 = (x^2)^3$ and $64 = 4^3$ (both perfect cubes).  If we follow the Difference of Squares approach, we get: \[ \begin{array}{rclr}

x^6 - 64 & = & (x^3)^2 - 8^2 & \\
         & = & (x^3 - 8)(x^3 + 8) & \text{Difference of Squares, $a = x^3$ and $b = 8$} \\ \end{array} \] At this point, we have an opportunity to use both the Difference and Sum of Cubes formulas: \[ \begin{array}{rclr}

x^6 - 64 & = & (x^3 - 2^3)(x^3 + 2^3) & \\
         & = & (x-2)(x^2+2x+2^2)(x+2)(x^2 - 2x + 2^2) & \text{Sum / Difference of Cubes, $a = x$, $b = 2$} \\ 
				 & = & (x-2)(x+2)(x^2-2x+4)(x^2+2x+4) & \text{Rearrange factors} \\ 
\end{array} \] From this approach, our final answer is $(x-2)(x+2)(x^2-2x+4)(x^2+2x+4)$.  

\smallskip

Following the Difference of Cubes Formula approach, we get \[ \begin{array}{rclr}

x^6 - 64 & = & (x^2)^3 - 4^3 & \\
         & = & (x^2 - 4)((x^2)^2 + 4x^2 + 4^2) & \text{Difference of Cubes, $a = x^2$, $b = 4$} \\ 
         & = & (x^2 - 4)(x^4 + 4x^2 + 16) & \\ 
\end{array} \] At this point, we recognize $x^2 - 4$ as a difference of two squares: \[ \begin{array}{rclr}

x^6 - 64 & = & (x^2 - 2^2)(x^4 + 4x^2 + 16)  & \\
         & = & (x-2)(x+2)(x^4 + 4x^2 + 16) & \text{Difference of Squares, $a = x$, $b = 2$} \\ 
 
\end{array} \]

Unfortunately, the remaining factor $x^4 + 4x^2 + 16$ is not a perfect square trinomial - the middle term would have to be $8x^2$ for this to work - so our final answer using this approach is $(x-2)(x+2)(x^4 + 4x^2 + 16)$.   This isn't as factored as our result from the Difference of Squares approach which was $(x-2)(x+2)(x^2-2x+4)(x^2+2x+4)$.  While it is true that $x^4 + 4x^2 + 16 = (x^2-2x+4)(x^2+2x+4)$, there is no `intuitive' way to motivate this factorization at this point.\footnote{Of course, this begs the question, ``How do we know $x^2-2x+4$ and $x^2+2x+4$ are irreducible?'' (We were told so on page \pageref{CommonFactoringFormulas}, but no reason was given.)  Stay tuned!  We'll get back to this in due course.}  The moral of the story?  When given the option between using the Difference of Squares and Difference of Cubes, start with the Difference of Squares.  Our final answer to this problem is  $(x-2)(x+2)(x^2-2x+4)(x^2+2x+4)$.  The reader is strongly encouraged to show that this reduces down to $x^6 - 64$ after performing all of the  multiplication.\qed

\end{enumerate}
\end{ex}

The formulas on page \pageref{CommonFactoringFormulas}, while useful, can only take us so far, so we need to review some more advanced factoring strategies. 

\medskip

\phantomsection
\label{AdvancedReviewFactoring}

\colorbox{ResultColor}{\bbm
\centerline{\textbf{Advanced Factoring Formulas}}

\begin{itemize}

\item  \textbf{`un-F.O.I.L.ing':} Given a trinomial $Ax^2 + Bx + C$, try to reverse the F.O.I.L. process.  

That is, find  $a$, $b$, $c$ and $d$ such that $Ax^2 + Bx + C= (ax+b)(cx+d)$.  

\textbf{NOTE:}  This means $ac = A$, $bd = C$ and $B = ad+bc$.

\item \textbf{Factor by Grouping:} If the expression contains four terms with no common factors among the four terms, try `factor by grouping': \[ac + bc + ad + bd = (a +b)c + (a+b)d = (a+b)(c+d)\]

\end{itemize}

\ebm}

\medskip

The techniques of `un-F.O.I.L.ing' and `factoring by grouping' are difficult to describe in general but should make sense to you with enough practice.  Be forewarned - like all `Rules of Thumb', these strategies work just often enough to be useful, but you can be sure there are exceptions which will defy any advice given here and will require some `inspiration' to solve. Even though the Math 1010 textbook will give us more powerful factoring methods, we'll find that, in the end, there is no single algorithm for factoring which works for every polynomial. In other words, there will be times when you just have to try something and see what happens.

\begin{ex}\label{advfactoring}  Factor the following polynomials completely over the integers.\footnote{This means that all of the coefficients in the factors will be integers. In a rare departure from form, Carl decided to avoid fractions in this set of examples.  Don't get complacent, though, because fractions will return with a vengeance soon enough.}

\enlargethispage{.1in}

\begin{multicols}{3}

\begin{enumerate}

\item  $x^2 - x - 6$  

\item  $2t^2 - 11t + 5$

\item  $36 - 11y - 12y^2$


\setcounter{HW}{\value{enumi}}

\end{enumerate}

\end{multicols}

\vspace*{-.3in}

\begin{multicols}{3}

\begin{enumerate}

\setcounter{enumi}{\value{HW}}


\item  $18xy^2 - 54xy - 180x$

\item  $2t^3 - 10t^2 + 3t - 15$ 

\item  $x^4 + 4x^2 + 16$

\setcounter{HW}{\value{enumi}}

\end{enumerate}

\end{multicols}

\pagebreak

{\bf Solution.}

\begin{enumerate}

\item  The G.C.F. of the terms $x^2 - x - 6$  is $1$ and $x^2 - x - 6$ isn't a perfect square trinomial (Think about why not.) so we try to reverse the F.O.I.L. process and look for integers $a$, $b$, $c$ and $d$ such that $(ax + b)(cx + d) = x^2 - x - 6$.  To get started, we note that $ac = 1$.  Since $a$ and $c$ are meant to be integers, that leaves us with either $a$ and $c$ both being $1$, or $a$ and $c$ both being $-1$.  We'll go with $a = c = 1$, since we can factor\footnote{Pun intended!} the negatives into our choices for $b$ and $d$.  This yields $(x+b)(x+d) = x^2-x-6$.  Next, we use the fact that $bd = -6$.  The product is negative so we know that one of $b$ or $d$ is positive and the other is negative.  Since $b$ and $d$ are integers, one of $b$ or $d$ is $\pm 1$ and the other is $\mp 6$ OR one of $b$ or $d$ is $\pm 2$ and the other is $\mp 3$. After some guessing and checking,\footnote{The authors have seen some strange gimmicks that allegedly help students with this step.  We don't like them so we're sticking with good old-fashioned guessing and checking.} we find that $x^2 - x - 6 = (x+2)(x-3)$.

\item As with the previous example, we check the G.C.F. of the terms in $2t^2 - 11t + 5$, determine it to be $1$ and see that the polynomial doesn't fit the pattern for a perfect square trinomial.  We now try to find integers $a$, $b$, $c$ and $d$ such that $(at+b)(ct+d) = 2t^2 - 11t + 5$.  Since $ac = 2$, we have that one of $a$ or $c$ is $2$, and the other is $1$. (Once again, we ignore the negative options.)  At this stage, there is nothing really distinguishing $a$ from $c$ so we choose $a = 2$ and $c = 1$.  Now we look for $b$ and $d$ so that $(2t + b)(t+d) = 2t^2 - 11t + 5$.  We know $bd = 5$ so one of $b$ or $d$ is $\pm 1$ and the other $\pm 5$. Given that $bd$ is positive, $b$ and $d$ must have the same sign.  The negative middle term $-11t$ guides us to guess $b = -1$ and $d = -5$ so that we get $(2t -1)(t -5) = 2t^2 - 11t + 5$.  We verify our answer by multiplying.\footnote{That's the `checking' part of 'guessing and checking'.}

\item  Once again, we check for a nontrivial G.C.F. and see if $36 - 11y - 12y^2$ fits the pattern of a perfect square.  Twice disappointed, we rewrite $36 - 11y - 12y^2 = -12y^2 - 11y + 36$ for notational convenience.  We now look for integers $a$, $b$, $c$ and $d$ such that $-12y^2 - 11y + 36 = (ay + b)(cy + d)$.  Since $ac =-12$, we know that one of $a$ or $c$ is $\pm 1$ and the other $\pm 12$ OR one of them is $\pm 2$ and the other is $\pm 6$ OR one of them is $\pm 3$ while the other is $\pm 4$. Since their product is $-12$, however, we know one of them is positive, while the other is negative.   To make matters worse, the constant term $36$ has its fair share of factors, too.  Our answers for $b$ and $d$ lie among the pairs $\pm 1$ and $\pm 36$, $\pm 2$ and $\pm 18$, $\pm 4$ and $\pm 9$, or $\pm 6$.  Since we know one of $a$ or $c$ will be negative, we can simplify our choices for $b$ and $d$ and just look at the positive possibilities.  After some guessing and checking,\footnote{Some of these guesses can be more `educated' than others.  Since the middle term is relatively `small,' we don't expect the `extreme' factors of $36$ and $12$ to appear, for instance.} we find $(-3y + 4)(4y+9) = -12y^2 - 11y + 36$.

\item  Since the G.C.F. of the terms in $18xy^2 - 54xy - 180x$  is $18x$, we begin the problem by factoring it out first:  $18xy^2 - 54xy - 180x = 18x(y^2 - 3y - 10)$.  We now focus our attention on $y^2 - 3y - 10$.  We can take $a$ and $c$ to both be $1$ which yields $(y+b)(y+d) = y^2 - 3y - 10$.  Our choices for $b$ and $d$ are among the factor pairs of $-10$: $\pm 1$ and $\pm 10$ or $\pm 2$ and $\pm 5$, where one of $b$ or $d$ is positive and the other is negative.  We find $(y-5)(y+2) = y^2 - 3y - 10$.  Our final answer is $18xy^2 - 54xy - 180x = 18x(y-5)(y+2)$.

\item Since $2t^3 - 10t^2 - 3t + 15$  has four terms, we are pretty much resigned to factoring by grouping.  The strategy here is to factor out the G.C.F. from two \textit{pairs} of terms, and see if this reveals a common factor. If we group the first two terms, we can factor out a $2t^2$ to get $2t^3 - 10t^2 = 2t^2(t-5)$.  We now try to factor something out of the last two terms that will leave us with a factor of $(t-5)$.  Sure enough, we can factor out a $-3$ from both:  $-3t + 15 = -3(t-5)$.  Hence, we get \[ 2t^3 - 10t^2 - 3t + 15 = 2t^2(t-5) - 3(t-5) = (2t^2-3)(t-5)\] Now the question becomes can we factor $2t^2 - 3$ over the integers?  This would require integers $a, b, c$ and $d$ such that $(at + b)(ct + d) = 2t^2 - 3$.   Since $ab = 2$ and $cd = -3$, we aren't left with many options - in fact, we really have only four choices:  $(2t - 1)(t+3)$, $(2t+1)(t-3)$, $(2t - 3)(t+1)$ and $(2t+3)(t-1)$.  None of these produces $2t^2 - 3$ - which means it's irreducible over the integers - thus our final answer is $(2t^2-3)(t-5)$.

\item  Our last example, $x^4 + 4x^2 + 16$, is our old friend from Example \ref{FormulaFactoring}.  As noted there, it is not a perfect square trinomial, so we could try to reverse the F.O.I.L. process. This is complicated by the fact that our highest degree term is $x^4$, so we would have to look at factorizations of the form $(x+b)(x^3+d)$ as well as $(x^2 + b)(x^2 + d)$.  We leave it to the reader to show that neither of those work.  This is an example of where `trying something' pays off.  Even though we've stated that it is not a perfect square trinomial, it's pretty close.  Identifying $x^4 = (x^2)^2$ and $16 = 4^2$, we'd have $(x^2 + 4)^2 = x^4 + 8x^2 + 16$, but instead of $8x^2$ as our middle term, we only have $4x^2$. We could add in the extra $4x^2$ we need, but to keep the balance, we'd have to subtract it off.  Doing so produces and unexpected opportunity: \[ \begin{array}{rclr}

x^4 + 4x^2 + 16 & = & x^4 + 4x^2 + 16 + (4x^2 - 4x^2) & \text{Adding and subtracting the same term} \\
                & = & x^4 + 8x^2 + 16 - 4x^2 & \text{Rearranging terms} \\
                & = & (x^2 + 4)^2 - (2x)^2 & \text{Factoring perfect square trinomial} \\
								& = & [(x^2 +4) - 2x][ (x^2 + 4) + 2x] & \text{Difference of Squares:  $a= (x^2 + 4)$, $b = 2x$}\\
								& = & (x^2 - 2x + 4)(x^2 + 2x + 4) & \text{Rearraging terms} \\

\end{array}\]

We leave it to the reader to check that neither $x^2 - 2x + 4$ nor $x^2 + 2x + 4$ factor over the integers, so we are done. \qed

\end{enumerate}

\end{ex}

\subsection{Solving Equations by Factoring}
\label{solveeqnsbyfactoring}

Many students wonder why they are forced to learn how to factor.  Simply put, factoring is our main tool for solving the non-linear equations which arise in many of the applications of Mathematics.\footnote{Also known as `story problems' or `real-world examples'.}  We use factoring in conjunction with the Zero Product Property of Real Numbers:

\medskip
 
\colorbox{ResultColor}{\bbm

\textbf{The Zero Product Property of Real Numbers:}  If $a$ and $b$ are real numbers with $ab = 0$ then either $a = 0$ or $b = 0$ or both.

\ebm}

\medskip

For example, consider the equation $6x^2 + 11x = 10$.  To see how the Zero Product Property is used to help us solve this equation, we first set the equation equal to zero and then apply the techniques from Example \ref{advfactoring}: \[ \begin{array}{rclr}
6x^2 + 11x & = & 10 \\
6x^2 + 11x - 10 & = & 0 & \text{Subtract $10$ from both sides} \\
(2x+5)(3x-2) & = & 0 & \text{Factor} \\
2x +5 = 0 & \text{or} & 3x -2 = 0 & \text{Zero Product Property} \\ [-2pt]
          &           &           & a = 2x+5, b = 3x-2 \\ [-8pt]
x = -\frac{5}{2} & \text{or} & x = \frac{2}{3} & \\ \end{array} \] The reader should check that both of these solutions satisfy the original equation.

\smallskip

It is critical that you see the importance of setting the expression equal to $0$ before factoring. Otherwise, we'd get: \[ \begin{array}{rclr}

6x^2 + 11x & = & 10 \\
x(6x + 11) & = & 10 & \text{Factor} \\
\end{array} \] What we \textbf{cannot} deduce from this equation is that $x = 10$ or $6x+11 = 10$ or that $x = 2$ and $6x+11 = 5$, etc..  (It's wrong and you should feel bad if you do it.)  It is precisely because $0$ plays such a special role in the arithmetic of real numbers (as the Additive Identity) that we can assume a factor is $0$ when the product is $0$.  No other real number has that ability.

\smallskip

We summarize the {\bf correct} equation solving strategy below.

\medskip

\phantomsection
\label{solvenonlineareqns}

\colorbox{ResultColor}{\bbm
\centerline{\textbf{Strategy for Solving Non-linear Equations}}

\begin{enumerate}

\item  Put all of the nonzero terms on one side of the equation so that the other side is $0$.
\item  Factor.
\item  Use the Zero Product Property of Real Numbers and set each factor equal to $0$.
\item  Solve each of the resulting equations.

\end{enumerate}

\ebm}

\medskip

Let's finish the section with a collection of examples in which we use this strategy.

\begin{ex}\label{solveeqnbyfactoring}  Solve the following equations.

\begin{multicols}{3}

\begin{enumerate}

\item  $3x^2 = 35 - 16x$\vphantom{$t = \dfrac{1 + 4t^2}{4}$}

\item  $t = \dfrac{1+4t^2}{4}$

\item  $(y-1)^2 = 2(y-1)$\vphantom{$t = \dfrac{1 + 4t^2}{4}$}

\setcounter{HW}{\value{enumi}}

\end{enumerate}

\end{multicols}

\begin{multicols}{3}

\begin{enumerate}

\setcounter{enumi}{\value{HW}}

\item  $\dfrac{w^4}{3} = \dfrac{8w^3-12}{12} - \dfrac{w^2-4}{4}$

\item  $z(z(18z+9)-50) = 25$\vphantom{$\dfrac{w^4}{3} = \dfrac{8w^3-12}{12} - \dfrac{w^2-4}{4}$}

\item  $x^4-8x^2 - 9 = 0$\vphantom{$\dfrac{w^4}{3} = \dfrac{8w^3-12}{12} - \dfrac{w^2-4}{4}$}

\setcounter{HW}{\value{enumi}}

\end{enumerate}

\end{multicols}

\pagebreak

{\bf Solution.}

\begin{enumerate}

\item  We begin by gathering all of the nonzero terms to one side getting $0$ on the other and then we proceed to factor and apply the Zero Product Property. \[ \begin{array}{rclr}

3x^2 & = & 35 - 16x & \\

3x^2 + 16x - 35 & = & 0 & \text{Add $16x$, subtract $35$} \\

(3x-5)(x+7) & = & 0 & \text{Factor} \\

3x-5 = 0 & \text{or} & x+7 = 0 & \text{Zero Product Property} \\

x = \frac{5}{3} & \text{or} & x = -7 & \\ 

\end{array} \]

We check our answers by substituting each of them into the original equation.  Plugging in $x = \frac{5}{3}$ yields $\frac{25}{3}$ on both sides while $x = -7$ gives $147$ on both sides.

\item To solve $t = \frac{1+4t^2}{4}$, we first clear fractions then move all of the nonzero terms to one side of the equation, factor and apply the Zero Product Property.\[ \begin{array}{rclr}

t & = & \dfrac{1+4t^2}{4} & \\

4t & = & 1+4t^2 & \text{Clear fractions (multiply by $4$)} \\

0 & = & 1+4t^2 - 4t & \text{Subtract 4} \\

0 & = & 4t^2 - 4t + 1 & \text{Rearrange terms} \\

0 & = & (2t-1)^2 & \text{Factor (Perfect Square Trinomial)} \\

\end{array} \]

At this point, we get $(2t-1)^2 = (2t-1)(2t-1) = 0$, so, the Zero Product Property gives us $2t-1 =0$ in both cases.\footnote{More generally, given a positive power $p$,  the only solution to $X^p = 0$ is $X = 0$.}  Our final answer is $t = \frac{1}{2}$, which we invite the reader to check.

\item  Following the strategy outlined above, the first step to solving $(y-1)^2 = 2(y-1)$ is to gather the nonzero terms on one side of the equation with $0$ on the other side and factor.\[ \begin{array}{rclr}

(y-1)^2 & = & 2(y-1) & \\

(y-1)^2 - 2(y-1) & = & 0 & \text{Subtract $2(y-1)$} \\
(y-1)[(y-1) - 2] & = & 0 & \text{Factor out G.C.F.} \\
(y-1)(y-3) & = & 0 &  \text{Simplify} \\
y-1 = 0 & \text{or} & y - 3 = 0 & \\

y = 1 & \text{or} & y = 3 & \\  \end{array} \] Both of these answers are easily checked by substituting them into the original equation.  

\smallskip

An alternative method to solving this equation is to begin by dividing both sides by $(y-1)$ to simplify things outright.  However, whenever we divide by a variable quantity, we must make the explicit assumption that this quantity is nonzero.  Thus we must stipulate that $y - 1 \neq 0$.\[ \begin{array}{rclr}

\dfrac{(y-1)^2}{(y-1)} & = & \dfrac{2(y-1)}{(y-1)} & \text{Divide by $(y-1)$ - this assumes $(y-1) \neq 0$}\\
y - 1 & = & 2 & \\
y & = & 3 & \\  \end{array} \] Note that in this approach, we obtain the $y=3$ solution, but we `lose' the $y = 1$ solution. How did that happen?  Assuming $y - 1 \neq 0$ is equivalent to assuming $y \neq 1$.  This is an issue because $y = 1$ is a solution to the original equation and it was `divided out' too early.  The moral of the story?  If you decide to divide by a variable expression, double check that you aren't excluding any solutions.\footnote{You will see other examples throughout your textbook where dividing by a variable quantity does more harm than good.  Keep this basic one in mind as you move on in your studies - it's a good cautionary tale.}

\item Proceeding as before, we clear fractions, gather the nonzero terms on one side of the equation, have $0$ on the other and factor.\[ \begin{array}{rclr}

\dfrac{w^4}{3} & = & \dfrac{8w^3-12}{12} - \dfrac{w^2-4}{4} & \\ [5pt]

12 \left(\dfrac{w^4}{3}\right) & = & 12 \left(\dfrac{8w^3-12}{12} - \dfrac{w^2-4}{4} \right) & \text{Multiply by $12$}\\[8pt]

4w^4 & = & (8w^3 - 12) - 3(w^2-4) & \text{Distribute} \\
4w^4 & = & 8w^3 - 12 - 3w^2 + 12 & \text{Distribute} \\

0 & = & 8w^3 - 12 - 3w^2 + 12  - 4w^4 & \text{Subtract $4w^4$} \\
0 & = & 8w^3  - 3w^2 - 4w^4 & \text{Gather like terms} \\
0 & = & w^2(8w - 3 - 4w^2) & \text{Factor out G.C.F.} \\ \end{array} \] At this point, we  apply the Zero Product Property to deduce that $w^2 = 0$ or $8w - 3 - 4w^2 = 0$. From $w^2 = 0$, we get $w = 0$. To solve $8w - 3 - 4w^2 = 0$, we rearrange terms and factor:  $-4w^2 + 8w - 3= (2w - 1)(-2w+3) = 0$.  Applying the Zero Product Property again, we get $2w - 1= 0$ (which gives $w = \frac{1}{2}$), or $-2w+3 = 0$ (which gives $w = \frac{3}{2}$).  Our final answers are $w = 0$, $w = \frac{1}{2}$ and $w = \frac{3}{2}$.  The reader is encouraged to check each of these answers in the original equation.  (You need the practice with fractions!)

\item  For our next example, we begin by subtracting the $25$ from both sides then work out the indicated operations before factoring by grouping.\[ \begin{array}{rclr}

z(z(18z+9)-50) & = & 25 & \\

z(z(18z+9)-50) - 25 & = & 0 & \text{Subtract $25$} \\
z(18z^2 + 9z - 50) - 25 & = & 0 & \text{Distribute} \\
18z^3 + 9z^2 - 50z - 25 & = & 0 & \text{Distribute} \\

9z^2(2z + 1) - 25(2z + 1) & = & 0 & \text{Factor} \\

(9z^2 - 25)(2z+1) & = & 0 & \text{Factor} \\ \end{array} \]

At this point, we use the Zero Product Property and get $9z^2 - 25 = 0$ or $2z + 1 = 0$.  The latter gives $z = -\frac{1}{2}$ whereas the former  factors as $(3z - 5)(3z+5) = 0$.  Applying the Zero Product Property again gives $3z-5 = 0$ (so $z = \frac{5}{3}$) or $3z+5 = 0$ (so $z = -\frac{5}{3}$.) Our final answers are $z = -\frac{1}{2}$,  $z = \frac{5}{3}$ and $z = -\frac{5}{3}$, each of which good fun to check.

\item  The nonzero terms of the equation $x^4-8x^2 - 9= 0$ are already on one side of the equation so we proceed to factor.  This trinomial doesn't fit the pattern of a perfect square so we attempt to reverse the F.O.I.L.ing process.  With an $x^4$ term, we have two possible forms to try:  $(ax^2 + b)(cx^2 + d)$ and $(ax^3 + b)(cx +d)$.  We leave it to you to show that $(ax^3 + b)(cx +d)$ does not work and we show that  $(ax^2 + b)(cx^2 + d)$ does. 

\smallskip

Since the coefficient of $x^4$ is $1$, we take $a = c = 1$.  The constant term is $-9$ so we know $b$ and $d$ have opposite signs and our choices are limited to two options: either $b$ and $d$ come from $\pm 1$ and $\pm 9$ OR one is $3$ while the other is $-3$.  After some trial and error, we get $x^4 - 8x^2 - 9 = (x^2 - 9)(x^2+1)$.  Hence $x^4-8x^2 - 9= 0$ reduces to $(x^2 - 9)(x^2 + 1) = 0$.  The Zero Product Property tells us that either $x^2 - 9 = 0$ or $x^2+1 = 0$.  To solve the former, we factor: $(x-3)(x+3) = 0$, so $x-3 = 0$ (hence, $x = 3$) or $x+3 = 0$ (hence, $x = -3$).  The equation $x^2 + 1 = 0$ has no (real) solution, since for any real number $x$, $x^2$ is always $0$ or greater.  Thus $x^2 + 1$ is always positive.  Our final answers are $x = 3$ and $x = -3$. As always, the reader is invited to check both answers in the original equation. \qed

\end{enumerate}

\end{ex}

\newpage

\subsection{Exercises}

In Exercises \ref{factoringexfirst} - \ref{factoringexlast}, factor completely over the integers.  Check your answer by multiplication.

\begin{multicols}{3}
\begin{enumerate}

\item $2x - 10x^2$ \label{factoringexfirst}
\item $12t^5 - 8t^3$
\item $16xy^2 - 12x^2y$

\setcounter{HW}{\value{enumi}}
\end{enumerate}
\end{multicols}

\begin{multicols}{3}
\begin{enumerate}
\setcounter{enumi}{\value{HW}}

\item $5(m+3)^2- 4(m+3)^3$
\item $(2x-1)(x+3) - 4(2x-1)$
\item $t^2(t-5) + t - 5$

\setcounter{HW}{\value{enumi}}
\end{enumerate}
\end{multicols}

\begin{multicols}{3}
\begin{enumerate}
\setcounter{enumi}{\value{HW}}

\item $w^2 - 121$
\item $49 - 4t^2$
\item $81t^4 - 16$

\setcounter{HW}{\value{enumi}}
\end{enumerate}
\end{multicols}

\begin{multicols}{3}
\begin{enumerate}
\setcounter{enumi}{\value{HW}}

\item $9z^2 - 64y^4$
\item $(y+3)^2 - 4y^2$
\item $(x+h)^3 - (x+h)$

\setcounter{HW}{\value{enumi}}
\end{enumerate}
\end{multicols}

\begin{multicols}{3}
\begin{enumerate}
\setcounter{enumi}{\value{HW}}

\item $y^2 - 24y + 144$
\item $25t^2 + 10t + 1$
\item $12x^3 - 36x^2 + 27x$

\setcounter{HW}{\value{enumi}}
\end{enumerate}
\end{multicols}

\begin{multicols}{3}
\begin{enumerate}
\setcounter{enumi}{\value{HW}}

\item $m^4 + 10m^2 + 25$
\item $27 - 8x^3$
\item $t^6 +t^3$


\setcounter{HW}{\value{enumi}}
\end{enumerate}
\end{multicols}



\begin{multicols}{3}
\begin{enumerate}
\setcounter{enumi}{\value{HW}}

\item $x^2 - 5x - 14$
\item $y^2 - 12y + 27$
\item $3t^2 + 16t + 5$


\setcounter{HW}{\value{enumi}}
\end{enumerate}
\end{multicols}


\begin{multicols}{3}
\begin{enumerate}
\setcounter{enumi}{\value{HW}}

\item $6x^2 - 23x + 20$
\item $35+2m - m^2$
\item $7w - 2w^2 - 3$



\setcounter{HW}{\value{enumi}}
\end{enumerate}
\end{multicols}


\begin{multicols}{3}
\begin{enumerate}
\setcounter{enumi}{\value{HW}}

\item $3m^3 + 9m^2 - 12m$
\item $x^4 + x^2 - 20$
\item $4(t^2-1)^2 +3(t^2-1) - 10$


\setcounter{HW}{\value{enumi}}
\end{enumerate}
\end{multicols}

\begin{multicols}{3}
\begin{enumerate}
\setcounter{enumi}{\value{HW}}

\item $x^3 - 5x^2 - 9x + 45$
\item $3t^2 + t - 3 - t^3$
\item\hspace{-0.1in}\footnote{$y^4 + 5y^2 + 9 = (y^4 + 6y^2 + 9) - y^2$} $y^4 + 5y^2 + 9$\label{factoringexlast}


\setcounter{HW}{\value{enumi}}
\end{enumerate}
\end{multicols}

In Exercises \ref{solvebyfactorfirst} - \ref{solvebyfactorlast},  find all rational number solutions.  Check your answers.

\begin{multicols}{3}
\begin{enumerate}
\setcounter{enumi}{\value{HW}}

\item   $(7x+3)(x-5) = 0$ \label{solvebyfactorfirst}
\item   $(2t-1)^2 (t+4) = 0$
\item   $(y^2 + 4)(3y^2 +y - 10) = 0$

\setcounter{HW}{\value{enumi}}
\end{enumerate}
\end{multicols}


\begin{multicols}{3}
\begin{enumerate}
\setcounter{enumi}{\value{HW}}

\item   $4t = t^2$
\item   $y+3 = 2y^2$
\item   $26x = 8x^2 + 21$  

\setcounter{HW}{\value{enumi}}
\end{enumerate}
\end{multicols}


\begin{multicols}{3}
\begin{enumerate}
\setcounter{enumi}{\value{HW}}

\item $16x^4 = 9x^2$
\item $w(6w+11) = 10$
\item $2w^2 + 5w + 2 = - 3(2w+1)$

\setcounter{HW}{\value{enumi}}
\end{enumerate}
\end{multicols}


\begin{multicols}{3}
\begin{enumerate}
\setcounter{enumi}{\value{HW}}

\item $x^2(x-3) = 16(x-3)$
\item $(2t+1)^3 = (2t+1)$
\item $a^4 + 4 = 6 - a^2$

\setcounter{HW}{\value{enumi}}
\end{enumerate}
\end{multicols}

\begin{multicols}{3}
\begin{enumerate}
\setcounter{enumi}{\value{HW}}

\item $\dfrac{8t^2}{3} = 2t+3$\vphantom{$\dfrac{y^4}{3} - y^2 = \dfrac{3}{2} (y^2 + 3)$}
\item $\dfrac{x^3+x}{2} = \dfrac{x^2+1}{3}$\vphantom{$\dfrac{y^4}{3} - y^2 = \dfrac{3}{2} (y^2 + 3)$}
\item $\dfrac{y^4}{3} - y^2 = \dfrac{3}{2} (y^2 + 3)$ \label{solvebyfactorlast}

\setcounter{HW}{\value{enumi}}
\end{enumerate}
\end{multicols}

\begin{enumerate}
\setcounter{enumi}{\value{HW}}

\item  With help from your classmates, factor $4x^4 + 8x^2 + 9$.  

\item  With help from your classmates, find an equation which has $3$, $-\frac{1}{2}$, and $117$ as solutions.  

\setcounter{HW}{\value{enumi}}
\end{enumerate}

\newpage

\subsection{Answers}

\begin{multicols}{3}
\begin{enumerate}

\item $2x(1 - 5x)$ 
\item $4t^3(3t^2-2)$
\item $4xy(4y-3x)$

\setcounter{HW}{\value{enumi}}
\end{enumerate}
\end{multicols}

\begin{multicols}{3}
\begin{enumerate}
\setcounter{enumi}{\value{HW}}

\item $-(m+3)^2(4m+7)$
\item $(2x-1)(x-1)$
\item $(t-5)(t^2+1)$

\setcounter{HW}{\value{enumi}}
\end{enumerate}
\end{multicols}

\begin{multicols}{3}
\begin{enumerate}
\setcounter{enumi}{\value{HW}}

\item $(w-11)(w+11)$
\item $(7-2t)(7+2t)$
\item $(3t-2)(3t+2)(9t^2+4)$

\setcounter{HW}{\value{enumi}}
\end{enumerate}
\end{multicols}

\begin{multicols}{3}
\begin{enumerate}
\setcounter{enumi}{\value{HW}}

\item $(3z-8y^2)(3z+8y^2)$
\item $-3(y - 3)(y+1)$
\item $(x+h)(x+h-1)(x+h+1)$

\setcounter{HW}{\value{enumi}}
\end{enumerate}
\end{multicols}

\begin{multicols}{3}
\begin{enumerate}
\setcounter{enumi}{\value{HW}}

\item $(y-12)^2$
\item $(5t+1)^2$
\item $3x(2x-3)^2$

\setcounter{HW}{\value{enumi}}
\end{enumerate}
\end{multicols}

\begin{multicols}{3}
\begin{enumerate}
\setcounter{enumi}{\value{HW}}

\item $(m^2+5)^2$
\item $(3-2x)(9 + 6x + 4x^2)$
\item $t^3(t+1)(t^2 - t + 1)$


\setcounter{HW}{\value{enumi}}
\end{enumerate}
\end{multicols}



\begin{multicols}{3}
\begin{enumerate}
\setcounter{enumi}{\value{HW}}

\item $(x-7)(x+2)$
\item $(y-9)(y-3)$
\item $(3t+1)(t+5)$


\setcounter{HW}{\value{enumi}}
\end{enumerate}
\end{multicols}


\begin{multicols}{3}
\begin{enumerate}
\setcounter{enumi}{\value{HW}}

\item $(2x-5)(3x-4)$
\item $(7-m)(5+m)$
\item $(-2w+1)(w-3)$



\setcounter{HW}{\value{enumi}}
\end{enumerate}
\end{multicols}


\begin{multicols}{3}
\begin{enumerate}
\setcounter{enumi}{\value{HW}}

\item $3m(m-1)(m+4)$
\item $(x-2)(x+2)(x^2+5)$
\item $(2t-3)(2t+3)(t^2+1)$


\setcounter{HW}{\value{enumi}}
\end{enumerate}
\end{multicols}

\begin{multicols}{3}
\begin{enumerate}
\setcounter{enumi}{\value{HW}}

\item $(x-3)(x+3)(x-5)$
\item $(t-3)(1-t)(1+t)$
\item $(y^2-y+3)(y^2+y+3)$


\setcounter{HW}{\value{enumi}}
\end{enumerate}
\end{multicols}



\begin{multicols}{3}
\begin{enumerate}
\setcounter{enumi}{\value{HW}}

\item   $x = -\dfrac{3}{7}$ or $x = 5$ 
\item   $t = \dfrac{1}{2}$ or $t = -4$
\item   $y = \dfrac{5}{3}$ or $y = -2$

\setcounter{HW}{\value{enumi}}
\end{enumerate}
\end{multicols}


\begin{multicols}{3}
\begin{enumerate}
\setcounter{enumi}{\value{HW}}

\item   $t = 0$ or $t = 4$\vphantom{$x = \dfrac{3}{2}$ or $x = \dfrac{7}{4}$ }
\item   $y = -1$ or $y = \dfrac{3}{2}$ \vphantom{$x = \dfrac{3}{2}$ or $x = \dfrac{7}{4}$}
\item   $x = \dfrac{3}{2}$ or $x = \dfrac{7}{4}$ 

\setcounter{HW}{\value{enumi}}
\end{enumerate}
\end{multicols}


\begin{multicols}{3}
\begin{enumerate}
\setcounter{enumi}{\value{HW}}

\item $x = 0$ or $x = \pm \dfrac{3}{4}$ 
\item $w = -\dfrac{5}{2}$ or $w = \dfrac{2}{3}$
\item $w=-5$ or $w = -\dfrac{1}{2}$

\setcounter{HW}{\value{enumi}}
\end{enumerate}
\end{multicols}


\begin{multicols}{3}
\begin{enumerate}
\setcounter{enumi}{\value{HW}}

\item $x=3$ or $x = \pm 4$ \vphantom{$t = -1$, $t= -\dfrac{1}{2}$, or $t = 0$}
\item $t = -1$, $t= -\dfrac{1}{2}$, or $t = 0$
\item $a = \pm 1$\vphantom{$t = -1$, $t= -\dfrac{1}{2}$, or $t = 0$}

\setcounter{HW}{\value{enumi}}
\end{enumerate}
\end{multicols}

\begin{multicols}{3}
\begin{enumerate}
\setcounter{enumi}{\value{HW}}

\item $t = -\dfrac{3}{4}$ or $t = \dfrac{3}{2}$
\item $x = \dfrac{2}{3}$
\item $y = \pm 3$  \vphantom{$x = \dfrac{2}{3}$}
\setcounter{HW}{\value{enumi}}
\end{enumerate}
\end{multicols}

\end{document}

\newpage

\section{Quadratic Equations}

\documentclass[11pt]{article}
\usepackage[margin=1in,letterpaper]{geometry}
\usepackage{amssymb,amsmath,amsthm,fancyhdr,supertabular,longtable,hhline}
\usepackage{colortbl}
\usepackage{import, multicol,boxedminipage}
\usepackage{graphicx}
\usepackage[colorlinks, hyperindex, plainpages=false, linkcolor=blue, urlcolor=blue, pdfpagelabels]{hyperref}
\usepackage[all]{hypcap}
\definecolor{ResultColor}{gray}{0.9}
\theoremstyle{definition}  % this prevents the text in definitions, theorems, and corollaries from being italicized
\newtheorem{defn}{\bf Definition}
\newtheorem{thm}{\bf Theorem}
\newtheorem{cor}[thm]{\bf Corollary}
\newtheorem{eqn}{\bf Equation}
\newtheorem{ex}{\bf Example}
\newtheorem{fig}{\bf Figure}
\setlength{\parindent}{0in}
\newcommand{\bbm}{\begin{boxedminipage}{6.41in}}
\newcommand{\ebm}{\end{boxedminipage}}
\usepackage{array}
\setlength{\extrarowheight}{2pt}
\allowdisplaybreaks[2]
\usepackage{cancel}
\usepackage{sectsty}
\usepackage{textcomp}
\usepackage{multirow}
\usepackage[sfdefault,lf]{carlito}
	%% The 'lf' option for lining figures
	%% The 'sfdefault' option to make the base font sans serif
	%\usepackage[T1]{fontenc}
	\renewcommand*\oldstylenums[1]{\carlitoOsF #1}
\usepackage[nottoc]{tocbibind}
\allsectionsfont{\mdseries \scshape}
\makeatletter
\renewcommand\l@section{\@dottedtocline{1}{1.5em}{3em}}
\renewcommand\l@subsection{\@dottedtocline{2}{4.5em}{3.5em}}
\makeatother
\pagestyle{fancy}
\newcounter{HW}
\newcounter{HWindent}

\title{Review \#5: Quadratic Equations}
\author{Carl Stitz and Jeff Zeager\\
Edited by Sean Fitzpatrick}

\begin{document}
\maketitle


\renewcommand{\headrulewidth}{0pt}
\renewcommand{\headheight}{14pt}
\lhead[\fancyplain{}{\sc\thepage}]%
      {\fancyplain{}{\sc \nouppercase{\rightmark}}}
\rhead[\fancyplain{}{\sc \nouppercase{\leftmark}}]%
      {\fancyplain{}{\sc\thepage}}
\cfoot{}


In another handout, we review how to solve basic non-linear equations by factoring.  While factoring is an important skill, our ability to apply it by hand is limited. For example, we can solve $2x^2+5x-3=0$ by factoring, $(2x-1)(x+3) = 0$, from which we obtain $x = \frac{1}{2}$ and $x = -3$.  If we change the $5$ to a $6$ and try to solve $2x^2 + 6x - 3 = 0$, however, we find that this polynomial doesn't factor over the integers and we are stuck.  It turns out that there are two real number solutions to this equation, but they are \textit{irrational} numbers, and our aim in this handout is to review the techniques which allow us to find these solutions. In this handout, we focus our attention on \textbf{quadratic} equations.

\medskip

\colorbox{ResultColor}{\bbm

\begin{defn}\label{quadeqndefn} An equation is said to be \textbf{quadratic} in a variable $X$ if it can be written in the form $AX^2 + BX + C = 0$ where $A$, $B$ and $C$ are expressions which do not involve $X$ and $A \neq 0$.

\end{defn}

\ebm}

\medskip

Think of quadratic equations as equations that are one degree up from linear equations - instead of the highest power of $X$ being just $X = X^1$, it's $X^2$.  The simplest class of quadratic equations to solve are the ones in which $B = 0$.  In that case, we have the following.

\medskip

\phantomsection
\label{extractingthesquareroot}

\colorbox{ResultColor}{\bbm

\centerline{\textbf{Solving Quadratic Equations by Extracting Square Roots}}
\vspace{.05in}
If $c$ is a real number with $c \geq 0$, the solutions to $X^2 = c$ are $X = \pm \sqrt{c}$.

\vspace{.05in}
\textbf{Note:}  If $c < 0$, $X^2 = c$ has no real number solutions.

\ebm}

\medskip

There are a couple different ways to see why Extracting Square Roots works, both of which are demonstrated by solving the equation $x^2 = 3$.  If we follow the procedure outlined in the previous section, we subtract $3$ from both sides to get $x^2 - 3 = 0$ and we now try to factor $x^2 - 3$.   We could write $x^2 - 3 = x^2 - (\sqrt{3})^2$ and apply the Difference of Squares formula to factor $x^2 - 3 = (x-\sqrt{3})(x+\sqrt{3})$.  We solve $(x-\sqrt{3})(x+\sqrt{3}) = 0$ by using the Zero Product Property as before by setting each factor equal to zero:  $x - \sqrt{3} = 0$ and $x+\sqrt{3} - 0$.  We get the answers $x = \pm \sqrt{3}$.  In general,  if $c \geq 0$, then $\sqrt{c}$ is a real number, so  $x^2 - c = x^2 - (\sqrt{c})^2 = (x-\sqrt{c})(x+\sqrt{c})$.  Replacing the `$3$' with `$c$' in the above discussion gives the general result. 

\smallskip

Another way to view this result is to visualize  `taking the square root' of both sides:   since $x^2 = c$,  $\sqrt{x^2} = \sqrt{c}$.  How do we simplify $\sqrt{x^2}$? We have to exercise a bit of caution here.  Note that $\sqrt{(5)^2}$ and $\sqrt{(-5)^2}$ both simplify to  $\sqrt{25} = 5$.  In both cases, $\sqrt{x^2}$ returned a \textit{positive} number, since the negative in $-5$ was `squared away' \textit{before} we took the square root.  In other words, $\sqrt{x^2}$ is $x$ if $x$ is positive, or, if $x$ is negative, we make $x$ positive - that is, $\sqrt{x^2} = |x|$, the absolute value of $x$.  So from $x^2 = 3$, we `take the square root' of both sides of the equation to get $\sqrt{x^2} = \sqrt{3}$.  This simplifies to $|x| = \sqrt{3}$, which is equivalent to $x = \sqrt{3}$ or $x = -\sqrt{3}$.  (See the handout on absolute values.) Replacing the `$3$' in the previous argument with `$c$,' gives the general result.

\smallskip

As you might expect, Extracting Square Roots can be applied to more complicated equations.  Consider the equation below.  We can solve it by Extracting Square Roots provided we first isolate the perfect square quantity:\[ \begin{array}{rclr}

2\left(x + \dfrac{3}{2}\right)^2 - \dfrac{15}{2} & = & 0 & \\ [8pt]
2\left(x + \dfrac{3}{2}\right)^2 & = & \dfrac{15}{2} & \text{Add $\dfrac{15}{2}$} \\
\left(x + \dfrac{3}{2}\right)^2 & = & \dfrac{15}{4} & \text{Divide by $2$} \\
x + \dfrac{3}{2} & = & \pm \sqrt{\dfrac{15}{4}} & \text{Extract Square Roots} \\ [8pt]
x + \dfrac{3}{2} & = & \pm \dfrac{\sqrt{15}}{2} & \text{Property of Radicals} \\ [5pt]
x & = & -\dfrac{3}{2} \pm \dfrac{\sqrt{15}}{2}  & \text{Subtract $\dfrac{3}{2}$} \\ [8pt]
x & = & -\dfrac{3 \pm \sqrt{15}}{2}  & \text{Add fractions} \\

\end{array} \] Let's return to the equation $2x^2 + 6x - 3 = 0$ from the beginning of the section.  We leave it to the reader to show that \[2\left(x + \dfrac{3}{2}\right)^2 - \dfrac{15}{2} =  2x^2 + 6x - 3. \] (Hint: Expand the left side.)  In other words, we can solve $2x^2 + 6x - 3 = 0$  by \textit{transforming} into an equivalent equation. This process, you may recall, is called `Completing the Square.'  We'll discuss Completing the Square in more generality and for a different purpose in the textbook, but for now we revisit the steps needed to complete the square to solve a quadratic equation.

\medskip

\phantomsection
\label{completesquareeqns}

\colorbox{ResultColor}{\bbm

\centerline{\textbf{Solving Quadratic Equations:  Completing the Square}}
\vspace{0.05in}
To solve a quadratic equation $AX^2 + BX + C = 0$ by Completing the Square:

\begin{enumerate}

\item  Subtract the constant $C$ from both sides.
\item  Divide both sides by $A$,  the coefficient of $X^2$.  (Remember:  $A \neq 0$.)
\item  Add $\left(\frac{B}{2A}\right)^2$ to both sides of the equation. (That's half the coefficient of $X$, squared.)
\vspace{-0.1in}
\item  Factor the left hand side of the equation as $\left(X + \frac{B}{2A}\right)^2$.
\item  Extract Square Roots.
\item  Subtract $\frac{B}{2A}$ from both sides.

\end{enumerate}

\ebm}

\medskip

To refresh our memories, we apply this method to solve $3x^2 - 24x + 5 = 0$: \[ \begin{array}{rclr}

3x^2 - 24x + 5 & = & 0 & \\

3x^2 - 24x  & = & -5 & \text{Subtract  $C=5$} \\

x^2 - 8x & = & -\dfrac{5}{3} & \text{Divide by $A = 3$} \\ [8pt]

x^2 - 8x + 16 & = & -\dfrac{5}{3} + 16 & \text{Add $\left(\frac{B}{2A}\right)^2 = (-4)^2 = 16$} \\ [8pt]

(x - 4)^2 & = & \dfrac{43}{3} & \text{Factor: Perfect Square Trinomial} \\
x - 4 & = & \pm \sqrt{\dfrac{43}{3}} & \text{Extract Square Roots} \\ [5pt]

x & = & 4 \pm \sqrt{\dfrac{43}{3}} & \text{Add $4$} \\

\end{array}\]

At this point, we use properties of fractions and radicals to `rationalize' the denominator:\footnote{Recall that this means we want to get a denominator with rational (more specifically, integer) numbers.}  \[ \sqrt{\dfrac{43}{3}} = \sqrt{\dfrac{43 \cdot 3}{3 \cdot 3}} = \dfrac{\sqrt{129}}{\sqrt{9}} = \dfrac{\sqrt{129}}{3} \]

We can now get a common (integer) denominator which yields: \[x=  4 \pm \sqrt{\dfrac{43}{3}} = 4 \pm \dfrac{\sqrt{129}}{3} = \dfrac{12 \pm \sqrt{129}}{3} \]

The key to Completing the Square is that the procedure always produces a perfect square trinomial. To see why this works \textit{every single time}, we start with $AX^2 + BX + C = 0$ and follow the procedure:\[ \begin{array}{rclr}

AX^2 + BX + C & = & 0 & \\

AX^2 + BX & = & -C & \text{Subtract $C$} \\

X^2 + \dfrac{BX}{A} & = & -\dfrac{C}{A} & \text{Divide by $A \neq 0$} \\ [8pt]

X^2 + \dfrac{BX}{A} + \left(\dfrac{B}{2A}\right)^2 & = & -\dfrac{C}{A} + \left(\dfrac{B}{2A}\right)^2 & \text{Add $ \left(\dfrac{B}{2A}\right)^2$} \\

\end{array} \]

(Hold onto the line above for a moment.)  Here's the heart of the method - we need to show that \[ X^2 + \dfrac{BX}{A} + \left(\dfrac{B}{2A}\right)^2 = \left(X + \dfrac{B}{2A}\right)^2 \]

To show this, we start with the right side of the equation and apply the Perfect Square Formula  \[ \left(X + \dfrac{B}{2A}\right)^2 = X^2 + 2\left(\dfrac{B}{2A}\right)X + \left(\dfrac{B}{2A}\right)^2 = X^2 + \dfrac{BX}{A} + \left(\dfrac{B}{2A}\right)^2 \, \checkmark \]

With just a few more steps we can solve the general equation $AX^{2} + BX + C = 0$ so let's pick up the story where we left off. (The line on the previous page we told you to hold on to.)\[ \begin{array}{rclr}

X^2 + \dfrac{BX}{A} + \left(\dfrac{B}{2A}\right)^2 & = & -\dfrac{C}{A} + \left(\dfrac{B}{2A}\right)^2 & \\ [8pt]
\left(X + \dfrac{B}{2A}\right)^2 & = & -\dfrac{C}{A} + \dfrac{B^2}{4A^2} & \text{Factor: Perfect Square Trinomial} \\ [3pt]

\left(X + \dfrac{B}{2A}\right)^2 & = & -\dfrac{4AC}{4A^2} + \dfrac{B^2}{4A^2} & \text{Get a common denominator}\\

\left(X + \dfrac{B}{2A}\right)^2 & = & \dfrac{B^2 - 4AC}{4A^2} & \text{Add fractions}\\ [5pt]

X + \dfrac{B}{2A} & = & \pm \sqrt{\dfrac{B^2 - 4AC}{4A^2}} & \text{Extract Square Roots} \\ [8pt]

X + \dfrac{B}{2A} & = & \pm \dfrac{\sqrt{B^2 - 4AC}}{2A} & \text{Properties of Radicals} \\ [8pt]

X  & = & - \dfrac{B}{2A} \pm \dfrac{\sqrt{B^2 - 4AC}}{2A} & \text{Subtract $\dfrac{B}{2A}$} \\ [8pt]

X  & = & \dfrac{-B \pm \sqrt{B^2 - 4AC}}{2A} & \text{Add fractions.} \\

\end{array}\]

Lo and behold, we have derived the legendary \textbf{Quadratic Formula}!

\medskip

\colorbox{ResultColor}{\bbm

\begin{thm}\label{quadraticformula}  \textbf{Quadratic Formula:} The solution to $AX^2 + BX + C = 0$ with $A \neq 0$ is: \[X  = \dfrac{-B \pm \sqrt{B^2 - 4AC}}{2A} \]

\end{thm}

\ebm}

\medskip

We can check our earlier solutions to $2x^2 + 6x - 3 = 0$ and $3x^2 - 24x + 5 = 0$ using the Quadratic Formula.  For $2x^2 + 6x - 3 = 0$, we identify $A = 2$, $B = 6$ and $C = -3$.  The quadratic formula gives: \[ x = \dfrac{-6 \pm \sqrt{6^2 - 4(2)(-3)}}{2(2)} - \dfrac{-6 \pm \sqrt{36 + 24}}{4} = \dfrac{-6 \pm \sqrt{60}}{4} \] Using properties of radicals ($\sqrt{60} = 2 \sqrt{15}$), this reduces to $\frac{2(-3 \pm \sqrt{15})}{4} =\frac{-3 \pm \sqrt{15}}{2}$. We leave it to the reader to show these two answers are the same as $-\frac{3 \pm \sqrt{15}}{2}$,  as required.\footnote{Think about what $-(3 \pm \sqrt{15})$ is really telling you.}  

\smallskip

For $3x^2 - 24x + 5 = 0$, we identify $A = 3$, $B = -24$ and $C = 5$.  Here, we get: \[ x = \dfrac{-(-24) \pm \sqrt{(-24)^2 - 4(3)(5)}}{2(3)} = \dfrac{24 \pm \sqrt{516}}{6} \]

Since $\sqrt{516} = 2\sqrt{129}$, this reduces to  $x = \frac{12 \pm \sqrt{129}}{3}$. 

\smallskip

It is worth noting that the Quadratic Formula applies to all quadratic equations - even ones we could solve using other techniques.  For example, to solve $2x^2 + 5x - 3 = 0$  we identify $A = 2$, $B = 5$ and $C = -3$.  This yields: \[ x = \dfrac{-5 \pm \sqrt{5^2 - 4(2)(-3)}}{2(2)} = \dfrac{-5 \pm \sqrt{49}}{4} = \dfrac{-5 \pm 7}{4} \]

At this point, we have $x = \frac{-5+7}{4} = \frac{1}{2}$ and $x = \frac{-5-7}{4} = \frac{-12}{4} = -3$ - the same two answers we obtained factoring.  We can also use it to solve $x^2 = 3$, if we wanted to.  From $x^2 -3 = 0$, we have $A = 1$, $B = 0$ and $C = -3$.  The Quadratic Formula produces \[ x = \dfrac{-0 \pm \sqrt{0^2 - 4(1)(3)}}{2(1)} = \dfrac{\pm\sqrt{12}}{2} = \pm \dfrac{2\sqrt{3}}{2} = \pm \sqrt{3}\]

As this last example illustrates, while the  Quadratic Formula \textit{can} be used to solve every quadratic equation, that doesn't mean it \textit{should} be used.  Many times other methods are more efficient.  We now provide a more comprehensive approach to solving Quadratic Equations.  

\medskip

\phantomsection
\label{solvequadraticeqns}

\colorbox{ResultColor}{\bbm
\centerline{\textbf{Strategies for Solving Quadratic Equations}}

\begin{itemize}

\item  If the variable appears in the squared term only, isolate it and Extract Square Roots.
\item  Otherwise, put the nonzero terms on one side of the equation so that the other side is $0$.
\begin{itemize}
\item  Try factoring.  
\item  If the expression doesn't factor easily, use the Quadratic Formula.

\end{itemize}
\end{itemize}

\ebm}

\medskip

The reader is encouraged to pause for a moment to think about why `Completing the Square' doesn't appear in our list of strategies despite the fact that we've spent the majority of the section so far talking about it.\footnote{Unacceptable answers include ``Jeff and Carl are mean'' and ``It was one of Carl's Pedantic Rants''.}  Let's get some practice solving quadratic equations, shall we?

\begin{ex}\label{reviewquadraticex}  Find all real number solutions to the following equations.

\begin{multicols}{3}

\begin{enumerate}

\item $3 - (2w-1)^2 = 0$\vphantom{$(y-1)^2 = 2 - \dfrac{y+2}{3}$}

\item $5x - x(x-3) = 7$\vphantom{$(y-1)^2 = 2 - \dfrac{y+2}{3}$}

\item  $(y-1)^2 = 2 - \dfrac{y+2}{3}$ 

\setcounter{HW}{\value{enumi}}

\end{enumerate}
\end{multicols}

\begin{multicols}{3}

\begin{enumerate}

\setcounter{enumi}{\value{HW}}

\item $5(25 - 21x) = \dfrac{59}{4} - 25x^2$

\item $-4.9t^2 + 10t\sqrt{3} + 2 = 0$ \vphantom{$5(25 - 21x) = \dfrac{59}{4} - 25x^2$}



\item $2x^2 = 3x^4 - 6$\vphantom{$5(25 - 21x) = \dfrac{59}{4} - 25x^2$}



\setcounter{HW}{\value{enumi}}

\end{enumerate}
\end{multicols}

{\bf Solution.}

\begin{enumerate}

\item  Since $3 - (2w-1)^2 = 0$ contains a perfect square, we isolate it first then extract square roots: \[ \begin{array}{rclr}

3 - (2w-1)^2 & = & 0 & \\

3 & = & (2w-1)^2 & \text{Add $(2w-1)^2$} \\

\pm \sqrt{3} & = & 2w - 1 & \text{Extract Square Roots} \\

1 \pm \sqrt{3} & = & 2w & \text{Add $1$} \\

\dfrac{1 \pm \sqrt{3}}{2} & = & w & \text{Divide by $2$} \\

\end{array} \]

We find our two answers $w = \frac{1 \pm \sqrt{3}}{2}$.  The reader is encouraged to check both answers by substituting each into the original equation.\footnote{It's excellent  practice working with radicals fractions so we really, \emph{really} want you to take the time to do it.}

\item To solve $5x - x(x-3) = 7$, we begin performing the indicated operations and getting one side equal to $0$.\[ \begin{array}{rclr}

5x - x(x-3) & = & 7 & \\

5x - x^2 + 3x & = & 7 & \text{Distribute} \\

-x^2 + 8x & = & 7 & \text{Gather like terms} \\

-x^2 + 8x - 7 & = & 0& \text{Subtract $7$} \\

\end{array}\]

At this point, we attempt to factor and find $-x^2 + 8x - 7 = (x-1)(-x+7)$.  Using the Zero Product Property, we get $x-1 = 0$  or $-x+7 = 0$.  Our answers are $x = 1$ or $x = 7$, both of which are easy to check.

\item Even though we have a perfect square in $(y-1)^2 = 2 - \frac{y+2}{3}$, Extracting Square Roots won't help matters since we have a $y$ on the other side of the equation.  Our strategy here is to perform the indicated operations (and clear the fraction for good measure) and get $0$ on one side of the equation.\[ \begin{array}{rclr}

(y-1)^2 & = &  2 - \dfrac{y+2}{3} & \\ [8pt]

y^2 - 2y + 1 & = & 2 - \dfrac{y+2}{3} & \text{Perfect Square Trinomial}\\ [8pt]

3(y^2 - 2y + 1) & = & 3\left(2 - \dfrac{y+2}{3} \right) & \text{Multiply by $3$} \\ [10pt]
3y^2 - 6y + 3 & = & 6 - 3\left(\dfrac{y+2}{3}\right) & \text{Distribute} \\ [8pt]

3y^2 - 6y + 3 & = & 6 - (y+2) & \\

3y^2 - 6y + 3 - 6 + (y+2) & = & 0 & \text{Subtract $6$, Add $(y+2)$}\\

3y^2 - 5y - 1 & = & 0 & \\

\end{array}\]

A cursory attempt at factoring bears no fruit, so we run this through the Quadratic Formula with $A = 3$, $B = -5$ and $C = -1$. \[ \begin{array}{rclr}

y & = & \dfrac{-(-5) \pm \sqrt{(-5)^2 - 4(3)(-1)}}{2(3)} & \\ [8pt]

y & = & \dfrac{5 \pm \sqrt{25 + 12}}{6} & \\[8pt]

y & = & \dfrac{5 \pm \sqrt{37}}{6} & \\

\end{array} \] Since $37$ is prime, we have no way to reduce $\sqrt{37}$.  Thus, our final answers are $y = \frac{5 \pm \sqrt{37}}{6}$. The reader is encouraged to supply the details of the challenging verification of the answers.

\item  We proceed as before; our aim is to gather the nonzero terms on one side of the equation. \[ \begin{array}{rclr}

5(25 - 21x) & = &  \dfrac{59}{4} - 25x^2 & \\ [10pt]

125 - 105x & = & \dfrac{59}{4} - 25x^2 & \text{Distribute} \\ [10pt]

4(125 - 105x) & = & 4\left(\dfrac{59}{4} - 25x^2 \right) & \text{Multiply by $4$} \\ [10pt]

500 - 420x & = & 59 - 100x^2 & \text{Distribute} \\ [10pt]

500 - 420x - 59 + 100x^2 & = & 0 & \text{Subtract $59$, Add $100x^2$} \\ [10pt]

100x^2 - 420x + 441 & = & 0 & \text{Gather like terms} \\  

\end{array} \]

With highly composite numbers like $100$ and $441$, factoring seems inefficient at best,\footnote{This is actually the Perfect Square Trinomial $(10x - 21)^2$.} so we apply the Quadratic Formula with $A = 100$, $B = -420$ and $C = 441$:\[ \begin{array}{rclr}

x & = & \dfrac{-(-420) \pm \sqrt{(-420)^2 - 4(100)(441)}}{2(100)} & \\ [12pt]

 & = & \dfrac{420 \pm \sqrt{176000 - 176400}}{200} & \\ [12pt]

& = & \dfrac{420 \pm \sqrt{0}}{200} & \\ [12pt]

& = & \dfrac{420 \pm 0}{200} & \\ [12pt]
& = & \dfrac{420}{200} & \\ [12pt]

& = & \dfrac{21}{10} & \\

\end{array} \]

To our surprise and delight we obtain just one answer, $x = \frac{21}{10}$.

\item  Our next equation $-4.9t^2 + 10t\sqrt{3} + 2 = 0$, already has $0$ on one side of the equation, but with coefficients like $-4.9$ and $10\sqrt{3}$, factoring with integers is not an option.  We could make things a \textit{bit} easier on the eyes by clearing the decimal (by multiplying through by $10$) to get  $-49t^2 + 100t\sqrt{3} + 20 = 0$ but we simply cannot rid ourselves of the irrational number $\sqrt{3}$.  The Quadratic Formula is our only recourse.  With $A = -49$, $B = 100\sqrt{3}$ and $C = 20$ we get: 

\[ \begin{array}{rclr}

t & = &  \dfrac{-100\sqrt{3} \pm \sqrt{(100\sqrt{3})^2 - 4(-49)(20)}}{2(-49)} & \\ [10pt]

& = &  \dfrac{-100\sqrt{3} \pm \sqrt{30000 +3920}}{-98}  & \\ [10pt]

& = &  \dfrac{-100\sqrt{3} \pm \sqrt{33920}}{-98}  & \\ [10pt]

& = &  \dfrac{-100\sqrt{3} \pm 8\sqrt{530}}{-98}  & \\ [10pt]

& = &  \dfrac{2(-50\sqrt{3} \pm 4\sqrt{530})}{2(-49)}  & \\ [10pt]

& = &  \dfrac{-50\sqrt{3} \pm 4\sqrt{530}}{-49}  & \text{Reduce} \\ [10pt]

& = &  \dfrac{-(-50\sqrt{3} \pm 4\sqrt{530})}{49}  & \text{Properties of Negatives} \\ [10pt]

& = & \dfrac{50\sqrt{3} \mp 4\sqrt{530}}{49} & \text{Distribute} \\ 

\end{array}\]

You'll note that when we `distributed' the negative in the last step, we changed the `$\pm$' to a `$\mp$.'  While this is technically correct, at the end of the day both symbols mean `plus or minus',\footnote{There are instances where we need both symbols, however.  For example, the Sum and Difference of Cubes Formulas can be written as a single formula:  $a^3 \pm b^3 = (a \pm b) (a^2 \mp ab + b^2)$.  In this case, all of the `top' symbols are read to give the sum formula;  the `bottom' symbols give the difference formula.} so we can write our answers as $t =  \frac{50\sqrt{3} \pm 4\sqrt{530}}{49}$. Checking these answers are a true test of arithmetic mettle.

\item At first glance, the equation $2x^2 = 3x^4 - 6$ seems misplaced.  The highest power of the variable $x$ here is $4$, not $2$, so this equation isn't a quadratic equation - at least not in terms of the variable $x$.  It is, however, an example of an equation that is quadratic `in disguise.'\footnote{More formally, \textbf{quadratic in form.}} We introduce a new variable $u$ to help us see the pattern - specifically we let $u = x^2$.  Thus $u^2 = (x^2)^2 = x^4$.  So in terms of the variable $u$, the equation $2x^2 = 3x^4 - 6$ is $2u = 3u^2 - 6$.  The latter is a quadratic equation, which we can solve using the usual techniques:\[ \begin{array}{rclr}

2u & = & 3u^2 - 6 & \\

0 & = & 3u^2 - 2u - 6 & \text{Subtract $2u$} \\

\end{array}\] After a few attempts at factoring, we resort to the Quadratic Formula with $A = 3$, $B = -2$, $C = -6$ and get:\[ \begin{array}{rclr}

u & = & \dfrac{-(-2) \pm \sqrt{(-2)^2 - 4(3)(-6)}}{2(3)} & \\ [10pt]

& = & \dfrac{2 \pm \sqrt{4 + 72}}{6} & \\ [10pt]

& = & \dfrac{2 \pm \sqrt{76}}{6} & \\ [10pt]

& = & \dfrac{2 \pm \sqrt{4 \cdot 19}}{6} & \\ [10pt]

& = & \dfrac{2 \pm 2\sqrt{19}}{6} & \text{Properties of Radicals} \\ [10pt]

& = & \dfrac{2(1 \pm \sqrt{19})}{2(3)} & \text{Factor} \\ [10pt]

& = & \dfrac{1 \pm \sqrt{19}}{3} & \text{Reduce} \\

\end{array} \]

We've solved the equation for $u$, but what we still need to solve the original equation\footnote{Or, you've solved the equation for `you' ($u$), now you have to solve it for your instructor ($x$).} - which means we need to find the corresponding values of $x$.  Since $u = x^2$, we have two equations:\[ \begin{array}{rclr}  

x^2  =\dfrac{1 + \sqrt{19}}{3} & \text{or} & x^2  =\dfrac{1 - \sqrt{19}}{3} & \\

\end{array}\] We can solve the first equation by extracting square roots to get  $x = \pm \sqrt{\frac{1 + \sqrt{19}}{3}}$.  The second equation, however, has no real number solutions because $\frac{1 - \sqrt{19}}{3}$ is a negative number.  For our final answers we can rationalize the denominator to get: \[ x = \pm \sqrt{\dfrac{1 + \sqrt{19}}{3}} = \pm \sqrt{\dfrac{1 + \sqrt{19}}{3} \cdot \dfrac{3}{3}} = \pm \dfrac{\sqrt{3 + 3\sqrt{19}}}{3} \] As with the previous exercise, the very challenging check is left to the reader. \qed

\end{enumerate}

\end{ex}

Our last example above, the `Quadratic in Disguise', hints that the Quadratic Formula is applicable to a wider class of equations than those which are strictly quadratic.  We give some general guidelines to recognizing these beasts in the wild on the next page.

\phantomsection
\label{QuadinDisguise}
\colorbox{ResultColor}{\bbm

\centerline{\textbf{Identifying Quadratics in Disguise}}

An equation is a `Quadratic in Disguise' if it can be written in the form:  $AX^{2m} + BX^{m} + C = 0$.  

In other words:

\begin{itemize}

\item There are exactly three terms, two with variables and one constant term.

\item  The exponent on the variable in one term is \textit{exactly twice} the variable on the other term.

\end{itemize}

To transform a Quadratic in Disguise to a quadratic equation, let $u = X^m$ so $u^2 = (X^m)^2 = X^{2m}$. This transforms the equation into $Au^2 + Bu + C = 0$.

\ebm}

\medskip

For example, $3x^6 - 2x^3 + 1 = 0$ is a Quadratic in Disguise, since $6 = 2 \cdot 3$.  If we let $u = x^3$, we get $u^2 = (x^3)^2 = x^6$, so the equation becomes $3u^2 - 2u + 1 = 0$.  However, $3x^6 - 2x^2 + 1 = 0$ is \textit{not} a Quadratic in Disguise, since $6 \neq 2\cdot 2$. The substitution $u = x^2$ yields $u^2 = (x^2)^2 = x^4$, not $x^6$ as required.  We'll see more instances of `Quadratics in Disguise' in later sections.

\medskip

We close this section with a review of the \textbf{discriminant} of a quadratic equation as defined below.

\medskip

\colorbox{ResultColor}{\bbm
\begin{defn} \textbf{The Discriminant:} Given a quadratic equation $AX^2 + BX + C = 0$, the quantity $B^2 - 4AC$ is called the \textbf{discriminant} of the equation.

\end{defn}
\ebm}

\medskip

The discriminant is the radicand of the square root in the quadratic formula:  \[X  = \dfrac{-B \pm \sqrt{B^2 - 4AC}}{2A} \] It \textit{discriminates} between the nature and number of solutions we get from a quadratic equation.   The results are summarized below. 

\medskip

\colorbox{ResultColor}{\bbm
\begin{thm} \textbf{Discriminant Theorem:} \label{discriminanttheoremrealversion} Given a Quadratic Equation $AX^2 + BX + C = 0$, let $D = B^2 - 4AC$ be the discriminant.

\begin{itemize}

\item  If $D > 0$, there are two distinct real number solutions to the equation. 

\item  If $D = 0$, there is one repeated real number solution.  

\textbf{Note:}  `Repeated' here comes from the fact that `both' solutions $\frac{-B \pm 0}{2A}$ reduce to $-\frac{B}{2A}$.

\item  If $D < 0$, there are no real solutions.

\end{itemize}

\end{thm}
\ebm}

\medskip

For example, $x^2 + x - 1 = 0$ has two real number solutions since the discriminant works out to be $(1)^2 - 4(1)(-1) = 5 > 0$.  This results in a $\pm \sqrt{5}$ in the Quadratic Formula, generating two different answers.  On the other hand, $x^2 + x + 1 = 0$ has no real solutions since here, the discriminant is $(1)^2 - 4(1)(1) = -3 < 0$ which generates a $\pm \sqrt{-3}$ in the Quadratic Formula.  The equation $x^2 + 2x +1 = 0$ has discriminant $(2)^2 - 4(1)(1) = 0$ so in the Quadratic Formula we get a $\pm \sqrt{0} = 0$ thereby  generating just one solution.  More can be said as well.  For example, the discriminant of $6x^2 - x - 40 = 0$ is $961$.  This is a perfect square, $\sqrt{961} = 31$, which means our solutions are rational numbers.  When our solutions are rational numbers, the quadratic actually factors nicely. In our example  $6x^2 - x - 40 = (2x+5)(3x-8)$.  Admittedly,  if you've already computed the discriminant, you're most of the way done with the problem and probably wouldn't take the time to experiment with factoring the quadratic at this point -- but we'll see another use for this analysis of the discriminant in the next section.

\newpage

\section{Exercises}

In Exercises \ref{solvequadfirst} - \ref{solvequadlast}, find all real solutions.  Check your answers, as directed by your instructor.

\begin{multicols}{3}
\begin{enumerate}

\item  $3\left(x - \dfrac{1}{2}\right)^2 = \dfrac{5}{12}$ \label{solvequadfirst}
\item  $4 - (5t+3)^2 = 3$ \vphantom{$3\left(x - \dfrac{1}{2}\right)^2 = \dfrac{5}{12}$}
\item  $3(y^2-3)^2-2 = 10$ \vphantom{$3\left(x - \dfrac{1}{2}\right)^2 = \dfrac{5}{12}$}

\setcounter{HW}{\value{enumi}}
\end{enumerate}
\end{multicols}

\begin{multicols}{3}
\begin{enumerate}
\setcounter{enumi}{\value{HW}}

\item $x^2 + x - 1 = 0$
\item $3w^2 = 2-w$
\item $y(y+4) = 1$

\setcounter{HW}{\value{enumi}}
\end{enumerate}
\end{multicols}

\begin{multicols}{3}
\begin{enumerate}
\setcounter{enumi}{\value{HW}}

\item $\dfrac{z}{2} = 4z^2-1$
\item  $0.1v^2 + 0.2v = 0.3$ \vphantom{$\dfrac{z}{2} = 4z^2-1$}
\item $x^2 = x - 1$\vphantom{$\dfrac{z}{2} = 4z^2-1$}

\setcounter{HW}{\value{enumi}}
\end{enumerate}
\end{multicols}

\begin{multicols}{3}
\begin{enumerate}
\setcounter{enumi}{\value{HW}}

\item $3-t = 2(t+1)^2$
\item $(x-3)^2 = x^2+9$
\item $(3y-1)(2y+1) = 5y$

\setcounter{HW}{\value{enumi}}
\end{enumerate}
\end{multicols}

\begin{multicols}{3}
\begin{enumerate}
\setcounter{enumi}{\value{HW}}

\item $w^4 + 3w^2 - 1 = 0$
\item $2x^4 +x^2 = 3$ 
\item $(2-y)^4 = 3(2-y)^2 + 1$

\setcounter{HW}{\value{enumi}}
\end{enumerate}
\end{multicols}

\begin{multicols}{3}
\begin{enumerate}
\setcounter{enumi}{\value{HW}}

\item $3x^4 + 6x^2 = 15x^3$
\item $6p + 2 = p^2 + 3p^3$
\item $10v = 7v^3 - v^5$

\setcounter{HW}{\value{enumi}}
\end{enumerate}
\end{multicols}



\begin{multicols}{3}
\begin{enumerate}
\setcounter{enumi}{\value{HW}}

\item $y^2 - \sqrt{8} y = \sqrt{18} y - 1$\vphantom{$\dfrac{v^2}{3} = \dfrac{v \sqrt{3}}{2} + 1$}
\item $x^2 \sqrt{3} = x \sqrt{6} + \sqrt{12}$\vphantom{$\dfrac{v^2}{3} = \dfrac{v \sqrt{3}}{2} + 1$}
\item $\dfrac{v^2}{3} = \dfrac{v \sqrt{3}}{2} + 1$ \label{solvequadlast}

\setcounter{HW}{\value{enumi}}
\end{enumerate}
\end{multicols}

In Exercises \ref{solvequadcalcfirst} - \ref{solvequadcalclast}, find all real solutions and use a calculator to approximate your answers, rounded to two decimal places.

\begin{multicols}{3}
\begin{enumerate}
\setcounter{enumi}{\value{HW}}

\item $5.54^2 + b^2 = 36$\label{solvequadcalcfirst}
\item $\pi r^2 = 37$ 
\item $54 = 8r\sqrt{2} + \pi r^2$


\setcounter{HW}{\value{enumi}}
\end{enumerate}
\end{multicols}

\begin{multicols}{3}
\begin{enumerate}
\setcounter{enumi}{\value{HW}}

\item $-4.9t^2 + 100t = 410$
\item $x^2 = 1.65(3-x)^2$
\item $(0.5+2A)^2 = 0.7(0.1-A)^2$ \label{solvequadcalclast}

\setcounter{HW}{\value{enumi}}
\end{enumerate}
\end{multicols}



%\begin{enumerate}
%\setcounter{enumi}{\value{HW}}


%\item $(0.00623 + 2Q)^2 = 54.3(0.00414 - Q)(0.0224-Q)$\label{solvequadcalclast}

%\setcounter{HW}{\value{enumi}}
%\end{enumerate}

In Exercises \ref{absquadfirst} - \ref{absquadlast}, use properties of absolute values along with the techniques in this section to find all real solutions to the following.

\begin{multicols}{3}
\begin{enumerate}
\setcounter{enumi}{\value{HW}}

\item $|x^2 - 3x| = 2$ \label{absquadfirst}
\item $|2x-x^2| = |2x-1|$
\item $|x^2 -x + 3| = |4-x^2|$ \label{absquadlast}

\setcounter{HW}{\value{enumi}}
\end{enumerate}
\end{multicols}

\begin{enumerate}
\setcounter{enumi}{\value{HW}}

\item  Prove that for every nonzero number $p$, $x^2 + xp + p^2 = 0$  has no real solutions.

\item  Solve for $t$: $-\dfrac{1}{2} g t^2 + vt + h = 0$. Assume $g > 0$, $v \geq 0$ and $h \geq 0$.

\setcounter{HW}{\value{enumi}}
\end{enumerate}




\newpage

\section{Answers}

\begin{multicols}{3}
\begin{enumerate}

\item  $x = \dfrac{3 \pm \sqrt{5}}{6}$ 
\item  $t = -\dfrac{4}{5}, -\dfrac{2}{5}$ \vphantom{$x = \dfrac{3 \pm \sqrt{5}}{6}$ }
\item  $y = \pm 1$, $\pm \sqrt{5}$ \vphantom{$x = \dfrac{3 \pm \sqrt{5}}{6}$ }

\setcounter{HW}{\value{enumi}}
\end{enumerate}
\end{multicols}

\begin{multicols}{3}
\begin{enumerate}
\setcounter{enumi}{\value{HW}}

\item $x = \dfrac{-1 \pm \sqrt{5}}{2}$
\item $w = -1, \dfrac{2}{3}$ \vphantom{$x = \dfrac{-1 \pm \sqrt{5}}{2}$}
\item $y = -2 \pm \sqrt{5}$\vphantom{$x = \dfrac{-1 \pm \sqrt{5}}{2}$}

\setcounter{HW}{\value{enumi}}
\end{enumerate}
\end{multicols}

\begin{multicols}{3}
\begin{enumerate}
\setcounter{enumi}{\value{HW}}

\item $z = \dfrac{1 \pm \sqrt{65}}{16}$
\item  $v = -3, 1$\vphantom{$z = \dfrac{1 \pm \sqrt{65}}{16}$}
\item No real solution.\vphantom{$z = \dfrac{1 \pm \sqrt{65}}{16}$}

\setcounter{HW}{\value{enumi}}
\end{enumerate}
\end{multicols}

\begin{multicols}{3}
\begin{enumerate}
\setcounter{enumi}{\value{HW}}

\item $t = \dfrac{-5 \pm \sqrt{33}}{4}$
\item $x = 0$ \vphantom{$t = \dfrac{-5 \pm \sqrt{33}}{4}$}
\item $y = \dfrac{2 \pm \sqrt{10}}{6}$\vphantom{$t = \dfrac{-5 \pm \sqrt{33}}{4}$}

\setcounter{HW}{\value{enumi}}
\end{enumerate}
\end{multicols}

\begin{multicols}{3}
\begin{enumerate}
\setcounter{enumi}{\value{HW}}

\item $w = \pm \sqrt{\dfrac{\sqrt{13} - 3}{2}}$\vphantom{$y = \dfrac{4 \pm \sqrt{6 + 2 \sqrt{13}}}{2}$}
\item $x = \pm 1$\vphantom{$y = \dfrac{4 \pm \sqrt{6 + 2 \sqrt{13}}}{2}$}
\item $y = \dfrac{4 \pm \sqrt{6 + 2 \sqrt{13}}}{2}$

\setcounter{HW}{\value{enumi}}
\end{enumerate}
\end{multicols}

\begin{multicols}{3}
\begin{enumerate}
\setcounter{enumi}{\value{HW}}

\item $x = 0, \dfrac{5 \pm \sqrt{17}}{2}$
\item $p = -\dfrac{1}{3}, \pm \sqrt{2}$ \vphantom{$x = 0, \dfrac{5 \pm \sqrt{17}}{2}$}
\item $v = 0, \pm \sqrt{2}, \pm \sqrt{5}$\vphantom{$x = 0, \dfrac{5 \pm \sqrt{17}}{2}$}

\setcounter{HW}{\value{enumi}}
\end{enumerate}
\end{multicols}



\begin{multicols}{3}
\begin{enumerate}
\setcounter{enumi}{\value{HW}}

\item $y = \dfrac{5\sqrt{2} \pm \sqrt{46}}{2}$\vphantom{$x = \dfrac{\sqrt{2} \pm \sqrt{10}}{2}$}
\item $x = \dfrac{\sqrt{2} \pm \sqrt{10}}{2}$
\item $v = -\dfrac{\sqrt{3}}{2}, 2\sqrt{3}$\vphantom{$x = \dfrac{\sqrt{2} \pm \sqrt{10}}{2}$}

\setcounter{HW}{\value{enumi}}
\end{enumerate}
\end{multicols}


\begin{multicols}{2}
\begin{enumerate}
\setcounter{enumi}{\value{HW}}

\item $b = \pm \dfrac{\sqrt{13271}}{50} \approx \pm 2.30$
\item $r = \pm \sqrt{\dfrac{37}{\pi}} \approx \pm 3.43$ 



\setcounter{HW}{\value{enumi}}
\end{enumerate}
\end{multicols}

\begin{multicols}{2}
\begin{enumerate}
\setcounter{enumi}{\value{HW}}
\item $r = \dfrac{-4\sqrt{2} \pm \sqrt{54\pi + 32}}{\pi}$, $r \approx -6.32, 2.72$
\item $t = \dfrac{500 \pm 10\sqrt{491}}{49}$, $t \approx 5.68, 14.73$


\setcounter{HW}{\value{enumi}}
\end{enumerate}
\end{multicols}

\begin{multicols}{2}
\begin{enumerate}
\setcounter{enumi}{\value{HW}}

\item $x = \dfrac{99 \pm 6 \sqrt{165}}{13}$, $x \approx 1.69, 13.54$
\item $A = \dfrac{-107 \pm 7 \sqrt{70}}{330}$, $A \approx -0.50, -0.15$

\setcounter{HW}{\value{enumi}}
\end{enumerate}
\end{multicols}



%\begin{enumerate}
%\setcounter{enumi}{\value{HW}}


%\item $Q = \dfrac{-468 \pm 21 \sqrt{454}}{475}$, $Q \approx -1.93, -0.04$

%\setcounter{HW}{\value{enumi}}
%\end{enumerate}

\begin{multicols}{3}
\begin{enumerate}
\setcounter{enumi}{\value{HW}}

\item $x = 1, 2, \dfrac{3 \pm \sqrt{17}}{2}$
\item $x = \pm 1, 2 \pm \sqrt{3}$ \vphantom{$x = 1, 2, \dfrac{3 \pm \sqrt{17}}{2}$}
\item $x = -\dfrac{1}{2}, 1, 7$\vphantom{$x = 1, 2, \dfrac{3 \pm \sqrt{17}}{2}$}

\setcounter{HW}{\value{enumi}}
\end{enumerate}
\end{multicols}

\begin{enumerate}
\setcounter{enumi}{\value{HW}}

\item The discriminant is: $D = p^2 - 4p^2 = -3p^2 < 0$.  Since $D < 0$, there are no real solutions.  

\item $t = \dfrac{v \pm \sqrt{v^2 + 2gh}}{g}$

\setcounter{HW}{\value{enumi}}
\end{enumerate}



\end{document}

\newpage

\section{Rational Expressions and Equations}

\documentclass[11pt]{article}
\usepackage[margin=1in,letterpaper]{geometry}
\usepackage{amssymb,amsmath,amsthm,fancyhdr,supertabular,longtable,hhline}
\usepackage{colortbl}
\usepackage{import, multicol,boxedminipage}
\usepackage{graphicx}
\usepackage[colorlinks, hyperindex, plainpages=false, linkcolor=blue, urlcolor=blue, pdfpagelabels]{hyperref}
\usepackage[all]{hypcap}
\definecolor{ResultColor}{gray}{0.9}
\theoremstyle{definition}  % this prevents the text in definitions, theorems, and corollaries from being italicized
\newtheorem{defn}{\bf Definition}
\newtheorem{thm}{\bf Theorem}
\newtheorem{cor}[thm]{\bf Corollary}
\newtheorem{eqn}{\bf Equation}
\newtheorem{ex}{\bf Example}
\newtheorem{fig}{\bf Figure}
\setlength{\parindent}{0in}
\newcommand{\bbm}{\begin{boxedminipage}{6.41in}}
\newcommand{\ebm}{\end{boxedminipage}}
\usepackage{array}
\setlength{\extrarowheight}{2pt}
\allowdisplaybreaks[2]
\usepackage{cancel}
\usepackage{sectsty}
\usepackage{textcomp}
\usepackage{multirow}
\usepackage[sfdefault,lf]{carlito}
	%% The 'lf' option for lining figures
	%% The 'sfdefault' option to make the base font sans serif
	%\usepackage[T1]{fontenc}
	\renewcommand*\oldstylenums[1]{\carlitoOsF #1}
\usepackage[nottoc]{tocbibind}
\allsectionsfont{\mdseries \scshape}
\makeatletter
\renewcommand\l@section{\@dottedtocline{1}{1.5em}{3em}}
\renewcommand\l@subsection{\@dottedtocline{2}{4.5em}{3.5em}}
\makeatother
\pagestyle{fancy}
\newcounter{HW}
\newcounter{HWindent}

\title{Review \#8: Rational Equations and Expressions}
\author{Carl Stitz and Jeff Zeager\\
Edited by Sean Fitzpatrick}


\begin{document}
\maketitle


\renewcommand{\headrulewidth}{0pt}
\renewcommand{\headheight}{14pt}
\lhead[\fancyplain{}{\sc\thepage}]%
      {\fancyplain{}{\sc \nouppercase{\rightmark}}}
\rhead[\fancyplain{}{\sc \nouppercase{\leftmark}}]%
      {\fancyplain{}{\sc\thepage}}
\cfoot{}

In this handout we turn our attention to rational expressions - that is, algebraic fractions - and equations which contain them.  The reader is encouraged to keep in mind the properties of fractions listed in the handout on real number arithmetic, because we will need them along the way.  Before we launch into reviewing the basic arithmetic operations of rational expressions, we take a moment to review how to simplify them properly.  As with numeric fractions, we `cancel common \textit{factors},' not common \textit{terms}.  That is, in order to simplify rational expressions, we first \textit{factor} the numerator and denominator.  For example:  \[ \dfrac{x^4 + 5x^3}{x^3 - 25x} \neq \dfrac{x^4 + 5\cancel{x^3}}{\cancel{x^3} - 25x} \]

but, rather \[ \begin{array}{rclr}

\dfrac{x^4 + 5x^3}{x^3 - 25x} & = & \dfrac{x^3(x + 5)}{x(x^2-25)} & \text{Factor G.C.F.} \\ [12pt]
                             & = & \dfrac{x^3(x + 5)}{x(x-5)(x+5)} & \text{Difference of Squares} \\ [12pt]
														& = & \dfrac{\cancelto{x^2}{x^3}\cancel{(x + 5)}}{\cancel{x}(x-5)\cancel{(x+5)}} & \text{Cancel common factors}\\ [12pt]                         
														& = & \dfrac{x^2}{x-5} & \\ \end{array}\] This equivalence holds provided the factors being cancelled aren't $0$. Since a factor of $x$ and a factor of $x+5$ were cancelled, $x \neq 0$ and $x+5 \neq 0$, so $x \neq -5$.   We usually stipulate this as: \[ \dfrac{x^4 + 5x^3}{x^3 - 25x}  = \dfrac{x^2}{x-5}, \qquad \text{provided $x \neq 0$, $x \neq -5$} \]

While we're talking about common mistakes, please notice that 

\[ \dfrac{5}{x^2+9} \neq \dfrac{5}{x^2} + \dfrac{5}{9} \] 

Just like their numeric counterparts, you don't add algebraic fractions by \textit{adding denominators} of fractions with \textit{common numerators} - it's the other way around:\footnote{One of the most common errors students make on college Mathematics placement tests is that they forget how to add algebraic fractions correctly.  This places many students into remedial classes even though they are probably ready for college-level Math.  We urge you to really study this section with great care so that you don't fall into that trap.} 

\[ \dfrac{x^2+9}{5} = \dfrac{x^2}{5} + \dfrac{9}{5} \] 

It's time to review the basic arithmetic operations with rational expressions. 

\pagebreak

\begin{ex} \label{rationalexpressionreviewex} Perform the indicated operations and simplify.

\begin{multicols}{2}
\begin{enumerate}

\item  $\dfrac{2x^2-5x-3}{x^4 - 4} \div \dfrac{x^2-2x-3}{x^5 + 2x^3}$

\item  $\dfrac{5}{w^2 - 9} - \dfrac{w+2}{w^2-9}$\vphantom{$\dfrac{2x^2-5x-3}{x^4 - 16} \div \dfrac{x^2-2x-3}{x^5 + 2x^3}$}

\setcounter{HW}{\value{enumi}}
\end{enumerate}

\end{multicols}

\begin{multicols}{2}
\begin{enumerate}
\setcounter{enumi}{\value{HW}}

\item  $\dfrac{3}{y^2 - 8y + 16} + \dfrac{y+1}{16y - y^3}$\vphantom{$\dfrac{\dfrac{2}{4 - (x+h)} - \dfrac{2}{4-x}}{h}$}

\item  $\dfrac{\dfrac{2}{4 - (x+h)} - \dfrac{2}{4-x}}{h}$

\setcounter{HW}{\value{enumi}}
\end{enumerate}

\end{multicols}

\begin{multicols}{2}
\begin{enumerate}
\setcounter{enumi}{\value{HW}}

\item  $2t^{-3} - (3t)^{-2}$

\item  $10x(x-3)^{-1} + 5x^2(-1)(x-3)^{-2}$

\setcounter{HW}{\value{enumi}}
\end{enumerate}

\end{multicols}

{\bf Solution.}

\begin{enumerate}

\item As with numeric fractions, we divide rational expressions by `inverting and multiplying'.  Before we get too carried away however, we factor to see what, if any, factors cancel.\[ \begin{array}{rclr}

\dfrac{2x^2-5x-3}{x^4 - 4} \div \dfrac{x^2-2x-3}{x^5 + 2x^3} & = & \dfrac{2x^2-5x-3}{x^4 - 4} \cdot \dfrac{x^5 + 2x^3}{x^2-2x-3} & \text{Invert and multiply} \\ [10pt]

& = & \dfrac{(2x^2-5x-3)(x^5 + 2x^3)}{(x^4 - 4)(x^2-2x-3)} & \text{Multiply fractions}  \\ [10pt]

& = & \dfrac{(2x+1)(x-3)x^3(x^2+2)}{(x^2-2)(x^2+2)(x-3)(x+1)} & \text{Factor} \\ [10pt]

& = & \dfrac{(2x+1)\cancel{(x-3)}x^3\cancel{(x^2+2)}}{(x^2-2)\cancel{(x^2+2)}\cancel{(x-3)}(x+1)} & \text{Cancel common factors} \\ [10pt]

& = & \dfrac{x^3(2x+1)}{(x+1)(x^2-2)} & \text{Provided $x \neq 3$} \\

\end{array}\]

The `$x \neq 3$' is mentioned since a factor of $(x-3)$ was cancelled as we reduced the expression.  We also cancelled a factor of $(x^2+2)$.  Why is there no stipulation as a result of cancelling this factor? Because $x^2 + 2 \neq 0$.  (Can you see why?)  At this point, we \textit{could} go ahead and multiply out the numerator and denominator to get \[\dfrac{x^3(2x+1)}{(x+1)(x^2-2)}  = \dfrac{2x^4 + x^3}{x^3+x^2-2x-2}\] but for most of the applications where this kind of algebra is needed (solving equations, for instance), it is best to leave things factored.  Your instructor will let you know whether to leave your answer in factored form or not.\footnote{Speaking of factoring, do you remember why $x^2-2$ can't be factored over the integers?}

\item  As with numeric fractions we need common denominators in order to subtract.  This is the case here so we proceed by subtracting the numerators. \[ \begin{array}{rclr}

\dfrac{5}{w^2 - 9} - \dfrac{w+2}{w^2-9} & = & \dfrac{5 - (w+2)}{w^2 - 9}& \text{Subtract fractions}\\ [8pt]
                                        & = & \dfrac{5 - w - 2}{w^2-9} & \text{Distribute} \\ [8pt]
																				& = & \dfrac{3-w}{w^2-9} & \text{Combine like terms} \\ \end{array}\]
At this point, we need to see if we can reduce this expression so we proceed to factor.  It first appears as if we have no common factors among the numerator and denominator until we recall the property of `factoring negatives' from the handout on real number arithmetic:  $3-w = -(w-3)$. This yields:\[ \begin{array}{rclr}

\dfrac{3-w}{w^2-9} & = & \dfrac{-(w-3)}{(w-3)(w+3)} & \text{Factor} \\ [10pt]
                   & = & \dfrac{-\cancel{(w-3)}}{\cancel{(w-3)}(w+3)} & \text{Cancel common factors} \\ [10pt]
									 & = & \dfrac{-1}{w+3} & \text{Provided $w \neq 3$} \\ 
									\end{array}\]
The stipulation $w \neq 3$ comes from the cancellation of the $(w-3)$ factor.

\item  	In this next example, we are asked to add two rational expressions with \textit{different} denominators.  As with numeric fractions, we must first find a \textit{common denominator}. To do so, we start by factoring each of the denominators. \[ \begin{array}{rclr}

\dfrac{3}{y^2 - 8y + 16} + \dfrac{y+1}{16y - y^3} & = & \dfrac{3}{(y-4)^2} + \dfrac{y+1}{y(16 - y^2)} & \text{Factor} \\	[8pt]		
                                                  & = & \dfrac{3}{(y-4)^2} + \dfrac{y+1}{y(4-y)(4+y)} & \text{Factor some more} \\
																									
																									\end{array}\]
To find the common denominator, we examine the factors in the first denominator and note that we need a factor of $(y-4)^2$.  We now look at the second denominator to see what other factors we need. We need a factor of $y$ and $(4+y) = (y+4)$.  What about $(4-y)$?  As mentioned in the last example, we can factor this as: $(4-y) = -(y-4)$. Using properties of negatives, we `migrate' this negative out to the front of the fraction, turning the addition into subtraction.  We find the (least) common denominator to be $(y-4)^2 y (y+4)$.  We can now proceed to multiply the numerator and denominator of each fraction by whatever factors each is missing from their respective denominators to produce equivalent expressions with common denominators. \[ \begin{array}{rclr}

\dfrac{3}{(y-4)^2} + \dfrac{y+1}{y(4-y)(4+y)} & = & \dfrac{3}{(y-4)^2} + \dfrac{y+1}{y(-(y-4))(y+4)} &   \\ [8pt]
																							& = & \dfrac{3}{(y-4)^2} - \dfrac{y+1}{y(y-4)(y+4)} & \\ [10pt]
																							& = & \dfrac{3}{(y-4)^2} \cdot \dfrac{y(y+4)}{y(y+4)} - \dfrac{y+1}{y(y-4)(y+4)} \cdot \dfrac{(y-4)}{(y-4)} & \text{Equivalent} \\[-8pt]                                             &   &                                                                                                       & \text{Fractions} \\
																						& = & \dfrac{3y(y+4)}{(y-4)^2y(y+4)}  - \dfrac{(y+1)(y-4)}{y(y-4)^2(y+4)}  & \text{Multiply} \\ [-8pt]
																						&   &                                                                      & \text{Fractions} \\	\end{array}\] At this stage, we can subtract numerators and simplify. We'll keep the denominator factored (in case we can reduce down later), but in the numerator, since there are no common factors, we proceed to perform the indicated multiplication and combine like terms.\[ \begin{array}{rclr}

 \dfrac{3y(y+4)}{(y-4)^2y(y+4)}  - \dfrac{(y+1)(y-4)}{y(y-4)^2(y+4)} & = &  \dfrac{3y(y+4) -(y+1)(y-4)}{(y-4)^2y(y+4)}  & \text{Subtract numerators} \\ [10pt]

& = & \dfrac{3y^2 + 12y - (y^2 - 3y - 4)}{(y-4)^2 y (y+4)} & \text{Distribute} \\ [10pt]

& = & \dfrac{3y^2 + 12y - y^2 + 3y + 4}{(y-4)^2 y (y+4)} & \text{Distribute} \\ [10pt]

& = & \dfrac{2y^2 + 15y + 4}{y (y+4) (y-4)^2} & \text{Gather like terms} \\ \end{array}\] We would like to factor the numerator and cancel factors it has in common with the denominator.  After a few attempts, it appears as if the numerator doesn't factor, at least over the integers.  As a check, we compute the discriminant of $2y^2 + 15y + 4$ and get $15^2 - 4(2)(4) = 193$.  This isn't a perfect square so we know that the quadratic equation $2y^2 + 15y + 4=0$ has irrational solutions. This means $2y^2 + 15y + 4$  can't factor over the integers so we are done.  

\item  In this example, we have a compound fraction, and we proceed to simplify it as we did its numeric counterparts in the handout on real number arithmetic.  Specifically, we start by multiplying the numerator and denominator of the `big' fraction by the least common denominator of the `little' fractions inside of it - in this case we need to use $(4-(x+h))(4-x)$ - to remove the compound nature of the `big' fraction.  Once we have a more normal looking fraction, we can proceed as we have in the previous examples.\[ \begin{array}{rclr}

\dfrac{\dfrac{2}{4 - (x+h)} - \dfrac{2}{4-x}}{h} & = & \dfrac{\left(\dfrac{2}{4 - (x+h)} - \dfrac{2}{4-x}\right)}{h}  \cdot \dfrac{(4-(x+h))(4-x)}{(4-(x+h))(4-x)} & \text{Equivalent} \\ [-8pt]
                                                 &    &                                                                                                            & \text{fractions} \\

& = & \dfrac{\left(\dfrac{2}{4 - (x+h)} - \dfrac{2}{4-x}\right) \cdot (4-(x+h))(4-x) }{h (4-(x+h))(4-x)} & \text{Multiply} \\ [20pt]

& = & \dfrac{\dfrac{2(4-(x+h))(4-x)}{4 - (x+h)} - \dfrac{2(4-(x+h))(4-x)}{4-x}}{h (4-(x+h))(4-x)} & \text{Distribute} \\ [20pt]


& = & \dfrac{\dfrac{2\cancel{(4-(x+h))}(4-x)}{\cancel{(4 - (x+h))}} - \dfrac{2(4-(x+h))\cancel{(4-x)}}{\cancel{(4-x)}}}{h (4-(x+h))(4-x)} & \text{Reduce} \\ [20pt]


& = & \dfrac{2(4-x) - 2(4-(x+h))}{h(4-(x+h))(4-x)} & \\ 

\end{array}\]

Now we can clean up and factor the numerator to see if anything cancels.  (This why we kept the denominator factored.)\[ \begin{array}{rclr}

\dfrac{2(4-x) - 2(4-(x+h))}{h(4-(x+h))(4-x)} & = & \dfrac{2[(4-x) - (4-(x+h))]}{h(4-(x+h))(4-x)} & \text{Factor out G.C.F.} \\ [12pt]
																						 & = & \dfrac{2[4-x - 4+(x+h)]}{h(4-(x+h))(4-x)} & \text{Distribute} \\ [12pt]
																						 & = & \dfrac{2[4- 4 - x+x+h]}{h(4-(x+h))(4-x)} & \text{Rearrange terms} \\ [12pt]
																						 & = & \dfrac{2h}{h(4-(x+h))(4-x)} & \text{Gather like terms} \\ [12pt]
																						 & = & \dfrac{2\cancel{h}}{\cancel{h}(4-(x+h))(4-x)} & \text{Reduce} \\ [12pt]
																						& = & \dfrac{2}{(4-(x+h))(4-x)} & \text{Provided $h \neq 0$} \\
\end{array}\]

Your instructor will let you know if you are to multiply out the denominator or not.\footnote{We'll keep it factored because in Calculus it needs to be factored.}

\item  At first glance, it doesn't seem as if there is anything that can be done with $2t^{-3} - (3t)^{-2}$ because the exponents on the variables are different.  However, since the exponents are negative, these are actually rational expressions.  In the first term, the $-3$ exponent applies to the $t$ \textit{only} but in the second term, the exponent $-2$ applies to \textit{both} the $3$ and the $t$, as indicated by the parentheses.  One way to proceed is as follows:\[ \begin{array}{rclr}

 2t^{-3} - (3t)^{-2} & = & \dfrac{2}{t^3} - \dfrac{1}{(3t)^2} & \\ [10pt]
                     & = & \dfrac{2}{t^3} - \dfrac{1}{9t^2} & \\ \end{array}\]
										
We see that we are being asked to subtract two rational expressions with different denominators, so we need to find a common denominator.  The first fraction contributes a $t^3$ to the denominator, while the second contributes a factor of $9$.  Thus our common denominator is $9t^3$, so we are missing a factor of `$9$' in the first denominator and a factor of `$t$' in the second. \[ \begin{array}{rclr}

 \dfrac{2}{t^3} - \dfrac{1}{9t^2} & = &  \dfrac{2}{t^3} \cdot \dfrac{9}{9} - \dfrac{1}{9t^2} \cdot \dfrac{t}{t} & \text{Equivalent Fractions} \\ [10pt]

                                  & = &  \dfrac{18}{9t^3} - \dfrac{t}{9t^3} & \text{Multiply}\\ [10pt]
																	
																	& = & \dfrac{18 - t}{9t^3} & \text{Subtract} \\ \end{array}\]
We find no common factors among the numerator and denominator so we are done.  

A second way to approach this problem is by factoring.  We can extend the concept of the `Polynomial G.C.F.' to these types of expressions and we can follow the same guidelines as set forth in the handout on polynomial arithmetic to factor out the G.C.F. of these two terms.  The key ideas to remember are that we take out each factor with the \textit{smallest} exponent and factoring is the same as dividing.  We first note that $2t^{-3} - (3t)^{-2}=  2t^{-3} - 3^{-2} t^{-2}$ and we see that the smallest power on $t$ is $-3$. Thus we want to factor out $t^{-3}$ from both terms.  It's clear that this will leave $2$ in the first term, but what about the second term?  Since factoring is the same as dividing, we would be dividing the second term by $t^{-3}$ which thanks to the properties of exponents is the same as \textit{multiplying} by $\frac{1}{t^{-3}} = t^3$.  The same holds for $3^{-2}$.  Even though there are no factors of $3$ in the first term, we can factor out $3^{-2}$ by multiplying it by $\frac{1}{3^{-2}} = 3^2 = 9$. We put these ideas together below.\[ \begin{array}{rclr}

2t^{-3} - (3t)^{-2} & = & 2t^{-3} - 3^{-2} t^{-2} &  \text{Properties of Exponents} \\ [5pt]
                    & = & 3^{-2} t^{-3} (2(3)^2 - t^{1}) & \text{Factor} \\ [5pt]
										& = & \dfrac{1}{3^2} \dfrac{1}{t^3} (18 - t) & \text{Rewrite}\\ [10pt]
										& = & \dfrac{18-t}{9t^3} & \text{Multiply} \\ \end{array}\]
																							
While both ways are valid, one may be more of a natural fit than the other depending on the circumstances and temperament of the student.

\item As with the previous example, we show two different yet equivalent ways to approach simplifying $10x(x-3)^{-1} + 5x^2(-1)(x-3)^{-2}$. First up is what we'll call the `common denominator approach' where we rewrite the negative exponents as fractions and proceed from there.

\begin{itemize}

\item  \textit{Common Denominator Approach}: \[ \begin{array}{rclr}

10x(x-3)^{-1} + 5x^2(-1)(x-3)^{-2} & = & \dfrac{10x}{x-3} + \dfrac{5x^2(-1)}{(x-3)^2} & \\ [10pt]
                                   & = & \dfrac{10x}{x-3} \cdot \dfrac{x-3}{x-3} - \dfrac{5x^2}{(x-3)^2} & \text{Equivalent Fractions} \\ [10pt]
																	 & = & \dfrac{10x(x-3)}{(x-3)^2} - \dfrac{5x^2}{(x-3)^2} & \text{Multiply} \\ [10pt]
																	 & = & \dfrac{10x(x-3) - 5x^2}{(x-3)^2} & \text{Subtract} \\ [10pt]
																	 & = & \dfrac{5x(2(x-3) - x)}{(x-3)^2} & \text{Factor out G.C.F.} \\ [10pt]
																	 & = & \dfrac{5x(2x-6-x)}{(x-3)^2} & \text{Distribute} \\ [10pt]
																	 & = & \dfrac{5x(x-6)}{(x-3)^2} & \text{Combine like terms} \\
																	
\end{array} \]

Both the numerator and the denominator are completely factored with no common factors so we are done.

\item  \textit{`Factoring Approach'}: In this case, the G.C.F. is $5x(x-3)^{-2}$.  Factoring this out of both terms gives: \[ \begin{array}{rclr}

10x(x-3)^{-1} + 5x^2(-1)(x-3)^{-2} & = & 5x(x-3)^{-2}(2(x-3)^{1} - x) & \text{Factor} \\ [8pt]
                                  & = & \dfrac{5x}{(x-3)^2} (2x-6 - x) & \text{Rewrite, distribute}\\ [12pt]
																	& = & \dfrac{5x(x-6)}{(x-3)^2} & \text{Multiply}\\ \end{array}\]

As expected, we got the same reduced fraction as before. \qed
\end{itemize}
																
\end{enumerate}

\end{ex}


Next, we review the solving of equations which involve rational expressions.  As with equations involving numeric fractions, our first step in solving equations with algebraic fractions is to clear denominators.  In doing so, we run the risk of introducing what are known as \textbf{extraneous} solutions - `answers' which don't satisfy the original equation.  As we illustrate the techniques used to solve these basic equations, see if you can find the step which creates the problem for us.

\pagebreak

\begin{ex}\label{rateqnreviewex} Solve the following equations.

\begin{multicols}{2}
\begin{enumerate}

\item  $1 + \dfrac{1}{x} = x$\vphantom{$\dfrac{t^3-2t+1}{t-1} = \dfrac{1}{2}t-1$}

\item  $\dfrac{t^3-2t+1}{t-1} = \dfrac{1}{2}t-1$



\setcounter{HW}{\value{enumi}}
\end{enumerate}

\end{multicols}

\begin{multicols}{2}
\begin{enumerate}
\setcounter{enumi}{\value{HW}}


\item  $\dfrac{3}{1 - w\sqrt{2}} - \dfrac{1}{2w+5} = 0$

\item $3(x^2+4)^{-1} + 3x(-1)(x^2+4)^{-2}(2x) = 0$\vphantom{$\dfrac{3}{1 - y\sqrt{2}} - \dfrac{1}{2y+5} = 0$}

\setcounter{HW}{\value{enumi}}
\end{enumerate}

\end{multicols}

\begin{multicols}{2}
\begin{enumerate}
\setcounter{enumi}{\value{HW}}

\item  Solve $x = \dfrac{2y+1}{y-3}$ for $y$. \vphantom{$\dfrac{1}{f} = \dfrac{1}{f_{\text{\tiny $1$}}} + \dfrac{1}{f_{\text{\tiny $2$}}}$ for $f_{1}$}

\item  Solve $\dfrac{1}{f} = \dfrac{1}{S_{\text{\tiny $1$}}} + \dfrac{1}{S_{\text{\tiny $2$}}}$ for $S_{1}$.

\setcounter{HW}{\value{enumi}}
\end{enumerate}

\end{multicols}

{\bf Solution.} 

\begin{enumerate}

\item   Our first step is to clear the fractions by multiplying both sides of the equation by $x$. In doing so, we are implicitly assuming $x \neq 0$; otherwise, we would have no guarantee that the resulting equation is equivalent to our original equation.\[ \begin{array}{rclr}

1 + \dfrac{1}{x} & = & x & \\ [8pt]

\left(1 + \dfrac{1}{x}\right) x & = & (x)x & \text{Provided $x \neq 0$} \\ [10pt]


1(x) + \dfrac{1}{x} (x) & = & x^2 & \text{Distribute} \\ [8pt]

x + \dfrac{x}{x} & = & x^2 & \text{Multiply} \\ [8pt]

x + 1 & = & x^2 &  \\

0 & = & x^2 - x - 1 & \text{Subtract $x$, subtract $1$} \\ [5pt]

x & = & \dfrac{-(-1) \pm \sqrt{(-1)^2 - 4(1)(-1)}}{2(1)} & \text{Quadratic Formula} \\

x & = & \dfrac{1 \pm \sqrt{5}}{2} & \text{Simplify} \\

\end{array}\]

We obtain two answers, $x = \frac{1 \pm \sqrt{5}}{2}$.  Neither of these are $0$ thus neither contradicts our assumption that $x \neq 0$.  The reader is invited to check both of these solutions.\footnote{The check relies on being able to `rationalize' the denominator  - a skill we haven't reviewed yet. (Come back after you've read the handout on radical expressions if you want to!)  Additionally, the positive solution to this equation is the famous \href{http://en.wikipedia.org/wiki/Golden_ratio}{\underline{Golden Ratio}}.}

\item  To solve the equation, we clear denominators.  Here, we need to assume $t-1 \neq 0$, or $t \neq 1$.\[ \begin{array}{rclr}

\dfrac{t^3-2t+1}{t-1} & = & \dfrac{1}{2}t-1 & \\ [8pt]

\left(\dfrac{t^3-2t+1}{t-1}\right) \cdot 2(t-1) & = & \left( \dfrac{1}{2}t-1 \right) \cdot 2(t-1) & \text{Provided $t \neq 1$} \\ [12pt]

\dfrac{(t^3-2t+1)(2\cancel{(t-1)})}{\cancel{(t-1)}}  & = & \dfrac{1}{\cancel{2}} t (\cancel{2}(t-1)) - 1(2(t-1))  & \text{Multiply, distribute} \\ [8pt]

2(t^3-2t+1) & = & t^2 - t - 2t + 2 & \text{Distribute} \\ [2pt]

2t^3 - 4t + 2 & = & t^2 -3t + 2 & \text{Distribute, combine like terms} \\ [2pt]

2t^3 -t^2 - t & = & 0 & \text{Subtract $t^2$, add $3t$, subtract $2$}\\ [2pt]

t(2t^2 -t - 1) & = & 0 & \text{Factor} \\ [2pt]

t = 0 & \text{or} & 2t^2 - t - 1 = 0 & \text{Zero Product Property}\\ [2pt]

t = 0 & \text{or} & (2t+1)(t-1) = 0 & \text{Factor}\\ [2pt]

t = 0 & \text{or} & 2t+1 = 0 \quad \text{or} \quad t-1 = 0 &\\ [2pt]

t & = & 0, \; -\dfrac{1}{2} \text{ or } 1 & \\

\end{array}\] We assumed that $t \neq 1$ in order to clear denominators.  Sure enough, the `solution' $t = 1$ doesn't check in the original equation since it causes division by $0$.  In this case, we call $t = 1$ an \textit{extraneous} solution.  Note that $t=1$ \textit{does} work in every equation \textit{after} we clear denominators.  In general, multiplying by variable expressions can produce these `extra' solutions, which is why checking our answers is always encouraged.\footnote{Contrast this with the fact that dividing a polynomial equation by a variable causes us to `lose' a solution.}  The other two solutions, $t = 0$ and $t = -\frac{1}{2}$, both work.

\item  As before, we begin by clearing denominators.  Here, we assume $1 - w\sqrt{2} \neq 0$ (so $w \neq \frac{1}{\sqrt{2}}$) and $2w+5 \neq 0$ (so $w \neq -\frac{5}{2}$).\[ \begin{array}{rclr}

 \dfrac{3}{1 - w\sqrt{2}} - \dfrac{1}{2w+5} & = &  0 & \\

\left(\dfrac{3}{1 - w\sqrt{2}} - \dfrac{1}{2w+5}\right)(1 - w\sqrt{2})(2w+5) & = &  0 (1 - w\sqrt{2})(2w+5)  & w \neq \dfrac{1}{\sqrt{2}}, -\dfrac{5}{2} \\ [12pt]

\dfrac{3\cancel{(1 - w\sqrt{2})}(2w+5) }{\cancel{(1 - w\sqrt{2})}}- \dfrac{1(1 - w\sqrt{2})\cancel{(2w+5)}}{\cancel{(2w+5)}} & = & 0 & \text{Distribute} \\ [12pt]

3(2w+5) - (1-w\sqrt{2}) & = & 0 & \\  \end{array}\]

The result is a \textit{linear} equation in $w$ so we gather the terms with $w$ on one side of the equation and put everything else on the other.  We factor out $w$ and divide by its coefficient. \[ \begin{array}{rclr}

3(2w+5) - (1-w\sqrt{2}) & = & 0 & \\

6w + 15 - 1 + w\sqrt{2} & = & 0 & \text{Distribute} \\

6w + w\sqrt{2} & = & -14 & \text{Subtract $14$} \\

(6 + \sqrt{2})w & = & -14 & \text{Factor} \\

w & = & -\dfrac{14}{6 + \sqrt{2}} & \text{Divide by $6 + \sqrt{2}$} \\ 

\end{array}\] This solution is different than our excluded values, $\frac{1}{\sqrt{2}}$ and $-\frac{5}{2}$, so we keep $w = -\frac{14}{6 + \sqrt{2}}$ as our final answer.  The reader is invited to check this in the original equation.

\item  To solve our next equation, we have two approaches to choose from:  we could rewrite the quantities with negative exponents as fractions and clear denominators, or we can factor.  We showcase each technique below.

\begin{itemize}

\item \textit{Clearing Denominators Approach}:  We rewrite the negative exponents as fractions and clear denominators.  In this case, we multiply both sides of the equation by $(x^2+4)^2$, which is never $0$. (Think about that for a moment.)  As a result, we need not exclude any $x$ values from our solution set.\[ \begin{array}{rclr}

3(x^2+4)^{-1} + 3x(-1)(x^2+4)^{-2}(2x)& = &  0 & \\ [8pt]

\dfrac{3}{x^2+4} + \dfrac{3x(-1)(2x)}{(x^2+4)^2} & = & 0 & \text{Rewrite} \\ [12pt]
\left(\dfrac{3}{x^2+4} - \dfrac{6x^2}{(x^2+4)^2} \right)(x^2+4)^2 & = & 0 (x^2+4)^2 & \text{Multiply} \\[12pt]

\dfrac{3\cancelto{(x^2+4)}{(x^2 + 4)^2}}{\cancel{(x^2+4)}}  - \dfrac{6x^2\cancel{(x^2+4)^2}}{\cancel{(x^2+4)^2}} & = & 0 & \text{Distribute} \\ [12pt]

3(x^2+4) - 6x^2 & = & 0 & \\ [2pt]

3x^2 + 12 - 6x^2 & = & 0 & \text{Distribute} \\ [2pt]

-3x^2 & = & -12 & \text{Combine like terms, subtract $12$} \\ [2pt]

x^2 & = & 4 & \text{Divide by $-3$} \\ [2pt]

x & = & \pm \sqrt{4} = \pm 2 & \text{Extract square roots} \\ 

\end{array} \]

We leave it to the reader to show both $x = -2$ and $x = 2$ satisfy the original equation.

\item  \textit{Factoring Approach}:  Since the equation is already set equal to $0$, we're ready to factor. Following the guidelines presented in Example \ref{rationalexpressionreviewex}, we factor out $3(x^2+4)^{-2}$ from both terms and look to see if more factoring can be done:\[ \begin{array}{rclr}

3(x^2+4)^{-1} + 3x(-1)(x^2+4)^{-2}(2x)& = &  0 & \\ [2pt]

3(x^2+4)^{-2}( (x^2+4)^{1} + x(-1)(2x)) & = & 0 & \text{Factor} \\ [2pt]

3(x^2+4)^{-2}( x^2 + 4 - 2x^2 ) & = & 0 & \\ [2pt]

3(x^2+4)^{-2}(4 - x^2) & = & 0 & \text{Gather like terms} \\ [2pt]

3(x^2+4)^{-2} = 0 & \text{or} & 4 - x^2 = 0 & \text{Zero Product Property} \\ [2pt]

\dfrac{3}{x^2+4} = 0 & \text{or} & 4 = x^2 & \\ \end{array} \]

The first equation yields no solutions (Think about this for a moment.) while the second gives us $x = \pm \sqrt{4} = \pm 2$ as before.


\end{itemize}

\item  We are asked to solve this equation for $y$ so we begin by clearing fractions with the stipulation that $y-3 \neq 0$ or $y \neq 3$.   We are left with a linear equation in the variable $y$.  To solve this, we gather the terms containing $y$ on one side of the equation and everything else on the other.  Next, we factor out the $y$ and divide by its coefficient, which in this case turns out to be $x-2$.  In order to divide by $x-2$, we stipulate $x - 2 \neq 0$ or, said differently, $x \neq 2$. \[ \begin{array}{rclr}

 x & = & \dfrac{2y+1}{y-3} & \\ [12pt]

x(y-3) & = & \left(\dfrac{2y+1}{y-3}\right)(y-3) & \text{Provided $y \neq 3$} \\ [12pt]

xy - 3x & = & \dfrac{(2y+1)\cancel{(y-3)}}{\cancel{(y-3)}} & \text{Distribute, multiply} \\ [12pt]

xy - 3x & = & 2y + 1 & \\ [2pt]

xy - 2y & = & 3x+1 & \text{Add $3x$, subtract $2y$} \\ [2pt]

y(x-2) & = & 3x+1 & \text{Factor} \\ [2pt]

y & = & \dfrac{3x+1}{x-2} & \text{Divide provided $x \neq 2$} \\

\end{array}\]

We highly encourage the reader to check the answer algebraically to see where the restrictions on $x$ and $y$ come into play.\footnote{It involves simplifying a compound fraction!}

\item  Our last example comes from physics and the world of photography.\footnote{See this article on \href{https://en.wikipedia.org/wiki/Focal_length}{\underline{focal length}}.}  We take a moment here to note that while superscripts in mathematics indicate exponents (powers), subscripts are used primarily to distinguish one or more variables.  In this case, $S_{\text{\tiny $1$}}$ and $S_{\text{\tiny $2$}}$ are two \textit{different} variables (much like $x$ and $y$) and we treat them as such. Our first step is to clear denominators by multiplying both sides by $f S_{\text{\tiny $1$}} S_{\text{\tiny $2$}}$ - provided each is nonzero.  We end up with an equation which is linear in $S_{\text{\tiny $1$}}$ so we proceed as in the previous example.  \[ \begin{array}{rclr}

\dfrac{1}{f} & = & \dfrac{1}{S_{\text{\tiny $1$}}} + \dfrac{1}{S_{\text{\tiny $2$}}} & \\ [12pt]


\left(\dfrac{1}{f}\right)(fS_{\text{\tiny $1$}}S_{\text{\tiny $2$}}) & = & \left(\dfrac{1}{S_{\text{\tiny $1$}}} + \dfrac{1}{S_{\text{\tiny $2$}}}\right) (fS_{\text{\tiny $1$}}S_{\text{\tiny $2$}}) & \text{Provided $f \neq 0$, $S_{\text{\tiny $1$}} \neq 0$, $S_{\text{\tiny $2$}}\neq 0$} \\ [12pt]

\dfrac{fS_{\text{\tiny $1$}}S_{\text{\tiny $2$}}}{f} & = & \dfrac{fS_{\text{\tiny $1$}}S_{\text{\tiny $2$}}}{S_{\text{\tiny $1$}}} + \dfrac{fS_{\text{\tiny $1$}}S_{\text{\tiny $2$}}}{S_{\text{\tiny $2$}}} & \text{Multiply, distribute} \\ [12pt]


\dfrac{\cancel{f}S_{\text{\tiny $1$}}S_{\text{\tiny $2$}}}{\cancel{f}} & = & \dfrac{f\cancel{S_{\text{\tiny $1$}}}S_{\text{\tiny $2$}}}{\cancel{S_{\text{\tiny $1$}}}} + \dfrac{fS_{\text{\tiny $1$}}\cancel{S_{\text{\tiny $2$}}}}{\cancel{S_{\text{\tiny $2$}}}} & \text{Cancel} \\ [12pt]

S_{\text{\tiny $1$}}S_{\text{\tiny $2$}} & = & f S_{\text{\tiny $2$}} + fS_{\text{\tiny $1$}} & \\ [3pt]

S_{\text{\tiny $1$}}S_{\text{\tiny $2$}}  - fS_{\text{\tiny $1$}} & = & f S_{\text{\tiny $2$}}   &  \text{Subtract $fS_{\text{\tiny $1$}}$} \\ [3pt]

S_{\text{\tiny $1$}}(S_{\text{\tiny $2$}} - f) & = & f S_{\text{\tiny $2$}} & \text{Factor}  \\ [5pt]

S_{\text{\tiny $1$}} & = & \dfrac{f S_{\text{\tiny $2$}}}{S_{\text{\tiny $2$}} - f} & \text{Divide provided  $S_{\text{\tiny $2$}} \neq f$}  \\

\end{array}\]

As always, the reader is highly encouraged to check the answer.\footnote{\ldots and see what the restriction $S_{\text{\tiny $2$}} \neq f$ means in terms of focusing a camera!}  \qed

\end{enumerate}

\end{ex}

\newpage

\section{Exercises}

In Exercises \ref{ratsimpfirst} - \ref{ratsimplast}, perform the indicated operations and simplify.

\begin{multicols}{3}
\begin{enumerate}

\item $\dfrac{x^2-9}{x^2} \cdot \dfrac{3x}{x^2-x-6}$\vphantom{$\dfrac{4y-y^2}{2y+1} \div \dfrac{y^2-16}{2y^2-5y-3}$}\label{ratsimpfirst}
\item $\dfrac{t^2-2t}{t^2+1} \div (3t^2 - 2t - 8)$\vphantom{$\dfrac{4y-y^2}{2y+1} \div \dfrac{y^2-16}{2y^2-5y-3}$}
\item $\dfrac{4y-y^2}{2y+1} \div \dfrac{y^2-16}{2y^2-5y-3}$

\setcounter{HW}{\value{enumi}}
\end{enumerate}
\end{multicols}

\begin{multicols}{3}
\begin{enumerate}
\setcounter{enumi}{\value{HW}}

\item  $\dfrac{x}{3x-1} - \dfrac{1-x}{3x-1}$\vphantom{$\dfrac{2-y}{3y} - \dfrac{1-y}{3y} + \dfrac{y^2-1}{3y}$}
\item  $\dfrac{2}{w-1} - \dfrac{w^2+1}{w-1}$\vphantom{$\dfrac{2-y}{3y} - \dfrac{1-y}{3y} + \dfrac{y^2-1}{3y}$}
\item  $\dfrac{2-y}{3y} - \dfrac{1-y}{3y} + \dfrac{y^2-1}{3y}$
 

\setcounter{HW}{\value{enumi}}
\end{enumerate}
\end{multicols}

\begin{multicols}{3}
\begin{enumerate}
\setcounter{enumi}{\value{HW}}

\item  $b+ \dfrac{1}{b-3} - 2$\vphantom{$\dfrac{m^2}{m^2-4} + \dfrac{1}{2-m}$}
\item  $\dfrac{2x}{x-4} - \dfrac{1}{2x+1}$\vphantom{$\dfrac{m^2}{m^2-4} + \dfrac{1}{2-m}$}
\item  $\dfrac{m^2}{m^2-4} + \dfrac{1}{2-m}$

\setcounter{HW}{\value{enumi}}
\end{enumerate}
\end{multicols}

\begin{multicols}{3}
\begin{enumerate}
\setcounter{enumi}{\value{HW}}

\item $\dfrac{\dfrac{2}{x} - 2}{x-1}$\vphantom{$\dfrac{\dfrac{1}{x+h} - \dfrac{1}{x}}{h}$}
\item $\dfrac{\dfrac{3}{2-h} - \dfrac{3}{2}}{h}$\vphantom{$\dfrac{\dfrac{1}{x+h} - \dfrac{1}{x}}{h}$}
\item $\dfrac{\dfrac{1}{x+h} - \dfrac{1}{x}}{h}$

\setcounter{HW}{\value{enumi}}
\end{enumerate}
\end{multicols}


\begin{multicols}{3}
\begin{enumerate}
\setcounter{enumi}{\value{HW}}

\item  $3w^{-1} - (3w)^{-1}$
\item  $-2y^{-1}  + 2(3-y)^{-2}$
\item  $3(x-2)^{-1} - 3x(x-2)^{-2}$

 
\setcounter{HW}{\value{enumi}}
\end{enumerate}
\end{multicols}

\begin{multicols}{3}
\begin{enumerate}
\setcounter{enumi}{\value{HW}}

\item $\dfrac{t^{-1} + t^{-2}}{t^{-3}}$  
\item $\dfrac{2(3+h)^{-2} - 2(3)^{-2}}{h}$ \vphantom{$\dfrac{t^{-1} + t^{-2}}{t^{-3}}$}
\item $\dfrac{(7-x-h)^{-1} - (7-x)^{-1}}{h}$ \vphantom{$\dfrac{t^{-1} + t^{-2}}{t^{-3}}$} \label{ratsimplast}


\setcounter{HW}{\value{enumi}}
\end{enumerate}
\end{multicols}

\vspace{-0.15in}

In Exercises \ref{rateqnfirst} - \ref{rateqnlast}, find all real solutions.  Be sure to check for extraneous solutions.

\begin{multicols}{3}
\begin{enumerate}
\setcounter{enumi}{\value{HW}}

\item $\dfrac{x}{5x + 4} = 3$\vphantom{$\dfrac{1}{w + 3} + \dfrac{1}{w - 3} = \dfrac{w^{2} - 3}{w^{2} - 9}$} \label{rateqnfirst}
\item $\dfrac{3y - 1}{y^{2} + 1} = 1$\vphantom{$\dfrac{1}{w + 3} + \dfrac{1}{w - 3} = \dfrac{w^{2} - 3}{w^{2} - 9}$}
\item $\dfrac{1}{w + 3} + \dfrac{1}{w - 3} = \dfrac{w^{2} - 3}{w^{2} - 9}$

\setcounter{HW}{\value{enumi}}
\end{enumerate}
\end{multicols}

\begin{multicols}{3}
\begin{enumerate}
\setcounter{enumi}{\value{HW}}


\item $\dfrac{2x + 17}{x + 1} = x + 5$\vphantom{$\dfrac{-y^{3} + 4y}{y^{2} - 9} = 4y$}
\item $\dfrac{t^{2} - 2t + 1}{t^{3} + t^{2} - 2t} = 1$\vphantom{$\dfrac{-y^{3} + 4y}{y^{2} - 9} = 4y$}
\item $\dfrac{-y^{3} + 4y}{y^{2} - 9} = 4y$  

\setcounter{HW}{\value{enumi}}
\end{enumerate}
\end{multicols}

\begin{multicols}{3}
\begin{enumerate}
\setcounter{enumi}{\value{HW}}


\item $w + \sqrt{3} = \dfrac{3w - w^3}{w - \sqrt{3}}$\vphantom{ $\dfrac{x^2}{(1 + x\sqrt{3})^2} = 3$}
\item $\dfrac{2}{x\sqrt{2} - 1}  - 1 = \dfrac{3}{x \sqrt{2} + 1}$\vphantom{ $\dfrac{x^2}{(1 + x\sqrt{3})^2} = 3$}
\item $\dfrac{x^2}{(1 + x\sqrt{3})^2} = 3$ \label{rateqnlast}

\setcounter{HW}{\value{enumi}}
\end{enumerate}
\end{multicols}



In Exercises \ref{absratfirst} - \ref{absratlast}, use properties of absolute values along with the techniques in this section to find all real solutions.

\begin{multicols}{3}
\begin{enumerate}
\setcounter{enumi}{\value{HW}}

\item $\left|\dfrac{3n}{n-1}  \right| = 3$\vphantom{$\left| \dfrac{2t}{4-t^2}\right| = \left|\dfrac{2}{t-2}\right|$} \label{absratfirst}
\item $\left| \dfrac{2x}{x^2-1}\right| = 2$\vphantom{$\left| \dfrac{2t}{4-t^2}\right| = \left|\dfrac{2}{t-2}\right|$}
\item $\left| \dfrac{2t}{4-t^2}\right| = \left|\dfrac{2}{t-2}\right|$ \label{absratlast}

\setcounter{HW}{\value{enumi}}
\end{enumerate}
\end{multicols}


In Exercises \ref{solveratcalcfirst} - \ref{solveratcalclast}, find all real solutions and use a calculator to approximate your answers, rounded to two decimal places.


\begin{multicols}{3}
\begin{enumerate}
\setcounter{enumi}{\value{HW}}


\item $2.41 = \dfrac{0.08}{4 \pi R^2}$ \label{solveratcalcfirst}
\item $\dfrac{x^2}{(2.31 -x)^2} = 0.04$
\item $1 - \dfrac{6.75 \times 10^{16}}{c^2} = \dfrac{1}{4}$ \label{solveratcalclast}

\setcounter{HW}{\value{enumi}}
\end{enumerate}
\end{multicols}


\newpage

In Exercises \ref{litrateqnfirst} - \ref{litrateqnlast}, solve the given equation for the indicated variable.

\begin{multicols}{2}
\begin{enumerate}
\setcounter{enumi}{\value{HW}}


\item Solve for $y$:  $\dfrac{1-2y}{y+3} = x$ \label{litrateqnfirst}

\item Solve for $y$: $x = 3 - \dfrac{2}{1-y}$ \vphantom{$\dfrac{1-2y}{y+3} = x$}

\setcounter{HW}{\value{enumi}}
\end{enumerate}
\end{multicols}



\begin{multicols}{2}
\begin{enumerate}
\setcounter{enumi}{\value{HW}}

\item\hspace{-0.1in}\footnote{Recall: subscripts on variables have no intrinsic mathematical meaning; they're just used to distinguish one variable from another.  In other words, treat quantities like `$V_{\text{\tiny $1$}}$' and `$V_{\text{\tiny $2$}}$'  as two different variables as you would `$x$' and `$y$.'}Solve for $T_{\text{\tiny $2$}}$:  $\dfrac{V_{\text{\tiny $1$}}}{T_{\text{\tiny $1$}}} = \dfrac{V_{\text{\tiny $2$}}}{T_{\text{\tiny $2$}}}$


\item  Solve for $t_{\text{\tiny $0$}}$:  $\dfrac{t_{\text{\tiny $0$}}}{1-t_{\text{\tiny $0$}}t_{\text{\tiny $1$}}} = 2$ 

\setcounter{HW}{\value{enumi}}
\end{enumerate}
\end{multicols}

\begin{multicols}{2}
\begin{enumerate}
\setcounter{enumi}{\value{HW}}


\item  Solve for $x$:  $\dfrac{1}{x - v_{\text{\tiny $r$}}} + \dfrac{1}{x + v_{\text{\tiny $r$}}} = 5$

\item Solve for $R$:  $P = \dfrac{25R}{(R+4)^2}$ \label{litrateqnlast}

\setcounter{HW}{\value{enumi}}
\end{enumerate}
\end{multicols}

\newpage

\section{Answers}

\begin{multicols}{3}
\begin{enumerate}

\item $\dfrac{3(x+3)}{x(x+2)}$, $x \neq 3$\vphantom{$-\dfrac{y(y-3)}{y+4}$, $y \neq -\dfrac{1}{2}, 3, 4$}
\item $\dfrac{t}{(3t+4)(t^2+1)}$, $t \neq 2$\vphantom{$-\dfrac{y(y-3)}{y+4}$, $y \neq -\dfrac{1}{2}, 3, 4$}
\item $-\dfrac{y(y-3)}{y+4}$, $y \neq -\dfrac{1}{2}, 3, 4$ 

\setcounter{HW}{\value{enumi}}
\end{enumerate}
\end{multicols}

\begin{multicols}{3}
\begin{enumerate}
\setcounter{enumi}{\value{HW}}

\item  $\dfrac{2x-1}{3x-1}$
\item  $-w-1$, $w \neq 1$\vphantom{$\dfrac{2x-1}{3x-1}$}
\item  $\dfrac{y}{3}$, $y \neq 0$\vphantom{$\dfrac{2x-1}{3x-1}$}
 

\setcounter{HW}{\value{enumi}}
\end{enumerate}
\end{multicols}

\begin{multicols}{3}
\begin{enumerate}
\setcounter{enumi}{\value{HW}}

\item  $\dfrac{b^2-5b+7}{b-3}$\vphantom{$\dfrac{4x^2+x+4}{(x-4)(2x+1)}$}
\item  $\dfrac{4x^2+x+4}{(x-4)(2x+1)}$
\item  $\dfrac{m+1}{m+2}$, $m \neq 2$\vphantom{$\dfrac{4x^2+x+4}{(x-4)(2x+1)}$}

\setcounter{HW}{\value{enumi}}
\end{enumerate}
\end{multicols}

\begin{multicols}{3}
\begin{enumerate}
\setcounter{enumi}{\value{HW}}

\item $-\dfrac{2}{x}$, $x \neq 1$\vphantom{$\dfrac{3}{4-2h}$, $h \neq 0$}
\item $\dfrac{3}{4-2h}$, $h \neq 0$
\item $-\dfrac{1}{x(x+h)}$, $h \neq 0$\vphantom{$\dfrac{3}{4-2h}$, $h \neq 0$}

\setcounter{HW}{\value{enumi}}
\end{enumerate}
\end{multicols}


\begin{multicols}{3}
\begin{enumerate}
\setcounter{enumi}{\value{HW}}

\item  $\dfrac{8}{3w}$\vphantom{$-\dfrac{2(y^2-7y+9)}{y(y-3)^2}$}
\item  $-\dfrac{2(y^2-7y+9)}{y(y-3)^2}$
\item  $-\dfrac{6}{(x-2)^2}$\vphantom{$-\dfrac{2(y^2-7y+9}{y(y-3)^2}$}

 
\setcounter{HW}{\value{enumi}}
\end{enumerate}
\end{multicols}

\begin{multicols}{3}
\begin{enumerate}
\setcounter{enumi}{\value{HW}}

\item $t^2+t$, $t \neq 0$\vphantom{$\dfrac{1}{(7-x)(7-x-h)}$, $h \neq 0$}  
\item $-\dfrac{2(h+6)}{9(h+3)^2}$, $h \neq 0$ \vphantom{$\dfrac{1}{(7-x)(7-x-h)}$, $h \neq 0$}
\item $\dfrac{1}{(7-x)(7-x-h)}$, $h \neq 0$ 

\setcounter{HW}{\value{enumi}}
\end{enumerate}
\end{multicols}



\begin{multicols}{3}
\begin{enumerate}
\setcounter{enumi}{\value{HW}}

\item $x = -\dfrac{6}{7}$
\item $y = 1, 2$ \vphantom{$x = -\dfrac{6}{7}$}
\item $w = -1$ \vphantom{$x = -\dfrac{6}{7}$}

\setcounter{HW}{\value{enumi}}
\end{enumerate}
\end{multicols}

\begin{multicols}{3}
\begin{enumerate}
\setcounter{enumi}{\value{HW}}


\item $x=-6, 2$
\item No solution.
\item $y = 0, \pm 2\sqrt{2}$  

\setcounter{HW}{\value{enumi}}
\end{enumerate}
\end{multicols}

\begin{multicols}{3}
\begin{enumerate}
\setcounter{enumi}{\value{HW}}


\item $w = -\sqrt{3}, -1$\vphantom{$x = -\dfrac{\sqrt{3}}{2}, -\dfrac{\sqrt{3}}{4}$}
\item $x = -\dfrac{3\sqrt{2}}{2}, \sqrt{2}$\vphantom{$x = -\dfrac{\sqrt{3}}{2}, -\dfrac{\sqrt{3}}{4}$}
\item $x = -\dfrac{\sqrt{3}}{2}, -\dfrac{\sqrt{3}}{4}$

\setcounter{HW}{\value{enumi}}
\end{enumerate}
\end{multicols}


\begin{multicols}{3}
\begin{enumerate}
\setcounter{enumi}{\value{HW}}

\item $n = \dfrac{1}{2}$\vphantom{$x = \dfrac{1 \pm \sqrt{5}}{2}, \dfrac{-1 \pm \sqrt{5}}{2}$}
\item $x = \dfrac{1 \pm \sqrt{5}}{2}, \dfrac{-1 \pm \sqrt{5}}{2}$
\item $t = -1$\vphantom{$x = \dfrac{1 \pm \sqrt{5}}{2}, \dfrac{-1 \pm \sqrt{5}}{2}$}

\setcounter{HW}{\value{enumi}}
\end{enumerate}
\end{multicols}


\begin{multicols}{2}
\begin{enumerate}
\setcounter{enumi}{\value{HW}}


\item $R = \pm \sqrt{\dfrac{0.08}{9.64 \pi}} \approx \pm 0.05$ 
\item $x = -\dfrac{231}{400} \approx -0.58$, $x = \dfrac{77}{200} \approx 0.38$ \vphantom{ $R = \pm \sqrt{\dfrac{0.08}{9.64 \pi}} \approx \pm 0.05$ }


\setcounter{HW}{\value{enumi}}
\end{enumerate}
\end{multicols}

\begin{enumerate}
\setcounter{enumi}{\value{HW}}
\item $c = \pm \sqrt{\dfrac{4 \cdot 6.75 \times 10^{16}}{3}} = \pm 3.00 \times 10^{8}$ (You actually didn't 
\textit{need} a calculator for this!)

\setcounter{HW}{\value{enumi}}
\end{enumerate}


\begin{multicols}{2}
\begin{enumerate}
\setcounter{enumi}{\value{HW}}


\item $y = \dfrac{1 - 3x}{x+2}$, $y \neq -3$, $x \neq -2$

\item $y = \dfrac{x-1}{x-3}$, $y \neq 1$, $x \neq 3$

\setcounter{HW}{\value{enumi}}
\end{enumerate}
\end{multicols}



\begin{multicols}{2}
\begin{enumerate}
\setcounter{enumi}{\value{HW}}

\item $T_{\text{\tiny $2$}} = \dfrac{V_{\text{\tiny $2$}}T_{\text{\tiny $1$}}}{V_{\text{\tiny $1$}}}$, $T_{\text{\tiny $1$}} \neq 0, T_{\text{\tiny $2$}} \neq 0, V_{\text{\tiny $1$}} \neq 0$


\item  $t_{\text{\tiny $0$}} = \dfrac{2}{2t_{\text{\tiny $1$}} + 1}$, $t_{\text{\tiny $1$}} \neq -\dfrac{1}{2}$\vphantom{$T_{\text{\tiny $2$}} = \dfrac{V_{\text{\tiny $2$}}T_{\text{\tiny $1$}}}{V_{\text{\tiny $1$}}}$, $T_{\text{\tiny $1$}}, T_{\text{\tiny $2$}}, V_{\text{\tiny $1$}} \neq 0$}

\setcounter{HW}{\value{enumi}}
\end{enumerate}
\end{multicols}

\begin{enumerate}
\setcounter{enumi}{\value{HW}}


\item  $x = \dfrac{1 \pm \sqrt{25v_{\text{\tiny $r$}}^2+1}}{5}$, $x \neq \pm v_{\text{\tiny $r$}}$.

\item $R= \dfrac{-(8P-25) \pm \sqrt{(8P-25)^2 - 64P^2}}{2P} = \dfrac{(25-8P) \pm 5 \sqrt{25-16P}}{2P}$, $P \neq 0$, $R \neq -4$

\setcounter{HW}{\value{enumi}}
\end{enumerate}


\end{document}

\newpage

\section{Radicals and Equations}

\mfpicnumber{1}
\mfpverbtex{%
\documentclass[10pt]{article}
\usepackage{amsmath}
\usepackage[sfdefault,lf]{carlito}
\usepackage[T1]{fontenc}
\renewcommand*\oldstylenums[1]{\carlitoOsF #1}
\begin{document}
}
\opengraphsfile{RadEqus}

\setcounter{footnote}{0}

\label{RadEqus}

In this section we review simplifying expressions and solving equations involving radicals.  In addition to the product, quotient and power rules stated in Theorem \ref{radicalprops0} in Section \ref{RealNumberArithmetic},  we present the following result which states that $n^{\text{th}}$ roots and $n^{\text{th}}$ powers more or less `undo' each other.

\medskip

\colorbox{ResultColor}{\bbm

\begin{thm}\label{simplifyradicals} \textbf{Simplifying $n^{\text{th}}$ powers of $n^{\text{th}}$ roots}:  Suppose $n$ is a natural number, $a$ is a real number and $\sqrt[n]{a}$ is a real number.  Then

\begin{itemize}

\item $(\sqrt[n]{a})^{n} = a$

\item  if $n$ is odd, $\sqrt[n]{a^{n}} = a$; if $n$ is even, $\sqrt[n]{a^{n}} = |a|$.

\end{itemize}

\end{thm}

\ebm} 

\medskip

Since $\sqrt[n]{a}$ is \textit{defined} so that $(\sqrt[n]{a})^n = a$,  the first claim in the theorem is just a re-wording of  Definition \ref{principalnthrootdefn0}.  The second part of the theorem breaks down along odd/even exponent lines due to how exponents affect negatives. To see this, consider the specific cases of $\sqrt[3]{(-2)^3}$ and $\sqrt[4]{(-2)^{4}}$.  

\medskip

In the first case,  $\sqrt[3]{(-2)^3} =\sqrt[3]{-8} = -2$, so we have an instance of when $\sqrt[n]{a^{n}} = a$.  The reason that the cube root `undoes' the third power in $\sqrt[3]{(-2)^3} = -2$ is because the negative is preserved when raised to the third (odd) power.  In  $\sqrt[4]{(-2)^{4}}$,  the negative `goes away' when raised to the fourth (even) power:$\sqrt[4]{(-2)^{4}} = \sqrt[4]{16}$.  According to Definition \ref{principalnthrootdefn0}, the fourth root is defined to give only \textit{non-negative} numbers, so $\sqrt[4]{16} = 2$.  Here we have a case where $\sqrt[4]{(-2)^{4}} = 2 = |-2|$, not $-2$. 

\medskip

In general, we need the absolute values to simplify $\sqrt[n]{a^{n}}$ only when $n$ is even because a negative to an even power is always positive.  In particular, $\sqrt{x^2} = |x|$, not just `$x$' (unless we \textit{know} $x \geq 0$.)\footnote{If this discussion sounds familiar, see the discussion following Definition \ref{rationalexponentdefn0} and the discussion following `Extracting the Square Root' on page \pageref{extractingthesquareroot}.}  We practice these formulas in the following example.

\begin{ex}\label{simplifyradexpressions}  Perform the indicated operations and simplify.

\begin{multicols}{3}

\begin{enumerate}

\item  $\sqrt{x^{2} + 1}$\vphantom{$\sqrt[4]{\dfrac{\pi r^{4}}{L^{8}}}$}

\item  $\sqrt{t^2-10t+25}$\vphantom{$\sqrt[4]{\dfrac{\pi r^{4}}{L^{8}}}$}

%\item  $\sqrt[3]{48x^{14}}$\vphantom{$\sqrt[4]{\dfrac{\pi r^{4}}{L^{8}}}$}

\item  $\sqrt[4]{\dfrac{\pi r^{4}}{L^{8}}}$


\setcounter{HW}{\value{enumi}}

\end{enumerate}

\end{multicols}

\begin{multicols}{2}

\begin{enumerate}
\setcounter{enumi}{\value{HW}}

\item $2x \sqrt[3]{x^2-4} + 2\left(\dfrac{1}{2(\sqrt[3]{x^2-4})^2}\right)  (2x)$ 

\item  $\sqrt{(\sqrt{18y} - \sqrt{8y})^2 + (\sqrt{20} - \sqrt{80})^2}$ \vphantom{$2x \sqrt[3]{x^2-4} + 2\left(\dfrac{1}{2(\sqrt[3]{x^2-4})^2}\right)  (2x)$ }

\end{enumerate}

\end{multicols}

{\bf Solution.}

\begin{enumerate}

\item We told you back on page \pageref{donotdistributeexponents} that roots do not `distribute' across addition and since $x^{2} + 1$ cannot be factored over the real numbers, $\sqrt{x^{2} + 1}$ cannot be simplified.  It may seem silly to start with this example but it is extremely important that you understand what manoeuvres are legal and which ones are not.\footnote{You really do need to understand this otherwise horrible evil will plague your future studies in Math.  If you say something totally wrong like $\sqrt{x^{2} + 1} = x + 1$ then you may never pass Calculus.  PLEASE be careful!}

\item Again we note that $\sqrt{t^2-10t+25}  \neq \sqrt{t^2} - \sqrt{10t} + \sqrt{25}$, since radicals do \textit{not} distribute across addition and subtraction.\footnote{Let $t = 1$ and see what happens to $\sqrt{t^2-10t+25}$  versus $\sqrt{t^2} - \sqrt{10t} + \sqrt{25}$.}  In this case, however, we can factor the radicand and simplify as \[ \sqrt{t^2 - 10t + 25} = \sqrt{(t-5)^2} = |t-5| \]
Without knowing more about the value of $t$, we have no idea if $t-5$ is positive or negative so $|t-5|$ is our final answer.\footnote{In general,  $|t-5| \neq |t| - |5|$ and  $|t-5| \neq t + 5$ so watch what you're doing!}


\item  In this example, we are looking for perfect fourth powers in the radicand.  In the numerator $r^4$ is clearly a perfect fourth power.  For the denominator, we take the power on the $L$, namely $12$, and divide by $4$ to get $3$.  This means $L^{8} = L^{2\cdot 4} = (L^2)^{4}$.  We get \[ \begin{array}{rclr}

\sqrt[4]{\dfrac{\pi r^{4}}{L^{12}}} & = & \dfrac{\sqrt[4]{\pi r^{4}}}{\sqrt[4]{L^{12}}} & \text{Quotient Rule of Radicals} \\ [12pt]

                                    & = & \dfrac{\sqrt[4]{\pi}\sqrt[4]{r^{4}}}{\sqrt[4]{(L^2)^{4}}} & \text{Product Rule of Radicals} \\ [12pt]
																		& = & \dfrac{\sqrt[4]{\pi}|r|}{|L^2|} & \text{Simplify} \\
																	
\end{array}\]  Without more information about $r$, we cannot simplify $|r|$ any further.  However, we can simplify $|L^2|$.  Regardless of the choice of $L$, $L^2 \geq 0$. Actually, $L^2 > 0$ because $L$ is in the denominator which means $L \neq 0$. Hence, $|L^2| = L^2$.  Our answer simplifies to: \[ \dfrac{\sqrt[4]{\pi}|r|}{|L^2|} = \dfrac{|r|\sqrt[4]{\pi}}{L^2} \]

\item After a quick cancellation (two of the $2$'s in the second term) we need to obtain a common denominator.  Since we can view the first term as having a denominator of $1$,  the common denominator is precisely the denominator of the second term, namely $(\sqrt[3]{x^2-4})^2$.  With common denominators, we proceed to add the two fractions.  Our last step is to factor the numerator to see if there are any cancellation opportunities with the denominator.\[ \begin{array}{rclr}

2x \sqrt[3]{x^2-4} + 2\left(\dfrac{1}{2(\sqrt[3]{x^2-4})^2}\right)  (2x) & = & 2x \sqrt[3]{x^2-4} + \cancel{2}\left(\dfrac{1}{\cancel{2}(\sqrt[3]{x^2-4})^2}\right)  (2x) & \text{Reduce}\\ [12pt]

& = & 2x \sqrt[3]{x^2-4} + \dfrac{2x}{(\sqrt[3]{x^2-4})^2} & \text{Mutiply} \\[12pt]

& = & (2x \sqrt[3]{x^2-4}) \cdot \dfrac{(\sqrt[3]{x^2-4})^2}{(\sqrt[3]{x^2-4})^2} + \dfrac{2x}{(\sqrt[3]{x^2-4})^2} & \hspace*{-.1in}\text{\small Equivalent} \\ [-8pt]
&   &                                                                                                               & \text{\small fractions} \\

& = & \dfrac{2x(\sqrt[3]{x^2-4})^3}{(\sqrt[3]{x^2-4})^2} + \dfrac{2x}{(\sqrt[3]{x^2-4})^2} & \text{Multiply}\\[12pt]

& = & \dfrac{2x(x^2-4)}{(\sqrt[3]{x^2-4})^2} + \dfrac{2x}{(\sqrt[3]{x^2-4})^2} & \text{Simplify}\\ [12pt]

& = & \dfrac{2x(x^2-4) + 2x}{(\sqrt[3]{x^2-4})^2} & \text{Add} \\ [12pt]


& = & \dfrac{2x(x^2-4 +1)}{(\sqrt[3]{x^2-4})^2} & \text{Factor}\\ [12pt]


& = & \dfrac{2x(x^2-3)}{(\sqrt[3]{x^2-4})^2} & \\

\end{array}\] We cannot reduce this any further because $x^2 - 3$ is irreducible over the rational numbers. 


\item  We begin by working inside each set of parentheses, using the product rule for radicals and combining like terms. (Note that the square root does \textbf{not} cancel with the two squared terms!)\[ \begin{array}{rclr}


 \sqrt{(\sqrt{18y} - \sqrt{8y})^2 + (\sqrt{20} - \sqrt{80})^2} & = & \sqrt{(\sqrt{9\cdot 2y} - \sqrt{4 \cdot 2y})^2 + (\sqrt{4\cdot 5} - \sqrt{16 \cdot 5})^2} & \\[8pt]

& = & \sqrt{(\sqrt{9} \sqrt{2y} - \sqrt{4}\sqrt{2y})^2 + (\sqrt{4}\sqrt{5} - \sqrt{16}\sqrt{5})^2} & \\[8pt]


& = & \sqrt{(3\sqrt{2y} - 2\sqrt{2y})^2 + (2\sqrt{5} - 4\sqrt{5})^2} & \\[8pt]

& = & \sqrt{(\sqrt{2y})^2 + (-2\sqrt{5})^2} & \\[8pt]


& = & \sqrt{2y + (-2)^2(\sqrt{5})^2} & \\[8pt]


& = & \sqrt{2y + 4\cdot 5} & \\[8pt]

& = & \sqrt{2y + 20} & \\ 

\end{array} \]

To see if this simplifies any further, we factor the radicand:  $\sqrt{2y+20} = \sqrt{2(y+10)}$.  Finding no perfect square factors, we are done. \qed

\end{enumerate}


\end{ex}


Theorem \ref{simplifyradicals} allows us to generalize the process of `Extracting Square Roots' to `Extracting $n^{\text{th}}$ roots' which in turn allows us to solve equations of the form $X^n  = c$.

\phantomsection
\label{extractingnthroots}

\medskip

\colorbox{ResultColor}{\bbm

\centerline{\textbf{Extracting $n^{\text{th}}$ roots:}}

\begin{itemize}

\item If $c$ is a real number and $n$ is odd then the real number solution to $X^{n} = c$ is $X = \sqrt[n]{c}$.

\item  If $c \geq 0$ and $n$ is even then the real number solutions to $X^{n} = c$ are $X = \pm \sqrt[n]{c}$.

\textbf{Note:} If $c < 0$ and $n$ is even then $X^{n} = c$ has no real number solutions.

\end{itemize}

\ebm}

\medskip

Essentially, we solve $X^{n} = c$ by `taking the $n^{\text{th}}$ root' of both sides:  $\sqrt[n]{X^{n}} = \sqrt[n]{c}$. Simplifying the left side gives us just $X$ if $n$ is odd or $|X|$ if $n$ is even.  In the first case,  $X =  \sqrt[n]{c}$, and in the second, $X = \pm \sqrt[n]{c}$.  Putting this together with the other part of Theorem \ref{simplifyradicals}, namely $(\sqrt[n]{a})^n = a$, gives us a strategy for solving equations which involve $n^{\text{th}}$ powers and $n^{\text{th}}$ roots. 

\phantomsection
\label{solvepowerandradicaleqns}

\medskip

\colorbox{ResultColor}{\bbm
\centerline{\textbf{Strategies for Power and Radical Equations}}

\begin{itemize}

\item  If the equation involves an $n^{\text{th}}$ power and the variable appears in only one term, isolate the term with the $n^{\text{th}}$ power and extract $n^{\text{th}}$ roots.

\item  If the equation involves an $n^{\text{th}}$ root and the variable appears in that $n^{\text{th}}$ root, isolate the $n^{\text{th}}$ root and raise both sides of the equation to the $n^{\text{th}}$ power.

\textbf{Note:}  When raising both sides of an equation to an \textit{even} power, be sure to check for extraneous solutions.

\end{itemize}

\ebm}

\medskip

The note about `extraneous solutions' can be demonstrated by the basic equation: $\sqrt{x} = -2$.  This equation has no solution since, by definition, $\sqrt{x} \geq 0$ for all real numbers $x$.  However, if we square both sides of this equation, we get $(\sqrt{x})^2 = (-2)^2$ or $x = 4$.  However, $x = 4$ doesn't check in the original equation, since $\sqrt{4} = 2$, not $-2$.  Once again, the root\footnote{Pun intended!} of all of our problems lies in the fact that a \textit{negative} number to an \textit{even} power results in a \textit{positive} number. In other words, raising both sides of an equation to an even power does \textit{not} produce an equivalent equation, but rather, an equation which may possess \textit{more} solutions than the original.  Hence the cautionary remark above about extraneous solutions.

\pagebreak

\begin{ex}\label{radicaleqnreview}  Solve the following equations.


\begin{multicols}{3}
\begin{enumerate}


\item  $(5x +3)^{4} = 16$\vphantom{$1 - \dfrac{(5-2w)^3}{7} = 9$}

%\item  $1 - \dfrac{(5-2w)^3}{7} = 9$\vphantom{$1 - \dfrac{(5-2w)^3}{7} = 9$}

\item  $t + \sqrt{2t+3} = 6$\vphantom{$1 - \dfrac{(5-2w)^3}{7} = 9$}

\item  $\sqrt{4x-1}  + 2\sqrt{1 - 2x} = 1$


\setcounter{HW}{\value{enumi}}
\end{enumerate}
\end{multicols}
%\begin{enumerate}
%\setcounter{enumi}{\value{HW}}


%\setcounter{HW}{\value{enumi}}
%\end{enumerate}

{\bf Solution.}

\begin{enumerate}

\item  In our first equation, the quantity containing $x$ is already isolated, so we extract fourth roots. Since the exponent here is even, when the roots are extracted we need both the positive and negative roots. \[ \begin{array}{rclr}

(5x +3)^{4} & = & 16 & \\ [2pt]

5x+3 & = & \pm \sqrt[4]{16} & \text{Extract fourth roots} \\ [2pt]

5x + 3 & = & \pm 2 & \\ [2pt]

5x+3 = 2 & \text{or} & 5x+3 = -2 & \\

x = -\dfrac{1}{5} & \text{or} & x = -1 \\ \end{array} \] We leave it to the reader that both of these solutions satisfy the original equation.

%\item  In this example, we first need to isolate the quantity containing the variable $w$.  Here, third (cube) roots are required and since the exponent (index) is odd, we do not need the $\pm$:\[ \begin{array}{rclr} 

%1 - \dfrac{(5-2w)^3}{7} & = &  9 & \\ [8pt]


%- \dfrac{(5-2w)^3}{7} & = & 8 & \text{Subtract $1$} \\[8pt]

%(5-2w) ^ 3 & = & -56 & \text{Multiply by $-7$} \\[2pt]

%5 - 2w & = & \sqrt[3]{-56} & \text{Extract cube root} \\[2pt]

%5 - 2w & = & \sqrt[3]{(-8)(7)} & \\[2pt]

%5 - 2w & = & \sqrt[3]{-8} \sqrt[3]{7} & \text{Product Rule}\\[2pt]

%5 - 2w & = & -2\sqrt[3]{7} & \\[2pt]

%-2w & = & -5-2 \sqrt[3]{7} & \text{Subtract $5$} \\[2pt]

%w & = & \dfrac{-5 - 2\sqrt[3]{7}}{-2} & \text{Divide by $-2$} \\[8pt]

%w & = & \dfrac{5 + 2\sqrt[3]{7}}{2} & \text{Properties of Negatives} \\

%\end{array}\] The reader should check the answer because it provides a hearty review of arithmetic.

\item  To solve  $t + \sqrt{2t+3} = 6$, we first isolate the square root, then proceed to square both sides of the equation.  In doing so, we run the risk of introducing extraneous solutions so checking our answers here is a necessity. \[ \begin{array}{rclr}

t + \sqrt{2t+3}  & = & 6 & \\ [2pt]

\sqrt{2t+3} & = & 6 - t & \text{Subtract $t$} \\ [2pt]

(\sqrt{2t+3})^2 & = & (6-t)^2 & \text{Square both sides} \\ [2pt]

2t + 3 & = & 36-12t + t^2 & \text{F.O.I.L. / Perfect Square Trinomial} \\ [2pt]

0 & = & t^2 - 14t + 33 & \text{Subtract $2t$ and $3$} \\ [2pt]

0 & = & (t-3)(t-11) & \text{Factor} \\ \end{array} \] From the Zero Product Property, we know either $t - 3 = 0$ (which gives $t=3$) or $t-11 = 0$ (which gives $t=11$).  When checking our answers, we find $t = 3$ satisfies the original equation, but $t = 11$ does not.\footnote{It is worth noting that when $t=11$ is substituted into the original equation, we get $11 + \sqrt{25} = 6$.  If the $+\sqrt{25}$ were $-\sqrt{25}$, the solution would check. Once again, when squaring both sides of an equation, we lose track of $\pm$, which is what lets extraneous solutions in the door.}  So our final answer is $t = 3$ only.



\item In the equation $\sqrt{4x-1}  + 2\sqrt{1 - 2x} = 1$, we have not one but two square roots.  We begin by isolating one of the square roots and squaring both sides.\[ \begin{array}{rclr}

\sqrt{4x-1}  + 2\sqrt{1 - 2x} & = & 1 & \\ [2pt]

\sqrt{4x-1} & = & 1 - 2\sqrt{1-2x} & \hspace*{-.1in} \text{Subtract $2\sqrt{1 - 2x}$ from both sides} \\[2pt]

(\sqrt{4x-1})^2 & = & (1 - 2\sqrt{1-2x})^2 & \text{Square both sides} \\[2pt]

4x - 1 & = & 1 - 4\sqrt{1-2x} + (2\sqrt{1-2x})^2 & \hspace*{-.1in} \text{F.O.I.L. / Perfect Square Trinomial} \\[2pt]

4x - 1 & = & 1 - 4\sqrt{1-2x} + 4(1-2x) & \\[2pt]

4x - 1 & = & 1 - 4\sqrt{1-2x} + 4 - 8x & \text{Distribute} \\ [2pt]

4x - 1 & = & 5 - 8x - 4\sqrt{1-2x} & \text{Gather like terms} \\ \end{array} \] At this point, we have just one square root so we proceed to isolate it and square both sides a second time.\footnote{To avoid complications with fractions, we'll forego dividing by the coefficient of $\sqrt{1-2x}$, namely $-4$. This is perfectly fine so long as we don't forget to square it when we square both sides of the equation.} \[ \begin{array}{rclr}

4x - 1 & = & 5 - 8x - 4\sqrt{1-2x} &  \\ [2pt]

12x - 6 & = & -4\sqrt{1-2x} & \text{Subtract $5$, add $8x$}\\ [2pt]

(12x-6)^2 & = & (-4\sqrt{1-2x})^2 & \text{Square both sides} \\[2pt]

144x^2 - 144x + 36 & = & 16(1-2x) & \\ [2pt]

144x^2 -  144x + 36 & = & 16 - 32x & \\[2pt]

144x^2 - 112x + 20 & = & 0 & \text{Subtract $16$, add $32x$} \\[2pt]

4(36x^2 - 28x + 5) & = & 0 & \text{Factor} \\[2pt]

4(2x-1)(18x - 5) & = & 0 & \text{Factor some more} \\

\end{array} \] From the Zero Product Property, we know either $2x-1 = 0$ or $18x - 5 = 0$.  The former gives $x = \frac{1}{2}$ while the latter gives us $x = \frac{5}{18}$.  Since we squared both sides of the equation (twice!), we need to check for extraneous solutions.  We find $x = \frac{5}{18}$ to be extraneous, so our only solution is $x = \frac{1}{2}$.\qed




\end{enumerate}

\end{ex}

\subsection{Rationalizing Denominators and Numerators}

In Section \ref{QuadEqus}, there were a few instances where we needed to `rationalize' a denominator - that is, take a fraction with radical in the denominator and re-write it as an equivalent fraction without a radical in the denominator.  There are various reasons for wanting to do this,\footnote{Before the advent of the hand-held calculator, rationalizing denominators made it easier to get decimal approximations to fractions containing radicals.} but the most pressing reason is that rationalizing denominators - and numerators as well - gives us an opportunity for more practice with fractions and radicals. To help refresh your memory, we rationalize a denominator and then a numerator below: \[ \dfrac{1}{\sqrt{2}} = \dfrac{\sqrt{2}}{\sqrt{2} \sqrt{2}} = \dfrac{\sqrt{2}}{\sqrt{4}} = \dfrac{\sqrt{2}}{2} \quad \text{and} \quad \dfrac{7\sqrt[3]{4}}{3} = \dfrac{7 \sqrt[3]{4}\sqrt[3]{2}}{3\sqrt[3]{2}} = \dfrac{7\sqrt[3]{8}}{3\sqrt[3]{2}} = \dfrac{7 \cdot 2}{3\sqrt[3]{2}} =  \dfrac{14}{3\sqrt[3]{2}} \]

 In general, if the fraction contains either a single term numerator or denominator with an undesirable $n^{\text{th}}$ root, we multiply the numerator and denominator by whatever is required to obtain a perfect $n^{\text{th}}$ power in the radicand that we want to eliminate. If the fraction contains two terms the situation is somewhat more complicated.  To see why, consider the fraction $\frac{3}{4 - \sqrt{5}}$.  Suppose we wanted to rid the denominator of the $\sqrt{5}$ term.  We could try as above and multiply numerator and denominator by $\sqrt{5}$ but that just yields: \[ \dfrac{3}{4 - \sqrt{5}} = \dfrac{3\sqrt{5}}{(4 - \sqrt{5})\sqrt{5}} = \dfrac{3\sqrt{5}}{4\sqrt{5} - \sqrt{5}\sqrt{5}} = \dfrac{3\sqrt{5}}{4\sqrt{5} - 5}\] We haven't removed $\sqrt{5}$ from the denominator - we've just shuffled it over to the other term in the denominator.  As you may recall, the strategy here is to multiply both numerator and denominator by what's called the \textbf{conjugate}\index{conjugate}.  

\medskip

\colorbox{ResultColor}{\bbm

\begin{defn}\label{squarerootconj} \textbf{Congugate of a Square Root Expression:}  If $a$, $b$ and $c$ are real numbers with $c > 0$ then the quantities $(a + b \sqrt{c})$ and $(a - b\sqrt{c})$ are \textbf{conjugates} of one another.\footnote{As are $(b\sqrt{c} -a)$ and $(b\sqrt{c} + a)$: $(b\sqrt{c} -a)(b\sqrt{c} + a) = b^2c - a^2$.}  Conjugates multiply according to the Difference of Squares Formula:  \[ (a + b \sqrt{c})(a - b\sqrt{c}) = a^2 - (b \sqrt{c})^2 = a^2 - b^2c\]
\end{defn}
\ebm}

\medskip

That is, to get the conjugate of a two-term expression involving a square root, you change the `$-$' to a `$+$,' or vice-versa.  For example, the conjugate of $4 - \sqrt{5}$ is $4 + \sqrt{5}$, and when we multiply these two factors together, we get $(4 - \sqrt{5})(4 + \sqrt{5}) = 4^2 - (\sqrt{5})^2 = 16 - 5 = 11$.  Hence, to eliminate the $\sqrt{5}$ from the denominator of our original fraction, we multiply both the numerator and denominator by the \textit{conjugate} of $4-\sqrt{5}$: \[\dfrac{3}{4 - \sqrt{5}} = \dfrac{3 (4 + \sqrt{5})}{(4 - \sqrt{5})(4 + \sqrt{5})} = \dfrac{3 (4 + \sqrt{5})}{4^2 - (\sqrt{5})^2} = \dfrac{3(4 + \sqrt{5})}{16 - 5} = \dfrac{12 + 3\sqrt{5}}{11}\] 

What if we had $\sqrt[3]{5}$ instead of $\sqrt{5}$?  We could try multiplying $4 - \sqrt[3]{5}$ by $4 + \sqrt[3]{5}$ to get  \[(4 - \sqrt[3]{5})(4 + \sqrt[3]{5}) = 4^2 - (\sqrt[3]{5})^2 = 16 - \sqrt[3]{25},\]
which leaves us with a cube root.  What we need to undo the cube root is a perfect cube, which means we look to the Difference of Cubes Formula for inspiration:  $a^3 - b^3 = (a-b)(a^2+ab+b^2)$.  If we take $a = 4$ and $b = \sqrt[3]{5}$, we multiply \[ (4 - \sqrt[3]{5})(4^2 + 4\sqrt[3]{5} + (\sqrt[3]{5})^2) = 4^3 + 4^2\sqrt[3]{5} + 4 \sqrt[3]{5} - 4^2\sqrt[3]{5}-4(\sqrt[3]{5})^2 - (\sqrt[3]{5})^3 = 64 - 5 = 59\]
So if we were charged with rationalizing the denominator of $\frac{3}{4 - \sqrt[3]{5}}$, we'd have:

\[ \dfrac{3}{4 - \sqrt[3]{5}} = \dfrac{3(4^2 + 4\sqrt[3]{5} + (\sqrt[3]{5})^2)}{(4 - \sqrt[3]{5})(4^2 + 4\sqrt[3]{5} + (\sqrt[3]{5})^2)} = \dfrac{48 + 12\sqrt[3]{5}+ 3\sqrt[3]{25}}{59}\]

This sort of thing extends to $n^{\text{th}}$ roots since $(a-b)$ is a factor of $a^n - b^n$ for all natural numbers $n$, but in practice, we'll stick with square roots with just a few cube roots thrown in for a challenge.\footnote{To see what to do about fourth roots, use long division to find $(a^4 - b^4) \div (a-b)$, and apply this to $4 - \sqrt[4]{5}$.}

\begin{ex} \label{rationalizenumdenom} Rationalize the indicated numerator or denominator:

\begin{multicols}{2}
\begin{enumerate}

\item  Rationalize the denominator:  $\dfrac{3}{\sqrt[5]{24x^2}}$

\item  Rationalize the numerator: $\dfrac{\sqrt{9 + h} - 3}{h}$

\setcounter{HW}{\value{enumi}}

\end{enumerate}
\end{multicols}

{\bf Solution.}

\begin{enumerate}

\item We are asked to rationalize the denominator, which in this case contains a fifth root.  That means we need to work to create fifth powers of each of the factors of the radicand.  To do so, we first factor the radicand:  $24x^2 = 8 \cdot 3 \cdot x^2 = 2^3 \cdot 3 \cdot x^2$.  To obtain fifth powers, we need to multiply by $2^2 \cdot 3^4 \cdot x^3$ inside the radical.  \[ \begin{array}{rclr}

\dfrac{3}{\sqrt[6]{24x^2}} & = & \dfrac{3}{\sqrt[5]{2^3 \cdot 3 \cdot x^2}} & \\ [12pt]
                           & = & \dfrac{3 \sqrt[5]{2^2 \cdot 3^4 \cdot x^3}}{\sqrt[5]{2^3 \cdot 3 \cdot x^2}\sqrt[5]{2^2 \cdot 3^4 \cdot x^3}} & \text{Equivalent Fractions} \\[12pt]
												& = & \dfrac{3 \sqrt[5]{2^2 \cdot 3^4 \cdot x^3}}{\sqrt[5]{2^3 \cdot 3 \cdot x^2 \cdot 2^2 \cdot 3^4 \cdot x^3}} & \text{Product Rule} \\[12pt]
													& = & \dfrac{3 \sqrt[5]{2^2 \cdot 3^4 \cdot x^3}}{\sqrt[5]{2^5 \cdot 3^5 \cdot x^5}} & \text{Property of Exponents}\\[12pt]
													
													& = & \dfrac{3 \sqrt[5]{2^2 \cdot 3^4 \cdot x^3}}{\sqrt[5]{2^5} \sqrt[5]{3^5}\sqrt[5]{x^5}} & \text{Product Rule}\\[12pt]
                          & = & \dfrac{3 \sqrt[5]{2^2 \cdot 3^4\cdot x^3}}{2 \cdot 3 \cdot x} & \text{Product Rule}\\[12pt]
													& = & \dfrac{\cancel{3} \sqrt[5]{4 \cdot 81\cdot x^3}}{2 \cdot \cancel{3} \cdot x} & \text{Reduce}\\[12pt]
													& = & \dfrac{\sqrt[5]{324x^3}}{2x} & \text{Simplify}\\
													\end{array}\]
													
\item  Here, we are asked to rationalize the \textit{numerator}.  Since it is a two term numerator involving a square root, we multiply both numerator and denominator by the conjugate of $\sqrt{9 + h} - 3$, namely $\sqrt{9 + h} + 3$.  After simplifying, we find an opportunity to reduce the fraction:\[\begin{array}{rclr}

\dfrac{\sqrt{9 + h} - 3}{h} & = & \dfrac{(\sqrt{9 + h} - 3)(\sqrt{9 + h} + 3)}{h(\sqrt{9 + h} + 3)} & \text{Equivalent Fractions} \\[12pt]

                             & = & \dfrac{(\sqrt{9+h})^2 - 3^2}{h(\sqrt{9 + h} + 3)} & \text{Difference of Squares} \\[12pt]
														 & = & \dfrac{(9+h) - 9}{h(\sqrt{9 + h} + 3)} & \text{Simplify} \\[12pt]
														 & = & \dfrac{h}{h(\sqrt{9 + h} + 3)} & \text{Simplify} \\[12pt]
														 & = & \dfrac{\cancelto{1}{h}}{\cancel{h}(\sqrt{9 + h} + 3)} & \text{Reduce}\\[12pt]
														& = & \dfrac{1}{\sqrt{9 + h} + 3} & \\

\end{array} \]

\end{enumerate}

\end{ex}

We close this section with an awesome example from Calculus.%\footnote{Slaying this expert-level Algebra boss is not for the fainthearted.  As an added challenge, we are not giving you the labels on the operations we perform at each step.  You and your brave Precalculus warrior-classmates should fill in those gaps on your own.}

\begin{ex} \label{rationalizenumdenombosslevel}

Simplify the compound fraction $\dfrac{\dfrac{1}{\sqrt{2(x+h)+1}} - \dfrac{1}{\sqrt{2x+1}}}{h}$ then rationalize the numerator of the result.

\medskip

{\bf Solution.} We start by multiplying the top and bottom of the `big' fraction by $\sqrt{2x+2h+1} \sqrt{2x+1}$.  \[ \begin{array}{rcl}

\dfrac{\dfrac{1}{\sqrt{2(x+h)+1}} - \dfrac{1}{\sqrt{2x+1}}}{h} & = & \dfrac{\dfrac{1}{\sqrt{2x+2h+1}} - \dfrac{1}{\sqrt{2x+1}}}{h}\\[10pt]

                                                               & = & \dfrac{\left(\dfrac{1}{\sqrt{2x+2h+1}} - \dfrac{1}{\sqrt{2x+1}}\right) \sqrt{2x+2h+1} \sqrt{2x+1}}{h\sqrt{2x+2h+1} \sqrt{2x+1}}\\[22pt]
																															
																															& = & \dfrac{\dfrac{\cancel{\sqrt{2x+2h+1}}\sqrt{2x+1}}{\cancel{\sqrt{2x+2h+1}}} - \dfrac{\sqrt{2x+2h+1} \cancel{\sqrt{2x+1}}}{\cancel{\sqrt{2x+1}}}}{h\sqrt{2x+2h+1} \sqrt{2x+1}}\\[22pt]
																																			
																															& = & \dfrac{\sqrt{2x + 1}- \sqrt{2x+2h+1}}{h\sqrt{2x+2h+1} \sqrt{2x+1}}\\	
																															\end{array}\]
																															
Next, we multiply the numerator and denominator by the conjugate of $\sqrt{2x+1} - \sqrt{2x+2h+1}$; namely,\\
 $\sqrt{2x+1} + \sqrt{2x+2h+1}$, simplify and reduce:\[\begin{array}{rcl}

 \dfrac{\sqrt{2x + 1}- \sqrt{2x+2h+1}}{h\sqrt{2x+2h+1} \sqrt{2x+1}} & = & \dfrac{(\sqrt{2x+1} - \sqrt{2x+2h+1})(\sqrt{2x+1} + \sqrt{2x+2h+1})}{h\sqrt{2x+2h+1} \sqrt{2x+1}(\sqrt{2x+1} + \sqrt{2x+2h+1})} \\ [20pt]

 & = & \dfrac{(\sqrt{2x+1})^2 - (\sqrt{2x+2h+1})^2}{h\sqrt{2x+2h+1} \sqrt{2x+1}(\sqrt{2x+1} + \sqrt{2x+2h+1})} \\[20pt]

 & = & \dfrac{(2x+1) - (2x+2h+1)}{h\sqrt{2x+2h+1} \sqrt{2x+1}(\sqrt{2x+1} + \sqrt{2x+2h+1})} \\[20pt]

& = & \dfrac{2x+1 -2x-2h-1}{h\sqrt{2x+2h+1} \sqrt{2x+1}(\sqrt{2x+1} + \sqrt{2x+2h+1})} \\[20pt]

& = & \dfrac{-2\cancel{h}}{\cancel{h}\sqrt{2x+2h+1} \sqrt{2x+1}(\sqrt{2x+1} + \sqrt{2x+2h+1})} \\[20pt]
& = & \dfrac{-2}{\sqrt{2x+2h+1} \sqrt{2x+1}(\sqrt{2x+1} + \sqrt{2x+2h+1})} \\
\end{array}\] 

While the denominator is quite a bit more complicated than what we started with, we have done what was asked of us.  In the interest of full disclosure, the reason we did all of this was to cancel the original `$h$' from the denominator. That's an awful lot of effort to get rid of just one little $h$, but you'll see the significance of this in Calculus.\qed

\end{ex}

\newpage

\subsection{Exercises}

In Exercises \ref{simpradfirst} - \ref{simpradlast}, perform the indicated operations and simplify.

\begin{multicols}{3}
\begin{enumerate}

\item   $\sqrt{9x^2}$ \label{simpradfirst}

\item   $\sqrt[3]{8t^3}$

\item   $\sqrt{4t^2 + 4t + 1}$


\setcounter{HW}{\value{enumi}}
\end{enumerate}
\end{multicols}



\begin{multicols}{2}
\begin{enumerate}
\setcounter{enumi}{\value{HW}}

\item  $\sqrt{x} - \dfrac{x+1}{\sqrt{x}}$\vphantom{$3 \sqrt{1-t^2} + 3t\left(\dfrac{1}{2 \sqrt{1-t^2}}\right)(-2t)$}

\item  $3 \sqrt{1-t^2} + 3t\left(\dfrac{1}{2 \sqrt{1-t^2}}\right)(-2t)$

\setcounter{HW}{\value{enumi}}
\end{enumerate}
\end{multicols}

\begin{multicols}{2}
\begin{enumerate}
\setcounter{enumi}{\value{HW}}

\item  $\sqrt{(\sqrt{12x} - \sqrt{3x})^2+1}$

\item  $\dfrac{3}{\sqrt[3]{2x-1}} + (3x)\left(-\dfrac{1}{3 \left(\sqrt[3]{2x-1} \right)^4}\right)(2)$  \label{simpradlast}

\setcounter{HW}{\value{enumi}}
\end{enumerate}
\end{multicols}



In Exercises \ref{algineqexfirst0} - \ref{algineqexlast0}, find all real solutions.

\begin{multicols}{3}
\begin{enumerate}
\setcounter{enumi}{\value{HW}}

\item  $(2x+1)^3 + 8 = 0$ \label{algineqexfirst0} \vphantom{ $\dfrac{1}{1 + 2t^3} = 4$}
\item $\sqrt{3x+1} = 4$ \vphantom{ $\dfrac{1}{1 + 2t^3} = 4$}
\item  $\dfrac{1}{1 + 2t^3} = 4$ 


\setcounter{HW}{\value{enumi}}
\end{enumerate}
\end{multicols}

\begin{multicols}{3}
\begin{enumerate}
\setcounter{enumi}{\value{HW}}

\item  $y + \sqrt{3y+10} = -2$
\item $\sqrt{x - 2} + \sqrt{x - 5} = 3$
\item $x+1 = \sqrt{3x+7}$ \label{algineqexlast0}% $x=3$     

\setcounter{HW}{\value{enumi}}
\end{enumerate}
\end{multicols}




In Exercises \ref{rationalizefirst} - \ref{rationalizelast}, rationalize the numerator or denominator, and simplify.


\begin{multicols}{3}
\begin{enumerate}
\setcounter{enumi}{\value{HW}}


\item   $\dfrac{4}{3 - \sqrt{2}}$ \vphantom{$\dfrac{7}{\sqrt[3]{12x^7}}$} \label{rationalizefirst}

\item  $\dfrac{7}{\sqrt[3]{12x^7}}$

\item  $\dfrac{\sqrt{2x+2h+1} - \sqrt{2x+1}}{h}$ \label{rationalizelast}
 \vphantom{$\dfrac{7}{\sqrt[3]{12x^7}}$}

\setcounter{HW}{\value{enumi}}
\end{enumerate}
\end{multicols}






\closegraphsfile

\newpage

